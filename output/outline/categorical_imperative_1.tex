%
\begin{isabellebody}%
\setisabellecontext{categorical{\isacharunderscore}imperative{\isacharunderscore}{\isadigit{1}}}%
%
\isadelimtheory
%
\endisadelimtheory
%
\isatagtheory
\isacommand{theory}\isamarkupfalse%
\ categorical{\isacharunderscore}imperative{\isacharunderscore}{\isadigit{1}}\ \isakeyword{imports}\ carmojones{\isacharunderscore}DDL{\isacharunderscore}completeness\isanewline
\isanewline
\isakeyword{begin}%
\endisatagtheory
{\isafoldtheory}%
%
\isadelimtheory
%
\endisadelimtheory
%
\isadelimdocument
%
\endisadelimdocument
%
\isatagdocument
%
\isamarkupsection{The Categorical Imperative%
}
\isamarkuptrue%
%
\isamarkupsubsection{Simple Formulation of the Kingdom of Ends%
}
\isamarkuptrue%
%
\endisatagdocument
{\isafolddocument}%
%
\isadelimdocument
%
\endisadelimdocument
%
\begin{isamarkuptext}%
This is my first attempt at formalizing the concept of the Kingdom of Ends%
\end{isamarkuptext}\isamarkuptrue%
%
\begin{isamarkuptext}%
NOTE: this attempt revealed a bug in my embedding. I've included it as an artifact, but none of these theorems hold anymore (hence the oops).%
\end{isamarkuptext}\isamarkuptrue%
\isacommand{abbreviation}\isamarkupfalse%
\ ddlpermissable{\isacharcolon}{\isacharcolon}{\isachardoublequoteopen}t{\isasymRightarrow}t{\isachardoublequoteclose}\ {\isacharparenleft}{\isachardoublequoteopen}P{\isacharunderscore}{\isachardoublequoteclose}{\isacharparenright}\isanewline
\ \ \isakeyword{where}\ {\isachardoublequoteopen}{\isacharparenleft}P\ A{\isacharparenright}\ {\isasymequiv}\ {\isacharparenleft}\isactrlbold {\isasymnot}{\isacharparenleft}O\ {\isacharbraceleft}\isactrlbold {\isasymnot}A{\isacharbraceright}{\isacharparenright}{\isacharparenright}{\isachardoublequoteclose}\isanewline
%
\isamarkupcmt{This operator represents permissibility%
}\isanewline
%
\isamarkupcmt{Will be useful when discussing the categorical imperative%
}\isanewline
%
\isamarkupcmt{Something is permissible if it is not prohibited%
}\isanewline
%
\isamarkupcmt{Something is prohibited if its negation is obligatory%
}\isanewline
\isanewline
\isanewline
\isacommand{lemma}\isamarkupfalse%
\ kingdom{\isacharunderscore}of{\isacharunderscore}ends{\isacharunderscore}{\isadigit{1}}{\isacharcolon}\isanewline
\ \ \isakeyword{shows}\ {\isachardoublequoteopen}{\isasymTurnstile}\ {\isacharparenleft}{\isacharparenleft}O\ {\isacharbraceleft}A{\isacharbraceright}{\isacharparenright}\ \isactrlbold {\isasymrightarrow}\ {\isacharparenleft}{\isasymbox}\ {\isacharparenleft}P\ A{\isacharparenright}{\isacharparenright}{\isacharparenright}{\isachardoublequoteclose}\isanewline
%
\isadelimproof
\ \ %
\endisadelimproof
%
\isatagproof
\isacommand{oops}\isamarkupfalse%
\isanewline
%
\isamarkupcmt{One interpretation of the categorical imperative is that something is obligatory only if it is permissible in every ideal world%
}\isanewline
%
\isamarkupcmt{This formulation mirrors the kingdom of ends.%
}\isanewline
%
\isamarkupcmt{This formulation is already a theorem of carmo and jones' DDL!.%
}\isanewline
%
\isamarkupcmt{It can be shown using the O diamond rule, which just says that obligatory things must be possible.%
}\isanewline
%
\isamarkupcmt{There are two possibilities: either the logic is already quite powerful OR this formulation is ``empty".%
}%
\endisatagproof
{\isafoldproof}%
%
\isadelimproof
\isanewline
%
\endisadelimproof
\isanewline
\isanewline
\isacommand{lemma}\isamarkupfalse%
\ kingdom{\isacharunderscore}of{\isacharunderscore}ends{\isacharunderscore}{\isadigit{2}}{\isacharcolon}\isanewline
\ \ \isakeyword{shows}\ {\isachardoublequoteopen}{\isasymTurnstile}\ {\isacharparenleft}{\isacharparenleft}{\isasymbox}\ {\isacharparenleft}P\ A{\isacharparenright}{\isacharparenright}\ \isactrlbold {\isasymrightarrow}\ {\isacharparenleft}O\ {\isacharbraceleft}A{\isacharbraceright}{\isacharparenright}{\isacharparenright}{\isachardoublequoteclose}\isanewline
%
\isadelimproof
\ \ %
\endisadelimproof
%
\isatagproof
\isacommand{oops}\isamarkupfalse%
\isanewline
\isanewline
%
\isamarkupcmt{Notice also that ideally, this relationship does not hold in the reverse direction.%
}\isanewline
%
\isamarkupcmt{Plenty of things are necessarily permissible (drinking water) but not obligatory.%
}\isanewline
%
\isamarkupcmt{Very strange that this is a theorem in this logic.....%
}\isanewline
%
\isamarkupcmt{That being said, Isabelle seems quite upset with this proof and is very slow to resconstruct it%
}\isanewline
%
\isamarkupcmt{I am struggling to recreate this proof on paper%
}%
\endisatagproof
{\isafoldproof}%
%
\isadelimproof
\isanewline
%
\endisadelimproof
\isanewline
\isanewline
\isacommand{lemma}\isamarkupfalse%
\ permissible{\isacharunderscore}to{\isacharunderscore}ob{\isacharcolon}\isanewline
\ \ \isakeyword{shows}\ {\isachardoublequoteopen}{\isasymTurnstile}\ {\isacharparenleft}{\isacharparenleft}P\ A{\isacharparenright}\ \isactrlbold {\isasymrightarrow}\ {\isacharparenleft}O\ {\isacharbraceleft}A{\isacharbraceright}{\isacharparenright}{\isacharparenright}{\isachardoublequoteclose}\isanewline
%
\isadelimproof
\ \ %
\endisadelimproof
%
\isatagproof
\isacommand{oops}\isamarkupfalse%
\isanewline
%
\isamarkupcmt{Uh-oh.....this shouldn't be true...%
}\isanewline
%
\isamarkupcmt{Not all permissable things are obligatory.....%
}%
\endisatagproof
{\isafoldproof}%
%
\isadelimproof
\isanewline
%
\endisadelimproof
\isanewline
\isacommand{lemma}\isamarkupfalse%
\ weaker{\isacharunderscore}permissible{\isacharunderscore}to{\isacharunderscore}ob{\isacharcolon}\isanewline
\ \ \isakeyword{shows}\ {\isachardoublequoteopen}{\isasymTurnstile}\ {\isacharparenleft}{\isacharparenleft}{\isasymdiamond}\ {\isacharparenleft}P\ A{\isacharparenright}{\isacharparenright}\ \isactrlbold {\isasymrightarrow}\ O\ {\isacharbraceleft}A{\isacharbraceright}{\isacharparenright}{\isachardoublequoteclose}\isanewline
%
\isadelimproof
\ \ \ \ %
\endisadelimproof
%
\isatagproof
\isacommand{oops}\isamarkupfalse%
\isanewline
%
\isamarkupcmt{Makes sense that this follows from the reverse kingdom of ends.%
}\isanewline
%
\isamarkupcmt{Obligation and necessity/possibility are separated in this logic%
}\isanewline
%
\isamarkupcmt{Both the dyadic obligation and necessity operator are world agnostic%
}%
\endisatagproof
{\isafoldproof}%
%
\isadelimproof
\isanewline
%
\endisadelimproof
\isanewline
\isacommand{lemma}\isamarkupfalse%
\ contradictory{\isacharunderscore}obligations{\isacharcolon}\isanewline
\ \ \isakeyword{shows}\ {\isachardoublequoteopen}{\isasymTurnstile}{\isacharparenleft}\isactrlbold {\isasymnot}\ {\isacharparenleft}{\isacharparenleft}O\ {\isacharbraceleft}A{\isacharbraceright}{\isacharparenright}\ \isactrlbold {\isasymand}\ {\isacharparenleft}O\ {\isacharbraceleft}\isactrlbold {\isasymnot}\ A{\isacharbraceright}{\isacharparenright}{\isacharparenright}{\isacharparenright}{\isachardoublequoteclose}\isanewline
%
\isadelimproof
\ \ %
\endisadelimproof
%
\isatagproof
\isacommand{oops}\isamarkupfalse%
\isanewline
%
\isamarkupcmt{What is the cause of the above strangeness?%
}\isanewline
%
\isamarkupcmt{This very intuitive theorem holds in my logic but not in Benzmuller Parent's%
}\isanewline
%
\isamarkupcmt{It's clear that this theorem results in the strange results above.%
}\isanewline
%
\isamarkupcmt{Conclusion: There is a bug in my embedding%
}%
\endisatagproof
{\isafoldproof}%
%
\isadelimproof
%
\endisadelimproof
%
\begin{isamarkuptext}%
Sidebar: the above theorem is really intuitive - it seems like we wouldn't want 
contradictory things to be obligatory in any logic. But for some reason, not only is it not
a theorem of Carmo and Jones' logic, it actually implies some strange conclusions, including 
that everything is either permissible or obligatory. It's not clear to me from a semantic 
perspective why this would be the case. In fact this theorem seems like a desirable 
property. Potential avenue for exploration%
\end{isamarkuptext}\isamarkuptrue%
%
\begin{isamarkuptext}%
Did some debugging. What was the problem? A misplaced parentheses in the definition 
of ax5b that led to a term being on the wrong side of an implication. Computer Science :(%
\end{isamarkuptext}\isamarkuptrue%
%
\begin{isamarkuptext}%
After the debugging, all of this is no longer true! On to the next attempt :)%
\end{isamarkuptext}\isamarkuptrue%
%
\isadelimtheory
%
\endisadelimtheory
%
\isatagtheory
\isacommand{end}\isamarkupfalse%
%
\endisatagtheory
{\isafoldtheory}%
%
\isadelimtheory
%
\endisadelimtheory
%
\end{isabellebody}%
\endinput
%:%file=~/Desktop/cs91r/categorical_imperative_1.thy%:%
%:%10=1%:%
%:%11=1%:%
%:%12=2%:%
%:%13=3%:%
%:%27=5%:%
%:%31=7%:%
%:%43=9%:%
%:%47=10%:%
%:%49=12%:%
%:%50=12%:%
%:%51=13%:%
%:%53=14%:%
%:%54=14%:%
%:%56=15%:%
%:%57=15%:%
%:%59=16%:%
%:%60=16%:%
%:%62=17%:%
%:%63=17%:%
%:%64=18%:%
%:%65=19%:%
%:%66=20%:%
%:%67=20%:%
%:%68=21%:%
%:%71=22%:%
%:%75=22%:%
%:%76=22%:%
%:%78=23%:%
%:%79=23%:%
%:%81=24%:%
%:%82=24%:%
%:%84=25%:%
%:%85=25%:%
%:%87=26%:%
%:%88=26%:%
%:%90=27%:%
%:%96=27%:%
%:%99=28%:%
%:%100=29%:%
%:%101=30%:%
%:%102=30%:%
%:%103=31%:%
%:%106=32%:%
%:%110=32%:%
%:%111=32%:%
%:%112=33%:%
%:%114=34%:%
%:%115=34%:%
%:%117=35%:%
%:%118=35%:%
%:%120=36%:%
%:%121=36%:%
%:%123=37%:%
%:%124=37%:%
%:%126=38%:%
%:%132=38%:%
%:%135=39%:%
%:%136=40%:%
%:%137=41%:%
%:%138=41%:%
%:%139=42%:%
%:%142=43%:%
%:%146=43%:%
%:%147=43%:%
%:%149=44%:%
%:%150=44%:%
%:%152=45%:%
%:%158=45%:%
%:%161=46%:%
%:%162=47%:%
%:%163=47%:%
%:%164=48%:%
%:%167=49%:%
%:%171=49%:%
%:%172=49%:%
%:%174=50%:%
%:%175=50%:%
%:%177=51%:%
%:%178=51%:%
%:%180=52%:%
%:%186=52%:%
%:%189=53%:%
%:%190=54%:%
%:%191=54%:%
%:%192=55%:%
%:%195=56%:%
%:%199=56%:%
%:%200=56%:%
%:%202=57%:%
%:%203=57%:%
%:%205=58%:%
%:%206=58%:%
%:%208=59%:%
%:%209=59%:%
%:%211=60%:%
%:%221=63%:%
%:%222=64%:%
%:%223=65%:%
%:%224=66%:%
%:%225=67%:%
%:%226=68%:%
%:%230=70%:%
%:%231=71%:%
%:%235=73%:%
%:%243=75%:%