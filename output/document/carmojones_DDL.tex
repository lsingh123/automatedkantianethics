%
\begin{isabellebody}%
\setisabellecontext{carmojones{\isacharunderscore}DDL}%
%
\begin{isamarkuptext}%
Referencing Benzmuller and Parent's implementation: https://www.mi.fu-berlin.de/inf/groups/ag-ki/publications/dyadic-deontic-logic/C71.pdf%
\end{isamarkuptext}\isamarkuptrue%
%
\begin{isamarkuptext}%
This theory contains the axiomatization of the system and some useful abbreviations.%
\end{isamarkuptext}\isamarkuptrue%
%
\isadelimtheory
%
\endisadelimtheory
%
\isatagtheory
\isacommand{theory}\isamarkupfalse%
\ carmojones{\isacharunderscore}DDL\isanewline
\ \ \isakeyword{imports}\ \isanewline
\ \ \ \ Main\ \isanewline
\isanewline
\isakeyword{begin}%
\endisatagtheory
{\isafoldtheory}%
%
\isadelimtheory
%
\endisadelimtheory
%
\isadelimdocument
%
\endisadelimdocument
%
\isatagdocument
%
\isamarkupsection{System Definition%
}
\isamarkuptrue%
%
\isamarkupsubsection{Definitions%
}
\isamarkuptrue%
%
\endisatagdocument
{\isafolddocument}%
%
\isadelimdocument
%
\endisadelimdocument
%
\begin{isamarkuptext}%
This section contains definitions and constants necessary to construct a DDL model.%
\end{isamarkuptext}\isamarkuptrue%
\isacommand{typedecl}\isamarkupfalse%
\ i\ %
\isamarkupcmt{i is the type for a set of possible worlds."%
}\isanewline
\isanewline
\isacommand{type{\isacharunderscore}synonym}\isamarkupfalse%
\ t\ {\isacharequal}\ {\isachardoublequoteopen}{\isacharparenleft}i\ {\isasymRightarrow}\ bool{\isacharparenright}{\isachardoublequoteclose}\ \isanewline
%
\isamarkupcmt{t represents a set of DDL formulas.%
}\isanewline
%
\isamarkupcmt{this set is defined by its truth function, mapping the set of worlds to the formula set's truth value.%
}\isanewline
\isanewline
%
\isamarkupcmt{accessibility relations map a set of worlds to:%
}\isanewline
\isacommand{consts}\isamarkupfalse%
\ av{\isacharcolon}{\isacharcolon}{\isachardoublequoteopen}i\ {\isasymRightarrow}\ t{\isachardoublequoteclose}\ %
\isamarkupcmt{actual versions of that world set%
}\isanewline
\ \ %
\isamarkupcmt{these worlds represent what is "open to the agent"%
}\isanewline
\ \ %
\isamarkupcmt{for example, the agent eating pizza or pasta for dinner might constitute two different actual worlds%
}\isanewline
\isanewline
\isacommand{consts}\isamarkupfalse%
\ pv{\isacharcolon}{\isacharcolon}{\isachardoublequoteopen}i\ {\isasymRightarrow}\ t{\isachardoublequoteclose}\ %
\isamarkupcmt{possible versions of that world set%
}\isanewline
\ \ %
\isamarkupcmt{these worlds represent was was "potentially open to the agent"%
}\isanewline
\ %
\isamarkupcmt{for example, what someone across the world eats for dinner might constitute a possible world, 
 %
\isamarkupcmt{since the agent has no control over this%
}%
}\isanewline
\isanewline
\isacommand{consts}\isamarkupfalse%
\ ob{\isacharcolon}{\isacharcolon}{\isachardoublequoteopen}t\ {\isasymRightarrow}\ {\isacharparenleft}t\ {\isasymRightarrow}\ bool{\isacharparenright}{\isachardoublequoteclose}\ \ %
\isamarkupcmt{set of propositions obligatory in this "context"%
}\isanewline
\ %
\isamarkupcmt{ob(context)(term) is True if t is obligatory in the context%
}\isanewline
\isanewline
\isacommand{consts}\isamarkupfalse%
\ cw{\isacharcolon}{\isacharcolon}i\ \ %
\isamarkupcmt{current world%
}%
\isadelimdocument
%
\endisadelimdocument
%
\isatagdocument
%
\isamarkupsubsection{Axiomatization%
}
\isamarkuptrue%
%
\endisatagdocument
{\isafolddocument}%
%
\isadelimdocument
%
\endisadelimdocument
%
\begin{isamarkuptext}%
This subsection contains axioms. Because the embedding is semantic, these are just constraints on models.%
\end{isamarkuptext}\isamarkuptrue%
%
\begin{isamarkuptext}%
This axiomatization comes from Carmo and Jones p 6 and the HOL embedding defined in Benzmuller and Parent%
\end{isamarkuptext}\isamarkuptrue%
\isacommand{axiomatization}\isamarkupfalse%
\ \isakeyword{where}\isanewline
ax{\isacharunderscore}{\isadigit{3}}a{\isacharcolon}\ {\isachardoublequoteopen}{\isasymexists}x{\isachardot}\ av{\isacharparenleft}w{\isacharparenright}{\isacharparenleft}x{\isacharparenright}{\isachardoublequoteclose}\ \isanewline
\ %
\isamarkupcmt{every world has some actual version%
}\isanewline
\isanewline
\isakeyword{and}\ ax{\isacharunderscore}{\isadigit{4}}a{\isacharcolon}\ {\isachardoublequoteopen}{\isasymforall}x{\isachardot}\ av{\isacharparenleft}w{\isacharparenright}{\isacharparenleft}x{\isacharparenright}\ {\isasymlongrightarrow}\ pv{\isacharparenleft}w{\isacharparenright}{\isacharparenleft}x{\isacharparenright}{\isachardoublequoteclose}\ \isanewline
%
\isamarkupcmt{all actual versions of a world are also possible versions of it%
}\isanewline
\isanewline
\isakeyword{and}\ ax{\isacharunderscore}{\isadigit{4}}b{\isacharcolon}\ {\isachardoublequoteopen}pv{\isacharparenleft}w{\isacharparenright}{\isacharparenleft}w{\isacharparenright}{\isachardoublequoteclose}\isanewline
%
\isamarkupcmt{every world is a possible version of itself%
}\isanewline
\isanewline
\isakeyword{and}\ ax{\isacharunderscore}{\isadigit{5}}a{\isacharcolon}\ {\isachardoublequoteopen}{\isasymnot}ob{\isacharparenleft}X{\isacharparenright}{\isacharparenleft}{\isasymlambda}w{\isachardot}\ False{\isacharparenright}{\isachardoublequoteclose}\isanewline
%
\isamarkupcmt{in any arbitrary context X, something will be obligatory%
}\isanewline
\isanewline
\isakeyword{and}\ ax{\isacharunderscore}{\isadigit{5}}b{\isacharcolon}\ {\isachardoublequoteopen}{\isasymforall}w{\isachardot}\ {\isacharparenleft}{\isacharparenleft}X{\isacharparenleft}w{\isacharparenright}\ {\isasymand}\ Y{\isacharparenleft}w{\isacharparenright}{\isacharparenright}\ {\isasymlongleftrightarrow}\ {\isacharparenleft}X{\isacharparenleft}w{\isacharparenright}\ {\isasymand}\ Z{\isacharparenleft}w{\isacharparenright}{\isacharparenright}{\isacharparenright}\ {\isasymlongrightarrow}\ {\isacharparenleft}ob{\isacharparenleft}X{\isacharparenright}{\isacharparenleft}Y{\isacharparenright}\ {\isasymlongleftrightarrow}\ ob{\isacharparenleft}X{\isacharparenright}{\isacharparenleft}Z{\isacharparenright}{\isacharparenright}{\isachardoublequoteclose}\ %
\isamarkupcmt{note that X(w) denotes w is a member of X%
}\isanewline
%
\isamarkupcmt{X, Y, and Z are sets of formulas%
}\isanewline
%
\isamarkupcmt{If X $\cap$ Y = X $\cap$ Z then the context X obliges Y iff it obliges Z%
}\isanewline
\isanewline
%
\isamarkupcmt{ob(X)($\lambda$ w. Fw) can be read as F $\in$ ob(X)%
}\isanewline
\isakeyword{and}\ ax{\isacharunderscore}{\isadigit{5}}c{\isacharcolon}\ {\isachardoublequoteopen}{\isacharparenleft}{\isasymforall}Z{\isachardot}\ {\isasymbeta}{\isacharparenleft}Z{\isacharparenright}\ {\isasymlongrightarrow}\ ob{\isacharparenleft}X{\isacharparenright}{\isacharparenleft}Z{\isacharparenright}\ {\isasymand}\ {\isacharparenleft}{\isasymexists}Z{\isachardot}\ {\isasymbeta}{\isacharparenleft}Z{\isacharparenright}{\isacharparenright}{\isacharparenright}\ {\isasymlongrightarrow}\isanewline
{\isacharparenleft}{\isasymexists}y{\isachardot}{\isacharparenleft}{\isasymforall}Z{\isachardot}\ {\isacharparenleft}{\isasymbeta}{\isacharparenleft}Z{\isacharparenright}\ {\isasymlongrightarrow}\ Z{\isacharparenleft}y{\isacharparenright}{\isacharparenright}{\isacharparenright}\ {\isasymand}\ X{\isacharparenleft}y{\isacharparenright}{\isacharparenright}\ {\isasymlongrightarrow}\ ob{\isacharparenleft}X{\isacharparenright}{\isacharparenleft}{\isasymlambda}w{\isachardot}\ {\isasymforall}Z{\isachardot}{\isacharparenleft}{\isasymbeta}{\isacharparenleft}Z{\isacharparenright}\ {\isasymlongrightarrow}\ Z{\isacharparenleft}w{\isacharparenright}{\isacharparenright}{\isacharparenright}{\isachardoublequoteclose}\isanewline
%
\isamarkupcmt{For any nonempty subset $\beta$ of ob(X), if its members share members with X then it members are all in ob(X)%
}\isanewline
\isanewline
\isakeyword{and}\ ax{\isacharunderscore}{\isadigit{5}}d{\isacharcolon}\ {\isachardoublequoteopen}{\isacharparenleft}{\isasymforall}w{\isachardot}\ {\isacharparenleft}Y{\isacharparenleft}w{\isacharparenright}{\isasymlongrightarrow}X{\isacharparenleft}w{\isacharparenright}{\isacharparenright}\ {\isasymand}\ {\isacharparenleft}ob{\isacharparenleft}X{\isacharparenright}{\isacharparenleft}Y{\isacharparenright}{\isacharparenright}\ {\isasymand}\ {\isacharparenleft}{\isasymforall}w{\isachardot}\ {\isacharparenleft}X{\isacharparenleft}w{\isacharparenright}\ {\isasymlongrightarrow}Z{\isacharparenleft}w{\isacharparenright}{\isacharparenright}{\isacharparenright}{\isacharparenright}\ {\isasymlongrightarrow}\isanewline
{\isacharparenleft}ob{\isacharparenleft}Y{\isacharparenright}{\isacharparenleft}{\isasymlambda}w{\isachardot}\ Y{\isacharparenleft}w{\isacharparenright}\ {\isasymor}\ {\isacharparenleft}Z{\isacharparenleft}w{\isacharparenright}\ {\isasymand}\ {\isasymnot}X{\isacharparenleft}w{\isacharparenright}{\isacharparenright}{\isacharparenright}{\isacharparenright}{\isachardoublequoteclose}\isanewline
%
\isamarkupcmt{If some subset Y of X is in ob(X) then in a larger context Z, any obligatory proposition must either be in Y or in Z-X%
}\isanewline
\isanewline
\isakeyword{and}\ ax{\isacharunderscore}{\isadigit{5}}e{\isacharcolon}\ {\isachardoublequoteopen}{\isacharparenleft}{\isacharparenleft}{\isasymforall}w{\isachardot}\ {\isacharparenleft}Y{\isacharparenleft}w{\isacharparenright}{\isasymlongrightarrow}X{\isacharparenleft}w{\isacharparenright}{\isacharparenright}{\isacharparenright}\ {\isasymand}\ ob{\isacharparenleft}X{\isacharparenright}{\isacharparenleft}Z{\isacharparenright}\ {\isasymand}\ {\isacharparenleft}{\isasymexists}w{\isachardot}{\isacharparenleft}Y{\isacharparenleft}w{\isacharparenright}\ {\isasymand}\ Z{\isacharparenleft}w{\isacharparenright}{\isacharparenright}{\isacharparenright}{\isacharparenright}\ {\isasymlongrightarrow}\ ob{\isacharparenleft}Y{\isacharparenright}{\isacharparenleft}Z{\isacharparenright}{\isachardoublequoteclose}\isanewline
%
\isamarkupcmt{If Z is obligatory in context X, then Z is obligatory in a subset of X called Y, if Z shares some elements with Y%
}%
\isadelimdocument
%
\endisadelimdocument
%
\isatagdocument
%
\isamarkupsubsection{Abbreviations%
}
\isamarkuptrue%
%
\endisatagdocument
{\isafolddocument}%
%
\isadelimdocument
%
\endisadelimdocument
%
\begin{isamarkuptext}%
These abbreviations are defined in Benzmuller and Parent, p9%
\end{isamarkuptext}\isamarkuptrue%
%
\begin{isamarkuptext}%
These are all syntactic sugar for HOL expressions, so evaluating these symbols will be light-weight%
\end{isamarkuptext}\isamarkuptrue%
%
\isamarkupcmt{propositional logic symbols%
}\isanewline
\isacommand{abbreviation}\isamarkupfalse%
\ ddlneg{\isacharcolon}{\isacharcolon}{\isachardoublequoteopen}t{\isasymRightarrow}t{\isachardoublequoteclose}\ {\isacharparenleft}{\isachardoublequoteopen}\isactrlbold {\isasymnot}{\isachardoublequoteclose}{\isacharparenright}\ \isanewline
\ \ \isakeyword{where}\ {\isachardoublequoteopen}\isactrlbold {\isasymnot}\ A\ {\isasymequiv}\ {\isasymlambda}w{\isachardot}\ {\isasymnot}\ A{\isacharparenleft}w{\isacharparenright}{\isachardoublequoteclose}\ \isanewline
\isacommand{abbreviation}\isamarkupfalse%
\ ddlor{\isacharcolon}{\isacharcolon}{\isachardoublequoteopen}t{\isasymRightarrow}t{\isasymRightarrow}t{\isachardoublequoteclose}\ {\isacharparenleft}{\isachardoublequoteopen}\isactrlbold {\isasymor}{\isachardoublequoteclose}{\isacharparenright}\ \isanewline
\ \ \isakeyword{where}\ {\isachardoublequoteopen}\isactrlbold {\isasymor}\ A\ B\ {\isasymequiv}\ {\isasymlambda}w{\isachardot}\ {\isacharparenleft}A{\isacharparenleft}w{\isacharparenright}\ {\isasymor}\ B{\isacharparenleft}w{\isacharparenright}{\isacharparenright}{\isachardoublequoteclose}\isanewline
\isacommand{abbreviation}\isamarkupfalse%
\ ddland{\isacharcolon}{\isacharcolon}{\isachardoublequoteopen}t{\isasymRightarrow}t{\isasymRightarrow}t{\isachardoublequoteclose}\ {\isacharparenleft}{\isachardoublequoteopen}{\isacharunderscore}\isactrlbold {\isasymand}{\isacharunderscore}{\isachardoublequoteclose}{\isacharparenright}\isanewline
\ \ \isakeyword{where}\ {\isachardoublequoteopen}A\isactrlbold {\isasymand}\ B\ {\isasymequiv}\ {\isasymlambda}w{\isachardot}\ {\isacharparenleft}A{\isacharparenleft}w{\isacharparenright}\ {\isasymand}\ B{\isacharparenleft}w{\isacharparenright}{\isacharparenright}{\isachardoublequoteclose}\isanewline
\isacommand{abbreviation}\isamarkupfalse%
\ ddlif{\isacharcolon}{\isacharcolon}{\isachardoublequoteopen}t{\isasymRightarrow}t{\isasymRightarrow}t{\isachardoublequoteclose}\ {\isacharparenleft}{\isachardoublequoteopen}{\isacharunderscore}\isactrlbold {\isasymrightarrow}{\isacharunderscore}{\isachardoublequoteclose}{\isacharparenright}\isanewline
\ \ \isakeyword{where}\ {\isachardoublequoteopen}A\isactrlbold {\isasymrightarrow}B\ {\isasymequiv}\ {\isasymlambda}w{\isachardot}\ {\isacharparenleft}{\isasymnot}\ A{\isacharparenleft}w{\isacharparenright}\ {\isasymor}\ B{\isacharparenleft}w{\isacharparenright}{\isacharparenright}{\isachardoublequoteclose}\isanewline
\isacommand{abbreviation}\isamarkupfalse%
\ ddlequiv{\isacharcolon}{\isacharcolon}{\isachardoublequoteopen}t{\isasymRightarrow}t{\isasymRightarrow}t{\isachardoublequoteclose}\ {\isacharparenleft}{\isachardoublequoteopen}{\isacharunderscore}\isactrlbold {\isasymequiv}{\isacharunderscore}{\isachardoublequoteclose}{\isacharparenright}\isanewline
\ \ \isakeyword{where}\ {\isachardoublequoteopen}{\isacharparenleft}A\isactrlbold {\isasymequiv}B{\isacharparenright}\ {\isasymequiv}\ {\isacharparenleft}{\isacharparenleft}A\isactrlbold {\isasymrightarrow}B{\isacharparenright}\ \isactrlbold {\isasymand}\ {\isacharparenleft}\ B\isactrlbold {\isasymrightarrow}A{\isacharparenright}{\isacharparenright}{\isachardoublequoteclose}\isanewline
\isanewline
%
\isamarkupcmt{modal operators%
}\isanewline
\isacommand{abbreviation}\isamarkupfalse%
\ ddlbox{\isacharcolon}{\isacharcolon}{\isachardoublequoteopen}t{\isasymRightarrow}t{\isachardoublequoteclose}\ {\isacharparenleft}{\isachardoublequoteopen}{\isasymbox}{\isachardoublequoteclose}{\isacharparenright}\ \isanewline
\ \ \isakeyword{where}\ {\isachardoublequoteopen}{\isasymbox}\ A\ {\isasymequiv}\ {\isasymlambda}w{\isachardot}{\isasymforall}y{\isachardot}\ A{\isacharparenleft}y{\isacharparenright}{\isachardoublequoteclose}\ \isanewline
\isacommand{abbreviation}\isamarkupfalse%
\ ddldiamond{\isacharcolon}{\isacharcolon}{\isachardoublequoteopen}t\ {\isasymRightarrow}\ t{\isachardoublequoteclose}\ {\isacharparenleft}{\isachardoublequoteopen}{\isasymdiamond}{\isachardoublequoteclose}{\isacharparenright}\isanewline
\ \ \isakeyword{where}\ {\isachardoublequoteopen}{\isasymdiamond}A\ {\isasymequiv}\ \isactrlbold {\isasymnot}{\isacharparenleft}{\isasymbox}{\isacharparenleft}\isactrlbold {\isasymnot}A{\isacharparenright}{\isacharparenright}{\isachardoublequoteclose}\isanewline
\isanewline
%
\isamarkupcmt{O$\{B|A\}$ can be read as ``B is obligatory in the context A"%
}\isanewline
\isacommand{abbreviation}\isamarkupfalse%
\ ddlob{\isacharcolon}{\isacharcolon}{\isachardoublequoteopen}t{\isasymRightarrow}t{\isasymRightarrow}t{\isachardoublequoteclose}\ {\isacharparenleft}{\isachardoublequoteopen}O{\isacharbraceleft}{\isacharunderscore}{\isacharbar}{\isacharunderscore}{\isacharbraceright}{\isachardoublequoteclose}{\isacharparenright}\isanewline
\ \ \isakeyword{where}\ {\isachardoublequoteopen}O{\isacharbraceleft}B{\isacharbar}A{\isacharbraceright}\ {\isasymequiv}\ {\isasymlambda}\ w{\isachardot}\ ob{\isacharparenleft}A{\isacharparenright}{\isacharparenleft}B{\isacharparenright}{\isachardoublequoteclose}\isanewline
\isanewline
%
\isamarkupcmt{modal symbols over the actual and possible worlds relations%
}\isanewline
\isacommand{abbreviation}\isamarkupfalse%
\ ddlboxa{\isacharcolon}{\isacharcolon}{\isachardoublequoteopen}t{\isasymRightarrow}t{\isachardoublequoteclose}\ {\isacharparenleft}{\isachardoublequoteopen}{\isasymbox}\isactrlsub a{\isachardoublequoteclose}{\isacharparenright}\isanewline
\ \ \isakeyword{where}\ {\isachardoublequoteopen}{\isasymbox}\isactrlsub aA\ {\isasymequiv}\ {\isasymlambda}x{\isachardot}{\isasymforall}y{\isachardot}\ {\isacharparenleft}{\isasymnot}\ av{\isacharparenleft}x{\isacharparenright}{\isacharparenleft}y{\isacharparenright}\ {\isasymor}\ A{\isacharparenleft}y{\isacharparenright}{\isacharparenright}{\isachardoublequoteclose}\isanewline
\isacommand{abbreviation}\isamarkupfalse%
\ ddldiamonda{\isacharcolon}{\isacharcolon}{\isachardoublequoteopen}t{\isasymRightarrow}t{\isachardoublequoteclose}\ {\isacharparenleft}{\isachardoublequoteopen}{\isasymdiamond}\isactrlsub a{\isachardoublequoteclose}{\isacharparenright}\isanewline
\ \ \isakeyword{where}\ {\isachardoublequoteopen}{\isasymdiamond}\isactrlsub aA\ {\isasymequiv}\ \isactrlbold {\isasymnot}{\isacharparenleft}{\isasymbox}\isactrlsub a{\isacharparenleft}\isactrlbold {\isasymnot}A{\isacharparenright}{\isacharparenright}{\isachardoublequoteclose}\isanewline
\isacommand{abbreviation}\isamarkupfalse%
\ ddlboxp{\isacharcolon}{\isacharcolon}{\isachardoublequoteopen}t{\isasymRightarrow}t{\isachardoublequoteclose}\ {\isacharparenleft}{\isachardoublequoteopen}{\isasymbox}\isactrlsub p{\isachardoublequoteclose}{\isacharparenright}\isanewline
\ \ \isakeyword{where}\ {\isachardoublequoteopen}{\isasymbox}\isactrlsub pA\ {\isasymequiv}\ {\isasymlambda}x{\isachardot}{\isasymforall}y{\isachardot}\ {\isacharparenleft}{\isasymnot}\ pv{\isacharparenleft}x{\isacharparenright}{\isacharparenleft}y{\isacharparenright}\ {\isasymor}\ A{\isacharparenleft}y{\isacharparenright}{\isacharparenright}{\isachardoublequoteclose}\isanewline
\isacommand{abbreviation}\isamarkupfalse%
\ ddldiamondp{\isacharcolon}{\isacharcolon}{\isachardoublequoteopen}t{\isasymRightarrow}t{\isachardoublequoteclose}\ {\isacharparenleft}{\isachardoublequoteopen}{\isasymdiamond}\isactrlsub p{\isachardoublequoteclose}{\isacharparenright}\isanewline
\ \ \isakeyword{where}\ {\isachardoublequoteopen}{\isasymdiamond}\isactrlsub pA\ {\isasymequiv}\ \isactrlbold {\isasymnot}{\isacharparenleft}{\isasymbox}\isactrlsub a{\isacharparenleft}\isactrlbold {\isasymnot}A{\isacharparenright}{\isacharparenright}{\isachardoublequoteclose}\isanewline
\isanewline
%
\isamarkupcmt{obligation symbols over the actual and possible worlds%
}\isanewline
\isacommand{abbreviation}\isamarkupfalse%
\ ddloba{\isacharcolon}{\isacharcolon}{\isachardoublequoteopen}t{\isasymRightarrow}t{\isachardoublequoteclose}\ {\isacharparenleft}{\isachardoublequoteopen}O\isactrlsub a{\isachardoublequoteclose}{\isacharparenright}\isanewline
\ \ \isakeyword{where}\ {\isachardoublequoteopen}O\isactrlsub a\ A\ {\isasymequiv}\ {\isasymlambda}x{\isachardot}\ ob{\isacharparenleft}av{\isacharparenleft}x{\isacharparenright}{\isacharparenright}{\isacharparenleft}A{\isacharparenright}\ {\isasymand}\ {\isacharparenleft}{\isasymexists}y{\isachardot}{\isacharparenleft}av{\isacharparenleft}x{\isacharparenright}{\isacharparenleft}y{\isacharparenright}\ {\isasymand}\ {\isasymnot}A{\isacharparenleft}y{\isacharparenright}{\isacharparenright}{\isacharparenright}{\isachardoublequoteclose}\isanewline
\isacommand{abbreviation}\isamarkupfalse%
\ ddlobp{\isacharcolon}{\isacharcolon}{\isachardoublequoteopen}t{\isasymRightarrow}t{\isachardoublequoteclose}\ {\isacharparenleft}{\isachardoublequoteopen}O\isactrlsub p{\isachardoublequoteclose}{\isacharparenright}\isanewline
\ \ \isakeyword{where}\ {\isachardoublequoteopen}O\isactrlsub p\ A\ {\isasymequiv}\ {\isasymlambda}x{\isachardot}\ ob{\isacharparenleft}pv{\isacharparenleft}x{\isacharparenright}{\isacharparenright}{\isacharparenleft}A{\isacharparenright}\ {\isasymand}\ {\isacharparenleft}{\isasymexists}y{\isachardot}{\isacharparenleft}pv{\isacharparenleft}x{\isacharparenright}{\isacharparenleft}y{\isacharparenright}\ {\isasymand}\ {\isasymnot}A{\isacharparenleft}y{\isacharparenright}{\isacharparenright}{\isacharparenright}{\isachardoublequoteclose}\isanewline
\isanewline
%
\isamarkupcmt{syntactic sugar for a ``monadic" obligation operator%
}\isanewline
\isacommand{abbreviation}\isamarkupfalse%
\ ddltrue{\isacharcolon}{\isacharcolon}{\isachardoublequoteopen}t{\isachardoublequoteclose}\ {\isacharparenleft}{\isachardoublequoteopen}{\isasymtop}{\isachardoublequoteclose}{\isacharparenright}\isanewline
\ \ \isakeyword{where}\ {\isachardoublequoteopen}{\isasymtop}\ {\isasymequiv}\ {\isasymlambda}w{\isachardot}\ True{\isachardoublequoteclose}\isanewline
\isacommand{abbreviation}\isamarkupfalse%
\ ddlob{\isacharunderscore}normal{\isacharcolon}{\isacharcolon}{\isachardoublequoteopen}t{\isasymRightarrow}t{\isachardoublequoteclose}\ {\isacharparenleft}{\isachardoublequoteopen}O{\isacharunderscore}{\isachardoublequoteclose}{\isacharparenright}\isanewline
\ \ \isakeyword{where}\ {\isachardoublequoteopen}{\isacharparenleft}O\ A{\isacharparenright}\ {\isasymequiv}\ {\isacharparenleft}O{\isacharbraceleft}A{\isacharbar}{\isasymtop}{\isacharbraceright}{\isacharparenright}\ {\isachardoublequoteclose}\isanewline
\isanewline
%
\isamarkupcmt{validity%
}\isanewline
\isacommand{abbreviation}\isamarkupfalse%
\ ddlvalid{\isacharcolon}{\isacharcolon}{\isachardoublequoteopen}t{\isasymRightarrow}bool{\isachardoublequoteclose}\ {\isacharparenleft}{\isachardoublequoteopen}{\isasymTurnstile}{\isacharunderscore}{\isachardoublequoteclose}{\isacharparenright}\isanewline
\ \ \isakeyword{where}\ {\isachardoublequoteopen}{\isasymTurnstile}A\ {\isasymequiv}\ {\isasymforall}w{\isachardot}\ A{\isacharparenleft}w{\isacharparenright}{\isachardoublequoteclose}\isanewline
\isacommand{abbreviation}\isamarkupfalse%
\ ddlvalidcw{\isacharcolon}{\isacharcolon}{\isachardoublequoteopen}t{\isasymRightarrow}bool{\isachardoublequoteclose}\ {\isacharparenleft}{\isachardoublequoteopen}{\isasymTurnstile}\isactrlsub c{\isacharunderscore}{\isachardoublequoteclose}{\isacharparenright}\isanewline
\ \ \isakeyword{where}\ {\isachardoublequoteopen}{\isasymTurnstile}\isactrlsub cA\ {\isasymequiv}\ A{\isacharparenleft}cw{\isacharparenright}{\isachardoublequoteclose}%
\isadelimdocument
%
\endisadelimdocument
%
\isatagdocument
%
\isamarkupsubsection{Consistency%
}
\isamarkuptrue%
%
\endisatagdocument
{\isafolddocument}%
%
\isadelimdocument
%
\endisadelimdocument
%
\begin{isamarkuptext}%
Consistency is so easy to show in Isabelle!%
\end{isamarkuptext}\isamarkuptrue%
\isacommand{lemma}\isamarkupfalse%
\ True\ \isacommand{nitpick}\isamarkupfalse%
\ {\isacharbrackleft}satisfy{\isacharcomma}user{\isacharunderscore}axioms{\isacharcomma}show{\isacharunderscore}all{\isacharcomma}format{\isacharequal}{\isadigit{2}}{\isacharbrackright}%
\isadelimproof
\ %
\endisadelimproof
%
\isatagproof
\isacommand{oops}\isamarkupfalse%
\ \isanewline
%
\isamarkupcmt{Nitpick successfully found a countermodel.%
}\isanewline
%
\isamarkupcmt{It's not shown in the document printout, hence the oops.%
}\isanewline
%
\isamarkupcmt{If you hover over "nitpick" in JEdit, the model will be printed to output.%
}%
\endisatagproof
{\isafoldproof}%
%
\isadelimproof
%
\endisadelimproof
\isanewline
%
\isadelimtheory
\isanewline
%
\endisadelimtheory
%
\isatagtheory
\isacommand{end}\isamarkupfalse%
%
\endisatagtheory
{\isafoldtheory}%
%
\isadelimtheory
%
\endisadelimtheory
%
\end{isabellebody}%
\endinput
%:%file=~/Desktop/cs91r/carmojones_DDL.thy%:%
%:%6=1%:%
%:%10=2%:%
%:%18=4%:%
%:%19=4%:%
%:%20=5%:%
%:%21=6%:%
%:%22=7%:%
%:%23=8%:%
%:%37=10%:%
%:%41=12%:%
%:%53=14%:%
%:%55=16%:%
%:%56=16%:%
%:%57=16%:%
%:%58=16%:%
%:%59=17%:%
%:%60=18%:%
%:%61=18%:%
%:%63=19%:%
%:%64=19%:%
%:%66=20%:%
%:%67=20%:%
%:%68=21%:%
%:%70=22%:%
%:%71=22%:%
%:%72=23%:%
%:%73=23%:%
%:%74=23%:%
%:%75=23%:%
%:%76=24%:%
%:%77=24%:%
%:%78=24%:%
%:%79=25%:%
%:%80=25%:%
%:%81=25%:%
%:%82=26%:%
%:%83=27%:%
%:%84=27%:%
%:%85=27%:%
%:%86=27%:%
%:%87=28%:%
%:%88=28%:%
%:%89=28%:%
%:%90=29%:%
%:%91=29%:%
%:%92=30%:%
%:%93=30%:%
%:%95=30%:%
%:%96=31%:%
%:%97=32%:%
%:%98=32%:%
%:%99=32%:%
%:%100=32%:%
%:%101=33%:%
%:%102=33%:%
%:%103=33%:%
%:%104=34%:%
%:%105=35%:%
%:%106=35%:%
%:%107=35%:%
%:%115=37%:%
%:%127=39%:%
%:%131=40%:%
%:%133=42%:%
%:%134=42%:%
%:%135=43%:%
%:%136=44%:%
%:%137=44%:%
%:%138=44%:%
%:%139=45%:%
%:%140=46%:%
%:%142=47%:%
%:%143=47%:%
%:%144=48%:%
%:%145=49%:%
%:%147=50%:%
%:%148=50%:%
%:%149=51%:%
%:%150=52%:%
%:%152=53%:%
%:%153=53%:%
%:%154=54%:%
%:%155=55%:%
%:%156=55%:%
%:%157=55%:%
%:%159=56%:%
%:%160=56%:%
%:%162=57%:%
%:%163=57%:%
%:%164=58%:%
%:%166=59%:%
%:%167=59%:%
%:%168=60%:%
%:%169=61%:%
%:%171=62%:%
%:%172=62%:%
%:%173=63%:%
%:%174=64%:%
%:%175=65%:%
%:%177=66%:%
%:%178=66%:%
%:%179=67%:%
%:%180=68%:%
%:%182=69%:%
%:%190=71%:%
%:%202=73%:%
%:%206=74%:%
%:%209=76%:%
%:%210=76%:%
%:%211=77%:%
%:%212=77%:%
%:%213=78%:%
%:%214=79%:%
%:%215=79%:%
%:%216=80%:%
%:%217=81%:%
%:%218=81%:%
%:%219=82%:%
%:%220=83%:%
%:%221=83%:%
%:%222=84%:%
%:%223=85%:%
%:%224=85%:%
%:%225=86%:%
%:%226=87%:%
%:%228=88%:%
%:%229=88%:%
%:%230=89%:%
%:%231=89%:%
%:%232=90%:%
%:%233=91%:%
%:%234=91%:%
%:%235=92%:%
%:%236=93%:%
%:%238=94%:%
%:%239=94%:%
%:%240=95%:%
%:%241=95%:%
%:%242=96%:%
%:%243=97%:%
%:%245=98%:%
%:%246=98%:%
%:%247=99%:%
%:%248=99%:%
%:%249=100%:%
%:%250=101%:%
%:%251=101%:%
%:%252=102%:%
%:%253=103%:%
%:%254=103%:%
%:%255=104%:%
%:%256=105%:%
%:%257=105%:%
%:%258=106%:%
%:%259=107%:%
%:%261=108%:%
%:%262=108%:%
%:%263=109%:%
%:%264=109%:%
%:%265=110%:%
%:%266=111%:%
%:%267=111%:%
%:%268=112%:%
%:%269=113%:%
%:%271=114%:%
%:%272=114%:%
%:%273=115%:%
%:%274=115%:%
%:%275=116%:%
%:%276=117%:%
%:%277=117%:%
%:%278=118%:%
%:%279=119%:%
%:%281=120%:%
%:%282=120%:%
%:%283=121%:%
%:%284=121%:%
%:%285=122%:%
%:%286=123%:%
%:%287=123%:%
%:%288=124%:%
%:%295=126%:%
%:%307=128%:%
%:%309=130%:%
%:%310=130%:%
%:%311=130%:%
%:%313=130%:%
%:%317=130%:%
%:%318=130%:%
%:%320=131%:%
%:%321=131%:%
%:%323=132%:%
%:%324=132%:%
%:%326=133%:%
%:%334=133%:%
%:%337=134%:%
%:%342=135%:%