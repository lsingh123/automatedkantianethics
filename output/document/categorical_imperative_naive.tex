%
\begin{isabellebody}%
\setisabellecontext{categorical{\isacharunderscore}imperative{\isacharunderscore}naive}%
%
\isadelimtheory
%
\endisadelimtheory
%
\isatagtheory
\isacommand{theory}\isamarkupfalse%
\ categorical{\isacharunderscore}imperative{\isacharunderscore}naive\ \isakeyword{imports}\ carmojones{\isacharunderscore}DDL{\isacharunderscore}completeness\isanewline
\isanewline
\isakeyword{begin}%
\endisatagtheory
{\isafoldtheory}%
%
\isadelimtheory
%
\endisadelimtheory
%
\isadelimdocument
%
\endisadelimdocument
%
\isatagdocument
%
\isamarkupsection{The Categorical Imperative%
}
\isamarkuptrue%
%
\isamarkupsubsection{Simple Formulation of the Formula of Universal Law%
}
\isamarkuptrue%
%
\endisatagdocument
{\isafolddocument}%
%
\isadelimdocument
%
\endisadelimdocument
%
\begin{isamarkuptext}%
This is my second attempt at formalizing the Formula of Universal Law%
\end{isamarkuptext}\isamarkuptrue%
\isacommand{abbreviation}\isamarkupfalse%
\ ddlpermissable{\isacharcolon}{\isacharcolon}{\isachardoublequoteopen}t{\isasymRightarrow}t{\isachardoublequoteclose}\ {\isacharparenleft}{\isachardoublequoteopen}P{\isacharunderscore}{\isachardoublequoteclose}{\isacharparenright}\isanewline
\ \ \isakeyword{where}\ {\isachardoublequoteopen}{\isacharparenleft}P\ A{\isacharparenright}\ {\isasymequiv}\ {\isacharparenleft}\isactrlbold {\isasymnot}{\isacharparenleft}O\ {\isacharbraceleft}\isactrlbold {\isasymnot}A{\isacharbraceright}{\isacharparenright}{\isacharparenright}{\isachardoublequoteclose}\isanewline
%
\isamarkupcmt{This operator represents permissibility%
}\isanewline
%
\isamarkupcmt{Will be useful when discussing the categorical imperative%
}\isanewline
%
\isamarkupcmt{Something is permissible if it is not prohibited%
}\isanewline
%
\isamarkupcmt{Something is prohibited if its negation is obligatory%
}%
\begin{isamarkuptext}%
Let's consider a naive reading of the Formula of Universal Law (FUL).
From the Groundwork, 'act only in accordance with that maxim through which you can at the same time will that it become a universal law'.
What does this mean in DDL? One interpretation is if A is not necessarily permissible, then its negation is obligated.%
\end{isamarkuptext}\isamarkuptrue%
\isacommand{axiomatization}\isamarkupfalse%
\ \isakeyword{where}\isanewline
FUL{\isacharunderscore}{\isadigit{1}}{\isacharcolon}\ {\isachardoublequoteopen}{\isasymTurnstile}\ {\isacharparenleft}{\isacharparenleft}\isactrlbold {\isasymnot}{\isacharparenleft}{\isasymbox}\ {\isacharparenleft}P\ A{\isacharparenright}{\isacharparenright}{\isacharparenright}\ \isactrlbold {\isasymrightarrow}\ {\isacharparenleft}O\ {\isacharbraceleft}{\isacharparenleft}\isactrlbold {\isasymnot}A{\isacharparenright}{\isacharbraceright}{\isacharparenright}{\isacharparenright}{\isachardoublequoteclose}%
\isadelimdocument
%
\endisadelimdocument
%
\isatagdocument
%
\isamarkupsubsection{Basic Tests%
}
\isamarkuptrue%
%
\endisatagdocument
{\isafolddocument}%
%
\isadelimdocument
%
\endisadelimdocument
\isacommand{lemma}\isamarkupfalse%
\ True\ \isacommand{nitpick}\isamarkupfalse%
\ {\isacharbrackleft}satisfy{\isacharcomma}user{\isacharunderscore}axioms{\isacharcomma}format{\isacharequal}{\isadigit{2}}{\isacharbrackright}%
\isadelimproof
\ %
\endisadelimproof
%
\isatagproof
\isacommand{oops}\isamarkupfalse%
\isanewline
%
\isamarkupcmt{``Nitpick found a model for card i = 1:

  Empty assignment"%
}\isanewline
%
\isamarkupcmt{Nitpick tells us that the FUL is consistent%
}\isanewline
%
\isamarkupcmt{``oops" after Nitpick does not mean Nitpick failed.%
}%
\endisatagproof
{\isafoldproof}%
%
\isadelimproof
%
\endisadelimproof
\isanewline
\isanewline
\isacommand{lemma}\isamarkupfalse%
\ something{\isacharunderscore}is{\isacharunderscore}obligatory{\isacharcolon}\isanewline
\ \ \isakeyword{shows}\ {\isachardoublequoteopen}{\isasymforall}\ w{\isachardot}\ {\isasymexists}\ A{\isachardot}\ O\ {\isacharbraceleft}A{\isacharbraceright}\ w{\isachardoublequoteclose}\isanewline
\ \ \isacommand{nitpick}\isamarkupfalse%
\ {\isacharbrackleft}user{\isacharunderscore}axioms{\isacharbrackright}\isanewline
%
\isadelimproof
\ \ %
\endisadelimproof
%
\isatagproof
\isacommand{oops}\isamarkupfalse%
\isanewline
%
\isamarkupcmt{We might think that in every world we want something to be obligated.%
}\isanewline
%
\isamarkupcmt{Sadly, Sledgehammer times out trying to prove this. Let's relax this%
}\isanewline
%
\isamarkupcmt{``Nitpick found a counterexample for card i = 1:

  Empty assignment"%
}\isanewline
%
\isamarkupcmt{Nitpick to the rescue! The formula is in fact not valid.%
}%
\endisatagproof
{\isafoldproof}%
%
\isadelimproof
\isanewline
%
\endisadelimproof
\isanewline
\isacommand{lemma}\isamarkupfalse%
\ something{\isacharunderscore}is{\isacharunderscore}obligatory{\isacharunderscore}{\isadigit{2}}{\isacharcolon}\isanewline
\ \ \isakeyword{shows}\ {\isachardoublequoteopen}{\isasymforall}\ w{\isachardot}\ {\isasymexists}\ A{\isachardot}\ O\ {\isacharbraceleft}A{\isacharbraceright}\ w{\isachardoublequoteclose}\isanewline
\ \ \isacommand{nitpick}\isamarkupfalse%
\ {\isacharbrackleft}user{\isacharunderscore}axioms{\isacharcomma}\ falsify{\isacharequal}false{\isacharbrackright}\isanewline
%
\isadelimproof
\ \ %
\endisadelimproof
%
\isatagproof
\isacommand{oops}\isamarkupfalse%
\isanewline
%
\isamarkupcmt{``Nitpick found a model for card i = 1:

  Skolem constant:
    A = ($\lambda x. \_$)($i_1$ := True)"%
}\isanewline
%
\isamarkupcmt{Nitpick tells us that the formula is consistent - it found a model where the formula is true.%
}\isanewline
%
\isamarkupcmt{This means that our model is underspecified - this formula is neither valid nor inconsistent.%
}%
\endisatagproof
{\isafoldproof}%
%
\isadelimproof
\isanewline
%
\endisadelimproof
\isanewline
\isacommand{lemma}\isamarkupfalse%
\ something{\isacharunderscore}is{\isacharunderscore}obligatory{\isacharunderscore}relaxed{\isacharcolon}\isanewline
\ \ \isakeyword{shows}\ {\isachardoublequoteopen}{\isasymexists}\ A\ w{\isachardot}\ O\ {\isacharbraceleft}A{\isacharbraceright}\ w{\isachardoublequoteclose}\isanewline
\ \ \isacommand{nitpick}\isamarkupfalse%
\ {\isacharbrackleft}user{\isacharunderscore}axioms{\isacharbrackright}\isanewline
%
\isadelimproof
\ \ %
\endisadelimproof
%
\isatagproof
\isacommand{oops}\isamarkupfalse%
\isanewline
%
\isamarkupcmt{``Nitpick found a counterexample for card i = 1:

  Empty assignment"%
}\isanewline
%
\isamarkupcmt{The relaxed version definitely isn't valid.%
}%
\endisatagproof
{\isafoldproof}%
%
\isadelimproof
\isanewline
%
\endisadelimproof
\isanewline
\isacommand{lemma}\isamarkupfalse%
\ something{\isacharunderscore}is{\isacharunderscore}obligatory{\isacharunderscore}relaxed{\isacharunderscore}{\isadigit{2}}{\isacharcolon}\isanewline
\ \ \isakeyword{shows}\ {\isachardoublequoteopen}{\isasymexists}\ A\ w{\isachardot}\ O\ {\isacharbraceleft}A{\isacharbraceright}\ w{\isachardoublequoteclose}\isanewline
\ \ \isacommand{nitpick}\isamarkupfalse%
\ {\isacharbrackleft}user{\isacharunderscore}axioms{\isacharcomma}\ falsify{\isacharequal}false{\isacharbrackright}\isanewline
%
\isadelimproof
\ \ %
\endisadelimproof
%
\isatagproof
\isacommand{oops}\isamarkupfalse%
\isanewline
%
\isamarkupcmt{``Nitpick found a model for card i = 1:

  Skolem constant:
    A = ($\lambda x. \_$)($i_1$ := True)"%
}\isanewline
%
\isamarkupcmt{Nitpick tells us that the formula is consistent - it found a model where the formula is true.%
}\isanewline
%
\isamarkupcmt{The model seems underspecified.%
}%
\endisatagproof
{\isafoldproof}%
%
\isadelimproof
%
\endisadelimproof
%
\isadelimdocument
%
\endisadelimdocument
%
\isatagdocument
%
\isamarkupsubsection{Specifying the Model%
}
\isamarkuptrue%
%
\endisatagdocument
{\isafolddocument}%
%
\isadelimdocument
%
\endisadelimdocument
%
\begin{isamarkuptext}%
Let's specify the model. What if we add something impermissible?%
\end{isamarkuptext}\isamarkuptrue%
\isacommand{consts}\isamarkupfalse%
\ M{\isacharcolon}{\isacharcolon}{\isachardoublequoteopen}t{\isachardoublequoteclose}\isanewline
\isacommand{abbreviation}\isamarkupfalse%
\ murder{\isacharunderscore}wrong{\isacharcolon}{\isacharcolon}{\isachardoublequoteopen}bool{\isachardoublequoteclose}\ \isakeyword{where}\ {\isachardoublequoteopen}murder{\isacharunderscore}wrong\ {\isasymequiv}\ {\isasymTurnstile}{\isacharparenleft}O\ {\isacharbraceleft}\isactrlbold {\isasymnot}\ M{\isacharbraceright}{\isacharparenright}{\isachardoublequoteclose}\isanewline
\isanewline
\isacommand{lemma}\isamarkupfalse%
\ something{\isacharunderscore}is{\isacharunderscore}obligatory{\isacharunderscore}{\isadigit{2}}{\isacharcolon}\isanewline
\ \ \isakeyword{assumes}\ murder{\isacharunderscore}wrong\isanewline
\ \ \isakeyword{shows}\ {\isachardoublequoteopen}{\isasymforall}\ w{\isachardot}\ {\isasymexists}\ A{\isachardot}\ O\ {\isacharbraceleft}A{\isacharbraceright}\ w{\isachardoublequoteclose}\isanewline
%
\isadelimproof
\ \ %
\endisadelimproof
%
\isatagproof
\isacommand{using}\isamarkupfalse%
\ assms\ \isacommand{by}\isamarkupfalse%
\ auto\isanewline
%
\isamarkupcmt{It works this time, but I think ``murder wrong" might be too strong of an assumption%
}%
\endisatagproof
{\isafoldproof}%
%
\isadelimproof
\isanewline
%
\endisadelimproof
\isanewline
\isacommand{abbreviation}\isamarkupfalse%
\ poss{\isacharunderscore}murder{\isacharunderscore}wrong{\isacharcolon}{\isacharcolon}{\isachardoublequoteopen}bool{\isachardoublequoteclose}\ \isakeyword{where}\ {\isachardoublequoteopen}poss{\isacharunderscore}murder{\isacharunderscore}wrong\ {\isasymequiv}\ {\isasymTurnstile}{\isacharparenleft}{\isasymdiamond}\ {\isacharparenleft}O\ {\isacharbraceleft}\isactrlbold {\isasymnot}\ M{\isacharbraceright}{\isacharparenright}{\isacharparenright}{\isachardoublequoteclose}\isanewline
\isanewline
\isacommand{lemma}\isamarkupfalse%
\ wrong{\isacharunderscore}if{\isacharunderscore}posibly{\isacharunderscore}wrong{\isacharcolon}\isanewline
\ \ \isakeyword{assumes}\ poss{\isacharunderscore}murder{\isacharunderscore}wrong\isanewline
\ \ \isakeyword{shows}\ murder{\isacharunderscore}wrong\isanewline
%
\isadelimproof
\ \ %
\endisadelimproof
%
\isatagproof
\isacommand{using}\isamarkupfalse%
\ assms\ \isacommand{by}\isamarkupfalse%
\ blast\isanewline
%
\isamarkupcmt{This lemma holds and uses a slightly weaker assumption. This also seems to get closer to the ``heart" of 
this naive interpretation. We really want to say that if something isn't necessarily obligated, it's not obligated anywhere."%
}%
\endisatagproof
{\isafoldproof}%
%
\isadelimproof
%
\endisadelimproof
%
\begin{isamarkuptext}%
Let's try an even weaker assumption: Not everyone can lie.%
\end{isamarkuptext}\isamarkuptrue%
\isacommand{typedecl}\isamarkupfalse%
\ person\isanewline
\isacommand{consts}\isamarkupfalse%
\ lies{\isacharcolon}{\isacharcolon}{\isachardoublequoteopen}person{\isasymRightarrow}t{\isachardoublequoteclose}\isanewline
\isacommand{consts}\isamarkupfalse%
\ me{\isacharcolon}{\isacharcolon}{\isachardoublequoteopen}person{\isachardoublequoteclose}\isanewline
\isanewline
\isacommand{lemma}\isamarkupfalse%
\ breaking{\isacharunderscore}promises{\isacharcolon}\isanewline
\ \ \isakeyword{assumes}\ {\isachardoublequoteopen}{\isasymnot}\ {\isacharparenleft}{\isasymforall}x{\isachardot}\ lie{\isacharparenleft}x{\isacharparenright}\ cw{\isacharparenright}\ {\isasymand}\ {\isacharparenleft}lie{\isacharparenleft}me{\isacharparenright}\ cw{\isacharparenright}{\isachardoublequoteclose}\isanewline
\ \ \isakeyword{shows}\ {\isachardoublequoteopen}{\isacharparenleft}O\ {\isacharbraceleft}\isactrlbold {\isasymnot}\ {\isacharparenleft}lie{\isacharparenleft}me{\isacharparenright}{\isacharparenright}{\isacharbraceright}{\isacharparenright}\ cw{\isachardoublequoteclose}\isanewline
\ \ \isacommand{nitpick}\isamarkupfalse%
\ {\isacharbrackleft}user{\isacharunderscore}axioms{\isacharbrackright}\isanewline
%
\isadelimproof
\ \ %
\endisadelimproof
%
\isatagproof
\isacommand{oops}\isamarkupfalse%
\isanewline
%
\isamarkupcmt{No proof found. Makes sense:%
}\isanewline
%
\isamarkupcmt{This version of FUL simply universalizes across worlds (using the $\Box$ operator),%
}\isanewline
%
\isamarkupcmt{But NOT across people, which is really what the most obvious reading of FUL implies%
}\isanewline
%
\isamarkupcmt{``Nitpick found a counterexample for card person = 2 and card i = 2:

  Free variable:
    lie = ($\lambda x. \_$)($p_1$ := ($\lambda x. \_$)($i_1$ := True, $i_2$ := False), $p_2$ := ($\lambda x. \_$)($i_1$ := False, $i_2$ := False))"%
}%
\endisatagproof
{\isafoldproof}%
%
\isadelimproof
\isanewline
%
\endisadelimproof
\isanewline
\isacommand{lemma}\isamarkupfalse%
\ universalizability{\isacharcolon}\isanewline
\ \ \isakeyword{assumes}\ {\isachardoublequoteopen}{\isasymTurnstile}\ O\ {\isacharbraceleft}{\isacharparenleft}lie{\isacharparenleft}me{\isacharparenright}{\isacharparenright}{\isacharbraceright}{\isachardoublequoteclose}\isanewline
\ \ \isakeyword{shows}\ {\isachardoublequoteopen}{\isasymforall}x{\isachardot}\ {\isasymTurnstile}\ {\isacharparenleft}O\ {\isacharbraceleft}{\isacharparenleft}lie{\isacharparenleft}x{\isacharparenright}{\isacharparenright}{\isacharbraceright}{\isacharparenright}{\isachardoublequoteclose}\isanewline
\ \ \isacommand{nitpick}\isamarkupfalse%
\ {\isacharbrackleft}user{\isacharunderscore}axioms{\isacharbrackright}%
\isadelimproof
\ %
\endisadelimproof
%
\isatagproof
\isacommand{oops}\isamarkupfalse%
\isanewline
%
\isamarkupcmt{Nitpick found a counterexample for card person = 2 and card i = 2:

  Free variable:
    lie = ($\lambda x. \_$)($p_1$ := ($\lambda x. \_$)($i_1$ := False, $i_2$ := True), $p_2$ := ($\lambda x. \_$)($i_1$ := False, $i_2$ := False))
  Skolem constant:
    x = $p_2$%
}\isanewline
%
\isamarkupcmt{This lemma demonstrates the point even more clearly - we really want to think that obligations are consistent 
across people, but because we don't have a notion of agency, we don't have that property.%
}%
\endisatagproof
{\isafoldproof}%
%
\isadelimproof
%
\endisadelimproof
%
\isadelimdocument
%
\endisadelimdocument
%
\isatagdocument
%
\isamarkupsubsection{Consistent Sentences%
}
\isamarkuptrue%
%
\endisatagdocument
{\isafolddocument}%
%
\isadelimdocument
%
\endisadelimdocument
%
\begin{isamarkuptext}%
The above section tested validity. We might also be interested in some weaker properties%
\end{isamarkuptext}\isamarkuptrue%
%
\begin{isamarkuptext}%
Let's test whether certain sentences are consistent - can we find a model that makes them true?%
\end{isamarkuptext}\isamarkuptrue%
\isacommand{lemma}\isamarkupfalse%
\ permissible{\isacharcolon}\isanewline
\ \ \isakeyword{fixes}\ A\isanewline
\ \ \isakeyword{shows}\ {\isachardoublequoteopen}{\isacharparenleft}{\isacharparenleft}\isactrlbold {\isasymnot}\ {\isacharparenleft}O\ {\isacharbraceleft}A{\isacharbraceright}{\isacharparenright}{\isacharparenright}\ \isactrlbold {\isasymand}\ {\isacharparenleft}\isactrlbold {\isasymnot}\ {\isacharparenleft}O\ {\isacharbraceleft}\isactrlbold {\isasymnot}\ A{\isacharbraceright}{\isacharparenright}{\isacharparenright}{\isacharparenright}\ w{\isachardoublequoteclose}\isanewline
\ \ \isacommand{nitpick}\isamarkupfalse%
\ {\isacharbrackleft}user{\isacharunderscore}axioms{\isacharcomma}\ falsify{\isacharequal}false{\isacharbrackright}%
\isadelimproof
\ %
\endisadelimproof
%
\isatagproof
\isacommand{oops}\isamarkupfalse%
\isanewline
%
\isamarkupcmt{``Nitpick found a model for card i = 1:

  Free variable:
    A = ($\lambda x. \_$)($i_1$ := False)"%
}\isanewline
%
\isamarkupcmt{Awesome! Permissible things are consistent - clearly we've fixed the bug from categorical\_imperative\_1%
}\isanewline
%
\isamarkupcmt{Note that apparently it's not clear {@cite kitcher} if Kant actually thought that permissibility was a coherent concept. Either way, 
I think permissibility is a pretty widely accepted ethical phenomenon.%
}%
\endisatagproof
{\isafoldproof}%
%
\isadelimproof
%
\endisadelimproof
\isanewline
\isanewline
\isacommand{lemma}\isamarkupfalse%
\ conflicting{\isacharunderscore}obligations{\isacharcolon}\isanewline
\ \ \isakeyword{fixes}\ A\isanewline
\ \ \isakeyword{shows}\ {\isachardoublequoteopen}{\isacharparenleft}O\ {\isacharbraceleft}A{\isacharbraceright}\ \isactrlbold {\isasymand}\ O\ {\isacharbraceleft}\isactrlbold {\isasymnot}\ A{\isacharbraceright}{\isacharparenright}\ w{\isachardoublequoteclose}\isanewline
\ \ \isacommand{nitpick}\isamarkupfalse%
\ {\isacharbrackleft}user{\isacharunderscore}axioms{\isacharcomma}\ falsify{\isacharequal}false{\isacharbrackright}%
\isadelimproof
\ %
\endisadelimproof
%
\isatagproof
\isacommand{oops}\isamarkupfalse%
\isanewline
%
\isamarkupcmt{``Nitpick found a model for card i = 2:

  Free variable:
    A = ($\lambda x. \_$)($i_1$ := False, $i_2$ := True)"%
}\isanewline
%
\isamarkupcmt{Oh no! Nitpick found a model with conflicting obligations - that's bad!%
}%
\endisatagproof
{\isafoldproof}%
%
\isadelimproof
%
\endisadelimproof
%
\isadelimdocument
%
\endisadelimdocument
%
\isatagdocument
%
\isamarkupsubsection{Metaethical Tests%
}
\isamarkuptrue%
%
\endisatagdocument
{\isafolddocument}%
%
\isadelimdocument
%
\endisadelimdocument
\isacommand{lemma}\isamarkupfalse%
\ FUL{\isacharunderscore}alternate{\isacharcolon}\isanewline
\ \ \isakeyword{shows}\ {\isachardoublequoteopen}{\isasymTurnstile}\ {\isacharparenleft}{\isacharparenleft}{\isasymdiamond}\ {\isacharparenleft}O\ {\isacharbraceleft}\isactrlbold {\isasymnot}\ A{\isacharbraceright}{\isacharparenright}{\isacharparenright}\ \isactrlbold {\isasymrightarrow}\ {\isacharparenleft}O\ {\isacharbraceleft}\isactrlbold {\isasymnot}\ A{\isacharbraceright}{\isacharparenright}{\isacharparenright}{\isachardoublequoteclose}\isanewline
%
\isadelimproof
\ \ %
\endisadelimproof
%
\isatagproof
\isacommand{by}\isamarkupfalse%
\ simp\isanewline
%
\isamarkupcmt{One problem becomes obvious if we look at the definition of permissible%
}\isanewline
%
\isamarkupcmt{Expanding the FUL gives us: $\sim \Box \sim O(\sim A) \longrightarrow O(\sim A)$%
}\isanewline
%
\isamarkupcmt{By modal duals we get that $\diamond O(\sim A) \longrightarrow O(\sim A)$%
}\isanewline
%
\isamarkupcmt{This means that if something is possibly prohibited, it is in fact prohibited.%
}\isanewline
%
\isamarkupcmt{I'm not convinced that this is a desirable property of an ethical theory.%
}%
\endisatagproof
{\isafoldproof}%
%
\isadelimproof
\isanewline
%
\endisadelimproof
\isanewline
\isacommand{lemma}\isamarkupfalse%
\ arbitrary{\isacharunderscore}obligations{\isacharcolon}\isanewline
\ \ \isakeyword{fixes}\ A{\isacharcolon}{\isacharcolon}{\isachardoublequoteopen}t{\isachardoublequoteclose}\isanewline
\ \ \isakeyword{shows}\ {\isachardoublequoteopen}O\ {\isacharbraceleft}A{\isacharbraceright}\ w{\isachardoublequoteclose}\isanewline
\ \ \isacommand{nitpick}\isamarkupfalse%
\ {\isacharbrackleft}user{\isacharunderscore}axioms{\isacharequal}true{\isacharbrackright}%
\isadelimproof
\ %
\endisadelimproof
%
\isatagproof
\isacommand{oops}\isamarkupfalse%
\isanewline
%
\isamarkupcmt{``Nitpick found a counterexample for card i = 1:

  Free variable:
    A = ($\lambda x. \_$)($i_1$ := False)"%
}\isanewline
%
\isamarkupcmt{This is good! Shows us that any arbitrary term isn't obligatory.%
}%
\endisatagproof
{\isafoldproof}%
%
\isadelimproof
%
\endisadelimproof
\isanewline
\isanewline
\isacommand{lemma}\isamarkupfalse%
\ removing{\isacharunderscore}conflicting{\isacharunderscore}obligations{\isacharcolon}\isanewline
\ \ \isakeyword{assumes}\ {\isachardoublequoteopen}{\isasymforall}A{\isachardot}\ {\isasymTurnstile}\ {\isacharparenleft}\isactrlbold {\isasymnot}\ {\isacharparenleft}O\ {\isacharbraceleft}A{\isacharbraceright}\ \isactrlbold {\isasymand}\ O\ {\isacharbraceleft}\isactrlbold {\isasymnot}\ A{\isacharbraceright}{\isacharparenright}{\isacharparenright}{\isachardoublequoteclose}\isanewline
\ \ \isakeyword{shows}\ True\isanewline
\ \ \isacommand{nitpick}\isamarkupfalse%
\ {\isacharbrackleft}satisfy{\isacharcomma}user{\isacharunderscore}axioms{\isacharcomma}format{\isacharequal}{\isadigit{2}}{\isacharbrackright}%
\isadelimproof
\ %
\endisadelimproof
%
\isatagproof
\isacommand{oops}\isamarkupfalse%
\isanewline
%
\isamarkupcmt{`` Nitpick found a model for card i = 1:

  Empty assignment"%
}\isanewline
%
\isamarkupcmt{We can disallow conflicting obligations and the system is still consistent - that's good.%
}%
\endisatagproof
{\isafoldproof}%
%
\isadelimproof
%
\endisadelimproof
\isanewline
\isanewline
\isacommand{lemma}\isamarkupfalse%
\ implied{\isacharunderscore}contradiction{\isacharcolon}\isanewline
\ \ \isakeyword{fixes}\ A{\isacharcolon}{\isacharcolon}{\isachardoublequoteopen}t{\isachardoublequoteclose}\isanewline
\ \ \isakeyword{fixes}\ B{\isacharcolon}{\isacharcolon}{\isachardoublequoteopen}t{\isachardoublequoteclose}\ \isanewline
\ \ \isakeyword{assumes}\ {\isachardoublequoteopen}{\isasymTurnstile}{\isacharparenleft}\isactrlbold {\isasymnot}\ {\isacharparenleft}A\ \isactrlbold {\isasymand}\ B{\isacharparenright}{\isacharparenright}{\isachardoublequoteclose}\isanewline
\ \ \isakeyword{shows}\ {\isachardoublequoteopen}{\isasymTurnstile}{\isacharparenleft}\isactrlbold {\isasymnot}\ {\isacharparenleft}O\ {\isacharbraceleft}A{\isacharbraceright}\ \isactrlbold {\isasymand}\ O\ {\isacharbraceleft}B{\isacharbraceright}{\isacharparenright}{\isacharparenright}{\isachardoublequoteclose}\isanewline
\ \ \isacommand{nitpick}\isamarkupfalse%
\ {\isacharbrackleft}user{\isacharunderscore}axioms{\isacharbrackright}\isanewline
%
\isadelimproof
%
\endisadelimproof
%
\isatagproof
\isacommand{proof}\isamarkupfalse%
\ {\isacharminus}\ \isanewline
\ \ \isacommand{have}\isamarkupfalse%
\ {\isachardoublequoteopen}{\isasymTurnstile}{\isacharparenleft}\isactrlbold {\isasymnot}{\isacharparenleft}{\isasymdiamond}{\isacharparenleft}A\ \isactrlbold {\isasymand}\ B{\isacharparenright}{\isacharparenright}{\isacharparenright}{\isachardoublequoteclose}\isanewline
\ \ \ \ \isacommand{by}\isamarkupfalse%
\ {\isacharparenleft}simp\ add{\isacharcolon}\ assms{\isacharparenright}\isanewline
\ \ \isacommand{then}\isamarkupfalse%
\ \isacommand{have}\isamarkupfalse%
\ {\isachardoublequoteopen}{\isasymTurnstile}{\isacharparenleft}\isactrlbold {\isasymnot}\ {\isacharparenleft}O\ {\isacharbraceleft}A\ \isactrlbold {\isasymand}\ B{\isacharbraceright}{\isacharparenright}{\isacharparenright}{\isachardoublequoteclose}\ \isacommand{by}\isamarkupfalse%
\ {\isacharparenleft}smt\ carmojones{\isacharunderscore}DDL{\isacharunderscore}completeness{\isachardot}O{\isacharunderscore}diamond{\isacharparenright}\isanewline
\ \ \isacommand{thus}\isamarkupfalse%
\ {\isacharquery}thesis\ \isacommand{oops}\isamarkupfalse%
\isanewline
%
\isamarkupcmt{\cite{KorsgaardFUL} mentions that if two maxims imply a contradiction, they must not be willed.%
}\isanewline
%
\isamarkupcmt{Above is a natural interpretation of this fact that we are, so far, unable to prove.%
}\isanewline
%
\isamarkupcmt{``Nitpick found a counterexample for card i = 2:

  Free variables:
    A = ($\lambda x. \_$)($i_1$ := True, $i_2$ := False)
    B = ($\lambda x. \_$)($i_1$ := False, $i_2$ := True)"%
}\isanewline
%
\isamarkupcmt{This isn't actually a theorem of the logic as formed - clearly this is a problem.%
}%
\endisatagproof
{\isafoldproof}%
%
\isadelimproof
\isanewline
%
\endisadelimproof
\isanewline
\isacommand{lemma}\isamarkupfalse%
\ distribute{\isacharunderscore}obligations{\isacharunderscore}if{\isacharcolon}\isanewline
\ \ \isakeyword{assumes}\ {\isachardoublequoteopen}{\isasymTurnstile}\ O\ {\isacharbraceleft}A\ \isactrlbold {\isasymand}\ B{\isacharbraceright}{\isachardoublequoteclose}\isanewline
\ \ \isakeyword{shows}\ {\isachardoublequoteopen}{\isasymTurnstile}\ {\isacharparenleft}O\ {\isacharbraceleft}A{\isacharbraceright}\ \isactrlbold {\isasymand}\ O\ {\isacharbraceleft}B{\isacharbraceright}{\isacharparenright}{\isachardoublequoteclose}\isanewline
\ \ \isacommand{nitpick}\isamarkupfalse%
\ {\isacharbrackleft}user{\isacharunderscore}axioms{\isacharcomma}\ falsify{\isacharequal}true{\isacharcomma}\ verbose{\isacharbrackright}\isanewline
%
\isadelimproof
\ \ %
\endisadelimproof
%
\isatagproof
\isacommand{oops}\isamarkupfalse%
\isanewline
%
\isamarkupcmt{Nitpick can't find a countermodel for this theorem, and sledgehammer can't find a proof.%
}\isanewline
%
\isamarkupcmt{Super strange. I wonder if this is similar to $\Box (A \wedge B)$ vs $\Box A \wedge \Box B$%
}%
\endisatagproof
{\isafoldproof}%
%
\isadelimproof
\isanewline
%
\endisadelimproof
\isanewline
\isacommand{lemma}\isamarkupfalse%
\ distribute{\isacharunderscore}boxes{\isacharcolon}\isanewline
\ \ \isakeyword{assumes}\ {\isachardoublequoteopen}{\isasymTurnstile}{\isacharparenleft}\ {\isasymbox}{\isacharparenleft}A\ \isactrlbold {\isasymand}\ B{\isacharparenright}{\isacharparenright}{\isachardoublequoteclose}\isanewline
\ \ \isakeyword{shows}\ {\isachardoublequoteopen}{\isasymTurnstile}\ {\isacharparenleft}{\isacharparenleft}{\isasymbox}A{\isacharparenright}\ \isactrlbold {\isasymand}\ {\isacharparenleft}{\isasymbox}B{\isacharparenright}{\isacharparenright}{\isachardoublequoteclose}\isanewline
%
\isadelimproof
\ \ %
\endisadelimproof
%
\isatagproof
\isacommand{using}\isamarkupfalse%
\ assms\ \isacommand{by}\isamarkupfalse%
\ blast\isanewline
%
\isamarkupcmt{We really expect the O operator to be acting like the $\square$ operator. It's like a modal necessity operator,
like necessity across ideal worlds instead of actual worlds. Therefore, we'd expect theorems that hold of $\square$
to also hold of O.%
}%
\endisatagproof
{\isafoldproof}%
%
\isadelimproof
\isanewline
%
\endisadelimproof
\isanewline
\isanewline
\isacommand{lemma}\isamarkupfalse%
\ distribute{\isacharunderscore}obligations{\isacharunderscore}onlyif{\isacharcolon}\isanewline
\ \ \isakeyword{assumes}\ \ {\isachardoublequoteopen}{\isasymTurnstile}\ {\isacharparenleft}O\ {\isacharbraceleft}A{\isacharbraceright}\ \isactrlbold {\isasymand}\ O\ {\isacharbraceleft}B{\isacharbraceright}{\isacharparenright}{\isachardoublequoteclose}\isanewline
\ \ \isakeyword{shows}\ {\isachardoublequoteopen}{\isasymTurnstile}\ O\ {\isacharbraceleft}A\ \isactrlbold {\isasymand}\ B{\isacharbraceright}{\isachardoublequoteclose}\isanewline
\ \ \isacommand{nitpick}\isamarkupfalse%
\ {\isacharbrackleft}user{\isacharunderscore}axioms{\isacharbrackright}%
\isadelimproof
\ %
\endisadelimproof
%
\isatagproof
\isacommand{oops}\isamarkupfalse%
\isanewline
%
\isamarkupcmt{``Nitpick found a counterexample for card i = 2:

  Free variables:
    A = ($\lambda x. \_$)($i_1$ := True, $i_2$ := False)
    B = ($\lambda x. \_$)($i_1$ := False, $i_2$ := True)"%
}\isanewline
%
\isamarkupcmt{If this was a theorem, then contradictory obligations would be ruled out pretty immediately.%
}\isanewline
%
\isamarkupcmt{Note that all of this holds in CJ's original DDL as well, not just my modified version.%
}\isanewline
%
\isamarkupcmt{We might imagine adding this equivalence to our system.%
}%
\endisatagproof
{\isafoldproof}%
%
\isadelimproof
%
\endisadelimproof
\isanewline
\isanewline
\isacommand{lemma}\isamarkupfalse%
\ ought{\isacharunderscore}implies{\isacharunderscore}can{\isacharcolon}\isanewline
\ \ \isakeyword{shows}\ {\isachardoublequoteopen}{\isasymforall}A{\isachardot}\ {\isasymTurnstile}\ {\isacharparenleft}O\ {\isacharbraceleft}A{\isacharbraceright}\ \isactrlbold {\isasymrightarrow}\ {\isacharparenleft}{\isasymdiamond}A{\isacharparenright}{\isacharparenright}{\isachardoublequoteclose}\isanewline
%
\isadelimproof
\ \ %
\endisadelimproof
%
\isatagproof
\isacommand{using}\isamarkupfalse%
\ O{\isacharunderscore}diamond\ \isacommand{by}\isamarkupfalse%
\ blast\isanewline
%
\isamarkupcmt{``ought implies can" is often attributed to Kant and is a pretty basic principle - you can't be obligated to do the impossible.
I'm not surprised that our base logic has this as an axiom. It's often said to be the central motivation behind deontic logics.%
}%
\endisatagproof
{\isafoldproof}%
%
\isadelimproof
\isanewline
%
\endisadelimproof
\isanewline
%
\isadelimtheory
\isanewline
%
\endisadelimtheory
%
\isatagtheory
\isacommand{end}\isamarkupfalse%
%
\endisatagtheory
{\isafoldtheory}%
%
\isadelimtheory
%
\endisadelimtheory
%
\end{isabellebody}%
\endinput
%:%file=~/Desktop/cs91r/categorical_imperative_naive.thy%:%
%:%10=1%:%
%:%11=1%:%
%:%12=2%:%
%:%13=3%:%
%:%27=5%:%
%:%31=7%:%
%:%43=9%:%
%:%45=11%:%
%:%46=11%:%
%:%47=12%:%
%:%49=13%:%
%:%50=13%:%
%:%52=14%:%
%:%53=14%:%
%:%55=15%:%
%:%56=15%:%
%:%58=16%:%
%:%61=18%:%
%:%62=19%:%
%:%63=20%:%
%:%65=22%:%
%:%66=22%:%
%:%67=23%:%
%:%74=25%:%
%:%84=27%:%
%:%85=27%:%
%:%86=27%:%
%:%88=27%:%
%:%92=27%:%
%:%93=27%:%
%:%95=28%:%
%:%96=29%:%
%:%97=30%:%
%:%98=30%:%
%:%100=31%:%
%:%101=31%:%
%:%103=32%:%
%:%111=32%:%
%:%112=33%:%
%:%113=34%:%
%:%114=34%:%
%:%115=35%:%
%:%116=36%:%
%:%117=36%:%
%:%120=37%:%
%:%124=37%:%
%:%125=37%:%
%:%127=38%:%
%:%128=38%:%
%:%130=39%:%
%:%131=39%:%
%:%133=40%:%
%:%134=41%:%
%:%135=42%:%
%:%136=42%:%
%:%138=43%:%
%:%144=43%:%
%:%147=44%:%
%:%148=45%:%
%:%149=45%:%
%:%150=46%:%
%:%151=47%:%
%:%152=47%:%
%:%155=48%:%
%:%159=48%:%
%:%160=48%:%
%:%162=49%:%
%:%163=50%:%
%:%164=51%:%
%:%165=52%:%
%:%166=52%:%
%:%168=53%:%
%:%169=53%:%
%:%171=54%:%
%:%177=54%:%
%:%180=55%:%
%:%181=56%:%
%:%182=56%:%
%:%183=57%:%
%:%184=58%:%
%:%185=58%:%
%:%188=59%:%
%:%192=59%:%
%:%193=59%:%
%:%195=60%:%
%:%196=61%:%
%:%197=62%:%
%:%198=62%:%
%:%200=63%:%
%:%206=63%:%
%:%209=64%:%
%:%210=65%:%
%:%211=65%:%
%:%212=66%:%
%:%213=67%:%
%:%214=67%:%
%:%217=68%:%
%:%221=68%:%
%:%222=68%:%
%:%224=69%:%
%:%225=70%:%
%:%226=71%:%
%:%227=72%:%
%:%228=72%:%
%:%230=73%:%
%:%231=73%:%
%:%233=74%:%
%:%248=76%:%
%:%260=78%:%
%:%262=80%:%
%:%263=80%:%
%:%264=81%:%
%:%265=81%:%
%:%266=82%:%
%:%267=83%:%
%:%268=83%:%
%:%269=84%:%
%:%270=85%:%
%:%273=86%:%
%:%277=86%:%
%:%278=86%:%
%:%279=86%:%
%:%281=87%:%
%:%287=87%:%
%:%290=88%:%
%:%291=89%:%
%:%292=89%:%
%:%293=90%:%
%:%294=91%:%
%:%295=91%:%
%:%296=92%:%
%:%297=93%:%
%:%300=94%:%
%:%304=94%:%
%:%305=94%:%
%:%306=94%:%
%:%308=95%:%
%:%309=96%:%
%:%319=99%:%
%:%321=101%:%
%:%322=101%:%
%:%323=102%:%
%:%324=102%:%
%:%325=103%:%
%:%326=103%:%
%:%327=104%:%
%:%328=105%:%
%:%329=105%:%
%:%330=106%:%
%:%331=107%:%
%:%332=108%:%
%:%333=108%:%
%:%336=109%:%
%:%340=109%:%
%:%341=109%:%
%:%343=110%:%
%:%344=110%:%
%:%346=111%:%
%:%347=111%:%
%:%349=112%:%
%:%350=112%:%
%:%352=113%:%
%:%353=114%:%
%:%354=115%:%
%:%355=116%:%
%:%361=116%:%
%:%364=117%:%
%:%365=118%:%
%:%366=118%:%
%:%367=119%:%
%:%368=120%:%
%:%369=121%:%
%:%370=121%:%
%:%372=121%:%
%:%376=121%:%
%:%377=121%:%
%:%379=122%:%
%:%380=123%:%
%:%381=124%:%
%:%382=125%:%
%:%383=126%:%
%:%384=127%:%
%:%385=127%:%
%:%387=128%:%
%:%388=129%:%
%:%403=131%:%
%:%415=133%:%
%:%419=134%:%
%:%421=136%:%
%:%422=136%:%
%:%423=137%:%
%:%424=138%:%
%:%425=139%:%
%:%426=139%:%
%:%428=139%:%
%:%432=139%:%
%:%433=139%:%
%:%435=140%:%
%:%436=141%:%
%:%437=142%:%
%:%438=143%:%
%:%439=143%:%
%:%441=144%:%
%:%442=144%:%
%:%444=145%:%
%:%445=146%:%
%:%453=146%:%
%:%454=147%:%
%:%455=148%:%
%:%456=148%:%
%:%457=149%:%
%:%458=150%:%
%:%459=151%:%
%:%460=151%:%
%:%462=151%:%
%:%466=151%:%
%:%467=151%:%
%:%469=152%:%
%:%470=153%:%
%:%471=154%:%
%:%472=155%:%
%:%473=155%:%
%:%475=156%:%
%:%490=158%:%
%:%500=160%:%
%:%501=160%:%
%:%502=161%:%
%:%505=162%:%
%:%509=162%:%
%:%510=162%:%
%:%512=163%:%
%:%513=163%:%
%:%515=164%:%
%:%516=164%:%
%:%518=165%:%
%:%519=165%:%
%:%521=166%:%
%:%522=166%:%
%:%524=167%:%
%:%530=167%:%
%:%533=168%:%
%:%534=169%:%
%:%535=169%:%
%:%536=170%:%
%:%537=171%:%
%:%538=172%:%
%:%539=172%:%
%:%541=172%:%
%:%545=172%:%
%:%546=172%:%
%:%548=173%:%
%:%549=174%:%
%:%550=175%:%
%:%551=176%:%
%:%552=176%:%
%:%554=177%:%
%:%562=177%:%
%:%563=178%:%
%:%564=179%:%
%:%565=179%:%
%:%566=180%:%
%:%567=181%:%
%:%568=182%:%
%:%569=182%:%
%:%571=182%:%
%:%575=182%:%
%:%576=182%:%
%:%578=183%:%
%:%579=184%:%
%:%580=185%:%
%:%581=185%:%
%:%583=186%:%
%:%591=186%:%
%:%592=187%:%
%:%593=188%:%
%:%594=188%:%
%:%595=189%:%
%:%596=190%:%
%:%597=191%:%
%:%598=192%:%
%:%599=193%:%
%:%600=193%:%
%:%607=194%:%
%:%608=194%:%
%:%609=195%:%
%:%610=195%:%
%:%611=196%:%
%:%612=196%:%
%:%613=197%:%
%:%614=197%:%
%:%615=197%:%
%:%616=197%:%
%:%617=198%:%
%:%618=198%:%
%:%619=198%:%
%:%621=199%:%
%:%622=199%:%
%:%624=200%:%
%:%625=200%:%
%:%627=201%:%
%:%628=202%:%
%:%629=203%:%
%:%630=204%:%
%:%631=205%:%
%:%632=205%:%
%:%634=206%:%
%:%640=206%:%
%:%643=207%:%
%:%644=208%:%
%:%645=208%:%
%:%646=209%:%
%:%647=210%:%
%:%648=211%:%
%:%649=211%:%
%:%652=212%:%
%:%656=212%:%
%:%657=212%:%
%:%659=213%:%
%:%660=213%:%
%:%662=214%:%
%:%668=214%:%
%:%671=215%:%
%:%672=216%:%
%:%673=216%:%
%:%674=217%:%
%:%675=218%:%
%:%678=219%:%
%:%682=219%:%
%:%683=219%:%
%:%684=219%:%
%:%686=220%:%
%:%687=221%:%
%:%688=222%:%
%:%694=222%:%
%:%697=223%:%
%:%698=224%:%
%:%699=225%:%
%:%700=225%:%
%:%701=226%:%
%:%702=227%:%
%:%703=228%:%
%:%704=228%:%
%:%706=228%:%
%:%710=228%:%
%:%711=228%:%
%:%713=229%:%
%:%714=230%:%
%:%715=231%:%
%:%716=232%:%
%:%717=233%:%
%:%718=233%:%
%:%720=234%:%
%:%721=234%:%
%:%723=235%:%
%:%724=235%:%
%:%726=236%:%
%:%734=236%:%
%:%735=237%:%
%:%736=238%:%
%:%737=238%:%
%:%738=239%:%
%:%741=240%:%
%:%745=240%:%
%:%746=240%:%
%:%747=240%:%
%:%749=241%:%
%:%750=242%:%
%:%756=242%:%
%:%759=243%:%
%:%762=244%:%
%:%767=245%:%