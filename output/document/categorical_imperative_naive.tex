%
\begin{isabellebody}%
\setisabellecontext{categorical{\isacharunderscore}imperative{\isacharunderscore}naive}%
%
\isadelimtheory
%
\endisadelimtheory
%
\isatagtheory
\isacommand{theory}\isamarkupfalse%
\ categorical{\isacharunderscore}imperative{\isacharunderscore}naive\ \isakeyword{imports}\ carmojones{\isacharunderscore}DDL{\isacharunderscore}completeness\isanewline
\isanewline
\isakeyword{begin}%
\endisatagtheory
{\isafoldtheory}%
%
\isadelimtheory
%
\endisadelimtheory
%
\isadelimdocument
%
\endisadelimdocument
%
\isatagdocument
%
\isamarkupsection{The Categorical Imperative%
}
\isamarkuptrue%
%
\isamarkupsubsection{Simple Formulation of the Formula of Universal Law%
}
\isamarkuptrue%
%
\endisatagdocument
{\isafolddocument}%
%
\isadelimdocument
%
\endisadelimdocument
%
\begin{isamarkuptext}%
This is my second attempt at formalizing the Formula of Universal Law%
\end{isamarkuptext}\isamarkuptrue%
\isacommand{abbreviation}\isamarkupfalse%
\ ddlpermissable{\isacharcolon}{\isacharcolon}{\isachardoublequoteopen}t{\isasymRightarrow}t{\isachardoublequoteclose}\ {\isacharparenleft}{\isachardoublequoteopen}P{\isacharunderscore}{\isachardoublequoteclose}{\isacharparenright}\isanewline
\ \ \isakeyword{where}\ {\isachardoublequoteopen}{\isacharparenleft}P\ A{\isacharparenright}\ {\isasymequiv}\ {\isacharparenleft}\isactrlbold {\isasymnot}{\isacharparenleft}O\ {\isacharbraceleft}\isactrlbold {\isasymnot}A{\isacharbraceright}{\isacharparenright}{\isacharparenright}{\isachardoublequoteclose}\isanewline
%
\isamarkupcmt{This operator represents permissibility%
}\isanewline
%
\isamarkupcmt{Will be useful when discussing the categorical imperative%
}\isanewline
%
\isamarkupcmt{Something is permissible if it is not prohibited%
}\isanewline
%
\isamarkupcmt{Something is prohibited if its negation is obligatory%
}%
\begin{isamarkuptext}%
Let's consider a naive reading of the Formula of Universal Law (FUL).
From the Groundwork, 'act only in accordance with that maxim through which you can at the same time will that it become a universal law'.
What does this mean in DDL? One interpretation is if A is not necessarily permissible, then its negation is obligated.%
\end{isamarkuptext}\isamarkuptrue%
\isacommand{axiomatization}\isamarkupfalse%
\ \isakeyword{where}\isanewline
FUL{\isacharunderscore}{\isadigit{1}}{\isacharcolon}\ {\isachardoublequoteopen}{\isasymTurnstile}\ {\isacharparenleft}{\isacharparenleft}\isactrlbold {\isasymnot}{\isacharparenleft}{\isasymbox}\ {\isacharparenleft}P\ A{\isacharparenright}{\isacharparenright}{\isacharparenright}\ \isactrlbold {\isasymrightarrow}\ {\isacharparenleft}O\ {\isacharbraceleft}{\isacharparenleft}\isactrlbold {\isasymnot}A{\isacharparenright}{\isacharbraceright}{\isacharparenright}{\isacharparenright}{\isachardoublequoteclose}%
\isadelimdocument
%
\endisadelimdocument
%
\isatagdocument
%
\isamarkupsubsection{Basic Tests%
}
\isamarkuptrue%
%
\endisatagdocument
{\isafolddocument}%
%
\isadelimdocument
%
\endisadelimdocument
\isacommand{lemma}\isamarkupfalse%
\ True\ \isacommand{nitpick}\isamarkupfalse%
\ {\isacharbrackleft}satisfy{\isacharcomma}user{\isacharunderscore}axioms{\isacharcomma}format{\isacharequal}{\isadigit{2}}{\isacharbrackright}%
\isadelimproof
\ %
\endisadelimproof
%
\isatagproof
\isacommand{oops}\isamarkupfalse%
\isanewline
%
\isamarkupcmt{``Nitpick found a model for card i = 1:

  Empty assignment"%
}\isanewline
%
\isamarkupcmt{Nitpick tells us that the FUL is consistent%
}\isanewline
%
\isamarkupcmt{``oops" after Nitpick does not mean Nitpick failed.%
}%
\endisatagproof
{\isafoldproof}%
%
\isadelimproof
%
\endisadelimproof
\isanewline
\isanewline
\isacommand{lemma}\isamarkupfalse%
\ something{\isacharunderscore}is{\isacharunderscore}obligatory{\isacharcolon}\isanewline
\ \ \isakeyword{shows}\ {\isachardoublequoteopen}{\isasymforall}\ w{\isachardot}\ {\isasymexists}\ A{\isachardot}\ O\ {\isacharbraceleft}A{\isacharbraceright}\ w{\isachardoublequoteclose}\isanewline
\ \ \isacommand{nitpick}\isamarkupfalse%
\ {\isacharbrackleft}user{\isacharunderscore}axioms{\isacharbrackright}\isanewline
%
\isadelimproof
\ \ %
\endisadelimproof
%
\isatagproof
\isacommand{oops}\isamarkupfalse%
\isanewline
%
\isamarkupcmt{We might think that in every world we want something to be obligated.%
}\isanewline
%
\isamarkupcmt{Sadly, Sledgehammer times out trying to prove this. Let's relax this%
}\isanewline
%
\isamarkupcmt{``Nitpick found a counterexample for card i = 1:

  Empty assignment"%
}\isanewline
%
\isamarkupcmt{Nitpick to the rescue! The formula is in fact not valid.%
}%
\endisatagproof
{\isafoldproof}%
%
\isadelimproof
\isanewline
%
\endisadelimproof
\isanewline
\isacommand{lemma}\isamarkupfalse%
\ something{\isacharunderscore}is{\isacharunderscore}obligatory{\isacharunderscore}{\isadigit{2}}{\isacharcolon}\isanewline
\ \ \isakeyword{shows}\ {\isachardoublequoteopen}{\isasymforall}\ w{\isachardot}\ {\isasymexists}\ A{\isachardot}\ O\ {\isacharbraceleft}A{\isacharbraceright}\ w{\isachardoublequoteclose}\isanewline
\ \ \isacommand{nitpick}\isamarkupfalse%
\ {\isacharbrackleft}user{\isacharunderscore}axioms{\isacharcomma}\ falsify{\isacharequal}false{\isacharbrackright}\isanewline
%
\isadelimproof
\ \ %
\endisadelimproof
%
\isatagproof
\isacommand{oops}\isamarkupfalse%
\isanewline
%
\isamarkupcmt{``Nitpick found a model for card i = 1:

  Skolem constant:
    A = ($\lambda x. \_$)($i_1$ := True)"%
}\isanewline
%
\isamarkupcmt{Nitpick tells us that the formula is consistent - it found a model where the formula is true.%
}\isanewline
%
\isamarkupcmt{This means that our model is underspecified - this formula is neither valid nor inconsistent.%
}%
\endisatagproof
{\isafoldproof}%
%
\isadelimproof
\isanewline
%
\endisadelimproof
\isanewline
\isacommand{lemma}\isamarkupfalse%
\ something{\isacharunderscore}is{\isacharunderscore}obligatory{\isacharunderscore}relaxed{\isacharcolon}\isanewline
\ \ \isakeyword{shows}\ {\isachardoublequoteopen}{\isasymexists}\ A\ w{\isachardot}\ O\ {\isacharbraceleft}A{\isacharbraceright}\ w{\isachardoublequoteclose}\isanewline
\ \ \isacommand{nitpick}\isamarkupfalse%
\ {\isacharbrackleft}user{\isacharunderscore}axioms{\isacharbrackright}\isanewline
%
\isadelimproof
\ \ %
\endisadelimproof
%
\isatagproof
\isacommand{oops}\isamarkupfalse%
\isanewline
%
\isamarkupcmt{``Nitpick found a counterexample for card i = 1:

  Empty assignment"%
}\isanewline
%
\isamarkupcmt{The relaxed version definitely isn't valid.%
}%
\endisatagproof
{\isafoldproof}%
%
\isadelimproof
\isanewline
%
\endisadelimproof
\isanewline
\isacommand{lemma}\isamarkupfalse%
\ something{\isacharunderscore}is{\isacharunderscore}obligatory{\isacharunderscore}relaxed{\isacharunderscore}{\isadigit{2}}{\isacharcolon}\isanewline
\ \ \isakeyword{shows}\ {\isachardoublequoteopen}{\isasymexists}\ A\ w{\isachardot}\ O\ {\isacharbraceleft}A{\isacharbraceright}\ w{\isachardoublequoteclose}\isanewline
\ \ \isacommand{nitpick}\isamarkupfalse%
\ {\isacharbrackleft}user{\isacharunderscore}axioms{\isacharcomma}\ falsify{\isacharequal}false{\isacharbrackright}\isanewline
%
\isadelimproof
\ \ %
\endisadelimproof
%
\isatagproof
\isacommand{oops}\isamarkupfalse%
\isanewline
%
\isamarkupcmt{``Nitpick found a model for card i = 1:

  Skolem constant:
    A = ($\lambda x. \_$)($i_1$ := True)"%
}\isanewline
%
\isamarkupcmt{Nitpick tells us that the formula is consistent - it found a model where the formula is true.%
}\isanewline
%
\isamarkupcmt{The model seems underspecified.%
}%
\endisatagproof
{\isafoldproof}%
%
\isadelimproof
%
\endisadelimproof
%
\isadelimdocument
%
\endisadelimdocument
%
\isatagdocument
%
\isamarkupsubsection{Specifying the Model%
}
\isamarkuptrue%
%
\endisatagdocument
{\isafolddocument}%
%
\isadelimdocument
%
\endisadelimdocument
%
\begin{isamarkuptext}%
Let's specify the model. What if we add something impermissible?%
\end{isamarkuptext}\isamarkuptrue%
\isacommand{consts}\isamarkupfalse%
\ M{\isacharcolon}{\isacharcolon}{\isachardoublequoteopen}t{\isachardoublequoteclose}\isanewline
\isacommand{abbreviation}\isamarkupfalse%
\ murder{\isacharunderscore}wrong{\isacharcolon}{\isacharcolon}{\isachardoublequoteopen}bool{\isachardoublequoteclose}\ \isakeyword{where}\ {\isachardoublequoteopen}murder{\isacharunderscore}wrong\ {\isasymequiv}\ {\isasymTurnstile}{\isacharparenleft}O\ {\isacharbraceleft}\isactrlbold {\isasymnot}\ M{\isacharbraceright}{\isacharparenright}{\isachardoublequoteclose}\isanewline
\isanewline
\isacommand{lemma}\isamarkupfalse%
\ something{\isacharunderscore}is{\isacharunderscore}obligatory{\isacharunderscore}{\isadigit{2}}{\isacharcolon}\isanewline
\ \ \isakeyword{assumes}\ murder{\isacharunderscore}wrong\isanewline
\ \ \isakeyword{shows}\ {\isachardoublequoteopen}{\isasymforall}\ w{\isachardot}\ {\isasymexists}\ A{\isachardot}\ O\ {\isacharbraceleft}A{\isacharbraceright}\ w{\isachardoublequoteclose}\isanewline
%
\isadelimproof
\ \ %
\endisadelimproof
%
\isatagproof
\isacommand{using}\isamarkupfalse%
\ assms\ \isacommand{by}\isamarkupfalse%
\ auto\isanewline
%
\isamarkupcmt{It works this time, but I think ``murder wrong" might be too strong of an assumption%
}%
\endisatagproof
{\isafoldproof}%
%
\isadelimproof
%
\endisadelimproof
%
\begin{isamarkuptext}%
Let's try a weaker assumption: Not everyone can lie.%
\end{isamarkuptext}\isamarkuptrue%
\isacommand{typedecl}\isamarkupfalse%
\ person\isanewline
\isacommand{consts}\isamarkupfalse%
\ lies{\isacharcolon}{\isacharcolon}{\isachardoublequoteopen}person{\isasymRightarrow}t{\isachardoublequoteclose}\isanewline
\isacommand{consts}\isamarkupfalse%
\ me{\isacharcolon}{\isacharcolon}{\isachardoublequoteopen}person{\isachardoublequoteclose}\isanewline
\isanewline
\isacommand{lemma}\isamarkupfalse%
\ breaking{\isacharunderscore}promises{\isacharcolon}\isanewline
\ \ \isakeyword{assumes}\ {\isachardoublequoteopen}{\isasymnot}\ {\isacharparenleft}{\isasymforall}x{\isachardot}\ lie{\isacharparenleft}x{\isacharparenright}\ cw{\isacharparenright}\ {\isasymand}\ {\isacharparenleft}lie{\isacharparenleft}me{\isacharparenright}\ cw{\isacharparenright}{\isachardoublequoteclose}\isanewline
\ \ \isakeyword{shows}\ {\isachardoublequoteopen}{\isacharparenleft}O\ {\isacharbraceleft}\isactrlbold {\isasymnot}\ {\isacharparenleft}lie{\isacharparenleft}me{\isacharparenright}{\isacharparenright}{\isacharbraceright}{\isacharparenright}\ cw{\isachardoublequoteclose}\isanewline
\ \ \isacommand{nitpick}\isamarkupfalse%
\ {\isacharbrackleft}user{\isacharunderscore}axioms{\isacharbrackright}\isanewline
%
\isadelimproof
\ \ %
\endisadelimproof
%
\isatagproof
\isacommand{oops}\isamarkupfalse%
\isanewline
%
\isamarkupcmt{No proof found. Makes sense:%
}\isanewline
%
\isamarkupcmt{This version of FUL simply universalizes across worlds (using the $\Box$ operator),%
}\isanewline
%
\isamarkupcmt{But NOT across people, which is really what the most obvious reading of FUL implies%
}\isanewline
%
\isamarkupcmt{``Nitpick found a counterexample for card person = 2 and card i = 2:

  Free variable:
    lie = ($\lambda x. \_$)($p_1$ := ($\lambda x. \_$)($i_1$ := True, $i_2$ := False), $p_2$ := ($\lambda x. \_$)($i_1$ := False, $i_2$ := False))"%
}%
\endisatagproof
{\isafoldproof}%
%
\isadelimproof
%
\endisadelimproof
%
\isadelimdocument
%
\endisadelimdocument
%
\isatagdocument
%
\isamarkupsubsection{Consistent Sentences%
}
\isamarkuptrue%
%
\endisatagdocument
{\isafolddocument}%
%
\isadelimdocument
%
\endisadelimdocument
%
\begin{isamarkuptext}%
The above section tested validity. We might also be interested in some weaker properties%
\end{isamarkuptext}\isamarkuptrue%
%
\begin{isamarkuptext}%
Let's test whether certain sentences are consistent - can we find a model that makes them true?%
\end{isamarkuptext}\isamarkuptrue%
\isacommand{lemma}\isamarkupfalse%
\ permissible{\isacharcolon}\isanewline
\ \ \isakeyword{fixes}\ A\isanewline
\ \ \isakeyword{shows}\ {\isachardoublequoteopen}{\isacharparenleft}{\isacharparenleft}\isactrlbold {\isasymnot}\ {\isacharparenleft}O\ {\isacharbraceleft}A{\isacharbraceright}{\isacharparenright}{\isacharparenright}\ \isactrlbold {\isasymand}\ {\isacharparenleft}\isactrlbold {\isasymnot}\ {\isacharparenleft}O\ {\isacharbraceleft}\isactrlbold {\isasymnot}\ A{\isacharbraceright}{\isacharparenright}{\isacharparenright}{\isacharparenright}\ w{\isachardoublequoteclose}\isanewline
\ \ \isacommand{nitpick}\isamarkupfalse%
\ {\isacharbrackleft}user{\isacharunderscore}axioms{\isacharcomma}\ falsify{\isacharequal}false{\isacharbrackright}%
\isadelimproof
\ %
\endisadelimproof
%
\isatagproof
\isacommand{oops}\isamarkupfalse%
\isanewline
%
\isamarkupcmt{``Nitpick found a model for card i = 1:

  Free variable:
    A = ($\lambda x. \_$)($i_1$ := False)"%
}\isanewline
%
\isamarkupcmt{Awesome! Permissible things are consistent - clearly we've fixed the bug from categorical\_imperative\_1%
}%
\endisatagproof
{\isafoldproof}%
%
\isadelimproof
%
\endisadelimproof
\isanewline
\isanewline
\isacommand{lemma}\isamarkupfalse%
\ conflicting{\isacharunderscore}obligations{\isacharcolon}\isanewline
\ \ \isakeyword{fixes}\ A\isanewline
\ \ \isakeyword{shows}\ {\isachardoublequoteopen}{\isacharparenleft}O\ {\isacharbraceleft}A{\isacharbraceright}\ \isactrlbold {\isasymand}\ O\ {\isacharbraceleft}\isactrlbold {\isasymnot}\ A{\isacharbraceright}{\isacharparenright}\ w{\isachardoublequoteclose}\isanewline
\ \ \isacommand{nitpick}\isamarkupfalse%
\ {\isacharbrackleft}user{\isacharunderscore}axioms{\isacharcomma}\ falsify{\isacharequal}false{\isacharbrackright}%
\isadelimproof
\ %
\endisadelimproof
%
\isatagproof
\isacommand{oops}\isamarkupfalse%
\isanewline
%
\isamarkupcmt{``Nitpick found a model for card i = 2:

  Free variable:
    A = ($\lambda x. \_$)($i_1$ := False, $i_2$ := True)"%
}\isanewline
%
\isamarkupcmt{Oh no! Nitpick found a model with conflicting obligations - that's bad!%
}%
\endisatagproof
{\isafoldproof}%
%
\isadelimproof
%
\endisadelimproof
%
\isadelimdocument
%
\endisadelimdocument
%
\isatagdocument
%
\isamarkupsubsection{Metaethical Tests%
}
\isamarkuptrue%
%
\endisatagdocument
{\isafolddocument}%
%
\isadelimdocument
%
\endisadelimdocument
\isacommand{lemma}\isamarkupfalse%
\ FUL{\isacharunderscore}alternate{\isacharcolon}\isanewline
\ \ \isakeyword{shows}\ {\isachardoublequoteopen}{\isasymTurnstile}\ {\isacharparenleft}{\isacharparenleft}{\isasymdiamond}\ {\isacharparenleft}O\ {\isacharbraceleft}\isactrlbold {\isasymnot}\ A{\isacharbraceright}{\isacharparenright}{\isacharparenright}\ \isactrlbold {\isasymrightarrow}\ {\isacharparenleft}O\ {\isacharbraceleft}\isactrlbold {\isasymnot}\ A{\isacharbraceright}{\isacharparenright}{\isacharparenright}{\isachardoublequoteclose}\isanewline
%
\isadelimproof
\ \ %
\endisadelimproof
%
\isatagproof
\isacommand{by}\isamarkupfalse%
\ simp\isanewline
%
\isamarkupcmt{One problem becomes obvious if we look at the definition of permissible%
}\isanewline
%
\isamarkupcmt{Expanding the FUL gives us: $\sim \Box \sim O(\sim A) \longrightarrow O(\sim A)$%
}\isanewline
%
\isamarkupcmt{By modal duals we get that $\diamond O(\sim A) \longrightarrow O(\sim A)$%
}\isanewline
%
\isamarkupcmt{This means that if something is possibly prohibited, it is in fact prohibited.%
}\isanewline
%
\isamarkupcmt{I'm not convinced that this is a desirable property of an ethical theory.%
}%
\endisatagproof
{\isafoldproof}%
%
\isadelimproof
\isanewline
%
\endisadelimproof
\isanewline
\isacommand{lemma}\isamarkupfalse%
\ arbitrary{\isacharunderscore}obligations{\isacharcolon}\isanewline
\ \ \isakeyword{fixes}\ A{\isacharcolon}{\isacharcolon}{\isachardoublequoteopen}t{\isachardoublequoteclose}\isanewline
\ \ \isakeyword{shows}\ {\isachardoublequoteopen}O\ {\isacharbraceleft}A{\isacharbraceright}\ w{\isachardoublequoteclose}\isanewline
\ \ \isacommand{nitpick}\isamarkupfalse%
\ {\isacharbrackleft}user{\isacharunderscore}axioms{\isacharequal}true{\isacharbrackright}%
\isadelimproof
\ %
\endisadelimproof
%
\isatagproof
\isacommand{oops}\isamarkupfalse%
\isanewline
%
\isamarkupcmt{``Nitpick found a counterexample for card i = 1:

  Free variable:
    A = ($\lambda x. \_$)($i_1$ := False)"%
}\isanewline
%
\isamarkupcmt{This is good! Shows us that any arbitrary term isn't obligatory.%
}%
\endisatagproof
{\isafoldproof}%
%
\isadelimproof
%
\endisadelimproof
\isanewline
\isanewline
\isacommand{axiomatization}\isamarkupfalse%
\ \isakeyword{where}\isanewline
ax{\isacharunderscore}morally{\isacharunderscore}neutral{\isacharcolon}\ {\isachardoublequoteopen}{\isasymexists}A{\isachardot}{\isacharparenleft}{\isacharparenleft}{\isacharparenleft}\isactrlbold {\isasymnot}\ {\isacharparenleft}O\ {\isacharbraceleft}A{\isacharbraceright}{\isacharparenright}{\isacharparenright}\ \isactrlbold {\isasymand}\ {\isacharparenleft}O\ {\isacharbraceleft}\isactrlbold {\isasymnot}\ A{\isacharbraceright}{\isacharparenright}{\isacharparenright}{\isacharparenright}\ w{\isachardoublequoteclose}\isanewline
\isanewline
\isacommand{lemma}\isamarkupfalse%
\ True\ \isacommand{nitpick}\isamarkupfalse%
\ {\isacharbrackleft}satisfy{\isacharcomma}user{\isacharunderscore}axioms{\isacharcomma}show{\isacharunderscore}all{\isacharcomma}format{\isacharequal}{\isadigit{2}}{\isacharbrackright}%
\isadelimproof
\ %
\endisadelimproof
%
\isatagproof
\isacommand{oops}\isamarkupfalse%
\ \isanewline
%
\isamarkupcmt{We might imagine that we want to allow for ``morally neutral" statements%
}\isanewline
%
\isamarkupcmt{Ex: it is neither obligated nor prohibited that I eat lunch today.%
}\isanewline
%
\isamarkupcmt{Nitpick successfully finds a model with morally neutral statements!%
}%
\endisatagproof
{\isafoldproof}%
%
\isadelimproof
%
\endisadelimproof
\isanewline
\isanewline
\isanewline
\isanewline
%
\isadelimtheory
\isanewline
%
\endisadelimtheory
%
\isatagtheory
\isacommand{end}\isamarkupfalse%
%
\endisatagtheory
{\isafoldtheory}%
%
\isadelimtheory
%
\endisadelimtheory
%
\end{isabellebody}%
\endinput
%:%file=~/Desktop/cs91r/categorical_imperative_naive.thy%:%
%:%10=1%:%
%:%11=1%:%
%:%12=2%:%
%:%13=3%:%
%:%27=5%:%
%:%31=7%:%
%:%43=9%:%
%:%45=11%:%
%:%46=11%:%
%:%47=12%:%
%:%49=13%:%
%:%50=13%:%
%:%52=14%:%
%:%53=14%:%
%:%55=15%:%
%:%56=15%:%
%:%58=16%:%
%:%61=18%:%
%:%62=19%:%
%:%63=20%:%
%:%65=22%:%
%:%66=22%:%
%:%67=23%:%
%:%74=25%:%
%:%84=27%:%
%:%85=27%:%
%:%86=27%:%
%:%88=27%:%
%:%92=27%:%
%:%93=27%:%
%:%95=28%:%
%:%96=29%:%
%:%97=30%:%
%:%98=30%:%
%:%100=31%:%
%:%101=31%:%
%:%103=32%:%
%:%111=32%:%
%:%112=33%:%
%:%113=34%:%
%:%114=34%:%
%:%115=35%:%
%:%116=36%:%
%:%117=36%:%
%:%120=37%:%
%:%124=37%:%
%:%125=37%:%
%:%127=38%:%
%:%128=38%:%
%:%130=39%:%
%:%131=39%:%
%:%133=40%:%
%:%134=41%:%
%:%135=42%:%
%:%136=42%:%
%:%138=43%:%
%:%144=43%:%
%:%147=44%:%
%:%148=45%:%
%:%149=45%:%
%:%150=46%:%
%:%151=47%:%
%:%152=47%:%
%:%155=48%:%
%:%159=48%:%
%:%160=48%:%
%:%162=49%:%
%:%163=50%:%
%:%164=51%:%
%:%165=52%:%
%:%166=52%:%
%:%168=53%:%
%:%169=53%:%
%:%171=54%:%
%:%177=54%:%
%:%180=55%:%
%:%181=56%:%
%:%182=56%:%
%:%183=57%:%
%:%184=58%:%
%:%185=58%:%
%:%188=59%:%
%:%192=59%:%
%:%193=59%:%
%:%195=60%:%
%:%196=61%:%
%:%197=62%:%
%:%198=62%:%
%:%200=63%:%
%:%206=63%:%
%:%209=64%:%
%:%210=65%:%
%:%211=65%:%
%:%212=66%:%
%:%213=67%:%
%:%214=67%:%
%:%217=68%:%
%:%221=68%:%
%:%222=68%:%
%:%224=69%:%
%:%225=70%:%
%:%226=71%:%
%:%227=72%:%
%:%228=72%:%
%:%230=73%:%
%:%231=73%:%
%:%233=74%:%
%:%248=76%:%
%:%260=78%:%
%:%262=80%:%
%:%263=80%:%
%:%264=81%:%
%:%265=81%:%
%:%266=82%:%
%:%267=83%:%
%:%268=83%:%
%:%269=84%:%
%:%270=85%:%
%:%273=86%:%
%:%277=86%:%
%:%278=86%:%
%:%279=86%:%
%:%281=87%:%
%:%291=89%:%
%:%293=91%:%
%:%294=91%:%
%:%295=92%:%
%:%296=92%:%
%:%297=93%:%
%:%298=93%:%
%:%299=94%:%
%:%300=95%:%
%:%301=95%:%
%:%302=96%:%
%:%303=97%:%
%:%304=98%:%
%:%305=98%:%
%:%308=99%:%
%:%312=99%:%
%:%313=99%:%
%:%315=100%:%
%:%316=100%:%
%:%318=101%:%
%:%319=101%:%
%:%321=102%:%
%:%322=102%:%
%:%324=103%:%
%:%325=104%:%
%:%326=105%:%
%:%327=106%:%
%:%342=108%:%
%:%354=110%:%
%:%358=111%:%
%:%360=113%:%
%:%361=113%:%
%:%362=114%:%
%:%363=115%:%
%:%364=116%:%
%:%365=116%:%
%:%367=116%:%
%:%371=116%:%
%:%372=116%:%
%:%374=117%:%
%:%375=118%:%
%:%376=119%:%
%:%377=120%:%
%:%378=120%:%
%:%380=121%:%
%:%388=121%:%
%:%389=122%:%
%:%390=123%:%
%:%391=123%:%
%:%392=124%:%
%:%393=125%:%
%:%394=126%:%
%:%395=126%:%
%:%397=126%:%
%:%401=126%:%
%:%402=126%:%
%:%404=127%:%
%:%405=128%:%
%:%406=129%:%
%:%407=130%:%
%:%408=130%:%
%:%410=131%:%
%:%425=133%:%
%:%435=135%:%
%:%436=135%:%
%:%437=136%:%
%:%440=137%:%
%:%444=137%:%
%:%445=137%:%
%:%447=138%:%
%:%448=138%:%
%:%450=139%:%
%:%451=139%:%
%:%453=140%:%
%:%454=140%:%
%:%456=141%:%
%:%457=141%:%
%:%459=142%:%
%:%465=142%:%
%:%468=143%:%
%:%469=144%:%
%:%470=144%:%
%:%471=145%:%
%:%472=146%:%
%:%473=147%:%
%:%474=147%:%
%:%476=147%:%
%:%480=147%:%
%:%481=147%:%
%:%483=148%:%
%:%484=149%:%
%:%485=150%:%
%:%486=151%:%
%:%487=151%:%
%:%489=152%:%
%:%497=152%:%
%:%498=153%:%
%:%499=154%:%
%:%500=154%:%
%:%501=155%:%
%:%502=156%:%
%:%503=157%:%
%:%504=157%:%
%:%505=157%:%
%:%507=157%:%
%:%511=157%:%
%:%512=157%:%
%:%514=158%:%
%:%515=158%:%
%:%517=159%:%
%:%518=159%:%
%:%520=160%:%
%:%528=160%:%
%:%529=161%:%
%:%530=162%:%
%:%531=163%:%
%:%534=164%:%
%:%539=165%:%