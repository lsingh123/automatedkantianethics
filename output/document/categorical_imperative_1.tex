%
\begin{isabellebody}%
\setisabellecontext{categorical{\isacharunderscore}imperative{\isacharunderscore}{\isadigit{1}}}%
%
\isadelimtheory
%
\endisadelimtheory
%
\isatagtheory
\isacommand{theory}\isamarkupfalse%
\ categorical{\isacharunderscore}imperative{\isacharunderscore}{\isadigit{1}}\ \isakeyword{imports}\ carmojones{\isacharunderscore}DDL{\isacharunderscore}completeness\isanewline
\isanewline
\isakeyword{begin}%
\endisatagtheory
{\isafoldtheory}%
%
\isadelimtheory
%
\endisadelimtheory
%
\isadelimdocument
%
\endisadelimdocument
%
\isatagdocument
%
\isamarkupsection{The Categorical Imperative%
}
\isamarkuptrue%
%
\isamarkupsubsection{Simple Formulation of the Kingdom of Ends%
}
\isamarkuptrue%
%
\endisatagdocument
{\isafolddocument}%
%
\isadelimdocument
%
\endisadelimdocument
%
\begin{isamarkuptext}%
This is my initial attempt at formalizing the concept of the Kingdom of Ends%
\end{isamarkuptext}\isamarkuptrue%
\isacommand{abbreviation}\isamarkupfalse%
\ ddlpermissable{\isacharcolon}{\isacharcolon}{\isachardoublequoteopen}t{\isasymRightarrow}t{\isachardoublequoteclose}\ {\isacharparenleft}{\isachardoublequoteopen}P{\isacharunderscore}{\isachardoublequoteclose}{\isacharparenright}\isanewline
\ \ \isakeyword{where}\ {\isachardoublequoteopen}{\isacharparenleft}P\ A{\isacharparenright}\ {\isasymequiv}\ {\isacharparenleft}\isactrlbold {\isasymnot}{\isacharparenleft}O\ {\isacharparenleft}\isactrlbold {\isasymnot}A{\isacharparenright}{\isacharparenright}{\isacharparenright}{\isachardoublequoteclose}\isanewline
%
\isamarkupcmt{This operator represents permissibility%
}\isanewline
%
\isamarkupcmt{Will be useful when discussing the categorical imperative%
}\isanewline
%
\isamarkupcmt{Something is permissible if it is not prohibited%
}\isanewline
%
\isamarkupcmt{Something is prohibited if its negation is obligatory%
}\isanewline
\isanewline
\isacommand{lemma}\isamarkupfalse%
\ kingdom{\isacharunderscore}of{\isacharunderscore}ends{\isacharunderscore}{\isadigit{1}}{\isacharcolon}\isanewline
\ \ \isakeyword{shows}\ {\isachardoublequoteopen}{\isasymTurnstile}\ {\isacharparenleft}{\isacharparenleft}O\ A{\isacharparenright}\ \isactrlbold {\isasymrightarrow}\ {\isacharparenleft}{\isasymbox}\ {\isacharparenleft}P\ A{\isacharparenright}{\isacharparenright}{\isacharparenright}{\isachardoublequoteclose}\isanewline
%
\isadelimproof
\ \ %
\endisadelimproof
%
\isatagproof
\isacommand{by}\isamarkupfalse%
\ {\isacharparenleft}metis\ O{\isacharunderscore}diamond\ ax{\isacharunderscore}{\isadigit{5}}b{\isacharparenright}\isanewline
%
\isamarkupcmt{One interpretation of the categorical imperative is that something is obligatory only if it is permissible in every ideal world%
}\isanewline
%
\isamarkupcmt{This formulation mirrors the kingdom of ends.%
}\isanewline
%
\isamarkupcmt{This formulation is already a theorem of carmo and jones' DDL!.%
}\isanewline
%
\isamarkupcmt{It can be shown using the O diamond rule, which just says that obligatory things must be possible.%
}\isanewline
%
\isamarkupcmt{There are two possibilities: either the logic is already quite powerful OR this formulation is ``empty".%
}%
\endisatagproof
{\isafoldproof}%
%
\isadelimproof
\isanewline
%
\endisadelimproof
\isanewline
\isacommand{lemma}\isamarkupfalse%
\ kingdom{\isacharunderscore}of{\isacharunderscore}ends{\isacharunderscore}{\isadigit{2}}{\isacharcolon}\isanewline
\ \ \isakeyword{shows}\ {\isachardoublequoteopen}{\isasymTurnstile}\ {\isacharparenleft}{\isacharparenleft}{\isasymbox}\ {\isacharparenleft}P\ A{\isacharparenright}{\isacharparenright}\ \isactrlbold {\isasymrightarrow}\ {\isacharparenleft}O\ A{\isacharparenright}{\isacharparenright}{\isachardoublequoteclose}\isanewline
%
\isadelimproof
\ \ %
\endisadelimproof
%
\isatagproof
\isacommand{by}\isamarkupfalse%
\ {\isacharparenleft}metis\ O{\isacharunderscore}diamond\ ax{\isacharunderscore}{\isadigit{5}}a\ ax{\isacharunderscore}{\isadigit{5}}b\ ax{\isacharunderscore}{\isadigit{5}}c{\isacharparenright}\isanewline
%
\isamarkupcmt{Notice also that ideally, this relationship does not hold in the reverse direction.%
}\isanewline
%
\isamarkupcmt{Plenty of things are necessarily permissible (drinking water) but not obligatory.%
}\isanewline
%
\isamarkupcmt{Very strange that this is a theorem in this logic.....%
}\isanewline
%
\isamarkupcmt{That being said, Isabelle seems quite upset with this proof and is very slow to resconstruct it%
}\isanewline
%
\isamarkupcmt{I am struggling to recreate this proof on paper%
}%
\endisatagproof
{\isafoldproof}%
%
\isadelimproof
\isanewline
%
\endisadelimproof
\isanewline
\isanewline
\isacommand{lemma}\isamarkupfalse%
\ permissible{\isacharunderscore}to{\isacharunderscore}ob{\isacharcolon}\isanewline
\ \ \isakeyword{shows}\ {\isachardoublequoteopen}{\isasymTurnstile}\ {\isacharparenleft}{\isacharparenleft}P\ A{\isacharparenright}\ \isactrlbold {\isasymrightarrow}\ {\isacharparenleft}O\ A{\isacharparenright}{\isacharparenright}{\isachardoublequoteclose}\isanewline
%
\isadelimproof
%
\endisadelimproof
%
\isatagproof
\isacommand{proof}\isamarkupfalse%
\ {\isacharminus}\isanewline
\isacommand{have}\isamarkupfalse%
\ {\isachardoublequoteopen}ob\ {\isasymtop}\ {\isacharparenleft}\isactrlbold {\isasymnot}\ A{\isacharparenright}\ {\isasymor}\ ob\ {\isasymtop}\ A{\isachardoublequoteclose}\isanewline
\isacommand{using}\isamarkupfalse%
\ kingdom{\isacharunderscore}of{\isacharunderscore}ends{\isacharunderscore}{\isadigit{2}}\ \isacommand{by}\isamarkupfalse%
\ presburger\isanewline
\ \ \isacommand{then}\isamarkupfalse%
\ \isacommand{show}\isamarkupfalse%
\ {\isacharquery}thesis\isanewline
\isacommand{by}\isamarkupfalse%
\ meson\isanewline
\isacommand{qed}\isamarkupfalse%
\isanewline
%
\isamarkupcmt{Uh-oh.....this shouldn't be true...%
}\isanewline
%
\isamarkupcmt{Not all permissable things are obligatory.....%
}%
\endisatagproof
{\isafoldproof}%
%
\isadelimproof
\isanewline
%
\endisadelimproof
\isanewline
\isacommand{lemma}\isamarkupfalse%
\ weaker{\isacharunderscore}permissible{\isacharunderscore}to{\isacharunderscore}ob{\isacharcolon}\isanewline
\ \ \isakeyword{shows}\ {\isachardoublequoteopen}{\isasymTurnstile}\ {\isacharparenleft}{\isacharparenleft}{\isasymdiamond}\ {\isacharparenleft}P\ A{\isacharparenright}{\isacharparenright}\ \isactrlbold {\isasymrightarrow}\ O\ A{\isacharparenright}{\isachardoublequoteclose}\isanewline
%
\isadelimproof
\ \ %
\endisadelimproof
%
\isatagproof
\isacommand{using}\isamarkupfalse%
\ kingdom{\isacharunderscore}of{\isacharunderscore}ends{\isacharunderscore}{\isadigit{2}}\ \isacommand{by}\isamarkupfalse%
\ auto\isanewline
%
\isamarkupcmt{Makes sense that this follows from the reverse kingdom of ends.%
}\isanewline
%
\isamarkupcmt{Obligation and necessity/possibility are separated in this logic%
}\isanewline
%
\isamarkupcmt{Both the dyadic obligation and necessity operator are world agnostic%
}%
\endisatagproof
{\isafoldproof}%
%
\isadelimproof
\isanewline
%
\endisadelimproof
\isanewline
\isanewline
%
\isadelimtheory
\isanewline
%
\endisadelimtheory
%
\isatagtheory
\isacommand{end}\isamarkupfalse%
%
\endisatagtheory
{\isafoldtheory}%
%
\isadelimtheory
%
\endisadelimtheory
%
\end{isabellebody}%
\endinput
%:%file=~/Desktop/cs91r/categorical_imperative_1.thy%:%
%:%10=1%:%
%:%11=1%:%
%:%12=2%:%
%:%13=3%:%
%:%27=5%:%
%:%31=7%:%
%:%43=9%:%
%:%45=11%:%
%:%46=11%:%
%:%47=12%:%
%:%49=13%:%
%:%50=13%:%
%:%52=14%:%
%:%53=14%:%
%:%55=15%:%
%:%56=15%:%
%:%58=16%:%
%:%59=16%:%
%:%60=17%:%
%:%61=18%:%
%:%62=18%:%
%:%63=19%:%
%:%66=20%:%
%:%70=20%:%
%:%71=20%:%
%:%73=21%:%
%:%74=21%:%
%:%76=22%:%
%:%77=22%:%
%:%79=23%:%
%:%80=23%:%
%:%82=24%:%
%:%83=24%:%
%:%85=25%:%
%:%91=25%:%
%:%94=26%:%
%:%95=27%:%
%:%96=27%:%
%:%97=28%:%
%:%100=29%:%
%:%104=29%:%
%:%105=29%:%
%:%107=30%:%
%:%108=30%:%
%:%110=31%:%
%:%111=31%:%
%:%113=32%:%
%:%114=32%:%
%:%116=33%:%
%:%117=33%:%
%:%119=34%:%
%:%125=34%:%
%:%128=35%:%
%:%129=36%:%
%:%130=37%:%
%:%131=37%:%
%:%132=38%:%
%:%139=39%:%
%:%140=39%:%
%:%141=40%:%
%:%142=40%:%
%:%143=41%:%
%:%144=41%:%
%:%145=41%:%
%:%146=42%:%
%:%147=42%:%
%:%148=42%:%
%:%149=43%:%
%:%150=43%:%
%:%151=44%:%
%:%152=44%:%
%:%154=45%:%
%:%155=45%:%
%:%157=46%:%
%:%163=46%:%
%:%166=47%:%
%:%167=48%:%
%:%168=48%:%
%:%169=49%:%
%:%172=50%:%
%:%176=50%:%
%:%177=50%:%
%:%178=50%:%
%:%180=51%:%
%:%181=51%:%
%:%183=52%:%
%:%184=52%:%
%:%186=53%:%
%:%192=53%:%
%:%195=54%:%
%:%196=55%:%
%:%199=56%:%
%:%204=57%:%