%
\begin{isabellebody}%
\setisabellecontext{kroy}%
%
\isadelimtheory
%
\endisadelimtheory
%
\isatagtheory
\isacommand{theory}\isamarkupfalse%
\ kroy\isanewline
\ \ \isakeyword{imports}\ carmojones{\isacharunderscore}DDL\isanewline
\isanewline
\isakeyword{begin}%
\endisatagtheory
{\isafoldtheory}%
%
\isadelimtheory
%
\endisadelimtheory
%
\begin{isamarkuptext}%
This theory will contain a formalization of the CI based on Moshe Kroy's partial formalization. \cite{kroy}%
\end{isamarkuptext}\isamarkuptrue%
%
\isadelimdocument
%
\endisadelimdocument
%
\isatagdocument
%
\isamarkupsection{Kroy's Formalization of the Categorical Imperative%
}
\isamarkuptrue%
%
\isamarkupsubsection{The Substitution Operator%
}
\isamarkuptrue%
%
\isamarkupsubsubsection{Definition%
}
\isamarkuptrue%
%
\endisatagdocument
{\isafolddocument}%
%
\isadelimdocument
%
\endisadelimdocument
\isacommand{typedecl}\isamarkupfalse%
\ s\ %
\isamarkupcmt{s is the type for a ``subject", like the subject of a sentence%
}\isanewline
%
\isamarkupcmt{Intuitively, we need some notion of ``x does action", which we can write as ``x is the subject of the sentence 'does action'"%
}\isanewline
\isanewline
\isacommand{type{\isacharunderscore}synonym}\isamarkupfalse%
\ os\ {\isacharequal}\ {\isachardoublequoteopen}{\isacharparenleft}s\ {\isasymRightarrow}\ t{\isacharparenright}{\isachardoublequoteclose}\isanewline
%
\isamarkupcmt{An open sentence is a generalized version of Kroy's substitution operator \cite{kroy} 196%
}\isanewline
%
\isamarkupcmt{``does action" is an open sentence that can be instantiated with a subject%
}\isanewline
%
\isamarkupcmt{``P sub (d/e)" can be written as ``S(e)", where S(d) = P%
}\isanewline
%
\isamarkupcmt{So the terms that we substitute into are actually instantiations of an open sentence, and substitution just requires re-instantiating the open sentence with a different subject%
}%
\isadelimdocument
%
\endisadelimdocument
%
\isatagdocument
%
\isamarkupsubsubsection{Abbreviations%
}
\isamarkuptrue%
%
\endisatagdocument
{\isafolddocument}%
%
\isadelimdocument
%
\endisadelimdocument
\isacommand{abbreviation}\isamarkupfalse%
\ os{\isacharunderscore}neg{\isacharcolon}{\isacharcolon}{\isachardoublequoteopen}os\ {\isasymRightarrow}\ os{\isachardoublequoteclose}\ {\isacharparenleft}{\isachardoublequoteopen}\isactrlemph {\isasymnot}{\isacharunderscore}{\isachardoublequoteclose}{\isacharparenright}\isanewline
\ \ \isakeyword{where}\ {\isachardoublequoteopen}{\isacharparenleft}\isactrlemph {\isasymnot}A{\isacharparenright}\ {\isasymequiv}\ {\isasymlambda}x{\isachardot}\ \isactrlbold {\isasymnot}{\isacharparenleft}A{\isacharparenleft}x{\isacharparenright}{\isacharparenright}{\isachardoublequoteclose}\isanewline
\isacommand{abbreviation}\isamarkupfalse%
\ os{\isacharunderscore}and{\isacharcolon}{\isacharcolon}{\isachardoublequoteopen}os\ {\isasymRightarrow}\ os\ {\isasymRightarrow}\ os{\isachardoublequoteclose}\ {\isacharparenleft}{\isachardoublequoteopen}{\isacharunderscore}\isactrlemph {\isasymand}{\isacharunderscore}{\isachardoublequoteclose}{\isacharparenright}\isanewline
\ \ \isakeyword{where}\ {\isachardoublequoteopen}{\isacharparenleft}A\ \isactrlemph {\isasymand}\ B{\isacharparenright}\ {\isasymequiv}\ {\isasymlambda}x{\isachardot}\ {\isacharparenleft}{\isacharparenleft}A{\isacharparenleft}x{\isacharparenright}{\isacharparenright}\ \isactrlbold {\isasymand}\ {\isacharparenleft}B{\isacharparenleft}x{\isacharparenright}{\isacharparenright}{\isacharparenright}{\isachardoublequoteclose}\isanewline
\isacommand{abbreviation}\isamarkupfalse%
\ os{\isacharunderscore}or{\isacharcolon}{\isacharcolon}{\isachardoublequoteopen}os\ {\isasymRightarrow}\ os\ {\isasymRightarrow}\ os{\isachardoublequoteclose}\ {\isacharparenleft}{\isachardoublequoteopen}{\isacharunderscore}\isactrlemph {\isasymor}{\isacharunderscore}{\isachardoublequoteclose}{\isacharparenright}\isanewline
\ \ \isakeyword{where}\ {\isachardoublequoteopen}{\isacharparenleft}A\ \isactrlemph {\isasymor}\ B{\isacharparenright}\ {\isasymequiv}\ {\isasymlambda}x{\isachardot}\ {\isacharparenleft}{\isacharparenleft}A{\isacharparenleft}x{\isacharparenright}{\isacharparenright}\ \isactrlbold {\isasymor}\ {\isacharparenleft}B{\isacharparenleft}x{\isacharparenright}{\isacharparenright}{\isacharparenright}{\isachardoublequoteclose}\isanewline
\isacommand{abbreviation}\isamarkupfalse%
\ os{\isacharunderscore}ob{\isacharcolon}{\isacharcolon}{\isachardoublequoteopen}os\ {\isasymRightarrow}\ os{\isachardoublequoteclose}\ {\isacharparenleft}{\isachardoublequoteopen}\isactrlemph O{\isacharbraceleft}{\isacharunderscore}{\isacharbraceright}{\isachardoublequoteclose}{\isacharparenright}\isanewline
\ \ \isakeyword{where}\ {\isachardoublequoteopen}\isactrlemph O{\isacharbraceleft}A{\isacharbraceright}\ {\isasymequiv}\ {\isasymlambda}x{\isachardot}\ {\isacharparenleft}O\ {\isacharbraceleft}A{\isacharparenleft}x{\isacharparenright}{\isacharbraceright}{\isacharparenright}{\isachardoublequoteclose}\isanewline
%
\isamarkupcmt{We could probably do without these abbreviations, but they will simplify the notiation a bit and unify it with Kroy's original paper.%
}\isanewline
\isanewline
\isacommand{abbreviation}\isamarkupfalse%
\ ddl{\isacharunderscore}permissible{\isacharcolon}{\isacharcolon}{\isachardoublequoteopen}t{\isasymRightarrow}t{\isachardoublequoteclose}\ {\isacharparenleft}{\isachardoublequoteopen}P\ {\isacharbraceleft}{\isacharunderscore}{\isacharbraceright}{\isachardoublequoteclose}{\isacharparenright}\isanewline
\ \ \isakeyword{where}\ {\isachardoublequoteopen}P\ {\isacharbraceleft}A{\isacharbraceright}\ {\isasymequiv}\ \isactrlbold {\isasymnot}\ {\isacharparenleft}O\ {\isacharbraceleft}\isactrlbold {\isasymnot}\ A{\isacharbraceright}{\isacharparenright}{\isachardoublequoteclose}\isanewline
\isacommand{abbreviation}\isamarkupfalse%
\ os{\isacharunderscore}permissible{\isacharcolon}{\isacharcolon}{\isachardoublequoteopen}os{\isasymRightarrow}os{\isachardoublequoteclose}\ {\isacharparenleft}{\isachardoublequoteopen}\isactrlemph P\ {\isacharbraceleft}{\isacharunderscore}{\isacharbraceright}{\isachardoublequoteclose}{\isacharparenright}\isanewline
\ \ \isakeyword{where}\ {\isachardoublequoteopen}\isactrlemph P\ {\isacharbraceleft}A{\isacharbraceright}\ {\isasymequiv}\ {\isasymlambda}x{\isachardot}\ P\ {\isacharbraceleft}A{\isacharparenleft}x{\isacharparenright}{\isacharbraceright}{\isachardoublequoteclose}\isanewline
%
\isamarkupcmt{Carmo and Jones don't make much use of permissibility, but we will find it useful here.%
}%
\isadelimdocument
%
\endisadelimdocument
%
\isatagdocument
%
\isamarkupsubsection{Differences Between Kroy's Logic (Kr) and DDL%
}
\isamarkuptrue%
%
\endisatagdocument
{\isafolddocument}%
%
\isadelimdocument
%
\endisadelimdocument
%
\begin{isamarkuptext}%
\cite{kroy} uses a different logic than DDL. Let's see if the semantics that Kr requires hold in DDL%
\end{isamarkuptext}\isamarkuptrue%
\isacommand{lemma}\isamarkupfalse%
\ permissible{\isacharunderscore}semantic{\isacharunderscore}faithful{\isacharcolon}\isanewline
\ \ \isakeyword{fixes}\ A\ w\isanewline
\ \ \isakeyword{shows}\ {\isachardoublequoteopen}P\ {\isacharbraceleft}A{\isacharbraceright}\ w\ {\isasymlongrightarrow}\ {\isacharparenleft}{\isasymexists}x{\isachardot}\ A{\isacharparenleft}x{\isacharparenright}{\isacharparenright}{\isachardoublequoteclose}\isanewline
\ \ \isacommand{nitpick}\isamarkupfalse%
{\isacharbrackleft}user{\isacharunderscore}axioms{\isacharbrackright}%
\isadelimproof
\ %
\endisadelimproof
%
\isatagproof
\isacommand{oops}\isamarkupfalse%
\isanewline
%
\isamarkupcmt{The most faithful interpretation of Kr is that if A is permissible in a context, then 
it must be true at some world in that context. Kr operates under the ``deontic alternatives" view, 
summarized by Solt as ``A proposition of the sort OA is true at the actual world w if and
only if A is true at every deontic alternative world to t." Under this view, permissible propositions
are obligated at some deontic alternative, but not at all of them.

DDL does not adopt a deontic alternatives view, which is why this proposition seems wildly counterintuitive
in DDL. In DDL, the ob operator abstracts away the notion of deontic alternatives and completely determines
obligations. Even if one belives that permissible statements should be true at some deontic alternative, 
it's not clear that permissible statements must be realized at some world, hence the failure of this 
lemma in DDL.%
}\isanewline
%
\isamarkupcmt{Nitpick found a counterexample for card i = 1:

  Free variable:
    A = ($\lambda x. \_$)($i_1$ := False)%
}%
\endisatagproof
{\isafoldproof}%
%
\isadelimproof
%
\endisadelimproof
\isanewline
\isanewline
\isacommand{lemma}\isamarkupfalse%
\ permissible{\isacharunderscore}semantic{\isacharunderscore}faithful{\isacharunderscore}{\isadigit{2}}{\isacharcolon}\isanewline
\ \ \isakeyword{fixes}\ A\ w\isanewline
\ \ \isakeyword{shows}\ {\isachardoublequoteopen}P\ {\isacharbraceleft}A{\isacharbraceright}\ w\ {\isasymlongrightarrow}\ {\isacharparenleft}{\isasymexists}x{\isachardot}\ O\ {\isacharbraceleft}A{\isacharbraceright}\ x{\isacharparenright}{\isachardoublequoteclose}\isanewline
\ \ \isacommand{nitpick}\isamarkupfalse%
{\isacharbrackleft}user{\isacharunderscore}axioms{\isacharbrackright}%
\isadelimproof
\ %
\endisadelimproof
%
\isatagproof
\isacommand{oops}\isamarkupfalse%
\isanewline
%
\isamarkupcmt{Nitpick found a counterexample for card i = 1:

  Free variable:
    A = ($\lambda x. \_$)($i_1$ := False)%
}\isanewline
%
\isamarkupcmt{This is the most clear lemma that we would expect to hold under the deontic alternatives view. The 
fact that it doesn't shows DDL is not a logic of deontic alternatives. There are pros and cons to this
approach. The deontic alternatives view is quite simple to visualize and offers clear intuition. On
the other hand, DDL's ob function can encode more complex relations than the deontic alternatives view,
and can encode these in a more intuitive manner. The notion of a ``deontically perfect alternative"
is a squishy one, and an ob function more directly captures the idea of obligation.%
}%
\endisatagproof
{\isafoldproof}%
%
\isadelimproof
%
\endisadelimproof
\isanewline
\isanewline
\isanewline
\isacommand{lemma}\isamarkupfalse%
\ permissible{\isacharunderscore}semantic{\isacharunderscore}vacuous{\isacharcolon}\isanewline
\ \ \isakeyword{fixes}\ A\ w\isanewline
\ \ \isakeyword{shows}\ {\isachardoublequoteopen}P\ {\isacharbraceleft}A{\isacharbraceright}\ w\ {\isasymlongrightarrow}\ {\isacharparenleft}{\isasymexists}x{\isachardot}\ ob{\isacharparenleft}x{\isacharparenright}{\isacharparenleft}A{\isacharparenright}{\isacharparenright}{\isachardoublequoteclose}\isanewline
\ \ \isacommand{nitpick}\isamarkupfalse%
{\isacharbrackleft}user{\isacharunderscore}axioms{\isacharbrackright}%
\isadelimproof
\ %
\endisadelimproof
%
\isatagproof
\isacommand{oops}\isamarkupfalse%
\isanewline
%
\isamarkupcmt{Kr does not allow vasuously permissible statements -> if something is permissible it has to be obligated 
at some deontically perfect alternative. In DDL, this can be roughly translated as, if A is permissible, 
it is obligated in some context.%
}\isanewline
%
\isamarkupcmt{Nitpick found a counterexample for card i = 1:

  Free variable:
    A = ($\lambda x. \_$)($i_1$ := False)%
}\isanewline
%
\isamarkupcmt{In order to make this true, we'd have to require that everything is either obligatory or 
prohibited somewhere. But as found in the buggy version of DDL, that breaks everything and destroys the 
notion of permissibility everywhere. I am going to allow for vacuous permissibility. If something breaks later
in Kr, it may be because of this.%
}%
\endisatagproof
{\isafoldproof}%
%
\isadelimproof
%
\endisadelimproof
\isanewline
\isanewline
\isacommand{lemma}\isamarkupfalse%
\ permissible{\isacharunderscore}ob{\isacharcolon}\isanewline
\ \ \isakeyword{fixes}\ A\ w\isanewline
\ \ \isakeyword{shows}\ {\isachardoublequoteopen}O\ {\isacharbraceleft}A{\isacharbraceright}\ w\ {\isasymlongrightarrow}\ P\ {\isacharbraceleft}A{\isacharbraceright}\ w{\isachardoublequoteclose}\isanewline
\ \ \isacommand{nitpick}\isamarkupfalse%
\ {\isacharbrackleft}user{\isacharunderscore}axioms{\isacharbrackright}%
\isadelimproof
\ %
\endisadelimproof
%
\isatagproof
\isacommand{oops}\isamarkupfalse%
\isanewline
%
\isamarkupcmt{Nitpick found a counterexample for card i = 2:

  Free variable:
    A = ($\lambda x. \_$)($i_1$ := False, $i_2$ := True)%
}\isanewline
%
\isamarkupcmt{This one is definitely problematic. Being permissible should be a precondition for being obligatory.
Let's add this as an axiom - I can't see any ethical theory succeeding without this.%
}%
\endisatagproof
{\isafoldproof}%
%
\isadelimproof
%
\endisadelimproof
\isanewline
\isanewline
\isacommand{axiomatization}\isamarkupfalse%
\ \isakeyword{where}\ permissible{\isacharunderscore}ob{\isacharcolon}\ {\isachardoublequoteopen}{\isasymTurnstile}\ {\isacharparenleft}O\ {\isacharbraceleft}A{\isacharbraceright}\ \isactrlbold {\isasymrightarrow}\ P\ {\isacharbraceleft}A{\isacharbraceright}{\isacharparenright}{\isachardoublequoteclose}\isanewline
\isanewline
\isacommand{lemma}\isamarkupfalse%
\ permissible{\isacharunderscore}ob{\isacharunderscore}round{\isacharunderscore}{\isadigit{2}}{\isacharcolon}\isanewline
\ \ \isakeyword{fixes}\ A\ w\isanewline
\ \ \isakeyword{shows}\ {\isachardoublequoteopen}O\ {\isacharbraceleft}A{\isacharbraceright}\ w\ {\isasymlongrightarrow}\ P\ {\isacharbraceleft}A{\isacharbraceright}\ w{\isachardoublequoteclose}\isanewline
%
\isadelimproof
\ \ %
\endisadelimproof
%
\isatagproof
\isacommand{by}\isamarkupfalse%
\ {\isacharparenleft}simp\ add{\isacharcolon}\ permissible{\isacharunderscore}ob{\isacharparenright}%
\endisatagproof
{\isafoldproof}%
%
\isadelimproof
%
\endisadelimproof
%
\isadelimdocument
%
\endisadelimdocument
%
\isatagdocument
%
\isamarkupsubsection{The Categorical Imperative%
}
\isamarkuptrue%
%
\endisatagdocument
{\isafolddocument}%
%
\isadelimdocument
%
\endisadelimdocument
\isacommand{abbreviation}\isamarkupfalse%
\ CI{\isacharcolon}{\isacharcolon}{\isachardoublequoteopen}bool{\isachardoublequoteclose}\ \isakeyword{where}\ {\isachardoublequoteopen}CI\ {\isasymequiv}\ {\isasymforall}w\ A{\isachardot}\ {\isacharparenleft}{\isacharparenleft}{\isasymexists}p{\isachardot}\ {\isacharparenleft}\isactrlemph P\ {\isacharbraceleft}A{\isacharbraceright}p\ {\isacharparenright}w{\isacharparenright}\ {\isasymlongrightarrow}\ {\isacharparenleft}{\isasymforall}x{\isachardot}\ {\isacharparenleft}\isactrlemph P\ {\isacharbraceleft}A{\isacharbraceright}x{\isacharparenright}w{\isacharparenright}{\isacharparenright}{\isachardoublequoteclose}\isanewline
%
\isamarkupcmt{This is Kroy's formalization of the FUL in DDL. Recall that the FUL says
``act only in accordance with that maxim through which you can at the same time will that it become a universal law" \cite{groundwork}
Kroy interprets this to mean that if an action is permissible for me, then it must be permissible for everyone.
This formalizes an important moral intuition - the CI prohibits free-riding. No one is a moral exception%
}\isanewline
\isanewline
\isacommand{lemma}\isamarkupfalse%
\ CI{\isacharcolon}\isanewline
\ \ \isakeyword{shows}\ CI\isanewline
\ \ \isacommand{nitpick}\isamarkupfalse%
{\isacharbrackleft}user{\isacharunderscore}axioms{\isacharbrackright}%
\isadelimproof
\ %
\endisadelimproof
%
\isatagproof
\isacommand{oops}\isamarkupfalse%
\isanewline
%
\isamarkupcmt{Nitpick found a counterexample for card s = 2 and card i = 2:

  Skolem constants:
    A = ($\lambda x. \_$)($s_1$ := ($\lambda x. \_$)($i_1$ := True, $i_2$ := True), $s_2$ := ($\lambda x. \_$)($i_1$ := False, $i_2$ := False))
    p = $s_1$
    x = $s_2$
This formalization doesn't hold in DDL. Good - this means that adding it as an axiom will change the logic.%
}%
\endisatagproof
{\isafoldproof}%
%
\isadelimproof
%
\endisadelimproof
\isanewline
\isanewline
\isacommand{lemma}\isamarkupfalse%
\ {\isachardoublequoteopen}{\isasymTurnstile}\ {\isacharparenleft}\isactrlbold {\isasymnot}\ {\isacharparenleft}O\ {\isacharbraceleft}\isactrlbold {\isasymnot}\ \isactrlbold {\isasymtop}{\isacharbraceright}{\isacharparenright}{\isacharparenright}{\isachardoublequoteclose}\isanewline
%
\isadelimproof
\ \ %
\endisadelimproof
%
\isatagproof
\isacommand{by}\isamarkupfalse%
\ {\isacharparenleft}simp\ add{\isacharcolon}\ ax{\isacharunderscore}{\isadigit{5}}a{\isacharparenright}%
\endisatagproof
{\isafoldproof}%
%
\isadelimproof
\ \isanewline
%
\endisadelimproof
\isanewline
\isacommand{lemma}\isamarkupfalse%
\ complete{\isacharcolon}\isanewline
\ \ \isakeyword{shows}\ {\isachardoublequoteopen}{\isacharparenleft}{\isasymTurnstile}\ {\isacharparenleft}{\isacharparenleft}{\isacharparenleft}A\ \isactrlbold {\isasymrightarrow}\ \isactrlbold {\isasymnot}\isactrlbold {\isasymtop}{\isacharparenright}{\isacharparenright}{\isacharparenright}{\isacharparenright}\ {\isasymlongrightarrow}\ {\isacharparenleft}{\isasymTurnstile}\ {\isacharparenleft}\isactrlbold {\isasymnot}\ {\isacharparenleft}O\ {\isacharbraceleft}A{\isacharbraceright}{\isacharparenright}{\isacharparenright}{\isacharparenright}{\isachardoublequoteclose}\isanewline
%
\isadelimproof
%
\endisadelimproof
%
\isatagproof
\isacommand{proof}\isamarkupfalse%
\ {\isacharminus}\ \isanewline
\ \ \isacommand{have}\isamarkupfalse%
\ {\isachardoublequoteopen}{\isacharparenleft}{\isasymexists}x{\isachardot}\ {\isacharparenleft}\isactrlbold {\isasymnot}\ {\isacharparenleft}{\isasymdiamond}\ A{\isacharparenright}{\isacharparenright}\ x{\isacharparenright}\ {\isasymlongrightarrow}\ {\isacharparenleft}{\isasymexists}x{\isachardot}\ {\isacharparenleft}\isactrlbold {\isasymnot}\ O\ {\isacharbraceleft}A{\isacharbraceright}{\isacharparenright}\ x{\isacharparenright}{\isachardoublequoteclose}\isanewline
\ \ \ \ \isacommand{by}\isamarkupfalse%
\ {\isacharparenleft}simp\ add{\isacharcolon}\ ax{\isacharunderscore}{\isadigit{5}}a\ ax{\isacharunderscore}{\isadigit{5}}b{\isacharparenright}\isanewline
\ \ \isacommand{thus}\isamarkupfalse%
\ {\isacharquery}thesis\isanewline
\ \ \ \ \isacommand{by}\isamarkupfalse%
\ blast\isanewline
\isacommand{qed}\isamarkupfalse%
%
\endisatagproof
{\isafoldproof}%
%
\isadelimproof
\isanewline
%
\endisadelimproof
\isanewline
\isacommand{axiomatization}\isamarkupfalse%
\ \isakeyword{where}\ CI{\isacharcolon}\ CI\ \isakeyword{and}\ \isanewline
possible{\isacharcolon}\ {\isachardoublequoteopen}{\isacharparenleft}{\isasymforall}w{\isachardot}\ {\isasymnot}\ {\isasymdiamond}A\ w{\isacharparenright}\ {\isasymlongrightarrow}\ {\isasymTurnstile}{\isacharparenleft}O\ {\isacharbraceleft}\isactrlbold {\isasymnot}\ A{\isacharbraceright}{\isacharparenright}{\isachardoublequoteclose}\isanewline
%
\isamarkupcmt{We really need a way to add negative obligations - to related the concept of contradiction and 
obligation. This seems reasonable - if A is never possible at any world, then it's prohibited.%
}\isanewline
\isakeyword{and}\ hmm{\isacharcolon}\ {\isachardoublequoteopen}{\isasymTurnstile}\ {\isacharparenleft}\isactrlbold {\isasymnot}\ {\isacharparenleft}O\ {\isacharbraceleft}\isactrlbold {\isasymnot}\ \isactrlbold {\isasymtop}{\isacharbraceright}{\isacharparenright}{\isacharparenright}{\isachardoublequoteclose}%
\isadelimdocument
%
\endisadelimdocument
%
\isatagdocument
%
\isamarkupsubsection{Tests%
}
\isamarkuptrue%
%
\endisatagdocument
{\isafolddocument}%
%
\isadelimdocument
%
\endisadelimdocument
\isacommand{lemma}\isamarkupfalse%
\ True\ \isacommand{nitpick}\isamarkupfalse%
\ {\isacharbrackleft}satisfy{\isacharcomma}user{\isacharunderscore}axioms{\isacharbrackright}%
\isadelimproof
\ %
\endisadelimproof
%
\isatagproof
\isacommand{oops}\isamarkupfalse%
\ \isanewline
%
\isamarkupcmt{Nitpick found a model for card s = 1 and card i = 1:

  Empty assignment%
}\isanewline
%
\isamarkupcmt{The categorical imperative is consistent!%
}%
\endisatagproof
{\isafoldproof}%
%
\isadelimproof
%
\endisadelimproof
%
\isadelimdocument
%
\endisadelimdocument
%
\isatagdocument
%
\isamarkupsubsubsection{Specifying the Model%
}
\isamarkuptrue%
%
\endisatagdocument
{\isafolddocument}%
%
\isadelimdocument
%
\endisadelimdocument
\isacommand{lemma}\isamarkupfalse%
\ breaking{\isacharunderscore}promises{\isacharcolon}\isanewline
\ \ \isakeyword{fixes}\ me{\isacharcolon}{\isacharcolon}s\isanewline
\ \ \isakeyword{fixes}\ lie{\isacharcolon}{\isacharcolon}os\isanewline
\ \ \isakeyword{assumes}\ {\isachardoublequoteopen}{\isasymexists}x{\isachardot}\ {\isacharparenleft}\isactrlbold {\isasymnot}\ {\isacharparenleft}{\isasymdiamond}{\isacharparenleft}lie{\isacharparenleft}x{\isacharparenright}{\isacharparenright}{\isacharparenright}\ cw{\isacharparenright}{\isachardoublequoteclose}\isanewline
\ \ \isakeyword{shows}\ {\isachardoublequoteopen}{\isasymexists}x{\isachardot}\ {\isacharparenleft}\isactrlbold {\isasymnot}\ {\isacharparenleft}O\ {\isacharbraceleft}lie{\isacharparenleft}x{\isacharparenright}{\isacharbraceright}{\isacharparenright}{\isacharparenright}\ cw{\isachardoublequoteclose}\isanewline
%
\isadelimproof
\ \ %
\endisadelimproof
%
\isatagproof
\isacommand{by}\isamarkupfalse%
\ {\isacharparenleft}metis\ assms\ ax{\isacharunderscore}{\isadigit{5}}a\ ax{\isacharunderscore}{\isadigit{5}}b{\isacharparenright}%
\endisatagproof
{\isafoldproof}%
%
\isadelimproof
\isanewline
%
\endisadelimproof
\isanewline
\isanewline
\isacommand{lemma}\isamarkupfalse%
\ breaking{\isacharunderscore}promises{\isacharcolon}\isanewline
\ \ \isakeyword{fixes}\ me{\isacharcolon}{\isacharcolon}s\isanewline
\ \ \isakeyword{fixes}\ lie{\isacharcolon}{\isacharcolon}os\isanewline
\ \ \isakeyword{assumes}\ {\isachardoublequoteopen}{\isasymexists}x{\isachardot}\ {\isacharparenleft}\isactrlbold {\isasymnot}\ {\isacharparenleft}{\isasymdiamond}{\isacharparenleft}lie{\isacharparenleft}x{\isacharparenright}{\isacharparenright}{\isacharparenright}\ cw{\isacharparenright}{\isachardoublequoteclose}\isanewline
\ \ \isakeyword{shows}\ {\isachardoublequoteopen}{\isasymexists}x{\isachardot}\ {\isacharparenleft}\isactrlbold {\isasymnot}\ {\isacharparenleft}P\ {\isacharbraceleft}lie{\isacharparenleft}x{\isacharparenright}{\isacharbraceright}{\isacharparenright}{\isacharparenright}\ cw{\isachardoublequoteclose}\isanewline
\ \ \isacommand{nitpick}\isamarkupfalse%
{\isacharbrackleft}user{\isacharunderscore}axioms{\isacharbrackright}%
\isadelimproof
\ %
\endisadelimproof
%
\isatagproof
\isacommand{oops}\isamarkupfalse%
%
\endisatagproof
{\isafoldproof}%
%
\isadelimproof
%
\endisadelimproof
%
\isadelimdocument
%
\endisadelimdocument
%
\isatagdocument
%
\isamarkupsubsubsection{Metaethical Tests%
}
\isamarkuptrue%
%
\isamarkupsubsubsection{Kroy's Tests%
}
\isamarkuptrue%
%
\endisatagdocument
{\isafolddocument}%
%
\isadelimdocument
%
\endisadelimdocument
%
\isadelimtheory
%
\endisadelimtheory
%
\isatagtheory
\isacommand{end}\isamarkupfalse%
%
\endisatagtheory
{\isafoldtheory}%
%
\isadelimtheory
%
\endisadelimtheory
%
\end{isabellebody}%
\endinput
%:%file=~/Desktop/cs91r/kroy.thy%:%
%:%10=1%:%
%:%11=1%:%
%:%12=2%:%
%:%13=3%:%
%:%14=4%:%
%:%23=6%:%
%:%32=8%:%
%:%36=10%:%
%:%40=12%:%
%:%50=14%:%
%:%51=14%:%
%:%52=14%:%
%:%53=14%:%
%:%55=15%:%
%:%56=15%:%
%:%57=16%:%
%:%58=17%:%
%:%59=17%:%
%:%61=18%:%
%:%62=18%:%
%:%64=19%:%
%:%65=19%:%
%:%67=20%:%
%:%68=20%:%
%:%70=21%:%
%:%78=23%:%
%:%88=25%:%
%:%89=25%:%
%:%90=26%:%
%:%91=27%:%
%:%92=27%:%
%:%93=28%:%
%:%94=29%:%
%:%95=29%:%
%:%96=30%:%
%:%97=31%:%
%:%98=31%:%
%:%99=32%:%
%:%101=33%:%
%:%102=33%:%
%:%103=34%:%
%:%104=35%:%
%:%105=35%:%
%:%106=36%:%
%:%107=37%:%
%:%108=37%:%
%:%109=38%:%
%:%111=39%:%
%:%119=41%:%
%:%131=43%:%
%:%133=45%:%
%:%134=45%:%
%:%135=46%:%
%:%136=47%:%
%:%137=48%:%
%:%138=48%:%
%:%140=48%:%
%:%144=48%:%
%:%145=48%:%
%:%147=49%:%
%:%148=50%:%
%:%149=51%:%
%:%150=52%:%
%:%151=53%:%
%:%152=54%:%
%:%153=55%:%
%:%154=56%:%
%:%155=57%:%
%:%156=58%:%
%:%157=59%:%
%:%158=59%:%
%:%160=60%:%
%:%161=61%:%
%:%162=62%:%
%:%163=63%:%
%:%171=63%:%
%:%172=64%:%
%:%173=65%:%
%:%174=65%:%
%:%175=66%:%
%:%176=67%:%
%:%177=68%:%
%:%178=68%:%
%:%180=68%:%
%:%184=68%:%
%:%185=68%:%
%:%187=69%:%
%:%188=70%:%
%:%189=71%:%
%:%190=72%:%
%:%191=72%:%
%:%193=73%:%
%:%194=74%:%
%:%195=75%:%
%:%196=76%:%
%:%197=77%:%
%:%198=78%:%
%:%206=78%:%
%:%207=79%:%
%:%208=80%:%
%:%209=81%:%
%:%210=81%:%
%:%211=82%:%
%:%212=83%:%
%:%213=84%:%
%:%214=84%:%
%:%216=84%:%
%:%220=84%:%
%:%221=84%:%
%:%223=85%:%
%:%224=86%:%
%:%225=87%:%
%:%226=87%:%
%:%228=88%:%
%:%229=89%:%
%:%230=90%:%
%:%231=91%:%
%:%232=91%:%
%:%234=92%:%
%:%235=93%:%
%:%236=94%:%
%:%237=95%:%
%:%245=95%:%
%:%246=96%:%
%:%247=97%:%
%:%248=97%:%
%:%249=98%:%
%:%250=99%:%
%:%251=100%:%
%:%252=100%:%
%:%254=100%:%
%:%258=100%:%
%:%259=100%:%
%:%261=101%:%
%:%262=102%:%
%:%263=103%:%
%:%264=104%:%
%:%265=104%:%
%:%267=105%:%
%:%268=106%:%
%:%276=106%:%
%:%277=107%:%
%:%278=108%:%
%:%279=108%:%
%:%280=109%:%
%:%281=110%:%
%:%282=110%:%
%:%283=111%:%
%:%284=112%:%
%:%287=113%:%
%:%291=113%:%
%:%292=113%:%
%:%306=115%:%
%:%316=117%:%
%:%317=117%:%
%:%319=118%:%
%:%320=119%:%
%:%321=120%:%
%:%322=121%:%
%:%323=121%:%
%:%324=122%:%
%:%325=123%:%
%:%326=123%:%
%:%327=124%:%
%:%328=125%:%
%:%329=125%:%
%:%331=125%:%
%:%335=125%:%
%:%336=125%:%
%:%338=126%:%
%:%339=127%:%
%:%340=128%:%
%:%341=129%:%
%:%342=130%:%
%:%343=131%:%
%:%344=132%:%
%:%352=132%:%
%:%353=133%:%
%:%354=134%:%
%:%355=134%:%
%:%358=135%:%
%:%362=135%:%
%:%363=135%:%
%:%368=135%:%
%:%371=136%:%
%:%372=137%:%
%:%373=137%:%
%:%374=138%:%
%:%381=139%:%
%:%382=139%:%
%:%383=140%:%
%:%384=140%:%
%:%385=141%:%
%:%386=141%:%
%:%387=142%:%
%:%388=142%:%
%:%389=143%:%
%:%390=143%:%
%:%391=144%:%
%:%397=144%:%
%:%400=145%:%
%:%401=146%:%
%:%402=146%:%
%:%403=147%:%
%:%405=148%:%
%:%406=149%:%
%:%407=149%:%
%:%408=150%:%
%:%415=152%:%
%:%425=154%:%
%:%426=154%:%
%:%427=154%:%
%:%429=154%:%
%:%433=154%:%
%:%434=154%:%
%:%436=155%:%
%:%437=156%:%
%:%438=157%:%
%:%439=157%:%
%:%441=158%:%
%:%456=160%:%
%:%466=163%:%
%:%467=163%:%
%:%468=164%:%
%:%469=165%:%
%:%470=166%:%
%:%471=167%:%
%:%474=168%:%
%:%478=168%:%
%:%479=168%:%
%:%484=168%:%
%:%487=169%:%
%:%488=170%:%
%:%489=171%:%
%:%490=171%:%
%:%491=172%:%
%:%492=173%:%
%:%493=174%:%
%:%494=175%:%
%:%495=176%:%
%:%496=176%:%
%:%498=176%:%
%:%502=176%:%
%:%517=180%:%
%:%521=182%:%
%:%537=188%:%