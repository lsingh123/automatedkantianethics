\documentclass[11pt]{article}
\usepackage{xcolor}
\usepackage{times}
\usepackage[margin=1in]{geometry}

\begin{document}

\noindent \textbf{FAKE: Faithful Automated Kantian Ethics}

\medskip 

\noindent \textbf{Abstract}: Calls from regulators, philosophers, and computer scientists for the ethical
use and development of artifical intelligence have spurred interest in automated ethics. Such machine 
ethics will best cohere with intuitions and guide action when it is faithful to existing philosophical
literature. Automating philosophically sophisticated ethical theories poses technical and philosophical
challenges because it requires translating
complex ethical theories expressed in natural language to the rigid syntax of a computer program. In 
this paper, I present FAKE: an implementation of automated Kantian 
ethics that is faithful to the Kantian philosophical tradition. Of the three major ethical 
traditions, Kant's categorical imperative is most natural to formalize because it presents inviolable, 
context-agnostic, formal rules.  I formalize Kant's categorical imperative 
in Carmo and Jones's dyadic deontic logic, implement this formalization 
in the Isabelle/HOL theorem prover, and develop a testing framework to evaluate how well 
different formalizations cohere with expected properties of Kantian ethics, as established in the literature. 
FAKE is not only an early step towards philosophically mature ethical AI agents, but it can also help
philosophers reach new ethical insights, thus paving the way for computational ethics.

\smallskip
\noindent \emph{Areas: algorithm development, applications, philosophy}
\end{document}

