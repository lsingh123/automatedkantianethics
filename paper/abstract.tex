\documentclass[11pt]{article}
\usepackage{xcolor}
\usepackage{times}
\usepackage[margin=1in]{geometry}

\begin{document}

\noindent \textbf{A Faithful Implementation of Automated Kantian Ethics}

\medskip 

\noindent \textbf{Abstract}: Warnings from regulators, philosophers, and computer scientists about the 
dangers of unethical artificial intelligence have spurred interest in the development 
of machines that can perform ethical reasoning. However, previous work in automated ethics rarely 
engages with existing philosophical literature. Philosophically sophisticated ethical theories are necesssary
for nuanced and reliable judgements, but faithfully translating these complex ethical theories
from natural language to the rigid syntax of a computer program poses technical and philosophical 
challenges. In this paper, I present an implementation of automated Kantian 
ethics that is faithful to the Kantian philosophical tradition. Of the three major ethical 
traditions, Kant's categorical imperative is the most natural to formalize because it is an inviolable, 
context-agnostic, formal rule.  I formalize Kant's categorical imperative 
in Carmo and Jones's dyadic deontic logic, implement this formalization 
in the Isabelle/HOL theorem prover, and develop a testing framework to evaluate how well 
my implementation coheres with expected properties of Kantian ethics, as established in the literature. 
My system is an early step towards philosophically mature ethical AI agents and it can make nuanced 
judgements in complex ethical dilemmas because it is grounded in philosophical literature. Moreover, 
because my system uses an interactive theorem prover, its judgements are explainable.


\smallskip
\noindent \emph{Area: Algorithm Development}

\noindent \emph{Keywords: AI ethics, interactive theorem provers, deontic logic, Kant}
\end{document}

