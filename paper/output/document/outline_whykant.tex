%
\begin{isabellebody}%
\setisabellecontext{outline{\isacharunderscore}whykant}%
%
\isadelimtheory
%
\endisadelimtheory
%
\isatagtheory
%
\endisatagtheory
{\isafoldtheory}%
%
\isadelimtheory
%
\endisadelimtheory
%
\isadelimdocument
%
\endisadelimdocument
%
\isatagdocument
%
\isamarkupsection{Introduction%
}
\isamarkuptrue%
%
\endisatagdocument
{\isafolddocument}%
%
\isadelimdocument
%
\endisadelimdocument
%
\begin{isamarkuptext}%
In this section, I justify my choice to automate Kantian ethics, as opposed to virtue ethics or
consequentialism. Kantian ethics is easy to formalize for three reasons. First, evaluating a maxim requires little 
data about the world. Second, a maxim is relatively easy to represent to a computer. Third, while 
any ethical theory has debates that automated ethics would need to take a stance on, these debates are 
less frequent and less controversial for Kantian ethics. 

The argument is not that no form of consequentialism or virtue ethics is tractable to automate, but 
rather that Kantian ethics is easier. I do not present a comprehensive comparision
of ethical theories, but instead sketch the basic principles that most consequentialist and virtue 
ethical theories share.%
\end{isamarkuptext}\isamarkuptrue%
%
\isadelimdocument
%
\endisadelimdocument
%
\isatagdocument
%
\isamarkupsection{Kantian Ethics%
}
\isamarkuptrue%
%
\isamarkupsubsection{Kant Crash Course%
}
\isamarkuptrue%
%
\endisatagdocument
{\isafolddocument}%
%
\isadelimdocument
%
\endisadelimdocument
%
\begin{isamarkuptext}%
First, I explain the concepts of 
practical reason, the will, and maxims. I then present Kant's argument that because the will is autonomous,
only the will has authority over itself \cite{sources}. Finally, I argue that a will is definitionally only 
bound by those imperatives that are implied by practical reason itself \cite{velleman}. From there, I present 
the three formulations of the categorical imperative, focusing on the Formula of Universal Law (FUL) \cite{groundwork}. 
This is not a full defense of Kantian ethics but is instead a quick sketch of one argument for the FUL.%
\end{isamarkuptext}\isamarkuptrue%
%
\isadelimdocument
%
\endisadelimdocument
%
\isatagdocument
%
\isamarkupsubsection{Kant is Easy to Formalize%
}
\isamarkuptrue%
%
\endisatagdocument
{\isafolddocument}%
%
\isadelimdocument
%
\endisadelimdocument
%
\begin{isamarkuptext}%
The FUL is easy to formalize
because it is a purely formal principle that is only evaluates the form of the maxim that an agent 
acts on. No other information is relevant to the test, so moral judgement can proceed with a relatively small
amount of data.%
\end{isamarkuptext}\isamarkuptrue%
%
\isadelimdocument
%
\endisadelimdocument
%
\isatagdocument
%
\isamarkupsubsection{Common Debates%
}
\isamarkuptrue%
%
\endisatagdocument
{\isafolddocument}%
%
\isadelimdocument
%
\endisadelimdocument
%
\begin{isamarkuptext}%
I acknowledge the debates\footnote{I have a separate piece of writing that provides a literature 
review of these debates and justifies my stances. Would that fit here or somewhere else?} in Kantian 
ethics that my project takes a stance on: the correct way to interpret the FUL and the definition of a 
maxim. The stances I take are generally accepted by most Kantians, though some still disagree.%
\end{isamarkuptext}\isamarkuptrue%
%
\isadelimdocument
%
\endisadelimdocument
%
\isatagdocument
%
\isamarkupsection{Consequentialism%
}
\isamarkuptrue%
%
\isamarkupsubsection{Consequentialism Crash Course%
}
\isamarkuptrue%
%
\endisatagdocument
{\isafolddocument}%
%
\isadelimdocument
%
\endisadelimdocument
%
\begin{isamarkuptext}%
I define
a consequentialist theory as one that evaluates the consequences of an action, acknowledging that
this definition itself is controversial \cite{consequentialismsep}. I then present the debates over theories of the good, which 
consequences to evaluate, and the aggregation of consequences.%
\end{isamarkuptext}\isamarkuptrue%
%
\isadelimdocument
%
\endisadelimdocument
%
\isatagdocument
%
\isamarkupsubsection{Consequentialism is Hard to Formalize%
}
\isamarkuptrue%
%
\isamarkupsubsubsection{Requires Lots of Data About States of Affairs%
}
\isamarkuptrue%
%
\endisatagdocument
{\isafolddocument}%
%
\isadelimdocument
%
\endisadelimdocument
%
\begin{isamarkuptext}%
Making a consequentialist moral judgement requires data about the entire state of affairs following
an action, posing many challenges. First, collecting this data is difficult. Second, in order to trust 
the system's judgements, we have to trust the ethical theory, its theory of the good, and the many
judgements required to assign each state of affairs a goodness measurement \cite{utilsep}.%
\end{isamarkuptext}\isamarkuptrue%
%
\isadelimdocument
%
\endisadelimdocument
%
\isatagdocument
%
\isamarkupsubsubsection{Tradeoff Between Aggregation vs Wholistic Evaluation%
}
\isamarkuptrue%
%
\endisatagdocument
{\isafolddocument}%
%
\isadelimdocument
%
\endisadelimdocument
%
\begin{isamarkuptext}%
Consequentialism also faces the further problem of aggregating goodness across people. 
Consequentialists who abandon aggregation must instead find some wholistic evaluation function
for a state of affairs. There is a tradeoff between the difficulty of aggregation
and the complexity of making judgements about an entire state of affairs, as opposed to about a single person.
A reasoner will need large, complicated datasets to settle these questions.%
\end{isamarkuptext}\isamarkuptrue%
%
\isadelimdocument
%
\endisadelimdocument
%
\isatagdocument
%
\isamarkupsection{Virtue Ethics%
}
\isamarkuptrue%
%
\isamarkupsubsection{Virtue Ethics Crash Course%
}
\isamarkuptrue%
%
\endisatagdocument
{\isafolddocument}%
%
\isadelimdocument
%
\endisadelimdocument
%
\begin{isamarkuptext}%
I understand the concept of virtue as those traits that are 
good for the posessor \cite{vesep}. I briefly explain Aristotle's eudaimonistic conception of virtue and present
some examples of virtues (courage, temperance, equanimity).%
\end{isamarkuptext}\isamarkuptrue%
%
\isadelimdocument
%
\endisadelimdocument
%
\isatagdocument
%
\isamarkupsubsection{Virtue Ethics is Hard to Formalize%
}
\isamarkuptrue%
%
\isamarkupsubsubsection{What is Virtue?%
}
\isamarkuptrue%
%
\endisatagdocument
{\isafolddocument}%
%
\isadelimdocument
%
\endisadelimdocument
%
\begin{isamarkuptext}%
Automated virtue ethics will need to plant a flag in the messy, controversial debate over the exact
list of virtues. While most Kantians agree on one interpretation of the FUL, most virtue ethicists 
have their own interpretations of what the virtues are.%
\end{isamarkuptext}\isamarkuptrue%
%
\isadelimdocument
%
\endisadelimdocument
%
\isatagdocument
%
\isamarkupsubsubsection{Representing Moral Character is Difficult%
}
\isamarkuptrue%
%
\endisatagdocument
{\isafolddocument}%
%
\isadelimdocument
%
\endisadelimdocument
%
\begin{isamarkuptext}%
Automated virtue ethics has to evaluate moral character, which is much more challenging than 
evaluating a maxim. Moral character is a complex concept to precisely represent to a computer.%
\end{isamarkuptext}\isamarkuptrue%
%
\isadelimdocument
%
\endisadelimdocument
%
\isatagdocument
%
\isamarkupsubsubsection{Machine Learning and Virtue Ethics%
}
\isamarkuptrue%
%
\endisatagdocument
{\isafolddocument}%
%
\isadelimdocument
%
\endisadelimdocument
%
\begin{isamarkuptext}%
There is a connection between the ideas of cultivating habit and mimicking virtuous action
and machine learning, which mimics patterns in datasets\footnote{Would love feedback on whether this should
go here or in a Related Work section.}. There's been some
work using machine learning to learn moral behavior from a dataset of actions tagged as ethical \cite{delphi}. 
One drawback of this approach is the fact that machine learning algorithms have trouble
explaining why they made the judgements they made and often pick up on patterns that human beings would 
not see as indicative of causation. In contrast,
my system can explain exactly which axioms and principles resulted in a maxim being obligated or prohibited
and can even present human-reconstructable proofs of its results.%
\end{isamarkuptext}\isamarkuptrue%
%
\isadelimtheory
%
\endisadelimtheory
%
\isatagtheory
%
\endisatagtheory
{\isafoldtheory}%
%
\isadelimtheory
%
\endisadelimtheory
%
\end{isabellebody}%
\endinput
%:%file=~/Desktop/cs91r/paper/outline_whykant.thy%:%
%:%24=6%:%
%:%36=8%:%
%:%37=9%:%
%:%38=10%:%
%:%39=11%:%
%:%40=12%:%
%:%41=13%:%
%:%42=14%:%
%:%43=15%:%
%:%44=16%:%
%:%45=17%:%
%:%54=19%:%
%:%58=21%:%
%:%70=23%:%
%:%71=24%:%
%:%72=25%:%
%:%73=26%:%
%:%74=27%:%
%:%75=28%:%
%:%84=31%:%
%:%96=33%:%
%:%97=34%:%
%:%98=35%:%
%:%99=36%:%
%:%108=39%:%
%:%120=41%:%
%:%121=42%:%
%:%122=43%:%
%:%123=44%:%
%:%132=46%:%
%:%136=48%:%
%:%148=50%:%
%:%149=51%:%
%:%150=52%:%
%:%151=53%:%
%:%160=55%:%
%:%164=57%:%
%:%176=59%:%
%:%177=60%:%
%:%178=61%:%
%:%179=62%:%
%:%188=64%:%
%:%200=66%:%
%:%201=67%:%
%:%202=68%:%
%:%203=69%:%
%:%204=70%:%
%:%213=72%:%
%:%217=74%:%
%:%229=76%:%
%:%230=77%:%
%:%231=78%:%
%:%240=80%:%
%:%244=82%:%
%:%256=85%:%
%:%257=86%:%
%:%258=87%:%
%:%267=89%:%
%:%279=91%:%
%:%280=92%:%
%:%289=94%:%
%:%301=96%:%
%:%302=97%:%
%:%303=98%:%
%:%304=99%:%
%:%305=100%:%
%:%306=101%:%
%:%307=102%:%
%:%308=103%:%
%:%309=104%:%