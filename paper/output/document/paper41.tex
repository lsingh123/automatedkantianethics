%
\begin{isabellebody}%
\setisabellecontext{paper{\isadigit{4}}{\isadigit{1}}}%
%
\isadelimtheory
%
\endisadelimtheory
%
\isatagtheory
%
\endisatagtheory
{\isafoldtheory}%
%
\isadelimtheory
%
\endisadelimtheory
%
\isadelimdocument
%
\endisadelimdocument
%
\isatagdocument
%
\isamarkupsection{Novel Formalization of the Categorical Imperative%
}
\isamarkuptrue%
%
\endisatagdocument
{\isafolddocument}%
%
\isadelimdocument
%
\endisadelimdocument
%
\begin{isamarkuptext}%
In this section, I present a custom formalization of the categorical imperative, as inspired by 
the goals from the previous chapter.%
\end{isamarkuptext}\isamarkuptrue%
%
\isadelimdocument
%
\endisadelimdocument
%
\isatagdocument
%
\isamarkupsubsection{Logical Background%
}
\isamarkuptrue%
%
\endisatagdocument
{\isafolddocument}%
%
\isadelimdocument
%
\endisadelimdocument
%
\begin{isamarkuptext}%
The previous attempts to model the categorical imperative in Chapter 2 partially failed due to 
an inability to fully represent the complexity of a maxim. Specifically, they treated actions as a single, 
monolithic unit of evaluation, whereas most Kantians consider the unit of evaluation for the FUL to the more
complex notion of a maxim. In this section, I will present some logical background necessarily to fully 
capture the spirit of a maxim. I will begin by borrowing some machinery to handle ``subjects" who perform 
actions from Chapter 2.%
\end{isamarkuptext}\isamarkuptrue%
\isacommand{typedecl}\isamarkupfalse%
\ s\ %
\isamarkupcmt{s is the type for a ``subject," i.e. the subject of a sentence. In this interpretation, 
a subject is merely defined as ``that which can act." It does not include any other properties, such as 
rationality or dignity. As I will show, for the purposes of defining the universalizability test, this 
``thin" representation of a subject suffices.%
}\isanewline
\isanewline
\isacommand{type{\isacharunderscore}synonym}\isamarkupfalse%
\ os\ {\isacharequal}\ {\isachardoublequoteopen}{\isacharparenleft}s\ {\isasymRightarrow}\ t{\isacharparenright}{\isachardoublequoteclose}\ %
\isamarkupcmt{Recall that an open sentence maps a subject to a term to model the 
substitution operator.%
}\isanewline
\isanewline
\isacommand{type{\isacharunderscore}synonym}\isamarkupfalse%
\ maxim\ {\isacharequal}\ {\isachardoublequoteopen}{\isacharparenleft}t\ {\isacharasterisk}\ os\ {\isacharasterisk}\ t{\isacharparenright}{\isachardoublequoteclose}%
\begin{isamarkuptext}%
The central unit of evaluation for the universalizability test is a ``maxim," which Kant defines 
in a footnote in \emph{Groundwork} as ``the subjective principle of willing," or the principle that 
the agent acts on \cite[16]{groundwork}. Modern Kantians differ in their interpretations of this definition. The naive view 
is that a maxim is an act, but Korsgaard adopts the more sophisticated view that a maxim is composed
of an act and the agent's purpose for acting \cite{actingforareason}. She also compares a maxim 
to Aristotle's logos, which includes these components and information about the circumstances and methods 
of the act. O'Neill concludes that Kant's examples imply that a maxim must also include circumstances \cite{actingonprinciple}, and 
Kitcher \cite{whatisamaxim} uses textual evidence from the Groundwork to argue for the inclusion of a maxim's purpose 
or motivation. In order to formalize the notion of a maxim, I must adopt a specific definition and 
defend my choice.

I define a maxim as a circumstance, act, goal tuple (C, A, G), read 
as ``In circumstances C, act A for goal G." Isabelle's strict typing rules mean that the choice of the 
type of each member of this tuple is significant. A circumstance is represented as a set of worlds 
$t$ where that circumstance holds. A goal is also a term because it can be true or false at a world if it 
is realized or not. An act is an open sentence because an act itself is not the kind of thing that can 
be true or false (as in, an act is not truth-apt), but the combination of a subject performing an act 
can be true or false at a world depending on whether or not the act is indeed performed by that subject. 
For example, ``running" is not truth-apt, but ``Sara runs" is truth-apt.

My definition of a maxim is inspired by O'Neill's work on maxims. I will defend my representation
below and consider an additional component that Kitcher argues for.

$\emph{O'Niell's Original Schematic and The Role of Practical Judgement}$

O'Neill \cite[37]{actingonprinciple} presents what Kitcher \cite{whatisamaxim}  calls the widely accepted 
view that a maxim is a circumstance, act, goal tuple. A maxim 
is an action-guiding rule and thus naturally includes an act and the circumstances under which 
it should be performed, which are often referred to as ``morally relevant circumstances." 

She also includes a purpose, end, or goal in the maxim because Kant includes this in many of his 
example maxims and because Kant argues that human activity, because it is guided by a rational will, 
is inherently purposive \cite[4:428]{groundwork}. A rational will does not act randomly (else it would not be rational), 
but instead in the pursuit of ends which it deems valuable. This inclusion is also essential for the version of the universalizability test 
that I will implement, explained in Section ??.

O'Neill's inclusion of circumstances is potentially controversial because it leaves open the question of what qualifies as a 
relevant circumstance for a particular maxim. This is gives rise to ``the tailoring objection" \cite[217]{whatisamaxim} \footnote{Kitcher
cites \cite{kantsethicalthought}  as offering an example of a false positive due to this objection.}, 
under which maxims are arbitrarily specified to pass the FUL. For example, the maxim ``When my name is Lavanya Singh,
I will lie to get some easy money," is universalizable, but is clearly a false positive. One solution to 
this problem is to argue that the circumstance ``When my name is Lavanya Singh" is not morally relevant 
to the act and goal. This solution requires some discussion of what qualifies as a relevant circumstance.

O'Neill seems to acknowledge the difficulty of determining relevant circumstances when she concedes that a maxim cannot include all 
of the infinitely many circumstances in which the agent may perform the action\cite[4:428]{actingonprinciple}. She argues that this is 
an artifact of the fact that maxims are rules of practical reason, the kind of reason that helps us decide what to do 
and how to do it \cite{bok}. Like any practical rule, 
maxims require the exercise of practical judgement to determine in which circumstances they should be applied. 
This judgement, applied in both choosing when to exercise the maxim and in the formulation of the maxim 
itself, is what determines the ``morally relevant circumstances."

The upshot for computational ethics is that the computer cannot perform all ethical activity alone. 
Human judgement and the exercise of practical reason are essential to both formulate maxims and 
determine when the actual conditions of life coincide with the circumstances in which the maxim is relevant. 
Choosing when to exercise a maxim is less relevant to my project because analyzing a formal representation of the FUL requires 
making the circumstances in a given scenario precise, but will be important for applications of 
computational ethics to guiding AI agents. The difficulty in formulating a maxim, on the other hand, demonstrates 
the important fact that ethics, as presented here, is not a solely computational activity. A
human being must create a representation for the dilemma they wish to test, effectively translating 
a complex, real situation into a flat logical structure. This parallels the challenge that programmers 
face when translating the complexity of reality to a programming langauge or computational representation. Not only will some of the situation's complexity
inevitably be lost, the outcome of the universalizability test will depend on how the human formulates the maxim
and whether or not this formulation does indeed include morally relevant circumstances. If the human puts 
garbage into the test, the test will return garbage out.

While this may appear to be a weakness of my system, I believe that it actually
allows my system to retain some of the human complexity that many philosophers agree cannot be automated away.\footnote{Powers presents 
the determination of morally relevant circumstances as an obstacle to the automation of Kantian ethics \cite{powers}.}
Ethics is a fundamentally human activity. Kant argues that the categorical imperative is a statement 
about the properties of rational wills. In fact, Korsgaard argues that morality derives its authority over us, 
or normativity, only because is it a property of a rational will, and we, as human beings, are rational wills.
If ethics is meant to guide human behavior, the role of the computer becomes clear as not a replacement for our will,
but instead as a tool to help guide our wills and reason more efficiently 
and more effectively. Just as calculators don't render mathematicians obsolete, computational ethics
does not render human judgement or philosophy obsolete. Chapter 4 Section ?? will be devoted to a more complete discussion 
of this issue.

$\emph{Exclusion of Motive}$

Kitcher begins with O'Niell's circumstance, act, goal view and expands it to include the motive 
behind performing the maxim \cite{whatisamaxim}. This additional component is read 
as ``In circumstance C, I will do A in order to G because of M," where M may be ``duty" or ``self-love."
Kitcher argues that the inclusion of motive is necessary for the fullest, most general form of a maxim
in order to capture Kant's idea that an action derives its moral worth from being done for the sake of duty itself.
Under this view, the FUL would obligate maxims of the form 
``In circumstance C, I will do A in order to G because I can will that I and everyone else simultaneously
will do A in order to G in circumstance C." In other words, if Kant is correct in arguing that moral 
actions must be done from the motive of duty, the affirmative result of the FUL becomes 
the motive for a moral action.

While Kitcher's conception of a maxim captures Kant's idea of acting for duty's own sake, I will not implement it 
because it is not necessary for putting maxims through the FUL. Indeed, Kitcher acknowledges that 
O'Neill's formulation suffices for the universalizability test, but is not the general notion of a maxim.
In order to pass the maxim through the FUL, it suffices to know the circumstance, act, and goal. The FUL
derives the motive that Kitcher bundles into the maxim, so automating the FUL does not require 
including a motive. The ``input" to the FUL is the circumstance, act, goal tuple. My project takes 
this input and returns the motivation that the dutiful, moral agent would adopt. Additionally, doing
justice to the rich notion of motive requires modelling the operation of practical reason itself, 
which is outside the scope of this project. My work focuses on the universalizability test, but future work that 
models the process of practical reason may use my implementation of the FUL as a ``library." Combined 
with a logic of practical reason, an implementation of the FUL can move from evaluating a maxim to 
evaluating an agent's behavior, since that's when ``acting from duty" starts to matter.%
\end{isamarkuptext}\isamarkuptrue%
\isacommand{abbreviation}\isamarkupfalse%
\ will\ {\isacharcolon}{\isacharcolon}\ {\isachardoublequoteopen}maxim\ {\isasymRightarrow}\ s{\isasymRightarrow}\ \ t{\isachardoublequoteclose}\ {\isacharparenleft}{\isachardoublequoteopen}W\ {\isacharunderscore}\ {\isacharunderscore}{\isachardoublequoteclose}{\isacharparenright}\isanewline
\ \ \isakeyword{where}\ {\isachardoublequoteopen}will\ {\isasymequiv}\ {\isasymlambda}{\isacharparenleft}c{\isacharcomma}\ a{\isacharcomma}\ g{\isacharparenright}\ s{\isachardot}\ {\isacharparenleft}c\ \isactrlbold {\isasymrightarrow}\ {\isacharparenleft}a\ s{\isacharparenright}{\isacharparenright}{\isachardoublequoteclose}%
\begin{isamarkuptext}%
Korsgaard claims that ``to will an end, rather than just
wishing for it or wanting it, is to set yourself to be its cause" \cite[38]{sources}. To will a maxim
is to set yourself to be the cause of its goal by taking the means 
specified in the maxim in the relevant circumstances. This coheres with 
Kitcher's and Korsgaard's understanding of a maxim as a principle or rule to live by. 

At worlds
where the circumstances do not hold, a maxim is vacuously willed. If you decide to act on the rule ``I will 
do X in these cirumstances", then you are vacuously obeying it when the circumstances don't hold.  

The above discussion implies that willing a maxim is particular to the agent, justifying my choice to 
require that a particular subject will a maxim. O'Neill argues for this interpretation when she distinguishes 
between the evaluation of a principle, which is generic, and a maxim, which she views as ``individuated only 
by referring to a person"\cite[13]{actingonprinciple}. I adopt the spirit of this interpretation but modify it slightly 
by representing the general maxim as a principle that anyone could adopt, and the act of willing the maxim 
as a person-particular instantiation of the maxim.

I additionally represent a subject as willing a maxim because I use the word `will' as a verb, to mean committing oneself to living by
the principle of a maxim. This coheres with the FUL, which tests the act willing 
of a maxim by determining if the maxim could be a universal law that everyone committed to. Formalizing this idea,
the type of a willed maxim is a term, allowing me
to use DDL's obligation operator on the notion of willing a maxim. Concretely, my system will prove 
or disprove statements of the form ``Lavanya is obligated to will the maxim M." 

Worlds where the circumstances do not hold are not relevant for determining obligation. Recall that in 
Benzmueller et. al's definition of the obligation operator,  $O \{B|A\}$ is true at all worlds iff ob(B)(A), or 
if the obligation function maps A to obligatory in context B (where the context is a set of worlds) \cite{BFP}. This 
definition implies that worlds outside of B have no bearing on the moral status of A in context B, which 
coheres with intuitions about contextual obligation. Thus, the dyadic obligation operator 
disqualifies worlds where the context does not hold, so the vacuous truth of the will statement in 
these worlds does not matter. 

Given that the will abbreviation already excludes worlds where the circumstances fail (by rendering 
the statement vacuously true at them), one may conclude that the dyadic obligation operator is now useless. 
Using the dyadic obligation operator allows me to take advantage of the power of DDL to represent the bearing 
that circumstances have on obligation. DDL has powerful axioms expressing the relationship between circumstances 
and obligation, such as the fact that obligations are monotonically increasing with respect to broader 
circumstances. Using the monadic obligation operator would require me to either operate with an empty 
notion of context or to redefine these axioms. The dyadic obligation operator lets me take advantage of the full 
power of DDL in expressing contrary-to-duty obligations. This is particularly important for Kantian ethics 
and the FUL specifically because many critiques of the FUL rely on attention to circumstances (tailoring 
objection) or lack thereof (ideal theory). This is also an innovation that my custom formalization presents 
over the prior work. By formally including the notion of a circumstance or context, I am able to represent 
these objections that Kantian scholars study. Formalizing Kantian ethics in a dyadic deontic logic 
instead of a monadic deontic logic is a key contribution of this thesis.%
\end{isamarkuptext}\isamarkuptrue%
\isacommand{abbreviation}\isamarkupfalse%
\ effective\ {\isacharcolon}{\isacharcolon}\ {\isachardoublequoteopen}maxim{\isasymRightarrow}s{\isasymRightarrow}\ t{\isachardoublequoteclose}\ {\isacharparenleft}{\isachardoublequoteopen}E\ {\isacharunderscore}\ {\isacharunderscore}{\isachardoublequoteclose}{\isacharparenright}\isanewline
\ \ \isakeyword{where}\ {\isachardoublequoteopen}effective\ \ {\isasymequiv}\ {\isasymlambda}{\isacharparenleft}c{\isacharcomma}\ a{\isacharcomma}\ g{\isacharparenright}\ s{\isachardot}\ {\isacharparenleft}{\isacharparenleft}will\ {\isacharparenleft}c{\isacharcomma}\ a{\isacharcomma}\ g{\isacharparenright}\ s{\isacharparenright}\ \isactrlbold {\isasymequiv}\ g{\isacharparenright}{\isachardoublequoteclose}\isanewline
\isacommand{print{\isacharunderscore}theorems}\isamarkupfalse%
%
\begin{isamarkuptext}%
A maxim is effective for a subject when, if the subject wills it then the goal is achieved, and
when the subject does not act on it, the goal is not achieved$\footnote{Thank you to Jeremy D. Zucker for helping me think through this.}$ \cite{sepcausation}. 
The former direction of the implication 
is intuitive: if the act results in the goal, it was effective in causing the goal. This represents `necessary'
causality. 

The latter direction represents `sufficient' causality, or the idea that, counterfactually,
if the maxim were not willed, then the goal is not achieved \cite{lewiscausality}. Note that nothing else changes about this
counterfactual world—the circumstances are identical and we neither added additional theorems nor 
specified the model any further. This represents Lewis's idea of "comparative similarity,"  where 
a counterfactual is true if it holds at the most similar world \cite{lewiscounterfactuals}. In our case, this is just the world 
where everything is the same except the maxim is not acted on.

Combining these ideas, this definition of effective states that a maxim is effective if the 
maxim being acted on by a subject is the necessary and sufficient cause of the goal.\footnote{Should I wave a hand at critiques of counterfactual causality?}

If the circumstances do not hold and the goal is achieved, then the maxim is vacuously effective, since 
it is vacuously willed (as described above). While this scenario is counterintuitive, it is not very 
interesting for my purposes because, when the circumstances do not hold, a maxim is not applicable. It 
doesn't really make sense to evaluate a maxim when it's not supposed to be applied. The maxim ``When on Jupiter,
read a book to one-up your nemesis" is vacuously effective because it can never be disproven.%
\end{isamarkuptext}\isamarkuptrue%
\isacommand{abbreviation}\isamarkupfalse%
\ universalized{\isacharcolon}{\isacharcolon}{\isachardoublequoteopen}maxim{\isasymRightarrow}t{\isachardoublequoteclose}\ \isakeyword{where}\ \isanewline
{\isachardoublequoteopen}universalized\ {\isasymequiv}\ {\isasymlambda}M{\isachardot}\ {\isacharparenleft}{\isasymlambda}w{\isachardot}\ {\isacharparenleft}{\isasymforall}p{\isachardot}\ {\isacharparenleft}W\ M\ p{\isacharparenright}\ w{\isacharparenright}{\isacharparenright}{\isachardoublequoteclose}\isanewline
\isanewline
\isacommand{abbreviation}\isamarkupfalse%
\ holds{\isacharcolon}{\isacharcolon}{\isachardoublequoteopen}maxim{\isasymRightarrow}t{\isachardoublequoteclose}\ \isakeyword{where}\ \isanewline
{\isachardoublequoteopen}holds\ {\isasymequiv}\ {\isasymlambda}{\isacharparenleft}c{\isacharcomma}\ a{\isacharcomma}\ g{\isacharparenright}{\isachardot}\ c{\isachardoublequoteclose}\isanewline
\isanewline
\isacommand{abbreviation}\isamarkupfalse%
\ not{\isacharunderscore}universalizable\ {\isacharcolon}{\isacharcolon}\ {\isachardoublequoteopen}maxim{\isasymRightarrow}s{\isasymRightarrow}bool{\isachardoublequoteclose}\ \isakeyword{where}\ \isanewline
{\isachardoublequoteopen}not{\isacharunderscore}universalizable\ {\isasymequiv}\ {\isasymlambda}M\ s{\isachardot}\ {\isasymforall}w{\isachardot}\ {\isacharparenleft}{\isacharparenleft}universalized\ M{\isacharparenright}\ \ \isactrlbold {\isasymrightarrow}\ {\isacharparenleft}\isactrlbold {\isasymnot}\ {\isacharparenleft}E\ M\ s{\isacharparenright}{\isacharparenright}{\isacharparenright}\ w{\isachardoublequoteclose}\isanewline
%
\isamarkupcmt{The formula above reads ``at world $w$, if M is universalized and M is acted on (i.e. the circumstances 
of M hold), then M is not effective." 
Notice that the antecedent specifies that the circumstances hold at the given world. 
When evaluating if a maxim is universalizable or not, we want to ignore worlds where the circumstance 
do not hold. At these worlds, the maxim is trivially effective and thus trivially universalizable. If we didn't exclude such worlds from 
consideration, a maxim with circumstances that ever fail to hold would be universalizable. Clearly 
this is not a desirable conclusion, since maxims like ``When you need money, lie to get easy money"
would be universalizable.%
}%
\begin{isamarkuptext}%
As before, the concepts of prohibition and permissibility will be helpful here. The unit of 
evaluation for my formalization of the FUL is the act of willing a maxim, which entails performing 
the maxim's act in the relevant circumstances. Therefore, I will say that, just as the act of willing a
 maxim can be obligatory for a subject, it can be prohibited or permissible for a subject.\footnote{In 
the rest of this section, for convenience, I will use the phrase ``subject s willing maxim M is obligatory" 
interchangeably with ``maxim M is obligatory for subject s." I will use ``maxim M is obligatory" to 
refer to M being obligatory for any arbitrary subject, which I will show to be equivalent to M being 
obligatory for a specific subject.}%
\end{isamarkuptext}\isamarkuptrue%
\isacommand{abbreviation}\isamarkupfalse%
\ prohibited{\isacharcolon}{\isacharcolon}{\isachardoublequoteopen}maxim{\isasymRightarrow}s{\isasymRightarrow}t{\isachardoublequoteclose}\ \isakeyword{where}\ \isanewline
{\isachardoublequoteopen}prohibited\ {\isasymequiv}\ {\isasymlambda}{\isacharparenleft}c{\isacharcomma}\ a{\isacharcomma}\ g{\isacharparenright}\ s{\isachardot}\ O{\isacharbraceleft}\isactrlbold {\isasymnot}\ {\isacharparenleft}will\ {\isacharparenleft}c{\isacharcomma}a{\isacharcomma}\ g{\isacharparenright}\ s{\isacharparenright}\ {\isacharbar}\ c{\isacharbraceright}{\isachardoublequoteclose}\isanewline
\isanewline
\isacommand{abbreviation}\isamarkupfalse%
\ permissible{\isacharcolon}{\isacharcolon}{\isachardoublequoteopen}maxim{\isasymRightarrow}s{\isasymRightarrow}t{\isachardoublequoteclose}\isanewline
\ \ \isakeyword{where}\ {\isachardoublequoteopen}permissible\ {\isasymequiv}\ {\isasymlambda}M\ s{\isachardot}\ \isactrlbold {\isasymnot}\ {\isacharparenleft}prohibited\ M\ s{\isacharparenright}{\isachardoublequoteclose}\isanewline
%
\isamarkupcmt{I will say that a maxim is permissible for a subject if it is not prohibited for that subject to 
will that maxim.%
}%
\begin{isamarkuptext}%
When analyzing the naive formalization and Kroy's formalization, I learned that DDL and the prior 
formalizations allow contradictory obligations. This is a major weakness of these systems, and my 
formalization should fix this. To do so, I will add as an axiom the idea that obligations cannot 
contradict each other or their internal circumstances. Formally, conflicting obligations are defined below.%
\end{isamarkuptext}\isamarkuptrue%
\isacommand{abbreviation}\isamarkupfalse%
\ non{\isacharunderscore}contradictory\ \isakeyword{where}\ \isanewline
{\isachardoublequoteopen}non{\isacharunderscore}contradictory\ A\ B\ c\ w\ {\isasymequiv}\ {\isacharparenleft}{\isacharparenleft}O{\isacharbraceleft}A{\isacharbar}c{\isacharbraceright}\ \isactrlbold {\isasymand}\ O{\isacharbraceleft}B{\isacharbar}c{\isacharbraceright}{\isacharparenright}\ w{\isacharparenright}\ {\isasymlongrightarrow}\ {\isasymnot}{\isacharparenleft}{\isacharparenleft}A\ \isactrlbold {\isasymand}\ {\isacharparenleft}B\ \isactrlbold {\isasymand}\ c{\isacharparenright}{\isacharparenright}\ w\ {\isasymlongrightarrow}\ False{\isacharparenright}{\isachardoublequoteclose}\isanewline
%
\isamarkupcmt{Terms A and B are non contradictory in circumstances c if, when A and B are obligated in circumstances 
c, the conjunction of A, B, and c, does not imply False.%
}\isanewline
\isanewline
\isacommand{axiomatization}\isamarkupfalse%
\ \isakeyword{where}\ no{\isacharunderscore}contradictions{\isacharcolon}{\isachardoublequoteopen}{\isasymforall}A{\isacharcolon}{\isacharcolon}t{\isachardot}\ {\isasymforall}B{\isacharcolon}{\isacharcolon}t{\isachardot}\ {\isasymforall}c{\isacharcolon}{\isacharcolon}t{\isachardot}\ {\isasymforall}w{\isacharcolon}{\isacharcolon}i{\isachardot}\ non{\isacharunderscore}contradictory\ A\ B\ c\ w{\isachardoublequoteclose}\isanewline
%
\isamarkupcmt{This axiom formalizes the idea that, for any terms A, B, and circumstances c, A and B must be 
non-contradictory in circumstances c at all worlds. Intuitively, this axiom requires that obligations 
do not conflict.%
}%
\isadelimdocument
%
\endisadelimdocument
%
\isatagdocument
%
\isamarkupsubsection{Formalizing the FUL%
}
\isamarkuptrue%
%
\endisatagdocument
{\isafolddocument}%
%
\isadelimdocument
%
\endisadelimdocument
%
\begin{isamarkuptext}%
Below is my first attempt at formalizing Korgsaard's definition of the practical contradiction
interpretation:  a maxim is not universalizable 
if, in the world where the maxim becomes the standard practice (i.e. everyone acts on the maxim), the
agent's attempt to use the maxim's act to achieve the maxim's goal is frustrated. In other words, if 
the maxim is universally willed (captured by applying a universal qunatifier and the will function 
to the maxim on the LHS), then it is no longer effective for the subject $s$ (RHS above).%
\end{isamarkuptext}\isamarkuptrue%
\isacommand{abbreviation}\isamarkupfalse%
\ FUL{\isadigit{0}}{\isacharcolon}{\isacharcolon}bool\ \isakeyword{where}\ {\isachardoublequoteopen}FUL{\isadigit{0}}\ {\isasymequiv}\ {\isasymforall}\ c\ a\ g\ s{\isachardot}\ not{\isacharunderscore}universalizable\ {\isacharparenleft}c{\isacharcomma}\ a{\isacharcomma}\ g{\isacharparenright}\ s\ {\isasymlongrightarrow}\ {\isasymTurnstile}{\isacharparenleft}{\isacharparenleft}prohibited\ {\isacharparenleft}c{\isacharcomma}\ a{\isacharcomma}\ g{\isacharparenright}\ s{\isacharparenright}{\isacharparenright}{\isachardoublequoteclose}\isanewline
%
\isamarkupcmt{This representation of the Formula of Universal Law reads, ``For all circumstances, goals, acts, 
and subjects, if the maxim of the subject performing the act for the goal in the circumstances is not 
universalizable (as defined above), then, at all worlds, in those circumstances, the subject 
is prohibited (obligated not to) from willing the maxim.%
}\isanewline
\isanewline
\isacommand{lemma}\isamarkupfalse%
\ {\isachardoublequoteopen}FUL{\isadigit{0}}\ {\isasymlongrightarrow}\ False{\isachardoublequoteclose}%
\isadelimproof
\ %
\endisadelimproof
%
\isatagproof
\isacommand{using}\isamarkupfalse%
\ O{\isacharunderscore}diamond\isanewline
\ \ \isacommand{using}\isamarkupfalse%
\ case{\isacharunderscore}prod{\isacharunderscore}conv\ no{\isacharunderscore}contradictions\ old{\isachardot}prod{\isachardot}case\ old{\isachardot}prod{\isachardot}case\ \isacommand{by}\isamarkupfalse%
\ fastforce%
\endisatagproof
{\isafoldproof}%
%
\isadelimproof
%
\endisadelimproof
%
\begin{isamarkuptext}%
FUL0 is not consistent, and sledgehammer is able to prove this by showing that it implies a contradiction 
usig axiom O\_diamond, which is \isa{{\isasymTurnstile}{\isasymlambda}w{\isachardot}\ ob\ {\isacharquery}B\ {\isacharquery}A\ {\isasymlongrightarrow}\ {\isasymnot}\ {\isasymTurnstile}\isactrlbold {\isasymnot}\ {\isacharquery}B\isactrlbold {\isasymand}{\isacharquery}A}. This axiom captures 
the idea that an obligation can't contradict its context. This is particularly problematic if the goal or 
action of a maxim are equivalent to its circumstances. In other words, if the maxim has already been 
acted on or the goal has already been achieved, then prohibiting it is impossible. 
In any model that has at least one term, it is possible to construct a maxim where the circumstances, goal, 
and act (once a subject acts on it) are all that same term, resulting in a contradiction. 

To get around this, I will exclude what I call ``badly formed maxims," which are those maxims such that the goal has already been 
achieved or the act has already been acted on. Under my formalization, such maxims are 
not well-formed. To understand why, I return to Korsgaard's and O'Neill's interpretations of a maxim as a practical
guide to action. A maxim is a practical principle that guides how we behave in everyday life. A 
principle of the form ``When you are eating breakfast, eat breakfast in order to eat breakfast," is not 
practically relevant. No agent would ever need to act on such a principle. It is not contradictory
or prohibited, but it is the wrong kind of question to be asking. It is not a 
well-formed maxim, so the categorical imperative does not apply to it. (more explanation in philosophical 
writing collection)%
\end{isamarkuptext}\isamarkuptrue%
\isacommand{abbreviation}\isamarkupfalse%
\ well{\isacharunderscore}formed{\isacharcolon}{\isacharcolon}{\isachardoublequoteopen}maxim{\isasymRightarrow}s{\isasymRightarrow}i{\isasymRightarrow}bool{\isachardoublequoteclose}\ \isakeyword{where}\ \isanewline
{\isachardoublequoteopen}well{\isacharunderscore}formed\ {\isasymequiv}\ {\isasymlambda}{\isacharparenleft}c{\isacharcomma}\ a{\isacharcomma}\ g{\isacharparenright}{\isachardot}\ {\isasymlambda}s{\isachardot}\ {\isasymlambda}w{\isachardot}\ {\isacharparenleft}{\isasymnot}\ \ {\isacharparenleft}c\ \isactrlbold {\isasymrightarrow}\ g{\isacharparenright}\ w{\isacharparenright}\ {\isasymand}\ {\isacharparenleft}{\isasymnot}\ \ {\isacharparenleft}c\ \isactrlbold {\isasymrightarrow}\ a\ s{\isacharparenright}\ w{\isacharparenright}{\isachardoublequoteclose}\isanewline
%
\isamarkupcmt{This abbreviation formalizes the well-formedness of a maxim for a subject. The goal cannot be 
already achieved in the circumstances and the subject cannot have already performed the act.%
}\isanewline
\isanewline
\isacommand{abbreviation}\isamarkupfalse%
\ FUL\ \isakeyword{where}\ \isanewline
{\isachardoublequoteopen}FUL\ {\isasymequiv}\ {\isasymforall}M{\isacharcolon}{\isacharcolon}maxim{\isachardot}\ {\isasymforall}s{\isacharcolon}{\isacharcolon}s{\isachardot}\ {\isacharparenleft}{\isasymforall}w{\isachardot}\ well{\isacharunderscore}formed\ M\ s\ w{\isacharparenright}\ {\isasymlongrightarrow}\ {\isacharparenleft}not{\isacharunderscore}universalizable\ M\ s\ {\isasymlongrightarrow}\ {\isasymTurnstile}\ {\isacharparenleft}prohibited\ M\ s{\isacharparenright}\ {\isacharparenright}{\isachardoublequoteclose}\isanewline
%
\isamarkupcmt{Let's try the exact same formalization of the FUL as above, except that it only applies to 
maxims that are well-formed at every world.%
}\isanewline
\isanewline
\isacommand{lemma}\isamarkupfalse%
\ {\isachardoublequoteopen}FUL{\isachardoublequoteclose}\isanewline
\ \ \isacommand{nitpick}\isamarkupfalse%
{\isacharbrackleft}user{\isacharunderscore}axioms{\isacharcomma}\ falsify{\isacharequal}true{\isacharbrackright}%
\isadelimproof
\ %
\endisadelimproof
%
\isatagproof
\isacommand{oops}\isamarkupfalse%
\isanewline
%
\isamarkupcmt{The FUL does not hold in DDL, because nitpick is able to find a model for my system in which it is 
false. If the FUL were already a theorem of the system, adding it wouldn't make the system any more 
powerful, so this is the desired result.

$\color{blue}$ Nitpick found a counterexample for card s = 1 and card i = 1:

  Skolem constants:
    M = (($\lambda x. \_$)($i_1$ := True), ($\lambda x. \_$)($s_1$ := ($\lambda x. \_$)($i_1$ := False)), ($\lambda x. \_$)
         ($i_1$ := False))
    $\lambda$ w. p = ($\lambda x. \_$)($i_1$ := $s_1$)
    s = $s_1$ $\color{black}$%
}%
\endisatagproof
{\isafoldproof}%
%
\isadelimproof
%
\endisadelimproof
\isanewline
\isanewline
\isacommand{axiomatization}\isamarkupfalse%
\ \isakeyword{where}\ FUL{\isacharcolon}FUL\isanewline
\isanewline
\isacommand{lemma}\isamarkupfalse%
\ True\isanewline
\ \ \isacommand{nitpick}\isamarkupfalse%
{\isacharbrackleft}user{\isacharunderscore}axioms{\isacharcomma}\ falsify{\isacharequal}false{\isacharbrackright}%
\isadelimproof
\ %
\endisadelimproof
%
\isatagproof
\isacommand{by}\isamarkupfalse%
\ simp\isanewline
%
\isamarkupcmt{Nitpick is able to find a model in which all axioms are satisfied, 
so this version of the FUL is consistent.

$\color{blue}$ Nitpick found a model for card i = 1 and card s = 1:

  Empty assignment $\color{black}$%
}%
\endisatagproof
{\isafoldproof}%
%
\isadelimproof
%
\endisadelimproof
%
\begin{isamarkuptext}%
During the process of making FUL0 consistent, I used Isabelle to gain philosophical insights 
about vacuous maxims. This process is an example of the power of computational tools to aid
philosophical progress. I used Nitpick and Sledgehammer to quickly test if a small tweak 
to FUL0 fixed the inconsistency or if I was still able to derive a contradiction.  I then realized that if 
I defined the circumstances, act, and goal as constants, then FUL0 was indeed consistent. After some 
experimentation, Prof. Amin correctly pointed out that as constants, these three entities were 
distinct. However, when merely quantifying over (c, a, g), all members of a tuple could be equivalent. Within
a minute, I could formalize this notion, add it to FUL0, and test if it solved the problem. The fact 
that it did spurred my philosophical insight about vacuous maxims. 

The logic confirmed that certain kinds
of circumstance, act, goal tuples are too badly formed for the categorical imperative to logically 
apply to them. The realization of this subtle problem would have been incredibly difficult without 
computational tools. The syntax and typing of Isabelle/HOL forced me to bind the free-variable $M$
in the FUL in different ways and allowed me to quickly test many bindings. The discovery of this 
logical inconsistency then enabled a philosophical insight about which kinds of maxims make sense as 
practical principles. This is one way to do computational ethics: model a system in a logic, use 
computational tools to refine and debug the logic, and then use insights about the logic to derive 
insights about the ethical phenonema it is modelling. This procedure parallels the use of proofs in 
theoretical math to understand the mathematical objects they model.%
\end{isamarkuptext}\isamarkuptrue%
%
\begin{isamarkuptext}%
One potential problem with my formalization is that it does not use the modal nature of the system. 
All of the properties that the FUL investigates hold at all worlds, in effect removing the modal nature 
of the system. This approach simplifies logical and therefore computational complexity, improving 
performance. On the other hand, it doesn't use the full expressivity of DDL. If I run into problems 
later on, one option is to tweak the FUL to use this expressivity.%
\end{isamarkuptext}\isamarkuptrue%
%
\isadelimdocument
%
\endisadelimdocument
%
\isatagdocument
%
\isamarkupsubsection{Application Tests%
}
\isamarkuptrue%
%
\endisatagdocument
{\isafolddocument}%
%
\isadelimdocument
%
\endisadelimdocument
%
\begin{isamarkuptext}%
As with the naive formalization and Kroy's formalization, I will apply my testing framework to 
my custom formalization of the FUL. I will begin with some basic application tests. In these tests, 
I specify particular maxims as constants with no properties and gradually add properties to understand 
how the system handles different kinds of maxims.%
\end{isamarkuptext}\isamarkuptrue%
%
\begin{isamarkuptext}%
I will show that the maxim, ``When strapped for cash, falsely promise to pay your friend back
to get some easy money." is prohibited. This example is due to Korsgaard and she uses it to highlight 
the strength of her preferred interpretation of the FUL, the practical contradiction interpretation \cite{KorsgaardFUL}.
There are two possible readings of this maxim, and I will show that my formalization can handle both. 
Under the first reading, the act of falsely promising is read as
as entering a pre-existing, implicit, social system of promising with no intention of upholding your 
promise. Under the second reading, the act of falsely promising is equivalent to uttering the worlds 
``I promise X" without intending to do X. The differences between these readings lies in the difference 
between promising as an act with meaning in a larger social structure and the utterance ``I promise."

Under the first reading, the maxim fails because falsely promising is no longer possible in a world where 
everyone everyone does so. This is how the logical contradiction interpretation reads this maxim—falsely 
promising is no longer possible when universalized because the institution of promising breaks down. 
The practical contradiction view also prohibits this maxim because if falsely promising is not longer 
possible, then it is no longer an effective way to achieve the end of getting some money. Below I 
define some logical tools to formalize this reading of this maxim.%
\end{isamarkuptext}\isamarkuptrue%
\isacommand{consts}\isamarkupfalse%
\ when{\isacharunderscore}strapped{\isacharunderscore}for{\isacharunderscore}cash{\isacharcolon}{\isacharcolon}t\isanewline
%
\isamarkupcmt{Constant representing the circumstances ``when strapped for cash." Recall that the type of circumstances 
is a term because circumstances can be true or false at a world.%
}\isanewline
\isacommand{consts}\isamarkupfalse%
\ falsely{\isacharunderscore}promise{\isacharcolon}{\isacharcolon}os\isanewline
%
\isamarkupcmt{Constant representing the act ``make a false promise to pay a loan back." Recall that the type of
an act is an open sentence because the sentence ``subject s performs act a" can be true or false at a world.%
}\isanewline
\isacommand{consts}\isamarkupfalse%
\ to{\isacharunderscore}get{\isacharunderscore}easy{\isacharunderscore}cash{\isacharcolon}{\isacharcolon}t\isanewline
%
\isamarkupcmt{Constant representing the goal ``to get some money." Recall that the type of a goal 
is a term because a goal can be true or false at a world depending on whether it is achieved or not.%
}\isanewline
\isanewline
\isacommand{abbreviation}\isamarkupfalse%
\ false{\isacharunderscore}promising{\isacharcolon}{\isacharcolon}maxim\ \isakeyword{where}\ \isanewline
{\isachardoublequoteopen}false{\isacharunderscore}promising\ {\isasymequiv}\ {\isacharparenleft}when{\isacharunderscore}strapped{\isacharunderscore}for{\isacharunderscore}cash{\isacharcomma}\ falsely{\isacharunderscore}promise{\isacharcomma}\ to{\isacharunderscore}get{\isacharunderscore}easy{\isacharunderscore}cash{\isacharparenright}{\isachardoublequoteclose}\isanewline
%
\isamarkupcmt{Armed with the circumstances, act, and goal above, I can define the example maxim as a tuple.%
}%
\begin{isamarkuptext}%
The logical objects above are ``empty," in the sense that I haven't specified any of their 
relevant properties. I will define these properties as assumptions and will show that, if the maxim 
above satisfies the assumed properties, it is prohibited.%
\end{isamarkuptext}\isamarkuptrue%
\isacommand{abbreviation}\isamarkupfalse%
\ everyone{\isacharunderscore}can{\isacharprime}t{\isacharunderscore}lie\ \isakeyword{where}\ \isanewline
{\isachardoublequoteopen}everyone{\isacharunderscore}can{\isacharprime}t{\isacharunderscore}lie\ {\isasymequiv}\ {\isasymforall}w{\isachardot}\ {\isasymnot}\ {\isacharparenleft}{\isasymforall}s{\isachardot}\ falsely{\isacharunderscore}promise{\isacharparenleft}s{\isacharparenright}\ w{\isacharparenright}\ {\isachardoublequoteclose}\isanewline
%
\isamarkupcmt{Under this reading, the problem with this maxim is that everyone can't
falsely promise simultaneously because the institution of promising will break down. It's probably 
possible to say something stronger than this (i.e. that if enough but not necessarily all people
falsely promise promising is no longer possible), but for my purposes this will suffice. The above 
formula reads, ``At all worlds, it is not the case that everyone falsely promises."%
}\isanewline
\isanewline
\isacommand{abbreviation}\isamarkupfalse%
\ circumstances{\isacharunderscore}hold\ \isakeyword{where}\ \isanewline
{\isachardoublequoteopen}circumstances{\isacharunderscore}hold\ {\isasymequiv}\ {\isasymforall}w{\isachardot}\ when{\isacharunderscore}strapped{\isacharunderscore}for{\isacharunderscore}cash\ w{\isachardoublequoteclose}\isanewline
%
\isamarkupcmt{This assumption narrows our scope of consideration to worlds where the circumstances of 
being strapped for cash hold. This is important because, at worlds where the circumstances do not hold, 
a maxim is trivially effective (since it's never acted on) and thus trivially universalizable. This 
assumption also makes practical sense; when evaluating a maxim, an agent would care about it specifically
at worlds where the circumstances hold, since these are the worlds where the maxim actually prescribes action.%
}\isanewline
\isanewline
\isacommand{abbreviation}\isamarkupfalse%
\ example{\isacharunderscore}is{\isacharunderscore}well{\isacharunderscore}formed\ \isakeyword{where}\ \isanewline
{\isachardoublequoteopen}example{\isacharunderscore}is{\isacharunderscore}well{\isacharunderscore}formed\ {\isasymequiv}\ {\isasymforall}s{\isachardot}\ {\isasymTurnstile}\ {\isacharparenleft}well{\isacharunderscore}formed\ false{\isacharunderscore}promising\ s{\isacharparenright}{\isachardoublequoteclose}\isanewline
%
\isamarkupcmt{This assumption states that the maxim of falsely promising is well-formed. This breaks down into
two individual assumptions. First, being strapped for cash can't imply falsely promising, which is plausible
because many people won't falsely promise under conditions of poverty. Second, being strapped for cash
can't imply getting ready cash, which is also plausible because people often fail to secure cash even 
when they need it.%
}%
\begin{isamarkuptext}%
Putting it all together, I want to show that if the three assumptions justified above hold, then 
the constructed maxim is prohibited. Below is the proof%
\end{isamarkuptext}\isamarkuptrue%
\isacommand{lemma}\isamarkupfalse%
\ lying{\isacharunderscore}bad{\isacharunderscore}{\isadigit{1}}{\isacharcolon}\isanewline
\ \ \isakeyword{assumes}\ everyone{\isacharunderscore}can{\isacharprime}t{\isacharunderscore}lie\isanewline
\ \ \isakeyword{assumes}\ circumstances{\isacharunderscore}hold\isanewline
\ \ \isakeyword{assumes}\ example{\isacharunderscore}is{\isacharunderscore}well{\isacharunderscore}formed\isanewline
\ \ \isakeyword{shows}\ {\isachardoublequoteopen}{\isasymforall}s{\isachardot}\ {\isasymTurnstile}\ {\isacharparenleft}prohibited\ false{\isacharunderscore}promising\ s{\isacharparenright}{\isachardoublequoteclose}\isanewline
%
\isadelimproof
%
\endisadelimproof
%
\isatagproof
\isacommand{proof}\isamarkupfalse%
{\isacharminus}\isanewline
\ \ \isacommand{have}\isamarkupfalse%
\ {\isachardoublequoteopen}{\isasymforall}s{\isachardot}\ not{\isacharunderscore}universalizable\ false{\isacharunderscore}promising\ s{\isachardoublequoteclose}\isanewline
\ \ \ \ \isacommand{by}\isamarkupfalse%
\ {\isacharparenleft}simp\ add{\isacharcolon}\ assms{\isacharparenleft}{\isadigit{1}}{\isacharparenright}\ assms{\isacharparenleft}{\isadigit{2}}{\isacharparenright}{\isacharparenright}\isanewline
%
\isamarkupcmt{I manually broke the proof into this intermediate lemma and the conclusion, and then Sledgehammer 
automatically found a proof.%
}\isanewline
\ \ \isacommand{thus}\isamarkupfalse%
\ {\isacharquery}thesis\isanewline
\ \ \ \ \isacommand{using}\isamarkupfalse%
\ FUL\ assms{\isacharparenleft}{\isadigit{3}}{\isacharparenright}\ \isacommand{by}\isamarkupfalse%
\ blast\ \isanewline
\isacommand{qed}\isamarkupfalse%
%
\endisatagproof
{\isafoldproof}%
%
\isadelimproof
%
\endisadelimproof
%
\begin{isamarkuptext}%
Under the second reading of this maxim, the act ``falsely promising" refers to uttering the 
sentence ``I promise to do X" with no intention of actually doing X\footnote{Note that under this 
reading, the maxim isn't prohibited under the logical contradiction interpretation because making an 
utterance is still possible even if eveyrone else makes that utterance. I will discuss this in detail 
later in this section in the context of the difference between natural and conventional acts.}. 
Under this reading, the practical contradiction interpretation prohibits this maxim because, in a world 
where false promising is universalized, no one believes promises anymore, so the utterance is no longer 
an effective way to get money. Below I formalize this reading of this maixm.%
\end{isamarkuptext}\isamarkuptrue%
\isacommand{consts}\isamarkupfalse%
\ believed{\isacharcolon}{\isacharcolon}os\ \isanewline
\isacommand{abbreviation}\isamarkupfalse%
\ false{\isacharunderscore}promising{\isacharunderscore}not{\isacharunderscore}believed\ \isakeyword{where}\ \isanewline
{\isachardoublequoteopen}false{\isacharunderscore}promising{\isacharunderscore}not{\isacharunderscore}believed\ {\isasymequiv}\ {\isasymforall}w\ s{\isachardot}\ {\isacharparenleft}falsely{\isacharunderscore}promise{\isacharparenleft}s{\isacharparenright}\ w\ {\isasymlongrightarrow}\ {\isasymnot}\ believed{\isacharparenleft}s{\isacharparenright}\ w{\isacharparenright}{\isachardoublequoteclose}\isanewline
%
\isamarkupcmt{This abbreviation formalizes the idea that if everyone falsely promises, then no one is believed
when promising.%
}\isanewline
\isanewline
\isacommand{abbreviation}\isamarkupfalse%
\ need{\isacharunderscore}to{\isacharunderscore}be{\isacharunderscore}believed\ \isakeyword{where}\ \isanewline
{\isachardoublequoteopen}need{\isacharunderscore}to{\isacharunderscore}be{\isacharunderscore}believed\ {\isasymequiv}\ {\isasymforall}w\ s{\isachardot}\ {\isacharparenleft}{\isasymnot}\ believed{\isacharparenleft}s{\isacharparenright}\ w\ {\isasymlongrightarrow}\ \isactrlbold {\isasymnot}{\isacharparenleft}{\isacharparenleft}falsely{\isacharunderscore}promise\ s{\isacharparenright}\ \isactrlbold {\isasymrightarrow}\ to{\isacharunderscore}get{\isacharunderscore}easy{\isacharunderscore}cash{\isacharparenright}w{\isacharparenright}{\isachardoublequoteclose}\isanewline
%
\isamarkupcmt{This abbreviation formalizes the idea that if a promise is not believed, then it is not an effective
way of getting easy cash.%
}\isanewline
\isanewline
\isacommand{lemma}\isamarkupfalse%
\ falsely{\isacharunderscore}promising{\isacharunderscore}bad{\isacharunderscore}{\isadigit{2}}{\isacharcolon}\isanewline
\ \ \isakeyword{assumes}\ false{\isacharunderscore}promising{\isacharunderscore}not{\isacharunderscore}believed\isanewline
\ \ \isakeyword{assumes}\ need{\isacharunderscore}to{\isacharunderscore}be{\isacharunderscore}believed\isanewline
%
\isamarkupcmt{The above two assumptions are specific to this reading and justified above.%
}\isanewline
\ \ \isakeyword{assumes}\ circumstances{\isacharunderscore}hold\isanewline
\ \ \isakeyword{assumes}\ example{\isacharunderscore}is{\isacharunderscore}well{\isacharunderscore}formed\isanewline
%
\isamarkupcmt{These two assumptions applied to the first reading as well and were justified there.%
}\isanewline
\ \ \isakeyword{shows}\ {\isachardoublequoteopen}{\isasymforall}s{\isachardot}\ {\isasymTurnstile}\ {\isacharparenleft}prohibited\ false{\isacharunderscore}promising\ s{\isacharparenright}{\isachardoublequoteclose}\isanewline
%
\isadelimproof
%
\endisadelimproof
%
\isatagproof
\isacommand{proof}\isamarkupfalse%
{\isacharminus}\isanewline
\ \ \isacommand{have}\isamarkupfalse%
\ {\isachardoublequoteopen}{\isasymforall}s{\isachardot}\ not{\isacharunderscore}universalizable\ false{\isacharunderscore}promising\ s{\isachardoublequoteclose}\isanewline
\ \ \ \ \isacommand{using}\isamarkupfalse%
\ assms{\isacharparenleft}{\isadigit{1}}{\isacharparenright}\ assms{\isacharparenleft}{\isadigit{2}}{\isacharparenright}\ assms{\isacharparenleft}{\isadigit{3}}{\isacharparenright}\ \isacommand{by}\isamarkupfalse%
\ auto\isanewline
\ \ \isacommand{thus}\isamarkupfalse%
\ {\isacharquery}thesis\isanewline
\ \ \ \ \isacommand{using}\isamarkupfalse%
\ FUL\ assms{\isacharparenleft}{\isadigit{4}}{\isacharparenright}\ \isacommand{by}\isamarkupfalse%
\ blast\isanewline
\isacommand{qed}\isamarkupfalse%
\isanewline
%
\isamarkupcmt{With some help, Isabelle is able to show that the maxim is prohibited under this reading as well.%
}%
\endisatagproof
{\isafoldproof}%
%
\isadelimproof
%
\endisadelimproof
%
\begin{isamarkuptext}%
This example demonstrates that my formalization is able to correctly prohibit this maxim, regardless
of its reading. This is additionally important because the two readings of this maxim represent reading 
the act as either a conventional or natural action, so my intrepretation can correctly handle both kinds
of actions. Korsgaard draws a distinction between conventional acts and natural acts. Conventional acts 
exist within a practice, which is "comprised of certain rules, and its existence (where it is not embodied in 
an institution with sanctions) consists in the general acknowledgement and following of those rules" 
\cite[10]{KorsgaardFUL}. For example, promising is a conventional act because it only exists as a 
practice. Murder, on the other hand, is an example of a natural act because its existence only depends
on the laws of nature\cite[11]{KorsgaardFUL}.

This distinction is important because Korsgaard argues that only the practical contradiction view can 
satisfactorily explain the wrongness of certain natural acts like murder\footnote{For more discussion 
of Korsgaard's argument for the practical contradiction view, see Section Philosophical Writing}. 
The practical contradiction view is thus stronger than the logical contradiction view because it can 
explain the wrongness of both conventional and natural acts. 

The fact that my interpretation can correctly show the wrongness of both conventional and natural acts
is evidence for its correctness as a formalization of the practical contradiction interpretation. 
The first reading of the example maxim reads the act 
``making a false promise" as entering into an agreement within a socially established system of promising. 
This is clearly a conventional act, and because it is a conventional act, it is not just contradictory
when universalized but literally impossible because the practice breaks down. I capture this idea in the 
assumption \isa{everyone{\isacharunderscore}can{\isacharprime}t{\isacharunderscore}lie\ {\isasymequiv}\ {\isasymforall}w{\isachardot}\ {\isasymnot}\ {\isacharparenleft}{\isasymforall}s{\isachardot}\ falsely{\isacharunderscore}promise\ s\ w{\isacharparenright}}, which states that, at all worlds, not everyone can falsely promise since 
otherwise the practice of promising would break down. The second reading, on the other hand, reads the 
act of making a false promise as uttering the statement ``I promise to pay you back," while never intending 
to fulfill this promise. This is a natural act because the act of uttering a sentence does not rely 
on any conventions, merely the laws of nature governing how your mouth and vocal cords behave\footnote{
Linguistic relativists may take issue with this claim and may argue that if the English language had 
never developed, then making this utterance would be impossible. Even if this is true, the laws of 
nature itself would not prohibit making the sounds corresponding to the English pronounciation of 
this phrase, so the act would still not be impossible in the way that a conventional act can be.} 

I show above that my formalization shows the wrongness of this maxim under both readings. Under the 
first reading, promising becomes impossible, so both the logical and 
practical contradiction interpretations prohibit the maxim. Under the second reading, promising is still
possible, but becomes ineffective because people no longer interpret the utterance as creating a commitment.
Under this view, only the practical contradiction interpretation succeeds in prohibiting the maxim. Thus, 
not only does my formalization likely capture the practical contradiction interpretation (as opposed to 
the teleological or logical contradiction interpretations), it also adequately handles both natural 
and conventional acts.%
\end{isamarkuptext}\isamarkuptrue%
%
\begin{isamarkuptext}%
I can also use Isabelle to confirm that the two readings are different. If they were the same, 
we would expect the assumptions corresponing to each to be equivalent. The RHS of the lemma below represents 
the second reading and the LHS represents the first reading.%
\end{isamarkuptext}\isamarkuptrue%
\isacommand{lemma}\isamarkupfalse%
\ readings{\isacharunderscore}are{\isacharunderscore}equivalent{\isacharcolon}\isanewline
\ \ \isakeyword{shows}\ {\isachardoublequoteopen}false{\isacharunderscore}promising{\isacharunderscore}not{\isacharunderscore}believed\ {\isasymand}\ need{\isacharunderscore}to{\isacharunderscore}be{\isacharunderscore}believed\ {\isasymequiv}\ everyone{\isacharunderscore}can{\isacharprime}t{\isacharunderscore}lie{\isachardoublequoteclose}\isanewline
\ \ \isacommand{nitpick}\isamarkupfalse%
{\isacharbrackleft}user{\isacharunderscore}axioms{\isacharbrackright}%
\isadelimproof
\ %
\endisadelimproof
%
\isatagproof
\isacommand{oops}\isamarkupfalse%
\isanewline
%
\isamarkupcmt{Nitpick finds a counterexample, showing that the two readings are different.
\color{blue}
Nitpick found a counterexample for card i = 1 and card s = 1:

  Empty assignment
\color{black}%
}%
\endisatagproof
{\isafoldproof}%
%
\isadelimproof
%
\endisadelimproof
%
\begin{isamarkuptext}%
This completes the application tests for my formalization. I showed that my formalization correctly 
handles an example from Korsgaard with two possible interpretations and also sufficiently handles both 
conventional and natural acts.%
\end{isamarkuptext}\isamarkuptrue%
%
\isadelimdocument
%
\endisadelimdocument
%
\isatagdocument
%
\isamarkupsubsection{Metaethical Tests%
}
\isamarkuptrue%
%
\endisatagdocument
{\isafolddocument}%
%
\isadelimdocument
%
\endisadelimdocument
%
\begin{isamarkuptext}%
Recall that metaethical tests test formal properties of the system that apply to any maxim, not 
just those specified in the application tests. In this section I adapt the metaethical tests developed 
in previous sections to my formalization of the categorical imperative. I preserved the philosophical 
goal of each test but modified them to test the stronger, richer notion of a maxim.

The first set of tests consider how obligation generalizes, first across worlds and then across
people. As expected, the tests below show that both wrongness (prohibition) and rightness (obligation)
generalize across both worlds and people. In other words, if something is obligated at some world, it 
is obligated at every world and if something is obligated for some person, then it is obligated for 
every person. 

Generalization across worlds is a consequence of the fact that my interpretation does not make use of the 
modal nature of DDL. In particular, I do not use any property of the world when prescribing obligations 
at that world.%
\end{isamarkuptext}\isamarkuptrue%
\isacommand{lemma}\isamarkupfalse%
\ wrong{\isacharunderscore}if{\isacharunderscore}wrong{\isacharunderscore}for{\isacharunderscore}someone{\isacharcolon}\isanewline
\ \ \isakeyword{shows}\ {\isachardoublequoteopen}{\isasymforall}w{\isachardot}\ {\isasymforall}c{\isacharcolon}{\isacharcolon}t{\isachardot}\ {\isasymforall}g{\isacharcolon}{\isacharcolon}t{\isachardot}\ {\isasymexists}s{\isacharcolon}{\isacharcolon}s{\isachardot}\ O{\isacharbraceleft}\isactrlbold {\isasymnot}\ {\isacharparenleft}W\ {\isacharparenleft}c{\isacharcomma}\ M{\isacharcomma}\ g{\isacharparenright}\ s{\isacharparenright}\ {\isacharbar}\ c{\isacharbraceright}\ w\ {\isasymlongrightarrow}\ {\isacharparenleft}{\isasymforall}p{\isachardot}\ \ O{\isacharbraceleft}\isactrlbold {\isasymnot}\ {\isacharparenleft}W\ {\isacharparenleft}c{\isacharcomma}\ M{\isacharcomma}\ g{\isacharparenright}\ p{\isacharparenright}\ {\isacharbar}\ c{\isacharbraceright}\ w{\isacharparenright}\ {\isachardoublequoteclose}\isanewline
%
\isadelimproof
\ \ %
\endisadelimproof
%
\isatagproof
\isacommand{by}\isamarkupfalse%
\ blast%
\endisatagproof
{\isafoldproof}%
%
\isadelimproof
\isanewline
%
\endisadelimproof
\isanewline
\isacommand{lemma}\isamarkupfalse%
\ right{\isacharunderscore}if{\isacharunderscore}right{\isacharunderscore}for{\isacharunderscore}someone{\isacharcolon}\isanewline
\ \ \isakeyword{shows}\ {\isachardoublequoteopen}{\isasymforall}w{\isachardot}\ {\isasymforall}c{\isacharcolon}{\isacharcolon}t{\isachardot}\ {\isasymforall}g{\isacharcolon}{\isacharcolon}t{\isachardot}\ {\isasymexists}s{\isacharcolon}{\isacharcolon}s{\isachardot}\ O{\isacharbraceleft}W\ {\isacharparenleft}c{\isacharcomma}\ M{\isacharcomma}\ g{\isacharparenright}\ s\ {\isacharbar}\ c{\isacharbraceright}\ w\ {\isasymlongrightarrow}\ {\isacharparenleft}{\isasymforall}p{\isachardot}\ \ O{\isacharbraceleft}W\ {\isacharparenleft}c{\isacharcomma}\ M{\isacharcomma}\ g{\isacharparenright}\ p\ {\isacharbar}\ c{\isacharbraceright}\ w{\isacharparenright}\ {\isachardoublequoteclose}\isanewline
%
\isadelimproof
\ \ %
\endisadelimproof
%
\isatagproof
\isacommand{by}\isamarkupfalse%
\ blast%
\endisatagproof
{\isafoldproof}%
%
\isadelimproof
\isanewline
%
\endisadelimproof
\isanewline
\isacommand{lemma}\isamarkupfalse%
\ wrong{\isacharunderscore}if{\isacharunderscore}wrong{\isacharunderscore}somewhere{\isacharcolon}\isanewline
\ \ \isakeyword{shows}\ {\isachardoublequoteopen}{\isasymforall}c\ g{\isachardot}\ {\isasymexists}w{\isadigit{1}}{\isachardot}\ O{\isacharbraceleft}\isactrlbold {\isasymnot}\ {\isacharparenleft}W\ {\isacharparenleft}c{\isacharcomma}\ M{\isacharcomma}\ g{\isacharparenright}\ s{\isacharparenright}\ {\isacharbar}\ c{\isacharbraceright}\ w{\isadigit{1}}\ {\isasymlongrightarrow}\ {\isacharparenleft}{\isasymforall}w{\isadigit{2}}{\isachardot}\ \ O{\isacharbraceleft}\isactrlbold {\isasymnot}\ {\isacharparenleft}W\ {\isacharparenleft}c{\isacharcomma}\ M{\isacharcomma}\ g{\isacharparenright}\ s{\isacharparenright}\ {\isacharbar}\ c{\isacharbraceright}\ w{\isadigit{2}}{\isacharparenright}{\isachardoublequoteclose}\isanewline
%
\isadelimproof
\ \ %
\endisadelimproof
%
\isatagproof
\isacommand{by}\isamarkupfalse%
\ blast%
\endisatagproof
{\isafoldproof}%
%
\isadelimproof
\isanewline
%
\endisadelimproof
\isanewline
\isacommand{lemma}\isamarkupfalse%
\ right{\isacharunderscore}if{\isacharunderscore}right{\isacharunderscore}somewhere{\isacharcolon}\isanewline
\ \ \isakeyword{shows}\ {\isachardoublequoteopen}{\isasymforall}c\ g{\isachardot}\ {\isasymexists}w{\isadigit{1}}{\isachardot}\ O{\isacharbraceleft}W\ {\isacharparenleft}c{\isacharcomma}\ M{\isacharcomma}\ g{\isacharparenright}\ s\ {\isacharbar}\ c{\isacharbraceright}\ w{\isadigit{1}}\ {\isasymlongrightarrow}\ {\isacharparenleft}{\isasymforall}w{\isadigit{2}}{\isachardot}\ \ O{\isacharbraceleft}W\ {\isacharparenleft}c{\isacharcomma}\ M{\isacharcomma}\ g{\isacharparenright}\ s\ {\isacharbar}\ c{\isacharbraceright}\ w{\isadigit{2}}{\isacharparenright}{\isachardoublequoteclose}\isanewline
%
\isadelimproof
\ \ %
\endisadelimproof
%
\isatagproof
\isacommand{by}\isamarkupfalse%
\ blast%
\endisatagproof
{\isafoldproof}%
%
\isadelimproof
%
\endisadelimproof
%
\begin{isamarkuptext}%
As expected, obligation generalizes across people and worlds. In the next set of tests, I will 
analyze basic properties of permissibility, obligation, and prohibition. 

First, I verify that the logic allows
for permissible maxims, as this is a problem that prior iterations ran into. Below, I use Nitpick to 
find a model in which there is a circumstance, act, goal tuple that is permissible but not 
obligated at some world.%
\end{isamarkuptext}\isamarkuptrue%
\isacommand{lemma}\isamarkupfalse%
\ permissible{\isacharcolon}\isanewline
\ \ \isakeyword{shows}\ {\isachardoublequoteopen}{\isacharparenleft}{\isacharparenleft}\isactrlbold {\isasymnot}\ {\isacharparenleft}O{\isacharbraceleft}{\isacharparenleft}W\ {\isacharparenleft}c{\isacharcomma}\ a{\isacharcomma}\ g{\isacharparenright}\ s{\isacharparenright}\ {\isacharbar}\ c{\isacharbraceright}{\isacharparenright}{\isacharparenright}\ \isactrlbold {\isasymand}\ {\isacharparenleft}\isactrlbold {\isasymnot}\ {\isacharparenleft}O{\isacharbraceleft}\isactrlbold {\isasymnot}\ {\isacharparenleft}W\ {\isacharparenleft}c{\isacharcomma}\ a{\isacharcomma}\ g{\isacharparenright}\ s{\isacharparenright}\ {\isacharbar}\ c{\isacharbraceright}{\isacharparenright}{\isacharparenright}{\isacharparenright}\ w{\isachardoublequoteclose}\isanewline
\ \ \isacommand{nitpick}\isamarkupfalse%
\ {\isacharbrackleft}user{\isacharunderscore}axioms{\isacharcomma}\ falsify{\isacharequal}false{\isacharbrackright}%
\isadelimproof
\ %
\endisadelimproof
%
\isatagproof
\isacommand{oops}\isamarkupfalse%
\isanewline
%
\isamarkupcmt{\color{blue}Nitpick found a model for card i = 1 and card s = 1:

  Free variables:
    a = ($\lambda x. \_$)($s_1$ := ($\lambda x. \_$)($i_1$ := False))
    c = ($\lambda x. \_$)($i_1$ := False)
    g = ($\lambda x. \_$)($i_1$ := False)
    s = $s_1$\color{black}
Recall that Nitpick is a model checker that finds models making certain formulae true or false. In this 
case, Nitpick finds a model satisfying the given formula (which simply requires that the sentence 
``s wills (c, a, g)'' is permissible but not obligator). This model consists of the above specifications 
of a, c, g, and s.%
}%
\endisatagproof
{\isafoldproof}%
%
\isadelimproof
%
\endisadelimproof
%
\begin{isamarkuptext}%
I also expect that any arbitrary maxim should be either permissible or prohibited, since all 
acts are either permissible or prohibited.%
\end{isamarkuptext}\isamarkuptrue%
\isacommand{lemma}\isamarkupfalse%
\ perm{\isacharunderscore}or{\isacharunderscore}prohibited{\isacharcolon}\isanewline
\ \ \isakeyword{shows}\ {\isachardoublequoteopen}{\isacharparenleft}{\isacharparenleft}permissible\ {\isacharparenleft}c{\isacharcomma}\ a{\isacharcomma}\ g{\isacharparenright}\ s{\isacharparenright}\ \isactrlbold {\isasymor}\ {\isacharparenleft}prohibited\ {\isacharparenleft}c{\isacharcomma}\ a{\isacharcomma}\ g{\isacharparenright}\ s{\isacharparenright}{\isacharparenright}\ w{\isachardoublequoteclose}\isanewline
%
\isadelimproof
\ \ %
\endisadelimproof
%
\isatagproof
\isacommand{by}\isamarkupfalse%
\ blast\isanewline
%
\isamarkupcmt{This simple test passes immediately by the definitions of  permissible and prohibited.%
}%
\endisatagproof
{\isafoldproof}%
%
\isadelimproof
%
\endisadelimproof
%
\begin{isamarkuptext}%
Obligation should be strictly stronger than permissibility. In other words, if a maxim is 
obligated at a world, it should be permissible at that world. Below I test this property.%
\end{isamarkuptext}\isamarkuptrue%
\isacommand{lemma}\isamarkupfalse%
\ obligated{\isacharunderscore}then{\isacharunderscore}permissible{\isacharcolon}\isanewline
\ \ \isakeyword{shows}\ {\isachardoublequoteopen}{\isacharparenleft}O{\isacharbraceleft}W{\isacharparenleft}c{\isacharcomma}\ a{\isacharcomma}\ g{\isacharparenright}\ s{\isacharbar}c{\isacharbraceright}\ \isactrlbold {\isasymrightarrow}\ {\isacharparenleft}{\isacharparenleft}permissible\ {\isacharparenleft}c{\isacharcomma}\ a{\isacharcomma}\ g{\isacharparenright}\ {\isacharparenright}\ s{\isacharparenright}{\isacharparenright}\ w\ {\isachardoublequoteclose}\isanewline
%
\isadelimproof
\ \ %
\endisadelimproof
%
\isatagproof
\isacommand{using}\isamarkupfalse%
\ no{\isacharunderscore}contradictions\ \isacommand{by}\isamarkupfalse%
\ auto\isanewline
%
\isamarkupcmt{This test passes and Isabelle is able to find a proof for the fact that all obligatory maxims are 
also permissible.%
}%
\endisatagproof
{\isafoldproof}%
%
\isadelimproof
%
\endisadelimproof
%
\begin{isamarkuptext}%
The above test failed under Kroy's formalization of the categorical imperative and is thus evidence 
that my formalization improves upon Kroy's. Interestingly, this new test passes because of the additional
added axiom that prohibits contradictory obligations (recall that Kroy's formalization allowed contradictory
obligations).%
\end{isamarkuptext}\isamarkuptrue%
%
\begin{isamarkuptext}%
Next, I will test if the formalization allows for vacuous obligations or modal collapse. These 
tests are sanity checks confirmed that the obligation operator is doesn't collapse. First, I will check 
that any arbitrary term isn't obligated.%
\end{isamarkuptext}\isamarkuptrue%
\isacommand{lemma}\isamarkupfalse%
\ arbitrary{\isacharunderscore}obligations{\isacharcolon}\isanewline
\ \ \isakeyword{fixes}\ c\ A{\isacharcolon}{\isacharcolon}{\isachardoublequoteopen}t{\isachardoublequoteclose}\isanewline
\ \ \isakeyword{shows}\ {\isachardoublequoteopen}O{\isacharbraceleft}A{\isacharbar}c{\isacharbraceright}\ w{\isachardoublequoteclose}\isanewline
\ \ \isacommand{nitpick}\isamarkupfalse%
\ {\isacharbrackleft}user{\isacharunderscore}axioms{\isacharequal}true{\isacharcomma}\ falsify{\isacharbrackright}%
\isadelimproof
\ %
\endisadelimproof
%
\isatagproof
\isacommand{oops}\isamarkupfalse%
\isanewline
%
\isamarkupcmt{This test passes—Nitpick finds a model where A isn't obligated in circumstances c.
\color{blue} Nitpick found a counterexample for card i = 1 and card s = 2:

  Free variables:
    A = ($\lambda x. \_$)($i_1$ := True)
    c = ($\lambda x. \_$)($i_1$ := False) \color{blue}
Previous iterations of this test used the monadic obligation operator, which tests the term in the 
context ``True" (equivalently the set of all worlds since True holds everywhere). In this iteration, 
I test the term in a context c, because my formalization uses the dyadic obligation operator and must 
thus specify circumstances.%
}%
\endisatagproof
{\isafoldproof}%
%
\isadelimproof
%
\endisadelimproof
%
\begin{isamarkuptext}%
This is exactly the expected result. Any arbitrary action $A$ isn't obligated. A slightly 
        stronger property is ``modal collapse," or whether or not `$A$ happens' implies `$A$ is obligated'. 
The proposition below should be falsifiable.%
\end{isamarkuptext}\isamarkuptrue%
\isacommand{lemma}\isamarkupfalse%
\ modal{\isacharunderscore}collapse{\isacharcolon}\isanewline
\ \ \isakeyword{shows}\ {\isachardoublequoteopen}{\isacharparenleft}{\isacharparenleft}W\ {\isacharparenleft}c{\isacharcomma}\ a{\isacharcomma}\ g{\isacharparenright}\ s{\isacharparenright}\ w{\isacharparenright}\ {\isasymlongrightarrow}\ O{\isacharbraceleft}W\ {\isacharparenleft}c{\isacharcomma}\ a{\isacharcomma}\ g{\isacharparenright}\ s{\isacharbar}c{\isacharbraceright}\ w{\isachardoublequoteclose}\isanewline
\ \ \isacommand{nitpick}\isamarkupfalse%
\ {\isacharbrackleft}user{\isacharunderscore}axioms{\isacharequal}true{\isacharcomma}\ falsify{\isacharbrackright}%
\isadelimproof
\ %
\endisadelimproof
%
\isatagproof
\isacommand{oops}\isamarkupfalse%
\isanewline
%
\isamarkupcmt{Nitpick finds a counterexample, so willing doesn't imply obligation, so this test passes. 
\color{blue}Nitpick found a counterexample for card i = 1 and card s = 2:

  Free variables:
    a = ($\lambda x. \_$)($s_1$ := ($\lambda x. \_$)($i_1$ := False), $s_2$ := ($\lambda x. \_$)($i_1$ := False))
    c = ($\lambda x. \_$)($i_1$ := False)
    g = ($\lambda x. \_$)($i_1$ := False)
    s = $s_2$
    w = $i_1$\color{black}
Once again, I modify this test to use the dyadic obligation operator instead of the monadic operator.%
}%
\endisatagproof
{\isafoldproof}%
%
\isadelimproof
%
\endisadelimproof
%
\begin{isamarkuptext}%
The final set of tests deal with ought implies can and conflicting obligations. Recall that I 
specifically added an axiom in my formalization to disallow contradictory obligations, so I expect 
these tests to pass. Kroy's formalization fails these tests, so this is another area of improvement 
over Kroy's formalization.%
\end{isamarkuptext}\isamarkuptrue%
\isacommand{lemma}\isamarkupfalse%
\ ought{\isacharunderscore}implies{\isacharunderscore}can{\isacharcolon}\isanewline
\ \ \isakeyword{shows}\ {\isachardoublequoteopen}O{\isacharbraceleft}W\ {\isacharparenleft}c{\isacharcomma}\ a{\isacharcomma}\ g{\isacharparenright}\ s{\isacharbar}c{\isacharbraceright}\ w\ {\isasymlongrightarrow}\ {\isacharparenleft}{\isasymdiamond}\ W\ {\isacharparenleft}c{\isacharcomma}\ a{\isacharcomma}\ g{\isacharparenright}\ s{\isacharparenright}\ w{\isachardoublequoteclose}\isanewline
%
\isadelimproof
\ \ %
\endisadelimproof
%
\isatagproof
\isacommand{using}\isamarkupfalse%
\ O{\isacharunderscore}diamond\ \isacommand{by}\isamarkupfalse%
\ blast\isanewline
%
\isamarkupcmt{This test is a lemma of DDL itself, so it's no surprise that this test passes.%
}%
\endisatagproof
{\isafoldproof}%
%
\isadelimproof
\isanewline
%
\endisadelimproof
\isanewline
\isacommand{lemma}\isamarkupfalse%
\ conflicting{\isacharunderscore}obligations{\isacharcolon}\isanewline
\ \ \isakeyword{shows}\ {\isachardoublequoteopen}{\isasymnot}\ {\isacharparenleft}O{\isacharbraceleft}W\ {\isacharparenleft}c{\isacharcomma}\ a{\isacharcomma}\ g{\isacharparenright}\ s{\isacharbar}c{\isacharbraceright}\ \isactrlbold {\isasymand}\ O{\isacharbraceleft}\isactrlbold {\isasymnot}{\isacharparenleft}W\ {\isacharparenleft}c{\isacharcomma}\ a{\isacharcomma}\ g{\isacharparenright}\ s{\isacharparenright}{\isacharbar}\ c{\isacharbraceright}{\isacharparenright}\ w{\isachardoublequoteclose}\isanewline
%
\isadelimproof
\ \ %
\endisadelimproof
%
\isatagproof
\isacommand{using}\isamarkupfalse%
\ no{\isacharunderscore}contradictions\ \isacommand{by}\isamarkupfalse%
\ blast\isanewline
%
\isamarkupcmt{This test passes immediately by the new axiom prohibited contradictory obligations.%
}%
\endisatagproof
{\isafoldproof}%
%
\isadelimproof
\isanewline
%
\endisadelimproof
\isanewline
\isacommand{lemma}\isamarkupfalse%
\ implied{\isacharunderscore}contradiction{\isacharcolon}\isanewline
\ \ \isakeyword{assumes}\ {\isachardoublequoteopen}{\isacharparenleft}{\isacharparenleft}{\isacharparenleft}W\ {\isacharparenleft}c{\isadigit{1}}{\isacharcomma}\ a{\isadigit{1}}{\isacharcomma}\ g{\isadigit{1}}{\isacharparenright}\ s{\isacharparenright}\ \isactrlbold {\isasymand}\ {\isacharparenleft}W\ {\isacharparenleft}c{\isadigit{2}}{\isacharcomma}\ a{\isadigit{2}}{\isacharcomma}\ g{\isadigit{2}}{\isacharparenright}\ s{\isacharparenright}{\isacharparenright}\ \isactrlbold {\isasymrightarrow}\ \isactrlbold {\isasymbottom}{\isacharparenright}\ w{\isachardoublequoteclose}\isanewline
\ \ \isakeyword{shows}\ {\isachardoublequoteopen}\isactrlbold {\isasymnot}\ {\isacharparenleft}O{\isacharbraceleft}W{\isacharparenleft}c{\isadigit{1}}{\isacharcomma}\ a{\isadigit{1}}{\isacharcomma}\ g{\isadigit{1}}{\isacharparenright}\ s{\isacharbar}c{\isacharbraceright}\ \isactrlbold {\isasymand}\ O{\isacharbraceleft}W{\isacharparenleft}c{\isadigit{2}}{\isacharcomma}\ a{\isadigit{2}}{\isacharcomma}\ g{\isadigit{2}}{\isacharparenright}\ s{\isacharbar}c{\isacharbraceright}{\isacharparenright}\ w{\isachardoublequoteclose}\isanewline
%
\isadelimproof
\ \ %
\endisadelimproof
%
\isatagproof
\isacommand{using}\isamarkupfalse%
\ assms\ no{\isacharunderscore}contradictions\ \isacommand{by}\isamarkupfalse%
\ blast\isanewline
%
\isamarkupcmt{Recall that the we also expect the stronger property that the combination of obligatory maxims can't
 imply a contradiction. The added axiom also makes this test pass.%
}%
\endisatagproof
{\isafoldproof}%
%
\isadelimproof
\isanewline
%
\endisadelimproof
\isanewline
\isacommand{lemma}\isamarkupfalse%
\ distribution{\isacharcolon}\isanewline
\ \ \isakeyword{assumes}\ {\isachardoublequoteopen}{\isasymTurnstile}\ {\isacharparenleft}O\ {\isacharbraceleft}A{\isacharbraceright}\ \isactrlbold {\isasymand}\ O\ {\isacharbraceleft}B{\isacharbraceright}{\isacharparenright}{\isachardoublequoteclose}\isanewline
\ \ \isakeyword{shows}\ {\isachardoublequoteopen}{\isasymTurnstile}\ O\ {\isacharbraceleft}A\ \isactrlbold {\isasymand}\ B{\isacharbraceright}{\isachardoublequoteclose}\isanewline
%
\isadelimproof
\ \ %
\endisadelimproof
%
\isatagproof
\isacommand{using}\isamarkupfalse%
\ assms\ no{\isacharunderscore}contradictions\ \isacommand{by}\isamarkupfalse%
\ fastforce%
\endisatagproof
{\isafoldproof}%
%
\isadelimproof
%
\endisadelimproof
%
\begin{isamarkuptext}%
The metaethical test suite ran on both Kroy's formalization and my formalizaion show two clear 
improvements. First, my formalization shows that obligatory maxims are permissible, whereas Kroy's 
paradoxically does not. Second, my formalization doesn't allow contradictory maxims, but Kroy's does. 
Both of these improvements are derived from the new axiom I added in my formalization that disallows 
contradictory obligations. Additionally, my formalization also improves on Kroy's by staying faithful to the 
strongest interpretation of the FUL, Korsgaard's practical contradiction interpretation. (maybe stick 
philosophical writing here or above?)%
\end{isamarkuptext}\isamarkuptrue%
%
\isadelimdocument
%
\endisadelimdocument
%
\isatagdocument
%
\isamarkupsubsection{Formalization Specific Tests%
}
\isamarkuptrue%
%
\endisatagdocument
{\isafolddocument}%
%
\isadelimdocument
%
\endisadelimdocument
%
\begin{isamarkuptext}%
In this section, I explore tests specific to my formalization of the categorical imperative. First, 
in my previous (buggy) implementation of DDL, prohibiting contradictory obligation led to the strange 
result that all permissible actions are obligatory. I will test if this bug appears in this implementation 
as well.%
\end{isamarkuptext}\isamarkuptrue%
\isacommand{lemma}\isamarkupfalse%
\ bug{\isacharcolon}\isanewline
\ \ \isakeyword{shows}\ {\isachardoublequoteopen}permissible\ {\isacharparenleft}c{\isacharcomma}\ a{\isacharcomma}\ g{\isacharparenright}\ s\ w\ {\isasymlongrightarrow}\ O{\isacharbraceleft}W{\isacharparenleft}c{\isacharcomma}\ a{\isacharcomma}\ g{\isacharparenright}\ s\ {\isacharbar}\ c{\isacharbraceright}\ w{\isachardoublequoteclose}\isanewline
\ \ \isacommand{nitpick}\isamarkupfalse%
{\isacharbrackleft}user{\isacharunderscore}axioms{\isacharbrackright}%
\isadelimproof
\ %
\endisadelimproof
%
\isatagproof
\isacommand{oops}\isamarkupfalse%
\isanewline
%
\isamarkupcmt{\color{blue}
Nitpick found a counterexample for card i = 1 and card s = 2:

  Free variables:
    a = ($\lambda x. \_$)($s_1$ := ($\lambda x. \_$)($i_1$ := False), $s_2$ := ($\lambda x. \_$)($i_1$ := False))
    c = ($\lambda x. \_$)($i_1$ := False)
    g = ($\lambda x. \_$)($i_1$ := False)
    s = $s_2$
    w = undefined
\color{black}
This strange result does not hold; good!%
}\isanewline
%
\endisatagproof
{\isafoldproof}%
%
\isadelimproof
%
\endisadelimproof
%
\isadelimtheory
%
\endisadelimtheory
%
\isatagtheory
%
\endisatagtheory
{\isafoldtheory}%
%
\isadelimtheory
%
\endisadelimtheory
%
\end{isabellebody}%
\endinput
%:%file=~/Desktop/cs91r/paper/paper41.thy%:%
%:%24=6%:%
%:%36=8%:%
%:%37=9%:%
%:%46=11%:%
%:%58=13%:%
%:%59=14%:%
%:%60=15%:%
%:%61=16%:%
%:%62=17%:%
%:%63=18%:%
%:%65=20%:%
%:%66=20%:%
%:%67=20%:%
%:%68=21%:%
%:%69=22%:%
%:%70=23%:%
%:%71=23%:%
%:%72=24%:%
%:%73=25%:%
%:%74=25%:%
%:%75=25%:%
%:%76=26%:%
%:%77=26%:%
%:%78=27%:%
%:%79=28%:%
%:%80=28%:%
%:%82=30%:%
%:%83=31%:%
%:%84=32%:%
%:%85=33%:%
%:%86=34%:%
%:%87=35%:%
%:%88=36%:%
%:%89=37%:%
%:%90=38%:%
%:%91=39%:%
%:%92=40%:%
%:%93=41%:%
%:%94=42%:%
%:%95=43%:%
%:%96=44%:%
%:%97=45%:%
%:%98=46%:%
%:%99=47%:%
%:%100=48%:%
%:%101=49%:%
%:%102=50%:%
%:%103=51%:%
%:%104=52%:%
%:%105=53%:%
%:%106=54%:%
%:%107=55%:%
%:%108=56%:%
%:%109=57%:%
%:%110=58%:%
%:%111=59%:%
%:%112=60%:%
%:%113=61%:%
%:%114=62%:%
%:%115=63%:%
%:%116=64%:%
%:%117=65%:%
%:%118=66%:%
%:%119=67%:%
%:%120=68%:%
%:%121=69%:%
%:%122=70%:%
%:%123=71%:%
%:%124=72%:%
%:%125=73%:%
%:%126=74%:%
%:%127=75%:%
%:%128=76%:%
%:%129=77%:%
%:%130=78%:%
%:%131=79%:%
%:%132=80%:%
%:%133=81%:%
%:%134=82%:%
%:%135=83%:%
%:%136=84%:%
%:%137=85%:%
%:%138=86%:%
%:%139=87%:%
%:%140=88%:%
%:%141=89%:%
%:%142=90%:%
%:%143=91%:%
%:%144=92%:%
%:%145=93%:%
%:%146=94%:%
%:%147=95%:%
%:%148=96%:%
%:%149=97%:%
%:%150=98%:%
%:%151=99%:%
%:%152=100%:%
%:%153=101%:%
%:%154=102%:%
%:%155=103%:%
%:%156=104%:%
%:%157=105%:%
%:%158=106%:%
%:%159=107%:%
%:%160=108%:%
%:%161=109%:%
%:%162=110%:%
%:%163=111%:%
%:%164=112%:%
%:%165=113%:%
%:%166=114%:%
%:%167=115%:%
%:%168=116%:%
%:%169=117%:%
%:%170=118%:%
%:%171=119%:%
%:%172=120%:%
%:%173=121%:%
%:%174=122%:%
%:%175=123%:%
%:%176=124%:%
%:%177=125%:%
%:%178=126%:%
%:%179=127%:%
%:%180=128%:%
%:%181=129%:%
%:%182=130%:%
%:%183=131%:%
%:%184=132%:%
%:%186=134%:%
%:%187=134%:%
%:%188=135%:%
%:%190=137%:%
%:%191=138%:%
%:%192=139%:%
%:%193=140%:%
%:%194=141%:%
%:%195=142%:%
%:%196=143%:%
%:%197=144%:%
%:%198=145%:%
%:%199=146%:%
%:%200=147%:%
%:%201=148%:%
%:%202=149%:%
%:%203=150%:%
%:%204=151%:%
%:%205=152%:%
%:%206=153%:%
%:%207=154%:%
%:%208=155%:%
%:%209=156%:%
%:%210=157%:%
%:%211=158%:%
%:%212=159%:%
%:%213=160%:%
%:%214=161%:%
%:%215=162%:%
%:%216=163%:%
%:%217=164%:%
%:%218=165%:%
%:%219=166%:%
%:%220=167%:%
%:%221=168%:%
%:%222=169%:%
%:%223=170%:%
%:%224=171%:%
%:%225=172%:%
%:%226=173%:%
%:%227=174%:%
%:%228=175%:%
%:%229=176%:%
%:%230=177%:%
%:%231=178%:%
%:%232=179%:%
%:%233=180%:%
%:%234=181%:%
%:%236=184%:%
%:%237=184%:%
%:%238=185%:%
%:%239=186%:%
%:%242=188%:%
%:%243=189%:%
%:%244=190%:%
%:%245=191%:%
%:%246=192%:%
%:%247=193%:%
%:%248=194%:%
%:%249=195%:%
%:%250=196%:%
%:%251=197%:%
%:%252=198%:%
%:%253=199%:%
%:%254=200%:%
%:%255=201%:%
%:%256=202%:%
%:%257=203%:%
%:%258=204%:%
%:%259=205%:%
%:%260=206%:%
%:%261=207%:%
%:%262=208%:%
%:%264=210%:%
%:%265=210%:%
%:%266=211%:%
%:%267=212%:%
%:%268=213%:%
%:%269=213%:%
%:%270=214%:%
%:%271=215%:%
%:%272=216%:%
%:%273=216%:%
%:%274=217%:%
%:%276=218%:%
%:%277=219%:%
%:%278=220%:%
%:%279=221%:%
%:%280=222%:%
%:%281=223%:%
%:%282=224%:%
%:%283=225%:%
%:%286=227%:%
%:%287=228%:%
%:%288=229%:%
%:%289=230%:%
%:%290=231%:%
%:%291=232%:%
%:%292=233%:%
%:%293=234%:%
%:%295=236%:%
%:%296=236%:%
%:%297=237%:%
%:%298=238%:%
%:%299=239%:%
%:%300=239%:%
%:%301=240%:%
%:%303=241%:%
%:%304=242%:%
%:%307=245%:%
%:%308=246%:%
%:%309=247%:%
%:%310=248%:%
%:%312=250%:%
%:%313=250%:%
%:%314=251%:%
%:%316=252%:%
%:%317=253%:%
%:%318=253%:%
%:%319=254%:%
%:%320=255%:%
%:%321=255%:%
%:%323=256%:%
%:%324=257%:%
%:%325=258%:%
%:%333=260%:%
%:%345=262%:%
%:%346=263%:%
%:%347=264%:%
%:%348=265%:%
%:%349=266%:%
%:%350=267%:%
%:%352=270%:%
%:%353=270%:%
%:%355=271%:%
%:%356=272%:%
%:%357=273%:%
%:%358=274%:%
%:%359=274%:%
%:%360=275%:%
%:%361=276%:%
%:%362=276%:%
%:%364=276%:%
%:%368=276%:%
%:%369=276%:%
%:%370=277%:%
%:%371=277%:%
%:%372=277%:%
%:%381=279%:%
%:%382=280%:%
%:%383=281%:%
%:%384=282%:%
%:%385=283%:%
%:%386=284%:%
%:%387=285%:%
%:%388=286%:%
%:%389=287%:%
%:%390=288%:%
%:%391=289%:%
%:%392=290%:%
%:%393=291%:%
%:%394=292%:%
%:%395=293%:%
%:%396=294%:%
%:%397=295%:%
%:%399=297%:%
%:%400=297%:%
%:%401=298%:%
%:%403=299%:%
%:%404=300%:%
%:%405=300%:%
%:%406=301%:%
%:%407=302%:%
%:%408=302%:%
%:%409=303%:%
%:%411=304%:%
%:%412=305%:%
%:%413=305%:%
%:%414=306%:%
%:%415=307%:%
%:%416=307%:%
%:%417=308%:%
%:%418=308%:%
%:%420=308%:%
%:%424=308%:%
%:%425=308%:%
%:%427=309%:%
%:%428=310%:%
%:%429=311%:%
%:%430=312%:%
%:%431=313%:%
%:%432=314%:%
%:%433=315%:%
%:%434=316%:%
%:%435=317%:%
%:%436=318%:%
%:%437=319%:%
%:%445=319%:%
%:%446=320%:%
%:%447=321%:%
%:%448=321%:%
%:%449=322%:%
%:%450=323%:%
%:%451=323%:%
%:%452=324%:%
%:%453=324%:%
%:%455=324%:%
%:%459=324%:%
%:%460=324%:%
%:%462=325%:%
%:%463=326%:%
%:%464=327%:%
%:%465=328%:%
%:%466=329%:%
%:%467=330%:%
%:%477=332%:%
%:%478=333%:%
%:%479=334%:%
%:%480=335%:%
%:%481=336%:%
%:%482=337%:%
%:%483=338%:%
%:%484=339%:%
%:%485=340%:%
%:%486=341%:%
%:%487=342%:%
%:%488=343%:%
%:%489=344%:%
%:%490=345%:%
%:%491=346%:%
%:%492=347%:%
%:%493=348%:%
%:%494=349%:%
%:%495=350%:%
%:%496=351%:%
%:%500=353%:%
%:%501=354%:%
%:%502=355%:%
%:%503=356%:%
%:%504=357%:%
%:%513=360%:%
%:%525=362%:%
%:%526=363%:%
%:%527=364%:%
%:%528=365%:%
%:%532=367%:%
%:%533=368%:%
%:%534=369%:%
%:%535=370%:%
%:%536=371%:%
%:%537=372%:%
%:%538=373%:%
%:%539=374%:%
%:%540=375%:%
%:%541=376%:%
%:%542=377%:%
%:%543=378%:%
%:%544=379%:%
%:%545=380%:%
%:%546=381%:%
%:%547=382%:%
%:%549=384%:%
%:%550=384%:%
%:%552=385%:%
%:%553=386%:%
%:%554=386%:%
%:%555=387%:%
%:%556=387%:%
%:%558=388%:%
%:%559=389%:%
%:%560=389%:%
%:%561=390%:%
%:%562=390%:%
%:%564=391%:%
%:%565=392%:%
%:%566=392%:%
%:%567=393%:%
%:%568=394%:%
%:%569=394%:%
%:%570=395%:%
%:%572=396%:%
%:%575=398%:%
%:%576=399%:%
%:%577=400%:%
%:%579=402%:%
%:%580=402%:%
%:%581=403%:%
%:%583=404%:%
%:%584=405%:%
%:%585=406%:%
%:%586=407%:%
%:%587=408%:%
%:%588=408%:%
%:%589=409%:%
%:%590=410%:%
%:%591=410%:%
%:%592=411%:%
%:%594=412%:%
%:%595=413%:%
%:%596=414%:%
%:%597=415%:%
%:%598=416%:%
%:%599=416%:%
%:%600=417%:%
%:%601=418%:%
%:%602=418%:%
%:%603=419%:%
%:%605=420%:%
%:%606=421%:%
%:%607=422%:%
%:%608=423%:%
%:%609=424%:%
%:%612=426%:%
%:%613=427%:%
%:%615=429%:%
%:%616=429%:%
%:%617=430%:%
%:%618=431%:%
%:%619=432%:%
%:%620=433%:%
%:%627=434%:%
%:%628=434%:%
%:%629=435%:%
%:%630=435%:%
%:%631=436%:%
%:%632=436%:%
%:%634=437%:%
%:%635=438%:%
%:%636=438%:%
%:%637=439%:%
%:%638=439%:%
%:%639=440%:%
%:%640=440%:%
%:%641=440%:%
%:%642=441%:%
%:%652=443%:%
%:%653=444%:%
%:%654=445%:%
%:%655=446%:%
%:%656=447%:%
%:%657=448%:%
%:%658=449%:%
%:%659=450%:%
%:%661=452%:%
%:%662=452%:%
%:%663=453%:%
%:%664=453%:%
%:%665=454%:%
%:%667=455%:%
%:%668=456%:%
%:%669=456%:%
%:%670=457%:%
%:%671=458%:%
%:%672=458%:%
%:%673=459%:%
%:%675=460%:%
%:%676=461%:%
%:%677=461%:%
%:%678=462%:%
%:%679=463%:%
%:%680=463%:%
%:%681=464%:%
%:%682=465%:%
%:%684=466%:%
%:%685=466%:%
%:%686=467%:%
%:%687=468%:%
%:%689=469%:%
%:%690=469%:%
%:%691=470%:%
%:%698=471%:%
%:%699=471%:%
%:%700=472%:%
%:%701=472%:%
%:%702=473%:%
%:%703=473%:%
%:%704=473%:%
%:%705=474%:%
%:%706=474%:%
%:%707=475%:%
%:%708=475%:%
%:%709=475%:%
%:%710=476%:%
%:%711=476%:%
%:%713=477%:%
%:%723=479%:%
%:%724=480%:%
%:%725=481%:%
%:%726=482%:%
%:%727=483%:%
%:%728=484%:%
%:%729=485%:%
%:%730=486%:%
%:%731=487%:%
%:%732=488%:%
%:%733=489%:%
%:%734=490%:%
%:%735=491%:%
%:%736=492%:%
%:%737=493%:%
%:%738=494%:%
%:%739=495%:%
%:%740=496%:%
%:%741=497%:%
%:%742=498%:%
%:%743=499%:%
%:%744=500%:%
%:%745=501%:%
%:%746=502%:%
%:%747=503%:%
%:%748=504%:%
%:%749=505%:%
%:%750=506%:%
%:%751=507%:%
%:%752=508%:%
%:%753=509%:%
%:%754=510%:%
%:%755=511%:%
%:%756=512%:%
%:%757=513%:%
%:%758=514%:%
%:%759=515%:%
%:%760=516%:%
%:%761=517%:%
%:%762=518%:%
%:%766=520%:%
%:%767=521%:%
%:%768=522%:%
%:%770=524%:%
%:%771=524%:%
%:%772=525%:%
%:%773=526%:%
%:%774=526%:%
%:%776=526%:%
%:%780=526%:%
%:%781=526%:%
%:%783=527%:%
%:%784=528%:%
%:%785=529%:%
%:%786=530%:%
%:%787=531%:%
%:%788=532%:%
%:%798=535%:%
%:%799=536%:%
%:%800=537%:%
%:%809=539%:%
%:%821=541%:%
%:%822=542%:%
%:%823=543%:%
%:%824=544%:%
%:%825=545%:%
%:%826=546%:%
%:%827=547%:%
%:%828=548%:%
%:%829=549%:%
%:%830=550%:%
%:%831=551%:%
%:%832=552%:%
%:%833=553%:%
%:%834=554%:%
%:%836=556%:%
%:%837=556%:%
%:%838=557%:%
%:%841=558%:%
%:%845=558%:%
%:%846=558%:%
%:%851=558%:%
%:%854=559%:%
%:%855=560%:%
%:%856=560%:%
%:%857=561%:%
%:%860=562%:%
%:%864=562%:%
%:%865=562%:%
%:%870=562%:%
%:%873=563%:%
%:%874=564%:%
%:%875=564%:%
%:%876=565%:%
%:%879=566%:%
%:%883=566%:%
%:%884=566%:%
%:%889=566%:%
%:%892=567%:%
%:%893=568%:%
%:%894=568%:%
%:%895=569%:%
%:%898=570%:%
%:%902=570%:%
%:%903=570%:%
%:%912=572%:%
%:%913=573%:%
%:%914=574%:%
%:%915=575%:%
%:%916=576%:%
%:%917=577%:%
%:%918=578%:%
%:%920=580%:%
%:%921=580%:%
%:%922=581%:%
%:%923=582%:%
%:%924=582%:%
%:%926=582%:%
%:%930=582%:%
%:%931=582%:%
%:%933=583%:%
%:%934=584%:%
%:%935=585%:%
%:%936=586%:%
%:%937=587%:%
%:%938=588%:%
%:%939=589%:%
%:%940=590%:%
%:%941=591%:%
%:%942=592%:%
%:%943=593%:%
%:%953=595%:%
%:%954=596%:%
%:%956=598%:%
%:%957=598%:%
%:%958=599%:%
%:%961=600%:%
%:%965=600%:%
%:%966=600%:%
%:%968=601%:%
%:%978=603%:%
%:%979=604%:%
%:%981=606%:%
%:%982=606%:%
%:%983=607%:%
%:%986=608%:%
%:%990=608%:%
%:%991=608%:%
%:%992=608%:%
%:%994=609%:%
%:%995=610%:%
%:%1005=612%:%
%:%1006=613%:%
%:%1007=614%:%
%:%1008=615%:%
%:%1012=617%:%
%:%1013=618%:%
%:%1014=619%:%
%:%1016=621%:%
%:%1017=621%:%
%:%1018=622%:%
%:%1019=623%:%
%:%1020=624%:%
%:%1021=624%:%
%:%1023=624%:%
%:%1027=624%:%
%:%1028=624%:%
%:%1030=625%:%
%:%1031=626%:%
%:%1032=627%:%
%:%1033=628%:%
%:%1034=629%:%
%:%1035=630%:%
%:%1036=631%:%
%:%1037=632%:%
%:%1038=633%:%
%:%1039=634%:%
%:%1049=636%:%
%:%1050=637%:%
%:%1051=638%:%
%:%1053=640%:%
%:%1054=640%:%
%:%1055=641%:%
%:%1056=642%:%
%:%1057=642%:%
%:%1059=642%:%
%:%1063=642%:%
%:%1064=642%:%
%:%1066=643%:%
%:%1067=644%:%
%:%1068=645%:%
%:%1069=646%:%
%:%1070=647%:%
%:%1071=648%:%
%:%1072=649%:%
%:%1073=650%:%
%:%1074=651%:%
%:%1075=652%:%
%:%1085=654%:%
%:%1086=655%:%
%:%1087=656%:%
%:%1088=657%:%
%:%1090=659%:%
%:%1091=659%:%
%:%1092=660%:%
%:%1095=661%:%
%:%1099=661%:%
%:%1100=661%:%
%:%1101=661%:%
%:%1103=662%:%
%:%1109=662%:%
%:%1112=663%:%
%:%1113=664%:%
%:%1114=664%:%
%:%1115=665%:%
%:%1118=666%:%
%:%1122=666%:%
%:%1123=666%:%
%:%1124=666%:%
%:%1126=667%:%
%:%1132=667%:%
%:%1135=668%:%
%:%1136=669%:%
%:%1137=669%:%
%:%1138=670%:%
%:%1139=671%:%
%:%1142=672%:%
%:%1146=672%:%
%:%1147=672%:%
%:%1148=672%:%
%:%1150=673%:%
%:%1151=674%:%
%:%1157=674%:%
%:%1160=675%:%
%:%1161=676%:%
%:%1162=676%:%
%:%1163=677%:%
%:%1164=678%:%
%:%1167=679%:%
%:%1171=679%:%
%:%1172=679%:%
%:%1173=679%:%
%:%1182=681%:%
%:%1183=682%:%
%:%1184=683%:%
%:%1185=684%:%
%:%1186=685%:%
%:%1187=686%:%
%:%1188=687%:%
%:%1197=689%:%
%:%1209=691%:%
%:%1210=692%:%
%:%1211=693%:%
%:%1212=694%:%
%:%1214=696%:%
%:%1215=696%:%
%:%1216=697%:%
%:%1217=698%:%
%:%1218=698%:%
%:%1220=698%:%
%:%1224=698%:%
%:%1225=698%:%
%:%1227=699%:%
%:%1228=700%:%
%:%1229=701%:%
%:%1230=702%:%
%:%1231=703%:%
%:%1232=704%:%
%:%1233=705%:%
%:%1234=706%:%
%:%1235=707%:%
%:%1236=708%:%
%:%1237=709%:%
%:%1238=709%:%