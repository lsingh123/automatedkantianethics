%
\begin{isabellebody}%
\setisabellecontext{paper{\isadigit{3}}{\isadigit{2}}}%
%
\isadelimtheory
%
\endisadelimtheory
%
\isatagtheory
%
\endisatagtheory
{\isafoldtheory}%
%
\isadelimtheory
%
\endisadelimtheory
%
\isadelimdocument
%
\endisadelimdocument
%
\isatagdocument
%
\isamarkupsubsection{Kroy's Formalization of the Categorical Imperative%
}
\isamarkuptrue%
%
\endisatagdocument
{\isafolddocument}%
%
\isadelimdocument
%
\endisadelimdocument
%
\begin{isamarkuptext}%
This section contains a formalization of the categorical imperative introduced by Moshe Kroy in 
1976 \cite{kroy}. Kroy used Hinktikka's deontic logic to formalize the Formula of Universal Law and
the Formula of Humanity. I will first import the additional logical tools that Hintikka's logic contains 
that Kroy relies on, then examine the differences between his logic and DDL, and finally implement 
and test both of Kroy's formalizations.%
\end{isamarkuptext}\isamarkuptrue%
%
\isadelimdocument
%
\endisadelimdocument
%
\isatagdocument
%
\isamarkupsubsubsection{Additional Logical Tools%
}
\isamarkuptrue%
%
\endisatagdocument
{\isafolddocument}%
%
\isadelimdocument
%
\endisadelimdocument
%
\begin{isamarkuptext}%
Kroy's logic relies heavily on some notion of identity or agency. We need some notion of ``x does action", 
which we can write as ``x is the subject of the sentence 'does action'". This requires defining a subject.%
\end{isamarkuptext}\isamarkuptrue%
\isacommand{typedecl}\isamarkupfalse%
\ s\ %
\isamarkupcmt{s is the type for a ``subject", like the subject of a sentence%
}%
\begin{isamarkuptext}%
Kroy also defines a substitution operator\footnote{see page 196 in \cite{kroy}}. $P (d/e)$ is read in his logic as ``P with e substituted 
for d." DDL has no such notion of substitution, so I will define a more generalized notion of an ``open 
sentence." An open sentence takes as input a subject and returns a complete or ``closed" DDL formula. For example, 
``does action" is an open sentence that can be instantiated with a subject.%
\end{isamarkuptext}\isamarkuptrue%
\isacommand{type{\isacharunderscore}synonym}\isamarkupfalse%
\ os\ {\isacharequal}\ {\isachardoublequoteopen}{\isacharparenleft}s\ {\isasymRightarrow}\ t{\isacharparenright}{\isachardoublequoteclose}\isanewline
%
\isamarkupcmt{``P sub (d/e)" can be written as ``S(e)", where S(d) = P%
}\isanewline
%
\isamarkupcmt{The terms that we substitute into are actually instantiations of an open sentence, and substitution 
just requires re-instantiating the open sentence with a different subject.%
}%
\begin{isamarkuptext}%
\textbf{New Operators}

Because Isabelle is strongly typed, we need to define new operators to handle open sentences. These operators 
are similar to DDL's original operators. We could probably do without these abbreviations, but they will 
simplify the notation and make it look more similar to Kroy's original paper.%
\end{isamarkuptext}\isamarkuptrue%
\isacommand{abbreviation}\isamarkupfalse%
\ os{\isacharunderscore}neg{\isacharcolon}{\isacharcolon}{\isachardoublequoteopen}os\ {\isasymRightarrow}\ os{\isachardoublequoteclose}\ {\isacharparenleft}{\isachardoublequoteopen}\isactrlemph {\isasymnot}{\isacharunderscore}{\isachardoublequoteclose}{\isacharparenright}\isanewline
\ \ \isakeyword{where}\ {\isachardoublequoteopen}{\isacharparenleft}\isactrlemph {\isasymnot}A{\isacharparenright}\ {\isasymequiv}\ {\isasymlambda}x{\isachardot}\ \isactrlbold {\isasymnot}{\isacharparenleft}A{\isacharparenleft}x{\isacharparenright}{\isacharparenright}{\isachardoublequoteclose}\isanewline
\isacommand{abbreviation}\isamarkupfalse%
\ os{\isacharunderscore}and{\isacharcolon}{\isacharcolon}{\isachardoublequoteopen}os\ {\isasymRightarrow}\ os\ {\isasymRightarrow}\ os{\isachardoublequoteclose}\ {\isacharparenleft}{\isachardoublequoteopen}{\isacharunderscore}\isactrlemph {\isasymand}{\isacharunderscore}{\isachardoublequoteclose}{\isacharparenright}\isanewline
\ \ \isakeyword{where}\ {\isachardoublequoteopen}{\isacharparenleft}A\ \isactrlemph {\isasymand}\ B{\isacharparenright}\ {\isasymequiv}\ {\isasymlambda}x{\isachardot}\ {\isacharparenleft}{\isacharparenleft}A{\isacharparenleft}x{\isacharparenright}{\isacharparenright}\ \isactrlbold {\isasymand}\ {\isacharparenleft}B{\isacharparenleft}x{\isacharparenright}{\isacharparenright}{\isacharparenright}{\isachardoublequoteclose}\isanewline
\isacommand{abbreviation}\isamarkupfalse%
\ os{\isacharunderscore}or{\isacharcolon}{\isacharcolon}{\isachardoublequoteopen}os\ {\isasymRightarrow}\ os\ {\isasymRightarrow}\ os{\isachardoublequoteclose}\ {\isacharparenleft}{\isachardoublequoteopen}{\isacharunderscore}\isactrlemph {\isasymor}{\isacharunderscore}{\isachardoublequoteclose}{\isacharparenright}\isanewline
\ \ \isakeyword{where}\ {\isachardoublequoteopen}{\isacharparenleft}A\ \isactrlemph {\isasymor}\ B{\isacharparenright}\ {\isasymequiv}\ {\isasymlambda}x{\isachardot}\ {\isacharparenleft}{\isacharparenleft}A{\isacharparenleft}x{\isacharparenright}{\isacharparenright}\ \isactrlbold {\isasymor}\ {\isacharparenleft}B{\isacharparenleft}x{\isacharparenright}{\isacharparenright}{\isacharparenright}{\isachardoublequoteclose}\isanewline
\isacommand{abbreviation}\isamarkupfalse%
\ os{\isacharunderscore}ob{\isacharcolon}{\isacharcolon}{\isachardoublequoteopen}os\ {\isasymRightarrow}\ os{\isachardoublequoteclose}\ {\isacharparenleft}{\isachardoublequoteopen}\isactrlemph O{\isacharbraceleft}{\isacharunderscore}{\isacharbraceright}{\isachardoublequoteclose}{\isacharparenright}\isanewline
\ \ \isakeyword{where}\ {\isachardoublequoteopen}\isactrlemph O{\isacharbraceleft}A{\isacharbraceright}\ {\isasymequiv}\ {\isasymlambda}x{\isachardot}\ {\isacharparenleft}O\ {\isacharbraceleft}A{\isacharparenleft}x{\isacharparenright}{\isacharbraceright}{\isacharparenright}{\isachardoublequoteclose}%
\begin{isamarkuptext}%
Once again, the notion of permissibility will be useful here.%
\end{isamarkuptext}\isamarkuptrue%
\isacommand{abbreviation}\isamarkupfalse%
\ ddl{\isacharunderscore}permissible{\isacharcolon}{\isacharcolon}{\isachardoublequoteopen}t{\isasymRightarrow}t{\isachardoublequoteclose}\ {\isacharparenleft}{\isachardoublequoteopen}P\ {\isacharbraceleft}{\isacharunderscore}{\isacharbraceright}{\isachardoublequoteclose}{\isacharparenright}\isanewline
\ \ \isakeyword{where}\ {\isachardoublequoteopen}P\ {\isacharbraceleft}A{\isacharbraceright}\ {\isasymequiv}\ \isactrlbold {\isasymnot}\ {\isacharparenleft}O\ {\isacharbraceleft}\isactrlbold {\isasymnot}\ A{\isacharbraceright}{\isacharparenright}{\isachardoublequoteclose}\isanewline
\isacommand{abbreviation}\isamarkupfalse%
\ os{\isacharunderscore}permissible{\isacharcolon}{\isacharcolon}{\isachardoublequoteopen}os{\isasymRightarrow}os{\isachardoublequoteclose}\ {\isacharparenleft}{\isachardoublequoteopen}\isactrlemph P\ {\isacharbraceleft}{\isacharunderscore}{\isacharbraceright}{\isachardoublequoteclose}{\isacharparenright}\isanewline
\ \ \isakeyword{where}\ {\isachardoublequoteopen}\isactrlemph P\ {\isacharbraceleft}A{\isacharbraceright}\ {\isasymequiv}\ {\isasymlambda}x{\isachardot}\ P\ {\isacharbraceleft}A{\isacharparenleft}x{\isacharparenright}{\isacharbraceright}{\isachardoublequoteclose}%
\isadelimdocument
%
\endisadelimdocument
%
\isatagdocument
%
\isamarkupsubsubsection{Differences Between Kroy's Logic (Kr) and DDL%
}
\isamarkuptrue%
%
\endisatagdocument
{\isafolddocument}%
%
\isadelimdocument
%
\endisadelimdocument
%
\isadelimproof
%
\endisadelimproof
%
\isatagproof
%
\endisatagproof
{\isafoldproof}%
%
\isadelimproof
%
\endisadelimproof
%
\isadelimproof
%
\endisadelimproof
%
\isatagproof
%
\endisatagproof
{\isafoldproof}%
%
\isadelimproof
%
\endisadelimproof
%
\isadelimproof
%
\endisadelimproof
%
\isatagproof
%
\endisatagproof
{\isafoldproof}%
%
\isadelimproof
%
\endisadelimproof
%
\isadelimproof
%
\endisadelimproof
%
\isatagproof
%
\endisatagproof
{\isafoldproof}%
%
\isadelimproof
%
\endisadelimproof
%
\isadelimproof
%
\endisadelimproof
%
\isatagproof
%
\endisatagproof
{\isafoldproof}%
%
\isadelimproof
%
\endisadelimproof
%
\isadelimdocument
%
\endisadelimdocument
%
\isatagdocument
%
\endisatagdocument
{\isafolddocument}%
%
\isadelimdocument
%
\endisadelimdocument
%
\isadelimproof
%
\endisadelimproof
%
\isatagproof
%
\endisatagproof
{\isafoldproof}%
%
\isadelimproof
%
\endisadelimproof
%
\isadelimproof
%
\endisadelimproof
%
\isatagproof
%
\endisatagproof
{\isafoldproof}%
%
\isadelimproof
%
\endisadelimproof
%
\isadelimproof
%
\endisadelimproof
%
\isatagproof
%
\endisatagproof
{\isafoldproof}%
%
\isadelimproof
%
\endisadelimproof
%
\isadelimdocument
%
\endisadelimdocument
%
\isatagdocument
%
\endisatagdocument
{\isafolddocument}%
%
\isadelimdocument
%
\endisadelimdocument
%
\isadelimproof
%
\endisadelimproof
%
\isatagproof
%
\endisatagproof
{\isafoldproof}%
%
\isadelimproof
%
\endisadelimproof
%
\isadelimdocument
%
\endisadelimdocument
%
\isatagdocument
%
\endisatagdocument
{\isafolddocument}%
%
\isadelimdocument
%
\endisadelimdocument
%
\isadelimproof
%
\endisadelimproof
%
\isatagproof
%
\endisatagproof
{\isafoldproof}%
%
\isadelimproof
%
\endisadelimproof
%
\isadelimproof
%
\endisadelimproof
%
\isatagproof
%
\endisatagproof
{\isafoldproof}%
%
\isadelimproof
%
\endisadelimproof
%
\isadelimdocument
%
\endisadelimdocument
%
\isatagdocument
%
\endisatagdocument
{\isafolddocument}%
%
\isadelimdocument
%
\endisadelimdocument
%
\isadelimtheory
%
\endisadelimtheory
%
\isatagtheory
%
\endisatagtheory
{\isafoldtheory}%
%
\isadelimtheory
%
\endisadelimtheory
%
\end{isabellebody}%
\endinput
%:%file=~/Desktop/cs91r/paper/paper32.thy%:%
%:%24=8%:%
%:%36=10%:%
%:%37=11%:%
%:%38=12%:%
%:%39=13%:%
%:%40=14%:%
%:%49=16%:%
%:%61=18%:%
%:%62=19%:%
%:%64=21%:%
%:%65=21%:%
%:%66=21%:%
%:%69=23%:%
%:%70=24%:%
%:%71=25%:%
%:%72=26%:%
%:%74=28%:%
%:%75=28%:%
%:%77=29%:%
%:%78=29%:%
%:%80=30%:%
%:%81=31%:%
%:%84=33%:%
%:%85=34%:%
%:%86=35%:%
%:%87=36%:%
%:%88=37%:%
%:%90=39%:%
%:%91=39%:%
%:%92=40%:%
%:%93=41%:%
%:%94=41%:%
%:%95=42%:%
%:%96=43%:%
%:%97=43%:%
%:%98=44%:%
%:%99=45%:%
%:%100=45%:%
%:%101=46%:%
%:%103=48%:%
%:%105=50%:%
%:%106=50%:%
%:%107=51%:%
%:%108=52%:%
%:%109=52%:%
%:%110=53%:%
%:%117=56%:%