%
\begin{isabellebody}%
\setisabellecontext{thesis{\isacharunderscore}{\isadigit{2}}{\isacharunderscore}methods}%
%
\isadelimtheory
%
\endisadelimtheory
%
\isatagtheory
%
\endisatagtheory
{\isafoldtheory}%
%
\isadelimtheory
%
\endisadelimtheory
%
\isadelimdocument
%
\endisadelimdocument
%
\isatagdocument
%
\isamarkupsection{System Components \label{methods}%
}
\isamarkuptrue%
%
\endisatagdocument
{\isafolddocument}%
%
\isadelimdocument
%
\endisadelimdocument
%
\begin{isamarkuptext}%
My system consists of three major components: an ethical theory (Kantian ethics), a logic in which
I formalize this ethical theory (Dyadic Deontic Logic), and an interactive theorem prover in which I 
implement the formalized ethical theory (Isabelle/HOL). In this section, I describe these components, 
present the philosophical, logic, and computational background that undergirds my system, and explain
the consequences of each of the three choices I make. 

These specific components determine the features
and limitations of the specific implementation of automated ethics that I present, but other choices of 
components, such as another ethical theory, a different logic, or a different theorem prover could be 
made. Given that my system is a proof-of-concept, I choose system components that are natural and 
intuitively fit together, but other choice could also be valid and perhaps even superior. 

I do not claim
that I have chosen the best ethical theory or the best logic or the best interactive theorem prover, 
and flaws with these components merely demonstrate that my specific choices were incorrect, but do not 
indict logic-programming-based automated ethics as I implement it in this thesis. My thesis seeks to 
both present a specific implementation of automated ethics but also to argue for a particular approach 
to automating ethical reasoning and these choices are relevant to the former goal but not to the latter.%
\end{isamarkuptext}\isamarkuptrue%
%
\isadelimdocument
%
\endisadelimdocument
%
\isatagdocument
%
\isamarkupsubsection{Kantian Ethics \label{whykant}%
}
\isamarkuptrue%
%
\endisatagdocument
{\isafolddocument}%
%
\isadelimdocument
%
\endisadelimdocument
%
\begin{isamarkuptext}%
In this thesis, I automate Kantian ethics. In 2006, Powers posited that deontological theories are 
attractive candidates for automation because rules are generally computationally tractable \cite[1]{powers}. 
Intuitively, algorithms are rules or procedures for problem solving and deontology offers one such 
procedure for the problem of making ethical judgements. I will make this intuition precise by
arguing that deontological ethics is natural to formalize because rules generally require little additional
data about the world and are usually easy to represent to a computer. All ethical traditions have debates that an 
automated ethical system will need to take a stance on, but these debates are less frequent and controversial
for deontological ethics than for consequentialism and virtue ethics.

I do not aim to show that deontology is the only tractable theory to automate or
to present a comprehensive overview of all consequentialist or virtue ethical theories. Instead, I 
present a sample of some approaches in each tradition and argue that deontology is more straightforward 
to formalize than these approaches. Future work could and should address the challenges I outline in 
this section. The more ethical theories that computational tools can handle, the more valuable 
computational philosophy becomes both for philosophers and for AI agents. Insofar as my project serves 
as an early proof-of-concept for computational ethics, I choose to automate an ethical theory that 
poses fewer challenges than others.

I first present deontological ethics, then consequentialism, and finally virtue ethics. For each 
tradition, I present a crash course for non-philosophers and then explain some obstacles to automation, 
arguing that these obstacles are weakest in the case of deontology. Finally, I will present the specific 
deontological theory I am automating (Kantian ethics) and will argue that it is comparatively easier
to formalize. I will also outline the specific debates in the literature that my formalization takes
a stance on and potential challenges for formalizing deontology.%
\end{isamarkuptext}\isamarkuptrue%
%
\isadelimdocument
%
\endisadelimdocument
%
\isatagdocument
%
\isamarkupsubsubsection{Deontological Ethics\label{deontology}%
}
\isamarkuptrue%
%
\endisatagdocument
{\isafolddocument}%
%
\isadelimdocument
%
\endisadelimdocument
%
\begin{isamarkuptext}%
Deontological ethical theories evaluate actions as permissible, obligatory, or prohibited. The 
deontological tradition argues that an action should not be judged on its consequences, 
but rather on ``its confirmity with a moral norm" \citep{sepdeont}. In other words, deontological theories
define a set of moral norms or rules and evaluate actions using
these rules. Deontologists do not believe that we should maximize the number of times that we 
conform to such rules, but instead that we should never violate any of the moral laws. A wrong
choice is wrong, regardless of its consequences.%
\end{isamarkuptext}\isamarkuptrue%
%
\begin{isamarkuptext}%
Deontology is immediately an attractive candidate for formalization because computers tend to 
understand rules; programming languages are designed to teach computers algorithms. Deontological
ethical theories give inviolable rules that an automated agent can apply. Moreover, because deontological 
theories focus on
the action itself, they require relatively little data. A deontological moral judgement does not require
as much information about context, consequences, or moral character as the other theories presented later in this
section. All that matters is the action and some limited set of circumstances in which it is performed.
I will later argue that, in the case of the specific deontological ethic that I implement (Kantian ethics), 
the action's representation is space efficient. 

Like all ethical traditions, deontology has debates that any implementation of automated deontological
ethics must resolve. Deontologists disagree about whether ethics should focus on agents' actions or 
on the rights of those impacted by an action. Different deontological theories have different conceptions 
of what an action is, from the physical act itself to the agent's mental state at the time of acting 
to the principle upon which the agent acted. 

While these debates are open, ``if any philosopher is regarded as central
to deontological moral theories, it is surely Immanuel Kant" \citep{sepdeont}. Out of the three ethical
traditions considered in this section, deontology has the most central representative in the form of 
Kant. Many modern deontologists agree on Kant's ethic but disagree in how to interpret it. In this paper, I will 
formalize Kantian ethics. Deontology's comparatively greater focus on Kant means that the choice of 
Kant as a guiding figure will be less controversial to deontologists than, for example, the choice of Bentham
as the guiding figure of consequentialism. Moreover, at the end of this section, I also argue that 
internal debates in the part of Kantian ethics that I focus on tend to be less controversial than those
in the consequentialist or virtue ethical traditions.%
\end{isamarkuptext}\isamarkuptrue%
%
\isadelimdocument
%
\endisadelimdocument
%
\isatagdocument
%
\isamarkupsubsubsection{Consequentialism%
}
\isamarkuptrue%
%
\endisatagdocument
{\isafolddocument}%
%
\isadelimdocument
%
\endisadelimdocument
%
\begin{isamarkuptext}%
A consequentialist ethical theory is an ethical theory that evaluates an 
action by evaluating its consequences.\footnote{There is long debate about what exactly makes an ethical theory consequentialist \citep{consequentialismsep}. 
For this thesis, I focus on theories that place the moral worth of an act in its the consequences.} For example, 
utilitarianism is a form of consequentialism in which the moral action 
is the action that produces the most good \citep{utilsep}. The focus
on the consequences of action distinguishes consequentialists from deontologists, who derive the moral worth
of an action from the action itself. Some debates in the consequentialist tradition include 
which consequences of an action matter, what exactly constitutes a ``good" consequence, and how we can 
aggregate the consequences of an action over all the individuals involved.%
\end{isamarkuptext}\isamarkuptrue%
%
\begin{isamarkuptext}%
\noindent \textbf{Which Consequences Matter}%
\end{isamarkuptext}\isamarkuptrue%
%
\begin{isamarkuptext}%
Because consequentialism evaluates the state of affairs following an action, this kind of ethical 
reasoning requires more knowledge about the state of the world than deontology. Consequentialism
requires knowledge about some or all consequences following an action. This requires that an automated 
ethical system somehow collect a subset of the infinite consquences of following an action, a difficult, 
if not impossible, task. Moreover, compiling this database of consequences requires 
answering difficult questions about which consequences were actually caused by an action.\footnote
{David Hume argues that many straightforward accounts of causation face difficulties \citep{hume}, 
and philosohpers continue to debate the possiblity of knowing an event's true cause. Kant even argued
that first causes, or noumena, are unknowable by human beings \citep{kantnoumena}.}
Evaluating an effect of an action requires knowledge about the 
state of the world before and after an action and knowledge about the action itself. Consequentialism requires 
knowledge about the situation in which the act is performed and following the act, whereas deontology mostly requires 
knowledge about the act itself. As acts become
more complex and affect more people, the computational time and space required to calculate and store
their consequences increases. Deontology, on the other hand, does not suffer this scaling
challenge because acts that affect 1 person and acts that affect 1 million people share the same
representation.

The challenge of representing the circumstances of action is not unique to consequentialism, but is particularly acute in this case. Kantian ethicists robustly debate which circumstances of an action are ``morally relevant'' when evaluating an action's moral worth.\footnote{Powers 
\citet{powers} identifies this as a challenge for automating Kantian ethics and briefly sketches 
solutions from \citet{constofreason}, \citet{silber}, and \citet{rawlsconstructivism}. For more on 
morally relevant circumstances, see Section WhatIsAMaxim.} Because deontology merely evaluates a 
single action, the surface of this debate is much smaller than the debate about circumstances and 
consequences in a consequentialist system. An automated consequentialist system must make such 
judgements about the act itself, the circumstances in which it is performed, and the circumstances 
following the act. All ethical theories relativize their judgements to the situation in which an act 
is performed, but consequentialism requires far more knowledge about the world than deontology.%
\end{isamarkuptext}\isamarkuptrue%
%
\begin{isamarkuptext}%
\noindent \textbf{Theory of the Good}%
\end{isamarkuptext}\isamarkuptrue%
%
\begin{isamarkuptext}%
An automated consequentialist reasoner must also take a stance on the debate over
what qualifies as a ``good consequence," or what the theory of the good is. For example, hedonists associate
good with the presence of pleasure and the absence of pain, while preference utiliarians believe that good is 
the satisfaction of individuals' desires. Other consequentialists, like Moore, adopt a pluralistic theory of value, under which 
many different kinds of things are good for different reasons \citep{moorepe}. 

Most of the above theories of good require that a moral reasoner understand complex features about
individuals' preferences, desires, or sensations in order to evaluate a moral action, making automated
consequentialist ethics difficult. Evaluating a state of affairs requires
judgements about whether a state of affairs actually satisifes the relevant criteria for goodness. 
These judgements are controversian, and any consequentialist decision requires many of 
these judgements for each individual involved. As systems involve more people, making these judgements 
quickly becomes difficult, posing a scaling challenger. Perfect knowledge of tens of thousands of 
people's pleasure or preferences or welfare or rights is impossible. Either a human being 
assigns values to states of affairs, which doesn't scale, or the machine does, 
which requires massive common-sense and increases room for doubting the system's judgements. This is 
a tractable problem, but it is much more difficult than the equivalent deontological task of formulating
and evaluating an action.%
\end{isamarkuptext}\isamarkuptrue%
%
\begin{isamarkuptext}%
\noindent \textbf{Aggregation}%
\end{isamarkuptext}\isamarkuptrue%
%
\begin{isamarkuptext}%
Once an automated consequentialist agent assigns a goodness measurement to each person in a state of affairs, it 
must also calculate an overall goodness measurement for the state of affairs. One approach to assigning
this value is to aggregate each person's individual goodness score into one complete score for a state. 
For example, under a simple welfare model, each person is assigned a welfare score and the total 
score for a state of affairs is the sum of the welfare scores for each involved person.
The more complex the theory of the good, the more difficult this aggregation becomes. For example, 
pluralistic theories struggle to explain how different kinds of value can be compared \citep{consequentialismsep}. 
How do we compare one unit of beauty to one unit of pleasure? Subjective theories of the good, such 
as those focused on the sensation of pleasure or an individual's preferences, present difficulties in 
comparing different people's subjective measures. Resolving this debate requires that the automated reasoner 
choose one specific aggregation algorithm, but those who disagree with this choice will not trust 
the reasoner's moral judgements. Moreover, for complex theories of the good, this aggregation algorithm
may be complex and may require a lot of data. 

To solve this problem, some consequentialists reject aggregation entirely and instead prefer wholistic
evaluations of a state of affairs. While this approach no longer requires that a reasoner define an 
aggregation algorithm, the reasoner still needs to calculate a goodness measurement for a state of 
affairs. Whereas before the reasoner could restrict analysis to a single person, the algorithm must now 
evaluate an entire state wholistically. As consequentialists modulate between aggregation 
and wholistic evaluation, they face a tradeoff between the difficulty of aggregation and the complexity 
of goodness measurements for large states of affairs.%
\end{isamarkuptext}\isamarkuptrue%
%
\begin{isamarkuptext}%
\noindent \textbf{Prior Attempts to Formalize Consequentialism}%
\end{isamarkuptext}\isamarkuptrue%
%
\begin{isamarkuptext}%
Because of its intuitive appeal, computer scientists have tried to formalize consequentialism in the past.
These efforts cannot escape the debates outlined above. For example, Abel et al. represent ethics as a
Markov Decision Process (MDP), with reward functions customized to particular ethical dilemmas 
\citep[3]{util1}. While this is a convenient representation, it either leaves unanswered or 
takes implicit stances on the debates above. It assumes that consequences can be aggregated just as 
reward is accumulated in an MDP.\footnote{Generally, reward for an MDP is accumulated according to a 
``discount factor" $\gamma < 1$, such that if $r_i$ is the reward at time $i$, the total reward is $\sum_{i=0}^{\infty}\gamma^i r_i$.} 
It leaves open the question of what the reward function is and thus 
leaves the theory of the good, arguably the defining trait of consequentialism, 
undefined. Similarly, Anderson and Anderson's proposal of a hedonistic act 
utilitarian automated reasoner chooses hedonism\footnote{Recall that hedonism views pleasure as good
and pain as bad.} as the theory of the good \citep[2]{util2}. Again, their proposal assumes that pleasure and pain can be 
given numeric values and that these values can be aggregated with a simple sum, taking an implicit
stance on the aggregation question. Other attempts to automate consequentialist ethics will suffer 
similar problems because, at some point, a useful automated consequentialist moral agent will need 
to resolve the above debates.%
\end{isamarkuptext}\isamarkuptrue%
%
\isadelimdocument
%
\endisadelimdocument
%
\isatagdocument
%
\isamarkupsubsubsection{Virtue Ethics\label{virtueethics}%
}
\isamarkuptrue%
%
\endisatagdocument
{\isafolddocument}%
%
\isadelimdocument
%
\endisadelimdocument
%
\begin{isamarkuptext}%
Virtue ethics places the virtues, or traits that constitute a good moral character and make 
their possessor good, at the center \cite{vesep}. For example, Aristotle describes virtues as the 
traits that enable human flourishing. Just as consequentialists define ``good'' consequences, virtue 
ethicists present a list of virtues. Such theories vary from Aristotle's virtues of courage and 
temperance \citet{aristotle} to the Buddhist virtue of equanimity \citet{mcrae}. An automated virtue 
ethical agent will need to commit to a particular theory of the virtues, a controversial choice. 
Virtue ethicists robustly debate which traits qualify as virtues, what each virtue actually means, and 
what kinds of feelings or attitudes must accompany virtuous action.%
\end{isamarkuptext}\isamarkuptrue%
%
\begin{isamarkuptext}%
Another difficulty with automating virtue ethics is that the unit of evaluation for a virtue ethical
theory is often a person's entire moral character. While deontologists evaluate the act itself and utilitarians
evaluate the consequences of an act, virtue ethicists evaluate the actor's moral character and their 
disposition towards the act. Virtues are character traits and evaluating an action as virtuous or 
not requires understanding the agent's character and disposition while acting. If states of affairs
require complex representations, an agent's ethical character and disposition are even more difficult
to represent to a computer. Consequentialism posed a data-collection problem in evaluating and representing states
of affairs, but virtue ethics poses a conceptual problem about the formal nature of moral character.
Formalizing the concept of character appears to require significant philosophical and computational
progress, whereas deontology immediately presents a formal rule to implement.%
\end{isamarkuptext}\isamarkuptrue%
%
\begin{isamarkuptext}%
\noindent \textbf{Prior Work in Machine Learning and Virtue Ethics}%
\end{isamarkuptext}\isamarkuptrue%
%
\begin{isamarkuptext}%
One potential appeal of virtue ethics is that many virtue ethical theories involve some form of 
moral habit, which seems to be amenable to a machine learning approach. Artistotle, for example, argued 
that cultivating virtuous action requires making such action habitual through moral education \citep{aristotle}. This
imply that ethical behavior can be learned from some dataset of virtuous acts, either those 
prescribed by a moral teacher or those that a virtuous ideal agent would undertake. Indeed, these 
theories seem to point towards a machine learning approach to computational ethics, in which ethics is 
learned from a dataset of acts tagged as virtuous or not virtuous. 

Just as prior work in consequentialism takes implicit or explicit stances on debates in consequentialist
literature, so does work in machine learning-based virtue ethics. For example, the training 
dataset with acts labelled as virtuous or not virtuous will contain an implicit view on what the virtues
are and how certain acts impact an agent's moral character. Because there is no canonical list of virtues
that virtue ethicists accept, this implicit view will likely be controversial.

Machine learning approaches like the Delphi system \citep{delphi} mentioned in \ref{intro} also may suffer explanability 
problems that my logic-programming, theorem-prover
approach does not face. Many machine learning algorithms cannot sufficiently explain their 
decisions to a human being, and often find patterns or correlations in datasets that don't actually 
cohere with the trends and causes that a human being would identify \citep{puiutta}. While there is significant activity 
and progress in explainable machine learning, interactive theorem provers are designed to be explainable 
at the outset. Indeed, Isabelle can show the axioms and lemmas it used in constructing a proof, 
allowing a human being to reconstruct the proof independently if they wish. This is not an 
intractable problem for machine learning approaches to computational ethics, but is one reason to 
prefer logical approaches.\footnote{This argument about explanability is in the context of virtue ethics and 
machine learning. It also applies to a broader class of work in automated ethics 
that uses ``bottom-up" approaches, in which a system learns moral judgements from prior judgements. 
I will extend this argument to general bottom-up approaches in Section Related Work.}%
\end{isamarkuptext}\isamarkuptrue%
%
\isadelimdocument
%
\endisadelimdocument
%
\isatagdocument
%
\isamarkupsubsubsection{Kantian Ethics \label{kantianethics}%
}
\isamarkuptrue%
%
\endisatagdocument
{\isafolddocument}%
%
\isadelimdocument
%
\endisadelimdocument
%
\begin{isamarkuptext}%
In this paper I focus on Kantian ethics, a specific branch of deontology. Kant's theory is centered 
on practical reason, which is the kind of reason that we 
use to decide what to do. In \emph{The Groundwork of the Metaphysics of Morals}, Kant's most influential 
text on ethics, he explains that rational beings are unique because we can act ``in accordance with 
the representations of laws" \citep[4:412]{groundwork}. In contrast, a ball thrown into the air acts 
according to the laws of physics. It cannot ask itself, ``Should I fall back to the ground?" 
It simply falls. A rational being, on the other hand, can ask, ``Should I act on this reason?" 
As Korsgaard describes it, when choosing which desire to act on, ``it is as if there is something over 
and above all of your desires, something which is you, and which chooses which desire to act on" \citep[100]{sources}. 
Rational beings are set apart by this reflective capacity. A rational being's behavior is purposive and 
their actions are guided by practical reason. They have reasons for acting, even when these reasons may be 
opaque to them. This operation of practical reason is what Kant calls the will. 

The will operates by adopting, or willing, maxims, which are its perceived reasons for acting. Kant defines a maxim as 
the ``subjective principle of willing," or the reason that the will \emph{subjectively} gives 
to itself for acting \citep[16 footnote 1]{groundwork}. There is debate about what exactly must be 
included in a maxim, but many philosophers agree that a maxim consists of some combination of circumstances, 
act, and goal.\footnote{For more discussion of the definition of a maxim, see Section What Is a Maxim}
One example of a maxim is ``When I am hungry, I will eat a doughnut in order to satisfy my sweet tooth." 
When an agent wills this maxim, they decide to act on it. They commit themselves to the end in the maxim 
(e.g. satisfying your sweet tooth). They represent their action, to themselves, as following the 
principle given by this maxim. Because a maxim captures an agent's principle of action, Kant evaluates
maxims as obligatory, prohibited, or permissible. He argues that certain maxims have a form or logical structure 
that requires any rational agent to will them, and these maxims are obligatory. 

The form of an obligatory maxim is given by the categorical imperative. 
An imperative is a command, such as ``Close the door" or ``Eat the doughnut in order to satisfy your 
sweet tooth." An imperative is categorical if it holds unconditionally for all rational agents under all 
circumstances. Kant argues that the moral law must be a categorical imperative, for otherwise it would 
not have the force that makes it a moral law \citep[5]{groundwork}. In order for an imperative to be 
categorical, it must be derived from the will's authority over itself. Our wills are autonomous, so 
the only thing that can have unconditional authority over a rational will is 
the rational will itself. In Velleman's version of this argument, he claims that no one else can tell you what 
to do because you can always ask why you 
should obey their authority. The only authority that you cannot question is the authority of your own 
practical reason. To question this authority is to demand a reason for acting for reasons, which 
concedes the authority of reason itself \citep[23]{velleman}. Therefore, the only possible candidates 
for the categorical imperative are those rules that are required of the will because it is a will. 
The categorical imperative must be a property of practical reason itself.

Armed with this understanding of practical reason, Kant presents the categorical 
imperative. He presents three ``formulations" or versions of the categorical imperative and goes on to 
argue that all three formulations are equivalent. In this project, I focus on the first formulation,
the Formula of Universal Law.\footnote{For more on this 
choice, see Section Why FUL.}

The first formulation of the categorical imperative is the
Formula of Universal Law (FUL), which reads, ``act only according to that maxim through which you can 
at the same time will that it become a universal law" \cite[34]{groundwork}. This formulation
generates the universalizability test, which tests the moral value of a maxim by 
imagining a world in which it becomes a universal law and attempting to will the maxim in that world.
If there is a contradiction in willing the maxim in a world in which everyone universally wills the maxim,
the maxim is prohibited. 
Velleman presents a concise argument for the FUL. He argues that reason is universally shared among reasoners. For 
example, all reasoners have equal access to the arithmetic logic that shows that ``2+2=4" \cite[29]{velleman}. The chain of 
reasoning that makes this statement true is not specific to any person, but is universal across people. 
Therefore, if I have sufficient reason to will a maxim, so does every other rational agent. There is 
nothing special about the operation of my practical reason that other reasoners don't have access to. 
Practical reason is shared, so in adopting a maxim, I implicitly state that all reasoners
across time also have reason to adopt that maxim. Therefore, because I act on reasons, I must obey the 
FUL. Notice that this fulfills the above criterion for a categorical imperative: the FUL is derived from 
a property of practical reason itself and thus derives authority from the will's authority over itself, 
as opposed to some external authority.

The above is not meant to serve as a full defense or articulation of Kant's ethical theory, as that is outside the scope
of this thesis. Instead, I briefly reconstruct a sketch of Kant's ethical theory in the hopes 
of offering context for the implementation of the FUL I present later in the thesis. Additionally, understanding 
the structure of Kant's theory also reveals why it is an ideal candidate 
for formalization.%
\end{isamarkuptext}\isamarkuptrue%
%
\begin{isamarkuptext}%
\noindent \textbf{Ease of Automation}%
\end{isamarkuptext}\isamarkuptrue%
%
\begin{isamarkuptext}%
Kantian ethics is an especially candidate for formalization because the categorical imperative, particularly the FUL, 
is a property of reason related to the form or structure of a maxim, or a formal principle of practical
reason. It does not require any situational knowledge or contingent beyond the circumstances included
in the maxim itself and thus requires far less contingent facts than other ethical theories.
Instead, it is purely a property of the proposed principle for action. This formalism makes Kantian
ethics an attractive candidate for formalization. While other ethical theories often rely on many facts about 
the world or the actor, Kantian ethics simply relies on the form of a given maxim. A computer evaluating 
a maxim doesn't require any knowledge about the world beyond what is contained in a maxim. A maxim 
is the only input that the computer needs to make a moral judgement. Automating 
Kantian ethics merely requires making the notion of a maxim precise and representing it to the computer. 
This distinguishes Kantian ethics from consequentialism and virtue ethics, which, as I argued above, 
require far more knowledge about the world or the agent to reach a moral decision.

Not only does evaluating Kantian ethics focus on a maxim, a maxim itself is an object
with a thin representation for a computer, as compares to more complex objects like states of 
affairs or moral character. Later in my project, I argue that a maxim can be represented simply as 
a tuple of circumstances, act, and goal.\footnote{For more, see Section What is a Maxim?} This representation
is simple and efficient, especially when compared to the representation of a causal chain or a state of 
affairs or moral character. A maxim is a principle with a well-defined form, so representing a maxim
to the computer merely requires capturing this form. This property not only reduces the computational complexity
(in terms of time and space) of representing a maxim, it also make the system easier for human reasoners
to interact with. A person crafting an input to a Kantian automated agent needs to reason about relatively
simple units of evaluation, as opposed to the more complex features that consequentialism and virtue
ethics require. I will make the comparision to consequentialism and virtue ethics explicit below.%
\end{isamarkuptext}\isamarkuptrue%
%
\begin{isamarkuptext}%
\textbf{Difficulties in Automation}%
\end{isamarkuptext}\isamarkuptrue%
%
\begin{isamarkuptext}%
My choices to interpret maxims and the Formula of Universal Law in a particular way represent debates
in Kantian ethics over the meanings of these terms that I take a stance on. Another debate in Kantian 
ethics is the role of ``common-sense" reasoning. Kantian ethics requires common-sense reasoning to 
determine which circumstances are ``morally relevant" in the formulation of a maxim. Many misunderstandings
in Kantian ethics are due to badly formulated maxims, so this question is important for an ethical 
reasoner to answer. My system does not need to answer this question because I assume a well-formed
maxim as input and apply the categorical imperative to this input, but if my system were ever to be used
in a faully automated agent, answering this question would require significant computational and philosophical
work. For more, see Section AI Ethics.

Common-sense reasoning is also relevant in applying the universalizability test itself. Consider an example
maxim tested using the Formula of Universal Law: ``When broke, I will falsely promise to repay a loan
to get some quick cash." This maxim fails the universalizability test because in a world where everyone
falsely promises to repay loans, no one will believe promises anymore, so the maxim will no longer serve
its intended purpose (getting some quick cash). Making this judgement requires understanding enough about
the system of promising to realize that it breaks down if everyone abuses it in this manner. This is a
kind of common sense reasoning that an automated Kantian agent would need. This need is not unique to 
Kantian ethics; consequentialists agents need this kind of common sense to determine the consequences of 
an action and virtue ethical agents need this kind of common sense to determine which virtues an action
reflects. Making any ethical judgement requires relatively robust conceptions of the action or situation
at hand, falsely promising in this case. The advantage of Kantian ethics is that this is all the common 
sense that it requires, whereas a consequentialist or virtue ethical agent will require much more. All
moral theories evaluating falsely promising will a robust definition of the convention of promising, 
but consequentialism and virtue ethics will also require additional information about consequences or character
that Kantian ethics will not. Thus, although the need for common sense poses a challenge to automated
Kantian ethics, this challenge is more acute for consequentialism or virtue ethics so Kantian ethics remains
 within the closest reach of automation.%
\end{isamarkuptext}\isamarkuptrue%
%
\isadelimdocument
%
\endisadelimdocument
%
\isatagdocument
%
\isamarkupsubsection{Dyadic Deontic Logic \label{ddl}%
}
\isamarkuptrue%
%
\endisatagdocument
{\isafolddocument}%
%
\isadelimdocument
%
\endisadelimdocument
%
\begin{isamarkuptext}%
I formalize Kantian ethics by representing it as an axiom on top of a base logic. In this section, 
I present the logical background necessary to understand my work and my choice of Dyadic Deontic Logic (DDL).%
\end{isamarkuptext}\isamarkuptrue%
%
\begin{isamarkuptext}%
Traditional modal logics include the necessitation operator, denoted as $\Box$. In simple modal logic
using the Kripke semantics, $\Box p$ is true at a world $w$ if $p$ is true at all of $w$'s neighbors \cite{cresswell}. 
These logics usually also contain the possibility operator $\diamond$, where
 $\diamond p \iff \sim \Box \sim p$. Additionally, modal logics include operators of propositional 
logic like $\sim, \wedge, \vee, \rightarrow$.

A deontic logic is a special kind of modal logic designed to reason about obligation. Standard deontic
logic \citep{cresswell, sep-logic-deontic} replaces $\Box$ with the obligation operator
$O$, and $\diamond$ with the permissibility operator $P$. Using the Kripke semantics for $O$, $O p$ 
is true at $w$ if $p$ is true at all  ideal deontic alternatives to $w$. The $O$ operator in SDL
takes a single argument (the formula that is obligatory), and is thus called a monadic deontic operator.

 While SDL is appreciable for its simplicity, it suffers a variety of well-documented paradoxes, 
including contrary-to-duty paradoxes \footnote{The paradigm case of a contrary-to-duty paradox is the 
Chisholm paradox. Consider the following statements: \begin{enumerate}
\item It ought to be that Tom helps his neighbors
\item It ought to be that if Tom helps his neighbors, he tells them he is coming
\item If Tom does not help his neighbors, he ought not tell them that he is coming
\item Tom does not help his neighbors
\end{enumerate} 
These premises contradict themselves, because items (2)-(4) imply that Tom ought not help his neighbors. The 
contradiction results because the logic cannot handle violations of duty mixed with
conditionals. \citep{chisholm, ctd}
}. 
In situations where duty is violated, the logic breaks down 
and produces paradoxical results. Thus, I use an improved deontic logic instead of SDL for this work.%
\end{isamarkuptext}\isamarkuptrue%
%
\begin{isamarkuptext}%
I use as my base logic Carmo and Jones's dyadic deontic logic, or DDL, which improves on SDL \cite{CJDDL}. 
It introduces a dyadic obligation operator $O\{A \vert B\}$ 
to represent the sentence ``A is obligated in the context B". This gracefully handles contrary-to-duty
conditionals. The obligation operator uses a neighborhood semantics \cite{neighborhood1, neighborhood2}, instead of the Kripke semantics. 
Carmo and Jones define a function $ob$ that maps from worlds to sets of sets of worlds. Intuitively, 
each world is mapped to the set of propositions obligated at that world, where a proposition $p$ is defined as 
the worlds at which the $p$ is true.

DDL also includes other modal operators. In addition to $\Box$ and $\diamond$, DDL also has a notion
of actual obligation and possible obligation, represented by operators $O_a$ and $O_p$ respectively. 
These notions are accompanied by the corresponding modal operators $\Box_a, \diamond_a, \Box_p, \diamond_p$. 
These operators use a Kripke semantics, with the functions $av$ and $pv$ mapping a world $w$ to the set 
of corresponding actual or possible versions of $w$. 

For more of fine-grained properties of DDL see \citep{CJDDL} or this project's source code. DDL is a heavy logic and contains modal operators 
that aren't necessary for my analysis. While this expressivity is powerful, it may also cause performance
impacts. DDL has a large set of axioms involving quantification over complex higher-order
logical expressions. Proofs involving these axioms will be computationally expensive. I do not run 
into performance issues in my system, but future work may choose to embed a less complicated logic.%
\end{isamarkuptext}\isamarkuptrue%
%
\isadelimdocument
%
\endisadelimdocument
%
\isatagdocument
%
\isamarkupsubsection{Isabelle/HOL \label{isabelle}%
}
\isamarkuptrue%
%
\endisatagdocument
{\isafolddocument}%
%
\isadelimdocument
%
\endisadelimdocument
%
\begin{isamarkuptext}%
The final component of my project is the automated theorem prover I use to automate my formalization.
Isabelle/HOL is an interactive proof assistant built on Haskell and Scala \cite{isabelle}. It 
allows the user to define types, functions, definitions, and axiom systems. It has built-in support for both
automatic and interactive/manual theorem proving. 

I started my project by reimplementing Benzmueller, Farjami, and Parent's implementation 
of DDL in Isabelle/HOL \cite{BFP, logikey}. This helped me learn how to use Isabelle/HOL, and the implementation showcased in the 
next few sections demonstrates the power of Isabelle.

Benzmueller, Farjami, and Parent use a shallow semantic embedding. This kind of embedding models the semantics of DDL as 
constants in HOL and axioms as constraints on DDL models. This document will contain a subset of my 
implementation that is particularly interesting and relevant to understanding the rest of the project. 
For the complete implementation, see the source code in \isatt{paper22.thy}.%
\end{isamarkuptext}\isamarkuptrue%
%
\isadelimdocument
%
\endisadelimdocument
%
\isatagdocument
%
\isamarkupsubsubsection{System Definition%
}
\isamarkuptrue%
%
\endisatagdocument
{\isafolddocument}%
%
\isadelimdocument
%
\endisadelimdocument
%
\begin{isamarkuptext}%
The first step in embedding a logic in Isabelle is defining the relevant terms and types.%
\end{isamarkuptext}\isamarkuptrue%
\isacommand{typedecl}\isamarkupfalse%
\ i\ %
\isamarkupcmt{i is the type for a set of worlds.%
}\isanewline
\isanewline
\isacommand{type{\isacharunderscore}synonym}\isamarkupfalse%
\ t\ {\isacharequal}\ {\isachardoublequoteopen}{\isacharparenleft}i\ {\isasymRightarrow}\ bool{\isacharparenright}{\isachardoublequoteclose}\ %
\isamarkupcmt{t represents a set of DDL formulae.%
}\isanewline
%
\isamarkupcmt{A set of formulae is defined by its truth value at a set of worlds. For example, the set \{True\}
would be true at any set of worlds.%
}\isanewline
%
\begin{isamarkuptext}%
The main accessibility relation that I will use is the $ob$ relation:%
\end{isamarkuptext}\isamarkuptrue%
\isacommand{consts}\isamarkupfalse%
\ ob{\isacharcolon}{\isacharcolon}{\isachardoublequoteopen}t\ {\isasymRightarrow}\ {\isacharparenleft}t\ {\isasymRightarrow}\ bool{\isacharparenright}{\isachardoublequoteclose}\ \ %
\isamarkupcmt{set of propositions obligatory in this context%
}\isanewline
\ %
\isamarkupcmt{ob(context)(term) is True if the term is obligatory in this context%
}\isanewline
%
\isadelimdocument
%
\endisadelimdocument
%
\isatagdocument
%
\isamarkupsubsubsection{Axiomatization%
}
\isamarkuptrue%
%
\endisatagdocument
{\isafolddocument}%
%
\isadelimdocument
%
\endisadelimdocument
%
\begin{isamarkuptext}%
For a semantic embedding, axioms are modelled as restrictions on models of the system. In this case,
a model is specificied by the relevant accessibility relations, so it suffices to place conditions on 
the accessibility relations. These axioms can be quite unweildy, so luckily I was able to lift BFP's 
implementation of Carmo and Jones's original axioms directly \citep{BFP}. Here's an example of an axiom:%
\end{isamarkuptext}\isamarkuptrue%
\isanewline
\isakeyword{and}\ ax{\isacharunderscore}{\isadigit{5}}d{\isacharcolon}\ {\isachardoublequoteopen}{\isasymforall}X\ Y\ Z{\isachardot}\ {\isacharparenleft}{\isacharparenleft}{\isasymforall}w{\isachardot}\ Y{\isacharparenleft}w{\isacharparenright}{\isasymlongrightarrow}X{\isacharparenleft}w{\isacharparenright}{\isacharparenright}\ {\isasymand}\ ob{\isacharparenleft}X{\isacharparenright}{\isacharparenleft}Y{\isacharparenright}\ {\isasymand}\ {\isacharparenleft}{\isasymforall}w{\isachardot}\ X{\isacharparenleft}w{\isacharparenright}{\isasymlongrightarrow}Z{\isacharparenleft}w{\isacharparenright}{\isacharparenright}{\isacharparenright}\ \isanewline
\ \ {\isasymlongrightarrow}ob{\isacharparenleft}Z{\isacharparenright}{\isacharparenleft}{\isasymlambda}w{\isachardot}{\isacharparenleft}Z{\isacharparenleft}w{\isacharparenright}\ {\isasymand}\ {\isasymnot}X{\isacharparenleft}w{\isacharparenright}{\isacharparenright}\ {\isasymor}\ Y{\isacharparenleft}w{\isacharparenright}{\isacharparenright}{\isachardoublequoteclose}\isanewline
%
\isamarkupcmt{If some subset Y of X is obligatory in the context X, then in a larger context Z,
 any obligatory proposition must either be in Y or in Z-X. Intuitively, expanding the context can't 
cause something unobligatory to become obligatory, so the obligation operator is monotonically increasing
 with respect to changing contexts.%
}\isanewline
%
\isadelimdocument
%
\endisadelimdocument
%
\isatagdocument
%
\isamarkupsubsubsection{Syntax%
}
\isamarkuptrue%
%
\endisatagdocument
{\isafolddocument}%
%
\isadelimdocument
%
\endisadelimdocument
%
\begin{isamarkuptext}%
The syntax that I will work with is defined as abbreviations. Each DDL operator is represented 
as a HOL formula. Isabelle automatically unfolds formulae defined with the \isatt{abbreviation} command 
whenever they are applied. While the shallow embedding is performant (because it uses Isabelle's original 
syntax tree), abbreviations may hurt performance. In some complicated proofs, we want to control definition
unfolding. Benzmueller and Parent told me that the performance cost of abbreviations can 
be mitigated using a definition instead.%
\end{isamarkuptext}\isamarkuptrue%
%
\begin{isamarkuptext}%
Modal operators will be useful for my purposes, but the implementation is pretty standard.%
\end{isamarkuptext}\isamarkuptrue%
\isacommand{abbreviation}\isamarkupfalse%
\ ddlbox{\isacharcolon}{\isacharcolon}{\isachardoublequoteopen}t{\isasymRightarrow}t{\isachardoublequoteclose}\ {\isacharparenleft}{\isachardoublequoteopen}{\isasymbox}{\isachardoublequoteclose}{\isacharparenright}\ \isanewline
\ \ \isakeyword{where}\ {\isachardoublequoteopen}{\isasymbox}\ A\ {\isasymequiv}\ {\isasymlambda}w{\isachardot}{\isasymforall}y{\isachardot}\ A{\isacharparenleft}y{\isacharparenright}{\isachardoublequoteclose}\ \isanewline
\isacommand{abbreviation}\isamarkupfalse%
\ ddldiamond{\isacharcolon}{\isacharcolon}{\isachardoublequoteopen}t\ {\isasymRightarrow}\ t{\isachardoublequoteclose}\ {\isacharparenleft}{\isachardoublequoteopen}{\isasymdiamond}{\isachardoublequoteclose}{\isacharparenright}\isanewline
\ \ \isakeyword{where}\ {\isachardoublequoteopen}{\isasymdiamond}A\ {\isasymequiv}\ \isactrlbold {\isasymnot}{\isacharparenleft}{\isasymbox}{\isacharparenleft}\isactrlbold {\isasymnot}A{\isacharparenright}{\isacharparenright}{\isachardoublequoteclose}%
\begin{isamarkuptext}%
The most important operator for our purposes is the obligation operator.%
\end{isamarkuptext}\isamarkuptrue%
\isacommand{abbreviation}\isamarkupfalse%
\ ddlob{\isacharcolon}{\isacharcolon}{\isachardoublequoteopen}t{\isasymRightarrow}t{\isasymRightarrow}t{\isachardoublequoteclose}\ {\isacharparenleft}{\isachardoublequoteopen}O{\isacharbraceleft}{\isacharunderscore}{\isacharbar}{\isacharunderscore}{\isacharbraceright}{\isachardoublequoteclose}{\isacharparenright}\isanewline
\ \ \isakeyword{where}\ {\isachardoublequoteopen}O{\isacharbraceleft}B{\isacharbar}A{\isacharbraceright}\ {\isasymequiv}\ {\isasymlambda}\ w{\isachardot}\ ob{\isacharparenleft}A{\isacharparenright}{\isacharparenleft}B{\isacharparenright}{\isachardoublequoteclose}\isanewline
%
\isamarkupcmt{O$\{B \vert A\}$ can be read as ``B is obligatory in the context A"%
}\isanewline
%
\begin{isamarkuptext}%
While DDL is powerful because of its support for a dyadic obligation operator, in many cases 
we need a monadic obligation operator. Below is some syntactic sugar for a monadic obligation operator.%
\end{isamarkuptext}\isamarkuptrue%
\isacommand{abbreviation}\isamarkupfalse%
\ ddltrue{\isacharcolon}{\isacharcolon}{\isachardoublequoteopen}t{\isachardoublequoteclose}\ {\isacharparenleft}{\isachardoublequoteopen}\isactrlbold {\isasymtop}{\isachardoublequoteclose}{\isacharparenright}\isanewline
\ \ \isakeyword{where}\ {\isachardoublequoteopen}\isactrlbold {\isasymtop}\ {\isasymequiv}\ {\isasymlambda}w{\isachardot}\ True{\isachardoublequoteclose}\isanewline
\isacommand{abbreviation}\isamarkupfalse%
\ ddlfalse{\isacharcolon}{\isacharcolon}{\isachardoublequoteopen}t{\isachardoublequoteclose}\ {\isacharparenleft}{\isachardoublequoteopen}\isactrlbold {\isasymbottom}{\isachardoublequoteclose}{\isacharparenright}\isanewline
\ \ \isakeyword{where}\ {\isachardoublequoteopen}\isactrlbold {\isasymbottom}\ {\isasymequiv}\ {\isasymlambda}w{\isachardot}\ False{\isachardoublequoteclose}\isanewline
\isacommand{abbreviation}\isamarkupfalse%
\ ddlob{\isacharunderscore}normal{\isacharcolon}{\isacharcolon}{\isachardoublequoteopen}t{\isasymRightarrow}t{\isachardoublequoteclose}\ {\isacharparenleft}{\isachardoublequoteopen}O\ {\isacharbraceleft}{\isacharunderscore}{\isacharbraceright}{\isachardoublequoteclose}{\isacharparenright}\isanewline
\ \ \isakeyword{where}\ {\isachardoublequoteopen}{\isacharparenleft}O\ {\isacharbraceleft}A{\isacharbraceright}{\isacharparenright}\ {\isasymequiv}\ {\isacharparenleft}O{\isacharbraceleft}A{\isacharbar}\isactrlbold {\isasymtop}{\isacharbraceright}{\isacharparenright}\ {\isachardoublequoteclose}\isanewline
%
\isamarkupcmt{Intuitively, the context \texttt{True} is the widest context possible because \texttt{True} holds at all worlds.%
}%
\begin{isamarkuptext}%
Validity will be useful when discussing metalogical/ethical properties.%
\end{isamarkuptext}\isamarkuptrue%
\isacommand{abbreviation}\isamarkupfalse%
\ ddlvalid{\isacharcolon}{\isacharcolon}{\isachardoublequoteopen}t{\isasymRightarrow}bool{\isachardoublequoteclose}\ {\isacharparenleft}{\isachardoublequoteopen}{\isasymTurnstile}{\isacharunderscore}{\isachardoublequoteclose}{\isacharparenright}\isanewline
\ \ \isakeyword{where}\ {\isachardoublequoteopen}{\isasymTurnstile}A\ {\isasymequiv}\ {\isasymforall}w{\isachardot}\ A\ w{\isachardoublequoteclose}\isanewline
%
\isadelimdocument
%
\endisadelimdocument
%
\isatagdocument
%
\isamarkupsubsubsection{Syntactic Properties%
}
\isamarkuptrue%
%
\endisatagdocument
{\isafolddocument}%
%
\isadelimdocument
%
\endisadelimdocument
%
\begin{isamarkuptext}%
One way to show that a semantic embedding is complete is to show that the syntactic specification
of the theory (axioms) are valid for this semantics - so to show that every axiom holds at every 
world. Benzmueller, Farjami, and Parent provide a complete treatment of the completeness of their embedding, but I 
will include selected axioms that are particularly interesting here. This section also demonstrates many
of the relevant features of Isabelle/HOL for my project.%
\end{isamarkuptext}\isamarkuptrue%
%
\begin{isamarkuptext}%
\textbf{Consistency}%
\end{isamarkuptext}\isamarkuptrue%
\isacommand{lemma}\isamarkupfalse%
\ True\ \isacommand{nitpick}\isamarkupfalse%
\ {\isacharbrackleft}satisfy{\isacharcomma}user{\isacharunderscore}axioms{\isacharcomma}format{\isacharequal}{\isadigit{2}}{\isacharbrackright}%
\isadelimproof
\ %
\endisadelimproof
%
\isatagproof
\isacommand{by}\isamarkupfalse%
\ simp\isanewline
%
\isamarkupcmt{Isabelle has built-in support for Nitpick, a model checker. 
Nitpick successfully found a model satisfying these axioms so the system is consistent.%
}\isanewline
%
\isamarkupcmt{\color{blue} Nitpick found a model for card i = 1:

  Empty assignment \color{black}%
}%
\endisatagproof
{\isafoldproof}%
%
\isadelimproof
%
\endisadelimproof
%
\begin{isamarkuptext}%
Nitpick \cite{nitpick} can generate models or countermodels, so it's useful to falsify potential
theorems, as well as to show consistency. {\color{red} by simp} indicates the proof method. In this 
case, {\color{red} simp} indicates the Simplification proof method, which involves unfolding definitions
and applying theorems directly. HOL has $True$ as a theorem, which is why this theorem was so easy to prove.%
\end{isamarkuptext}\isamarkuptrue%
%
\isadelimtheory
%
\endisadelimtheory
%
\isatagtheory
%
\endisatagtheory
{\isafoldtheory}%
%
\isadelimtheory
%
\endisadelimtheory
%
\end{isabellebody}%
\endinput
%:%file=~/Desktop/cs91r/paper/thesis_2_methods.thy%:%
%:%24=6%:%
%:%36=8%:%
%:%37=9%:%
%:%38=10%:%
%:%39=11%:%
%:%40=12%:%
%:%41=13%:%
%:%42=14%:%
%:%43=15%:%
%:%44=16%:%
%:%45=17%:%
%:%46=18%:%
%:%47=19%:%
%:%48=20%:%
%:%49=21%:%
%:%50=22%:%
%:%51=23%:%
%:%52=24%:%
%:%53=25%:%
%:%62=27%:%
%:%74=30%:%
%:%75=31%:%
%:%76=32%:%
%:%77=33%:%
%:%78=34%:%
%:%79=35%:%
%:%80=36%:%
%:%81=37%:%
%:%82=38%:%
%:%83=39%:%
%:%84=40%:%
%:%85=41%:%
%:%86=42%:%
%:%87=43%:%
%:%88=44%:%
%:%89=45%:%
%:%90=46%:%
%:%91=47%:%
%:%92=48%:%
%:%93=49%:%
%:%94=50%:%
%:%95=51%:%
%:%96=52%:%
%:%97=53%:%
%:%106=56%:%
%:%118=58%:%
%:%119=59%:%
%:%120=60%:%
%:%121=61%:%
%:%122=62%:%
%:%123=63%:%
%:%124=64%:%
%:%128=66%:%
%:%129=67%:%
%:%130=68%:%
%:%131=69%:%
%:%132=70%:%
%:%133=71%:%
%:%134=72%:%
%:%135=73%:%
%:%136=74%:%
%:%137=75%:%
%:%138=76%:%
%:%139=77%:%
%:%140=78%:%
%:%141=79%:%
%:%142=80%:%
%:%143=81%:%
%:%144=82%:%
%:%145=83%:%
%:%146=84%:%
%:%147=85%:%
%:%148=86%:%
%:%149=87%:%
%:%150=88%:%
%:%151=89%:%
%:%152=90%:%
%:%161=93%:%
%:%173=95%:%
%:%174=96%:%
%:%175=97%:%
%:%176=98%:%
%:%177=99%:%
%:%178=100%:%
%:%179=101%:%
%:%180=102%:%
%:%181=103%:%
%:%185=105%:%
%:%189=107%:%
%:%190=108%:%
%:%191=109%:%
%:%192=110%:%
%:%193=111%:%
%:%194=112%:%
%:%195=113%:%
%:%196=114%:%
%:%197=115%:%
%:%198=116%:%
%:%199=117%:%
%:%200=118%:%
%:%201=119%:%
%:%202=120%:%
%:%203=121%:%
%:%204=122%:%
%:%205=123%:%
%:%206=124%:%
%:%207=125%:%
%:%208=126%:%
%:%209=127%:%
%:%210=128%:%
%:%211=129%:%
%:%212=130%:%
%:%213=131%:%
%:%214=132%:%
%:%215=133%:%
%:%219=136%:%
%:%223=138%:%
%:%224=139%:%
%:%225=140%:%
%:%226=141%:%
%:%227=142%:%
%:%228=143%:%
%:%229=144%:%
%:%230=145%:%
%:%231=146%:%
%:%232=147%:%
%:%233=148%:%
%:%234=149%:%
%:%235=150%:%
%:%236=151%:%
%:%237=152%:%
%:%238=153%:%
%:%239=154%:%
%:%240=155%:%
%:%244=157%:%
%:%248=159%:%
%:%249=160%:%
%:%250=161%:%
%:%251=162%:%
%:%252=163%:%
%:%253=164%:%
%:%254=165%:%
%:%255=166%:%
%:%256=167%:%
%:%257=168%:%
%:%258=169%:%
%:%259=170%:%
%:%260=171%:%
%:%261=172%:%
%:%262=173%:%
%:%263=174%:%
%:%264=175%:%
%:%265=176%:%
%:%266=177%:%
%:%267=178%:%
%:%268=179%:%
%:%272=182%:%
%:%276=185%:%
%:%277=186%:%
%:%278=187%:%
%:%279=188%:%
%:%280=189%:%
%:%281=190%:%
%:%282=191%:%
%:%283=192%:%
%:%284=193%:%
%:%285=194%:%
%:%286=195%:%
%:%287=196%:%
%:%288=197%:%
%:%289=198%:%
%:%290=199%:%
%:%291=200%:%
%:%300=203%:%
%:%312=205%:%
%:%313=206%:%
%:%314=207%:%
%:%315=208%:%
%:%316=209%:%
%:%317=210%:%
%:%318=211%:%
%:%319=212%:%
%:%323=216%:%
%:%324=217%:%
%:%325=218%:%
%:%326=219%:%
%:%327=220%:%
%:%328=221%:%
%:%329=222%:%
%:%330=223%:%
%:%331=224%:%
%:%332=225%:%
%:%336=227%:%
%:%340=229%:%
%:%341=230%:%
%:%342=231%:%
%:%343=232%:%
%:%344=233%:%
%:%345=234%:%
%:%346=235%:%
%:%347=236%:%
%:%348=237%:%
%:%349=238%:%
%:%350=239%:%
%:%351=240%:%
%:%352=241%:%
%:%353=242%:%
%:%354=243%:%
%:%355=244%:%
%:%356=245%:%
%:%357=246%:%
%:%358=247%:%
%:%359=248%:%
%:%360=249%:%
%:%361=250%:%
%:%362=251%:%
%:%363=252%:%
%:%364=253%:%
%:%365=254%:%
%:%366=255%:%
%:%375=258%:%
%:%387=260%:%
%:%388=261%:%
%:%389=262%:%
%:%390=263%:%
%:%391=264%:%
%:%392=265%:%
%:%393=266%:%
%:%394=267%:%
%:%395=268%:%
%:%396=269%:%
%:%397=270%:%
%:%398=271%:%
%:%399=272%:%
%:%400=273%:%
%:%401=274%:%
%:%402=275%:%
%:%403=276%:%
%:%404=277%:%
%:%405=278%:%
%:%406=279%:%
%:%407=280%:%
%:%408=281%:%
%:%409=282%:%
%:%410=283%:%
%:%411=284%:%
%:%412=285%:%
%:%413=286%:%
%:%414=287%:%
%:%415=288%:%
%:%416=289%:%
%:%417=290%:%
%:%418=291%:%
%:%419=292%:%
%:%420=293%:%
%:%421=294%:%
%:%422=295%:%
%:%423=296%:%
%:%424=297%:%
%:%425=298%:%
%:%426=299%:%
%:%427=300%:%
%:%428=301%:%
%:%429=302%:%
%:%430=303%:%
%:%431=304%:%
%:%432=305%:%
%:%433=306%:%
%:%434=307%:%
%:%435=308%:%
%:%436=309%:%
%:%437=310%:%
%:%438=311%:%
%:%439=312%:%
%:%440=313%:%
%:%441=314%:%
%:%442=315%:%
%:%443=316%:%
%:%444=317%:%
%:%445=318%:%
%:%446=319%:%
%:%447=320%:%
%:%448=321%:%
%:%449=322%:%
%:%450=323%:%
%:%451=324%:%
%:%452=325%:%
%:%453=326%:%
%:%454=327%:%
%:%455=328%:%
%:%459=330%:%
%:%463=332%:%
%:%464=333%:%
%:%465=334%:%
%:%466=335%:%
%:%467=336%:%
%:%468=337%:%
%:%469=338%:%
%:%470=339%:%
%:%471=340%:%
%:%472=341%:%
%:%473=342%:%
%:%474=343%:%
%:%475=344%:%
%:%476=345%:%
%:%477=346%:%
%:%478=347%:%
%:%479=348%:%
%:%480=349%:%
%:%481=350%:%
%:%482=351%:%
%:%483=352%:%
%:%484=353%:%
%:%485=354%:%
%:%486=355%:%
%:%490=359%:%
%:%494=361%:%
%:%495=362%:%
%:%496=363%:%
%:%497=364%:%
%:%498=365%:%
%:%499=366%:%
%:%500=367%:%
%:%501=368%:%
%:%502=369%:%
%:%503=370%:%
%:%504=371%:%
%:%505=372%:%
%:%506=373%:%
%:%507=374%:%
%:%508=375%:%
%:%509=376%:%
%:%510=377%:%
%:%511=378%:%
%:%512=379%:%
%:%513=380%:%
%:%514=381%:%
%:%515=382%:%
%:%516=383%:%
%:%517=384%:%
%:%518=385%:%
%:%519=386%:%
%:%520=387%:%
%:%529=390%:%
%:%541=392%:%
%:%542=393%:%
%:%546=396%:%
%:%547=397%:%
%:%548=398%:%
%:%549=399%:%
%:%550=400%:%
%:%551=401%:%
%:%552=402%:%
%:%553=403%:%
%:%554=404%:%
%:%555=405%:%
%:%556=406%:%
%:%557=407%:%
%:%558=408%:%
%:%559=409%:%
%:%560=410%:%
%:%561=411%:%
%:%562=412%:%
%:%563=413%:%
%:%564=414%:%
%:%565=415%:%
%:%566=416%:%
%:%567=417%:%
%:%568=418%:%
%:%569=419%:%
%:%570=420%:%
%:%571=421%:%
%:%575=423%:%
%:%576=424%:%
%:%577=425%:%
%:%578=426%:%
%:%579=427%:%
%:%580=428%:%
%:%581=429%:%
%:%582=430%:%
%:%583=431%:%
%:%584=432%:%
%:%585=433%:%
%:%586=434%:%
%:%587=435%:%
%:%588=436%:%
%:%589=437%:%
%:%590=438%:%
%:%591=439%:%
%:%592=440%:%
%:%593=441%:%
%:%602=444%:%
%:%614=446%:%
%:%615=447%:%
%:%616=448%:%
%:%617=449%:%
%:%618=450%:%
%:%619=451%:%
%:%620=452%:%
%:%621=453%:%
%:%622=454%:%
%:%623=455%:%
%:%624=456%:%
%:%625=457%:%
%:%626=458%:%
%:%635=461%:%
%:%647=463%:%
%:%649=465%:%
%:%650=465%:%
%:%651=465%:%
%:%652=465%:%
%:%653=466%:%
%:%654=467%:%
%:%655=467%:%
%:%656=467%:%
%:%657=467%:%
%:%659=468%:%
%:%660=469%:%
%:%661=469%:%
%:%664=483%:%
%:%666=485%:%
%:%667=485%:%
%:%668=485%:%
%:%669=485%:%
%:%670=486%:%
%:%671=486%:%
%:%672=486%:%
%:%680=490%:%
%:%692=492%:%
%:%693=493%:%
%:%694=494%:%
%:%695=495%:%
%:%697=519%:%
%:%698=520%:%
%:%699=521%:%
%:%701=522%:%
%:%702=523%:%
%:%703=524%:%
%:%704=525%:%
%:%705=525%:%
%:%713=532%:%
%:%725=534%:%
%:%726=535%:%
%:%727=536%:%
%:%728=537%:%
%:%729=538%:%
%:%730=539%:%
%:%734=555%:%
%:%736=556%:%
%:%737=556%:%
%:%738=557%:%
%:%739=558%:%
%:%740=558%:%
%:%741=559%:%
%:%743=561%:%
%:%745=562%:%
%:%746=562%:%
%:%747=563%:%
%:%749=564%:%
%:%750=564%:%
%:%753=584%:%
%:%754=585%:%
%:%756=586%:%
%:%757=586%:%
%:%758=587%:%
%:%759=588%:%
%:%760=588%:%
%:%761=589%:%
%:%762=590%:%
%:%763=590%:%
%:%764=591%:%
%:%766=592%:%
%:%769=594%:%
%:%771=595%:%
%:%772=595%:%
%:%773=596%:%
%:%781=605%:%
%:%793=607%:%
%:%794=608%:%
%:%795=609%:%
%:%796=610%:%
%:%797=611%:%
%:%801=613%:%
%:%803=614%:%
%:%804=614%:%
%:%805=614%:%
%:%807=614%:%
%:%811=614%:%
%:%812=614%:%
%:%814=615%:%
%:%815=616%:%
%:%816=616%:%
%:%818=617%:%
%:%819=618%:%
%:%820=619%:%
%:%830=621%:%
%:%831=622%:%
%:%832=623%:%
%:%833=624%:%