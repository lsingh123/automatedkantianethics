%
\begin{isabellebody}%
\setisabellecontext{paper{\isadigit{2}}{\isadigit{2}}}%
%
\isadelimtheory
%
\endisadelimtheory
%
\isatagtheory
%
\endisatagtheory
{\isafoldtheory}%
%
\isadelimtheory
%
\endisadelimtheory
%
\isadelimdocument
%
\endisadelimdocument
%
\isatagdocument
%
\isamarkupsubsection{Isabelle/HOL Implementation%
}
\isamarkuptrue%
%
\endisatagdocument
{\isafolddocument}%
%
\isadelimdocument
%
\endisadelimdocument
%
\begin{isamarkuptext}%
Isabelle/HOL is an interactive proof assistant \cite{isabelle} built on Haskell and Scala. It 
allows the user to define types, functions, definitions, and axiom systems. It has built-in support for both
automatic and interactive/manual theorem proving. 

I started my project by reimplementing Benzmueller, Farjami, and Parent's \cite{BFP, logikey} implementation 
of DDL in Isabelle/HOL. This helped me learn how to use Isabelle/HOL, and the implementation showcased in the 
next few sections demonstrates the power of Isabelle.

BFP use a shallow semantic embedding. This kind of embedding models the semantics of DDL as 
constants in HOL and axioms as constraints on DDL models. This document will contain a subset of my 
implementation that is particularly interesting and relevant to understanding the rest of the project. 
For the complete implementation, see the source code in \isatt{paper22.thy}.%
\end{isamarkuptext}\isamarkuptrue%
%
\isadelimdocument
%
\endisadelimdocument
%
\isatagdocument
%
\isamarkupsubsubsection{System Definition%
}
\isamarkuptrue%
%
\endisatagdocument
{\isafolddocument}%
%
\isadelimdocument
%
\endisadelimdocument
%
\begin{isamarkuptext}%
The first step in embedding a logic in Isabelle is defining the relevant terms and types.%
\end{isamarkuptext}\isamarkuptrue%
\isacommand{typedecl}\isamarkupfalse%
\ i\ %
\isamarkupcmt{i is the type for a set of worlds.%
}\isanewline
\isanewline
\isacommand{type{\isacharunderscore}synonym}\isamarkupfalse%
\ t\ {\isacharequal}\ {\isachardoublequoteopen}{\isacharparenleft}i\ {\isasymRightarrow}\ bool{\isacharparenright}{\isachardoublequoteclose}\ %
\isamarkupcmt{t represents a set of DDL formulas.%
}\isanewline
%
\isamarkupcmt{A set of formulae is defined by its truth value at a set of worlds. For example, the set {``True"}
would be true at any set of worlds.%
}\isanewline
%
\begin{isamarkuptext}%
The main accessibility relation that I will use is the $ob$ relation:%
\end{isamarkuptext}\isamarkuptrue%
\isacommand{consts}\isamarkupfalse%
\ ob{\isacharcolon}{\isacharcolon}{\isachardoublequoteopen}t\ {\isasymRightarrow}\ {\isacharparenleft}t\ {\isasymRightarrow}\ bool{\isacharparenright}{\isachardoublequoteclose}\ \ %
\isamarkupcmt{set of propositions obligatory in this "context"%
}\isanewline
\ %
\isamarkupcmt{ob(context)(term) is True if t is obligatory in this context%
}\isanewline
%
\isadelimdocument
%
\endisadelimdocument
%
\isatagdocument
%
\isamarkupsubsubsection{Axiomatization%
}
\isamarkuptrue%
%
\endisatagdocument
{\isafolddocument}%
%
\isadelimdocument
%
\endisadelimdocument
%
\begin{isamarkuptext}%
For a semantic embedding, axioms are modelled as restrictions on models of the system. In this case,
a model is specificied by the relevant accessibility relations, so it suffices to place conditions on 
the accessibility relations. These axioms can be quite unweildy, so luckily I was able to lift BFP's \cite{BFP}
implementation of Carmo and Jones's original axioms directly. Here's an example of an axiom:%
\end{isamarkuptext}\isamarkuptrue%
\isanewline
\isakeyword{and}\ ax{\isacharunderscore}{\isadigit{5}}d{\isacharcolon}\ {\isachardoublequoteopen}{\isasymforall}X\ Y\ Z{\isachardot}\ {\isacharparenleft}{\isacharparenleft}{\isasymforall}w{\isachardot}\ Y{\isacharparenleft}w{\isacharparenright}{\isasymlongrightarrow}X{\isacharparenleft}w{\isacharparenright}{\isacharparenright}\ {\isasymand}\ ob{\isacharparenleft}X{\isacharparenright}{\isacharparenleft}Y{\isacharparenright}\ {\isasymand}\ {\isacharparenleft}{\isasymforall}w{\isachardot}\ X{\isacharparenleft}w{\isacharparenright}{\isasymlongrightarrow}Z{\isacharparenleft}w{\isacharparenright}{\isacharparenright}{\isacharparenright}\ \isanewline
\ \ {\isasymlongrightarrow}ob{\isacharparenleft}Z{\isacharparenright}{\isacharparenleft}{\isasymlambda}w{\isachardot}{\isacharparenleft}Z{\isacharparenleft}w{\isacharparenright}\ {\isasymand}\ {\isasymnot}X{\isacharparenleft}w{\isacharparenright}{\isacharparenright}\ {\isasymor}\ Y{\isacharparenleft}w{\isacharparenright}{\isacharparenright}{\isachardoublequoteclose}\isanewline
%
\isamarkupcmt{If some subset Y of X is obligatory in the context X, then in a larger context Z,
 any obligatory proposition must either be in Y or in Z-X. Intuitively, expanding the context can't 
cause something unobligatory to become obligatory, so the obligation operator is monotonically increasing
 with respect to changing contexts.%
}\isanewline
%
\isadelimdocument
%
\endisadelimdocument
%
\isatagdocument
%
\isamarkupsubsubsection{Syntax%
}
\isamarkuptrue%
%
\endisatagdocument
{\isafolddocument}%
%
\isadelimdocument
%
\endisadelimdocument
%
\begin{isamarkuptext}%
The syntax that I will work with is defined as abbreviations. Each DDL operator is represented 
as a HOL formula. A formula defined with the \isatt{abbreviation} command is unfolded in every 
instance. While the shallow embedding is generally performant (because it uses Isabelle's original 
syntax tree), abbreviations may possibly hurt performance. In some complicated proofs, we want controlled
unfolding. Benzmueller and Parent told me that eventually, the performance cost of abbreviations can 
be mitigated using a definition instead.%
\end{isamarkuptext}\isamarkuptrue%
%
\begin{isamarkuptext}%
Modal operators will be useful for my purposes, but the implementation is pretty standard.%
\end{isamarkuptext}\isamarkuptrue%
\isacommand{abbreviation}\isamarkupfalse%
\ ddlbox{\isacharcolon}{\isacharcolon}{\isachardoublequoteopen}t{\isasymRightarrow}t{\isachardoublequoteclose}\ {\isacharparenleft}{\isachardoublequoteopen}{\isasymbox}{\isachardoublequoteclose}{\isacharparenright}\ \isanewline
\ \ \isakeyword{where}\ {\isachardoublequoteopen}{\isasymbox}\ A\ {\isasymequiv}\ {\isasymlambda}w{\isachardot}{\isasymforall}y{\isachardot}\ A{\isacharparenleft}y{\isacharparenright}{\isachardoublequoteclose}\ \isanewline
\isacommand{abbreviation}\isamarkupfalse%
\ ddldiamond{\isacharcolon}{\isacharcolon}{\isachardoublequoteopen}t\ {\isasymRightarrow}\ t{\isachardoublequoteclose}\ {\isacharparenleft}{\isachardoublequoteopen}{\isasymdiamond}{\isachardoublequoteclose}{\isacharparenright}\isanewline
\ \ \isakeyword{where}\ {\isachardoublequoteopen}{\isasymdiamond}A\ {\isasymequiv}\ \isactrlbold {\isasymnot}{\isacharparenleft}{\isasymbox}{\isacharparenleft}\isactrlbold {\isasymnot}A{\isacharparenright}{\isacharparenright}{\isachardoublequoteclose}%
\begin{isamarkuptext}%
The most important operator for our purposes is the obligation operator.%
\end{isamarkuptext}\isamarkuptrue%
\isacommand{abbreviation}\isamarkupfalse%
\ ddlob{\isacharcolon}{\isacharcolon}{\isachardoublequoteopen}t{\isasymRightarrow}t{\isasymRightarrow}t{\isachardoublequoteclose}\ {\isacharparenleft}{\isachardoublequoteopen}O{\isacharbraceleft}{\isacharunderscore}{\isacharbar}{\isacharunderscore}{\isacharbraceright}{\isachardoublequoteclose}{\isacharparenright}\isanewline
\ \ \isakeyword{where}\ {\isachardoublequoteopen}O{\isacharbraceleft}B{\isacharbar}A{\isacharbraceright}\ {\isasymequiv}\ {\isasymlambda}\ w{\isachardot}\ ob{\isacharparenleft}A{\isacharparenright}{\isacharparenleft}B{\isacharparenright}{\isachardoublequoteclose}\isanewline
%
\isamarkupcmt{O$\{B \vert A\}$ can be read as ``B is obligatory in the context A"%
}\isanewline
%
\begin{isamarkuptext}%
While DDL is powerful because of its support for a dyadic obligation operator, in many cases 
we need a monadic obligation operator. Below is some syntactic sugar for a monadic obligation operator.%
\end{isamarkuptext}\isamarkuptrue%
\isacommand{abbreviation}\isamarkupfalse%
\ ddltrue{\isacharcolon}{\isacharcolon}{\isachardoublequoteopen}t{\isachardoublequoteclose}\ {\isacharparenleft}{\isachardoublequoteopen}\isactrlbold {\isasymtop}{\isachardoublequoteclose}{\isacharparenright}\isanewline
\ \ \isakeyword{where}\ {\isachardoublequoteopen}\isactrlbold {\isasymtop}\ {\isasymequiv}\ {\isasymlambda}w{\isachardot}\ True{\isachardoublequoteclose}\isanewline
\isacommand{abbreviation}\isamarkupfalse%
\ ddlob{\isacharunderscore}normal{\isacharcolon}{\isacharcolon}{\isachardoublequoteopen}t{\isasymRightarrow}t{\isachardoublequoteclose}\ {\isacharparenleft}{\isachardoublequoteopen}O\ {\isacharbraceleft}{\isacharunderscore}{\isacharbraceright}{\isachardoublequoteclose}{\isacharparenright}\isanewline
\ \ \isakeyword{where}\ {\isachardoublequoteopen}{\isacharparenleft}O\ {\isacharbraceleft}A{\isacharbraceright}{\isacharparenright}\ {\isasymequiv}\ {\isacharparenleft}O{\isacharbraceleft}A{\isacharbar}\isactrlbold {\isasymtop}{\isacharbraceright}{\isacharparenright}\ {\isachardoublequoteclose}\isanewline
%
\isamarkupcmt{Intuitively, the context \texttt{True} is the widest context possible because \texttt{True} holds at all worlds.%
}%
\begin{isamarkuptext}%
Validity will be useful when discussing metalogical/ethical properties.%
\end{isamarkuptext}\isamarkuptrue%
\isacommand{abbreviation}\isamarkupfalse%
\ ddlvalid{\isacharcolon}{\isacharcolon}{\isachardoublequoteopen}t{\isasymRightarrow}bool{\isachardoublequoteclose}\ {\isacharparenleft}{\isachardoublequoteopen}{\isasymTurnstile}{\isacharunderscore}{\isachardoublequoteclose}{\isacharparenright}\isanewline
\ \ \isakeyword{where}\ {\isachardoublequoteopen}{\isasymTurnstile}A\ {\isasymequiv}\ {\isasymforall}w{\isachardot}\ A\ w{\isachardoublequoteclose}\isanewline
%
\isadelimdocument
%
\endisadelimdocument
%
\isatagdocument
%
\isamarkupsubsubsection{Syntactic Properties%
}
\isamarkuptrue%
%
\endisatagdocument
{\isafolddocument}%
%
\isadelimdocument
%
\endisadelimdocument
%
\begin{isamarkuptext}%
One way to show that a semantic embedding is complete is to show that the syntactic specification
of the theory (axioms) are valid for this semantics - so to show that every axiom holds at every 
world. BFP \cite{BFP} provide a complete treatment of the completeness of their embedding, but I 
will include selected axioms that are particularly interesting here. This section also demonstrates many
of the relevant features of Isabelle/HOL for my project.%
\end{isamarkuptext}\isamarkuptrue%
%
\begin{isamarkuptext}%
\textbf{Consistency}%
\end{isamarkuptext}\isamarkuptrue%
\isacommand{lemma}\isamarkupfalse%
\ True\ \isacommand{nitpick}\isamarkupfalse%
\ {\isacharbrackleft}satisfy{\isacharcomma}user{\isacharunderscore}axioms{\isacharcomma}format{\isacharequal}{\isadigit{2}}{\isacharbrackright}%
\isadelimproof
\ %
\endisadelimproof
%
\isatagproof
\isacommand{by}\isamarkupfalse%
\ simp\isanewline
%
\isamarkupcmt{Isabelle has built-in support for Nitpick, a model checker. 
Nitpick successfully found a model satisfying these axioms so the system is consistent.\cite{hintikka}%
}\isanewline
%
\isamarkupcmt{\color{blue} Nitpick found a model for card i = 1:

  Empty assignment \color{black}%
}%
\endisatagproof
{\isafoldproof}%
%
\isadelimproof
%
\endisadelimproof
%
\begin{isamarkuptext}%
{\color{red} Nitpick} \cite{nitpick} can generate models or countermodels, so it's useful to falsify potential
theorems, as well as to show consistency. {\color{red} by simp} indicates the proof method. In this 
case, {\color{red} simp} indicates the Simplification proof method, which involves unfolding definitions
and applying theorems directly. HOL has $True$ as a theorem, which is why this theorem was so easy to prove.%
\end{isamarkuptext}\isamarkuptrue%
%
\isadelimtheory
%
\endisadelimtheory
%
\isatagtheory
%
\endisatagtheory
{\isafoldtheory}%
%
\isadelimtheory
%
\endisadelimtheory
%
\end{isabellebody}%
\endinput
%:%file=~/Desktop/cs91r/paper/paper22.thy%:%
%:%24=6%:%
%:%36=8%:%
%:%37=9%:%
%:%38=10%:%
%:%39=11%:%
%:%40=12%:%
%:%41=13%:%
%:%42=14%:%
%:%43=15%:%
%:%44=16%:%
%:%45=17%:%
%:%46=18%:%
%:%47=19%:%
%:%56=22%:%
%:%68=24%:%
%:%70=26%:%
%:%71=26%:%
%:%72=26%:%
%:%73=26%:%
%:%74=27%:%
%:%75=28%:%
%:%76=28%:%
%:%77=28%:%
%:%78=28%:%
%:%80=29%:%
%:%81=30%:%
%:%82=30%:%
%:%85=44%:%
%:%87=46%:%
%:%88=46%:%
%:%89=46%:%
%:%90=46%:%
%:%91=47%:%
%:%92=47%:%
%:%93=47%:%
%:%101=51%:%
%:%113=53%:%
%:%114=54%:%
%:%115=55%:%
%:%116=56%:%
%:%118=80%:%
%:%119=81%:%
%:%120=82%:%
%:%122=83%:%
%:%123=84%:%
%:%124=85%:%
%:%125=86%:%
%:%126=86%:%
%:%134=93%:%
%:%146=95%:%
%:%147=96%:%
%:%148=97%:%
%:%149=98%:%
%:%150=99%:%
%:%151=100%:%
%:%155=116%:%
%:%157=117%:%
%:%158=117%:%
%:%159=118%:%
%:%160=119%:%
%:%161=119%:%
%:%162=120%:%
%:%164=122%:%
%:%166=123%:%
%:%167=123%:%
%:%168=124%:%
%:%170=125%:%
%:%171=125%:%
%:%174=145%:%
%:%175=146%:%
%:%177=147%:%
%:%178=147%:%
%:%179=148%:%
%:%180=149%:%
%:%181=149%:%
%:%182=150%:%
%:%184=151%:%
%:%187=153%:%
%:%189=154%:%
%:%190=154%:%
%:%191=155%:%
%:%199=162%:%
%:%211=164%:%
%:%212=165%:%
%:%213=166%:%
%:%214=167%:%
%:%215=168%:%
%:%219=170%:%
%:%221=171%:%
%:%222=171%:%
%:%223=171%:%
%:%225=171%:%
%:%229=171%:%
%:%230=171%:%
%:%232=172%:%
%:%233=173%:%
%:%234=173%:%
%:%236=174%:%
%:%237=175%:%
%:%238=176%:%
%:%248=178%:%
%:%249=179%:%
%:%250=180%:%
%:%251=181%:%