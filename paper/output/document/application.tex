%
\begin{isabellebody}%
\setisabellecontext{application}%
%
\isadelimtheory
%
\endisadelimtheory
%
\isatagtheory
%
\endisatagtheory
{\isafoldtheory}%
%
\isadelimtheory
%
\endisadelimtheory
%
\isadelimdocument
%
\endisadelimdocument
%
\isatagdocument
%
\isamarkupsection{Applications \label{sec:applications}%
}
\isamarkuptrue%
%
\endisatagdocument
{\isafolddocument}%
%
\isadelimdocument
%
\endisadelimdocument
%
\begin{isamarkuptext}%
In this section, I use my system to perform ``applied ethical reasoning." In Chapter 2
and Chapter 3, I performed metaethical reasoning while testing different formalizations of the FUL. Metaethical
reasoning analyzes properties of moral thought itself \citep{sepme}. Because metaethics is about ethics itself,
and not about the actual dilemmas that ethics is supposed to help us resolve, this kind of reasoning requires
very little additional knowledge beyond that of the ethical theory in question. Indeed, in previous chapters,
I perform metaethical reasoning using my system (which formalizes an ethical theory) and basic logical
facts and objects. This kind of reasoning contrasts with ``applied ethical reasoning," which is the use of
ethics to resolve ethical dilemmas and make judgements about what an agent should or should not do.
Applied ethical reasoning is about the situation an agent is in, not about an ethical theory as a whole,
abstract entity. In the application tests in previous chapters, I perform some toy examples of applied
ethical reasoning, but in this section, I expore how my system can perform extended applied ethical reasoning.
Metaethical reasoning is most interesting to philosophers who are trying to formulate the ``best" or 
most correct theory of ethics. Applied ethical reasoning, on the other hand, is useful to ordinary people
who are trying to decide how to live their lives. In order for automated ethics to guide AI agents, 
it must perform applied ethical reasoning. 

One challenge of applied ethical reasoning is that it requires more common sense knowledge than metaethical
reasoning. Applied ethical reasoning focuses on a particular ethical dilemma and thus requires enough 
common sense to understand the dilemma and options at hand. For example, an applied ethicist 
evaluating the permissibility of lying needs some robust definition of the term lying and likely some
understanding about the activities of communication and truth telling. In this section, I attempt to
endow my system with this kind of common sense in the specific case of lying. Indeed, in order for my 
system to be used to perform applied ethical reasoning, it will need to be equipped with a database
of common sense facts and definitions as I present in this section. My system will contribute the core
reasoning about the Formula of Universal Law, but this reasoning must be applied to objects that are defined
in this common sense database. In this section, I try to understand what this common sense reasoning looks like
for the specific example of lying. I will examine how different sets of common sense facts impact the 
judgements that my system comes to. 

Because these common sense facts can entirely determine my system's judgement, they are part of the trusted
code base for my system. In order for someone to trust my system's judgements, they must trust that the
common sense database is correct. Indeed, changing these common sense facts will change the judgements 
that my system makes. For example, if we define truth telling as an act that is self-contradictory (perhaps
by defining lying as ``$p \wedge \neg p$), then my system will output that truth telling is prohibited.
Malicious common sense facts and definitions will result in bad ethical judgements. In other words, garbage in, 
garbage out. This common sense reasoning is also not a part of the system I contribute here—in order for my
system to be used, future work must develop a common sense database for the particular situation at hand. 
The challenge of endowing automated ethical reasoners with common sense reasoning is not unique to my 
system, and virtually all prior attempts in machine ethics face similar challenges. Most prior attempts
sidestep this question, whereas I contribute an prototype implementation of one kind of common sense reasoning.

The specific kind of common sense reasoning required for automated Kantian ethics appears to be a challenge 
for automating Kantian ethics, and perhaps a reason that consequentialist or virtue ethical automated
agents would be within closer reach for automation. Ultimately, I still argue that Kantian ethics is 
easiest to automate because it will require fewer, less controversial common facts than other ethical
systems. As the examples in this section demonstrate, Kantian ethics requires a definition of lying (which any other 
theory would also requires) and the knowledge that if everyone lies in a given context, no one will believe 
each other in that particular context. This latter fact may not be required for every ethical theory, 
but is relatively uncontroversial. It is a kind of intuition about human behavior that is generally accepted. 
Neither a definition of lying nor this property of lying seem like unreasonable prerequisites for ethical reasoning. 

Consequentialism, on the other hand, would require much more specific data about the consequences of 
a lie, perhaps for the specific person's credibility, for the victim of the lie, for the third-parties 
watching the lie unfold. Indeed, consequentialism more numerous and specific judgements, all of which are 
likely to be more controversial than the two outlined above for Kantian ethics. Similarly, virtue ethics
would likely require information about the actor's entire moral character, including their attitude
towards the lie and their moral history. Virtue ethics would also require robust definitions of the 
relevant virtues, in addition to a definition of lying. Thus, while Kantian ethics requires some 
common sense reasoning, it requires fewer and less controversial background facts than other ethical theories.

This section will provide additional examples of the kinds of common sense required to get my system
off the ground. I will aim to use as lean and uncontroversial of a common sense database as possible
to achieve robust and powerful results. This serves as evidence for the ease of automating
Kantian ethics, an example of what additional work my system requires, and a demonstration of the contributions
that I make. These examples demonstrate that nuanced common sense facts can cause my system to contribute 
nuanced judgements faithful to philosophical literature. On one hand, this means that my system can perform
sophisticated ethical reasoning, but on the other, the quality of this reasoning is heavily reliant on 
the common sense database. Thus, my system \emph{could} make smart ethical judgements 
in practice, but getting to that point will require a robust, trusted common sense database.

\emph{Stick a bit here about the philosophical work that will go in this section}%
\end{isamarkuptext}\isamarkuptrue%
%
\isadelimdocument
%
\endisadelimdocument
%
\isatagdocument
%
\isamarkupsubsection{Simple Lying Examples \label{sec:lyingprelim}%
}
\isamarkuptrue%
%
\endisatagdocument
{\isafolddocument}%
%
\isadelimdocument
%
\endisadelimdocument
%
\begin{isamarkuptext}%
This chapter focuses on the example of lying because this case is robustly debated in the Kantian
literature. I draw on examples of ethical reasoning as presented in Korsgaard's Right to Lie.
She analyzes the example of someone who shows up at your door and asks, "Is Sara home?" Unbeknownst to them, 
you know that they want to know Sara's location in order to murder her. Ordinary moral intuition seems
to imply that in this situation, if Sara is home, you should lie and say that she is not in order to 
protect her from the murderer. However, the categorical imperative seems to prohibit lying in all 
circumstances. In this section, I will formalize Korsgaard's treatment of lying and joking under the 
categorical imperative, focusing on the common sense assumptions necessary to achieve her conclusions. 
In the next section, I will formalize the core of Korsgaard's argument that the categorical imperative
coheres with ordinary intuition and does not prohibit lying to the murderer to protect Sara.

First, Korsgaard argues that the categorical imperative appears to prohibit all lies because, when universalized, 
lies will no longer be believed, so it seems like lying could never be an effective way of achieving any goal. 
Korsgaard points out that ``we believe what is said to us in a given context because most of the time 
people in that context say what they really think" \citep[4]{KorsgaardRTL}. In order to formalize this argument,
I first need to define some relevant terms.%
\end{isamarkuptext}\isamarkuptrue%
\isacommand{consts}\isamarkupfalse%
\ believe{\isacharcolon}{\isacharcolon}{\isachardoublequoteopen}s{\isasymRightarrow}t{\isasymRightarrow}t{\isachardoublequoteclose}\ {\isacharparenleft}{\isachardoublequoteopen}{\isacharunderscore}\ believes\ {\isacharunderscore}{\isachardoublequoteclose}{\isacharparenright}\isanewline
%
\isamarkupcmt{Person s::subject believes sentence t::term. The concept of belief will play a crucial role both in 
the arguments for lying being prohibited and for the maxim of lying to the murderer being permissible. 
Logicians and epistemologists have developed robust, complex logics of belief and knowledge, but for the 
sake of this project, I avoid this complexity by merely defining the concept of belief as an empty
constant that maps a subject, term pair to a term \cite{seplogicbelief}. For the examples in this section, this choice suffices, 
as my common sense beliefs encode enough properties of belief for my purposes. Future work could
integrate a much more complex logic of belief into my system.%
}\isanewline
\isanewline
\isacommand{consts}\isamarkupfalse%
\ utter{\isacharcolon}{\isacharcolon}{\isachardoublequoteopen}s{\isasymRightarrow}t{\isasymRightarrow}t{\isachardoublequoteclose}\isanewline
%
\isamarkupcmt{Person s::subject utters sentence t::term%
}\isanewline
\isacommand{abbreviation}\isamarkupfalse%
\ utter{\isacharunderscore}falsehood{\isacharcolon}{\isacharcolon}{\isachardoublequoteopen}s{\isasymRightarrow}t{\isasymRightarrow}t{\isachardoublequoteclose}\ \isakeyword{where}\isanewline
{\isachardoublequoteopen}utter{\isacharunderscore}falsehood\ s\ t\ {\isasymequiv}\ \ {\isacharparenleft}utter\ s\ t{\isacharparenright}\ \isactrlbold {\isasymand}\ {\isacharparenleft}\isactrlbold {\isasymnot}\ t{\isacharparenright}{\isachardoublequoteclose}\isanewline
%
\isamarkupcmt{Person s utters falsehood t if and only if s utters t and t is false.%
}\isanewline
\isacommand{abbreviation}\isamarkupfalse%
\ knowingly{\isacharunderscore}utter{\isacharunderscore}falsehood{\isacharcolon}{\isacharcolon}{\isachardoublequoteopen}s{\isasymRightarrow}t{\isasymRightarrow}t{\isachardoublequoteclose}\ \isakeyword{where}\isanewline
{\isachardoublequoteopen}knowingly{\isacharunderscore}utter{\isacharunderscore}falsehood\ s\ t\ {\isasymequiv}\ {\isacharparenleft}utter{\isacharunderscore}falsehood\ s\ t{\isacharparenright}\ \isactrlbold {\isasymand}\ {\isacharparenleft}\isactrlbold {\isasymnot}\ {\isacharparenleft}believe\ s\ t{\isacharparenright}{\isacharparenright}{\isachardoublequoteclose}\isanewline
%
\isamarkupcmt{Person s::subject knowingly utters a falsehood t, if they both utter t as a falsehood and don't
believe t. This and the above abbreviations are the core of my formalization of Korsgaard's definition
of lying. They are also relatively uncontroversial and have little normative content.%
}\isanewline
\isanewline
\isacommand{abbreviation}\isamarkupfalse%
\ lie{\isacharcolon}{\isacharcolon}{\isachardoublequoteopen}maxim{\isasymRightarrow}bool{\isachardoublequoteclose}\ \isakeyword{where}\ \isanewline
{\isachardoublequoteopen}lie\ {\isasymequiv}\ {\isasymlambda}\ {\isacharparenleft}c{\isacharcomma}\ a{\isacharcomma}\ g{\isacharparenright}{\isachardot}\ {\isasymexists}t{\isachardot}\ {\isacharparenleft}a\ \isactrlbold {\isasymlongrightarrow}\ {\isacharparenleft}{\isasymlambda}s{\isachardot}\ knowingly{\isacharunderscore}utter{\isacharunderscore}falsehood\ s\ t{\isacharparenright}{\isacharparenright}\ {\isasymand}\ {\isacharparenleft}{\isasymexists}p{\isachardot}\ {\isasymforall}w{\isachardot}\ {\isacharparenleft}g\ \isactrlbold {\isasymrightarrow}\ believe\ p\ t{\isacharparenright}\ w{\isacharparenright}{\isachardoublequoteclose}\isanewline
%
\isamarkupcmt{Using the above definitions, I can characterize a maxim as a lie if 
    (a) the act requires knowingly uttering a falsehood 
    (b) the end requires that some person $p$ believe the false statement $t$%
}%
\begin{isamarkuptext}%
To avoid unintentional wrongdoing, I focus on ``knowing lies," 
in which the speaker is aware that they are lying. It is uncontroversial that an act is a knowing lie
if the speaker utters a false statement that they do not believe. The second half of this definition, which 
requires that the goal of the action of lying be to decieve, is inspired by  Korsgaard's interpretation 
of a lie, which understands a lie as a kind of falsehood that is usually effective \emph{because} it decieves
\citep[4]{KorsgaardRTL}. In my formalization, this means that the purpose or goal of the maxim must
involve decieving someone, or, in other words, that someone believe the falsehood told.%
\end{isamarkuptext}\isamarkuptrue%
%
\begin{isamarkuptext}%
With the above logical background, I can now automate Korsgaard's argument that maxims that involve
lying are prohibited. First, I define the subject and maxim at hand.%
\end{isamarkuptext}\isamarkuptrue%
\isacommand{consts}\isamarkupfalse%
\ me{\isacharcolon}{\isacharcolon}s\isanewline
%
\isamarkupcmt{I am trying to reason about my obligations so I will define myself as a specific subject.%
}\isanewline
\isacommand{consts}\isamarkupfalse%
\ m{\isacharcolon}{\isacharcolon}maxim\isanewline
%
\isamarkupcmt{I also define a maxim $m$. My goal is to show that if $m$ is a maxim about lying, then $m$
is prohibited.%
}\isanewline
\isacommand{consts}\isamarkupfalse%
\ c{\isacharcolon}{\isacharcolon}t\ a{\isacharcolon}{\isacharcolon}os\ g{\isacharcolon}{\isacharcolon}t\isanewline
%
\isamarkupcmt{$m$ will be composed of the circumstances, act, and goal above.%
}%
\begin{isamarkuptext}%
In the following lemma, I use my system to show that lying is prohibited. The assumptions of 
this lemma represent the logical and common sense background necessary to reach this conclusion. This
common sense background is a direct formalization of the premises of Korsgaard's argument. Using these
relatively ``thin" premises about individual behavior, my system can derive a prohibition against lying.%
\end{isamarkuptext}\isamarkuptrue%
\isacommand{lemma}\isamarkupfalse%
\ lying{\isacharunderscore}prohibited{\isacharcolon}\isanewline
\ \ \isakeyword{assumes}\ {\isachardoublequoteopen}m\ {\isasymequiv}\ {\isacharparenleft}c{\isacharcolon}{\isacharcolon}t{\isacharcomma}\ a{\isacharcolon}{\isacharcolon}os{\isacharcomma}\ g{\isacharcolon}{\isacharcolon}t{\isacharparenright}{\isachardoublequoteclose}\isanewline
\ \ \isakeyword{assumes}\ {\isachardoublequoteopen}{\isasymforall}w{\isachardot}\ {\isasymforall}s{\isachardot}\ well{\isacharunderscore}formed\ m\ s\ w{\isachardoublequoteclose}\isanewline
%
\isamarkupcmt{Initial technical set-up, m is a well-formed maxim composed of some circumstances, act, and goal.%
}\isanewline
\ \ \isakeyword{assumes}\ {\isachardoublequoteopen}lie\ m{\isachardoublequoteclose}\isanewline
%
\isamarkupcmt{$m$ is a maxim about lying. Precisely, it is a maxim in which the action requires knowingly uttering a 
falsehood and the goal requires that someone believe this falsehood.%
}\isanewline
\ \ \isakeyword{assumes}\ {\isachardoublequoteopen}{\isasymforall}t\ w{\isachardot}\ {\isacharparenleft}{\isacharparenleft}{\isasymforall}p{\isachardot}\ utter{\isacharunderscore}falsehood\ p\ t\ w{\isacharparenright}\ {\isasymlongrightarrow}\ {\isacharparenleft}{\isasymforall}p{\isachardot}\ \isactrlbold {\isasymnot}\ {\isacharparenleft}believe\ p\ t{\isacharparenright}\ w{\isacharparenright}{\isacharparenright}{\isachardoublequoteclose}\isanewline
%
\isamarkupcmt{Assumption that if everyone utters false statement t, then no one will believe t. This assumption is 
Korsgaard's core piece of ``common sense" about lying \cite[5]{KorsgaardRTL}. This simple assumption encodes the common sense knowledge
that human communication involves an implicit trust, and that when this trust erodes, the convention of 
communication begins to break down and people no longer believe each other. Call this the ``convention of 
lying" fact. In the rest of this section, I will test versions of this assumption, effectively encoding 
different common sense understandings of lying.%
}\isanewline
\ \ \isakeyword{assumes}\ {\isachardoublequoteopen}{\isasymforall}w{\isachardot}\ c\ w{\isachardoublequoteclose}\isanewline
%
\isamarkupcmt{Restrict our focus to worlds in which the circumstances hold, as these are the morally interesting 
worlds for this example. A technical detail.%
}\isanewline
\ \ \isakeyword{shows}\ {\isachardoublequoteopen}{\isasymTurnstile}\ {\isacharparenleft}prohibited\ m\ me{\isacharparenright}{\isachardoublequoteclose}\isanewline
%
\isadelimproof
%
\endisadelimproof
%
\isatagproof
\isacommand{proof}\isamarkupfalse%
\ {\isacharminus}\ \isanewline
\ \ \isacommand{have}\isamarkupfalse%
\ {\isachardoublequoteopen}{\isacharparenleft}{\isasymforall}p\ w{\isachardot}\ {\isacharparenleft}W\ m\ p{\isacharparenright}\ w{\isacharparenright}\ {\isasymlongrightarrow}\ {\isacharparenleft}{\isasymTurnstile}\ {\isacharparenleft}c\ \isactrlbold {\isasymrightarrow}\ {\isacharparenleft}\isactrlbold {\isasymnot}\ g{\isacharparenright}{\isacharparenright}{\isacharparenright}{\isachardoublequoteclose}\isanewline
\ \ \ \ \isacommand{by}\isamarkupfalse%
\ {\isacharparenleft}smt\ assms{\isacharparenleft}{\isadigit{1}}{\isacharparenright}\ assms{\isacharparenleft}{\isadigit{2}}{\isacharparenright}\ assms{\isacharparenleft}{\isadigit{5}}{\isacharparenright}\ case{\isacharunderscore}prod{\isacharunderscore}beta\ fst{\isacharunderscore}conv\ old{\isachardot}prod{\isachardot}exhaust\ snd{\isacharunderscore}conv{\isacharparenright}\isanewline
%
\isamarkupcmt{Unlike many of the other proofs in this project, this proof was a little heavier and required some manual
work to produce. I divided the proof into the intermediate steps shown here, and Isabelle was able to do 
the rest! This step says that if $m$ is universalized, then the circumstances won't lead to the goal, 
which is quite close to the idea of the maxim not being universalizable.%
}\isanewline
\ \ \isacommand{have}\isamarkupfalse%
\ {\isachardoublequoteopen}not{\isacharunderscore}universalizable\ m\ me{\isachardoublequoteclose}\isanewline
\ \ \ \ \isacommand{by}\isamarkupfalse%
\ {\isacharparenleft}metis\ {\isacharparenleft}mono{\isacharunderscore}tags{\isacharcomma}\ lifting{\isacharparenright}\ assms{\isacharparenleft}{\isadigit{1}}{\isacharparenright}\ assms{\isacharparenleft}{\isadigit{2}}{\isacharparenright}\ case{\isacharunderscore}prod{\isacharunderscore}beta\ fst{\isacharunderscore}conv\ snd{\isacharunderscore}conv{\isacharparenright}\isanewline
\ \ \isacommand{thus}\isamarkupfalse%
\ {\isacharquery}thesis\isanewline
\ \ \ \ \isacommand{using}\isamarkupfalse%
\ FUL\ assms{\isacharparenleft}{\isadigit{2}}{\isacharparenright}\ \isacommand{by}\isamarkupfalse%
\ blast\ \isanewline
\isacommand{qed}\isamarkupfalse%
%
\endisatagproof
{\isafoldproof}%
%
\isadelimproof
%
\endisadelimproof
%
\begin{isamarkuptext}%
Now that I have formalized Korsgaard's argument for why lying is prohibited, I will 
implement her argument for why jokes are permissible. Specifically, she considers a joke a story that is 
false and argues that joking is permissible because ``the universal practice of lying in the context of jokes
does not interfere with the purpose of jokes, which is to amuse and does not depend on
deception" \citep[4]{KorsgaardRTL}. Let's define a joke.%
\end{isamarkuptext}\isamarkuptrue%
\isacommand{abbreviation}\isamarkupfalse%
\ joke{\isacharcolon}{\isacharcolon}{\isachardoublequoteopen}maxim{\isasymRightarrow}bool{\isachardoublequoteclose}\ \isakeyword{where}\ \isanewline
{\isachardoublequoteopen}joke\ {\isasymequiv}\ {\isasymlambda}\ {\isacharparenleft}c{\isacharcomma}\ a{\isacharcomma}\ g{\isacharparenright}{\isachardot}\ \ {\isasymexists}t{\isachardot}\ {\isacharparenleft}a\ \isactrlbold {\isasymlongrightarrow}\ {\isacharparenleft}{\isasymlambda}s{\isachardot}\ knowingly{\isacharunderscore}utter{\isacharunderscore}falsehood\ s\ t{\isacharparenright}{\isacharparenright}\ {\isasymand}\ {\isasymnot}\ {\isacharparenleft}{\isasymexists}p{\isachardot}\ {\isasymforall}w{\isachardot}\ {\isacharparenleft}g\ \isactrlbold {\isasymrightarrow}\ {\isacharparenleft}believe\ p\ t{\isacharparenright}{\isacharparenright}\ w{\isacharparenright}{\isachardoublequoteclose}%
\begin{isamarkuptext}%
This definition of a joke merely defines a joke as a falsehood uttered for some purpose that 
emph{doesn't} require deception, where deception involves someone believing the uttered falsehood. Notice 
that this is quite a thin definition of a joke; it doesn't require any conception of humor but merely
distinguishes jokes from lies. As far as common sense reasoning goes, this is a relatively thin proposition. 
I will now demonstrate that this conception of a joke is sufficient to show that joking is permissible.

Korsgaard argues that her above argument for a prohibition against lying also implies that joking is 
permissible, because its purpose is \emph{not} to decieve, but something else entirely. This means that, 
even armed with the same core convention of lying assumption as above, joking must be permissible. The 
lemma below shows exactly that.%
\end{isamarkuptext}\isamarkuptrue%
\isacommand{lemma}\isamarkupfalse%
\ joking{\isacharunderscore}not{\isacharunderscore}prohibited{\isacharcolon}\isanewline
\ \ \isakeyword{assumes}\ {\isachardoublequoteopen}m\ {\isasymequiv}\ {\isacharparenleft}c{\isacharcolon}{\isacharcolon}t{\isacharcomma}\ a{\isacharcolon}{\isacharcolon}os{\isacharcomma}\ g{\isacharcolon}{\isacharcolon}t{\isacharparenright}{\isachardoublequoteclose}\isanewline
\ \ \isakeyword{assumes}\ {\isachardoublequoteopen}{\isasymforall}w{\isachardot}\ {\isasymforall}s{\isachardot}\ well{\isacharunderscore}formed\ m\ s\ w{\isachardoublequoteclose}\isanewline
%
\isamarkupcmt{Initial set-up, m is a well-formed maxim composed of some circumstances, act, and goal.%
}\isanewline
\ \ \isakeyword{assumes}\ {\isachardoublequoteopen}joke\ m{\isachardoublequoteclose}\isanewline
%
\isamarkupcmt{$m$ is a maxim about joking. Precisely, it is a maxim in which the action is to knowingly utter a 
falsehood and the goal does not require that someone believe this falsehood.%
}\isanewline
\ \ \isakeyword{assumes}\ {\isachardoublequoteopen}{\isasymforall}t\ w{\isachardot}\ {\isacharparenleft}{\isacharparenleft}{\isasymforall}p{\isachardot}\ utter{\isacharunderscore}falsehood\ p\ t\ w{\isacharparenright}\ {\isasymlongrightarrow}\ {\isacharparenleft}{\isasymforall}p{\isachardot}\ \isactrlbold {\isasymnot}\ {\isacharparenleft}believe\ p\ t{\isacharparenright}\ w{\isacharparenright}{\isacharparenright}{\isachardoublequoteclose}\isanewline
%
\isamarkupcmt{The same convention of lying assumption as in the above example.%
}\isanewline
\ \ \isakeyword{assumes}\ {\isachardoublequoteopen}{\isasymforall}w{\isachardot}\ c\ w{\isachardoublequoteclose}\isanewline
%
\isamarkupcmt{Restrict our focus to worlds in which the circumstances hold, as these are the morally interesting 
worlds for this example. An irrelevant technical detail.%
}\isanewline
\ \ \isakeyword{shows}\ {\isachardoublequoteopen}{\isasymTurnstile}\ {\isacharparenleft}permissible\ m\ me{\isacharparenright}{\isachardoublequoteclose}\isanewline
%
\isadelimproof
\ \ %
\endisadelimproof
%
\isatagproof
\isacommand{by}\isamarkupfalse%
\ {\isacharparenleft}smt\ assms{\isacharparenleft}{\isadigit{1}}{\isacharparenright}\ assms{\isacharparenleft}{\isadigit{2}}{\isacharparenright}\ assms{\isacharparenleft}{\isadigit{3}}{\isacharparenright}\ case{\isacharunderscore}prod{\isacharunderscore}conv{\isacharparenright}%
\endisatagproof
{\isafoldproof}%
%
\isadelimproof
%
\endisadelimproof
%
\begin{isamarkuptext}%
This completes my implementation of Korsgaard's first interpretation of lying. Using the convention
of lying assumption formalized in this section, I show that lying is generally prohibited but joking
is permissible. Notice that this is already a sophisticated kind of ethical reasoning. The sophistication
necessary to distinguish between lying and joking is a direct consequence of my system's use of a robust
conception of a maxim, which encodes the goal of an act as part of the maxim being evaluated. This kind 
of ethical reasoning also required relatively few and uncontroversial common sense facts. The deepest
assumption required was that, if everyone lies about a given statement, no one will believe that 
statement. This assumption is not merely definitional; it does encode some synthetic knowledge about the 
world, but it is relatively uncontroversial. Indeed, it is so well-accepted that Korsgaard does not 
bother to justify it in her argument. These examples demonstrate that, while common sense reasoning 
is an obstacle that must be overcome for my system to be used in practice, it is surmountable.%
\end{isamarkuptext}\isamarkuptrue%
%
\isadelimdocument
%
\endisadelimdocument
%
\isatagdocument
%
\isamarkupsubsection{Lying to a Liar \label{sec:lyingtoliar}%
}
\isamarkuptrue%
%
\endisatagdocument
{\isafolddocument}%
%
\isadelimdocument
%
\endisadelimdocument
%
\begin{isamarkuptext}%
Once Korsgaard completes her preliminary work differentiating between lies and jokes, she begins her main
 argument, which examines the controversial case of the murderer at the door. Recall that the murderer
appears at your door and asks if their intended victim is at home. Ordinary intuition requires that it 
is at the very least permissible (if not obligatory) to lie to the murderer in order to protect the 
victim. Korsgaard notes that a murderer who wishes to find their victim cannot simply announce their
intentions to murder; instead, he must ``must suppose that you do not know who he is and
what he has in mind" \citep[5]{KorsgaaardRTL}.\footnote{Korsgaard assumes that the murderer will lie 
about his identity in order to take advantage of your honesty. In footnote 5, she accepts that her 
arguments will not apply in the case of the honest murderer who announces his attentions, so she 
restricts her focus to the case of lying to a liar. She claims that in this case, the correct act is 
to refuse to respond, but she does not argue for this in this paper.} Thus, she can modify the maxim in 
question and specifies that when someone lies to you, you are allowed to lie to them. The maxim of lying 
to the murderer is actually the maxim of lying to a liar, which she argues is permissible. In this section, I will 
formalize her argument for lying to a liar.

As usual, first I define my terms.%
\end{isamarkuptext}\isamarkuptrue%
\isacommand{consts}\isamarkupfalse%
\ murderer{\isacharcolon}{\isacharcolon}s\isanewline
%
\isamarkupcmt{This example involves one more subject: the murderer.%
}\isanewline
\isacommand{consts}\isamarkupfalse%
\ not{\isacharunderscore}a{\isacharunderscore}murderer{\isacharcolon}{\isacharcolon}t\isanewline
%
\isamarkupcmt{This statement represents the lie that the murderer tells you. By not announcing their
intention, they are implicitly telling you that they are not a murderer, as people normally assume that 
those knocking on their door are not murderers.%
}\isanewline
\isacommand{consts}\isamarkupfalse%
\ when{\isacharunderscore}at{\isacharunderscore}my{\isacharunderscore}door{\isacharcolon}{\isacharcolon}t\isanewline
%
\isamarkupcmt{These are the circumstances that the murderer is in.%
}\isanewline
\isacommand{consts}\isamarkupfalse%
\ find{\isacharunderscore}victim{\isacharcolon}{\isacharcolon}t\isanewline
%
\isamarkupcmt{This will be the murderer's goal: to find their victim.%
}\isanewline
\isacommand{abbreviation}\isamarkupfalse%
\ murderers{\isacharunderscore}maxim{\isacharcolon}{\isacharcolon}{\isachardoublequoteopen}maxim{\isachardoublequoteclose}\ \isakeyword{where}\ \isanewline
{\isachardoublequoteopen}murderers{\isacharunderscore}maxim\ {\isasymequiv}\ {\isacharparenleft}when{\isacharunderscore}at{\isacharunderscore}my{\isacharunderscore}door{\isacharcomma}\ {\isasymlambda}s{\isachardot}\ knowingly{\isacharunderscore}utter{\isacharunderscore}falsehood\ s\ not{\isacharunderscore}a{\isacharunderscore}murderer{\isacharcomma}\ find{\isacharunderscore}victim{\isacharparenright}{\isachardoublequoteclose}\isanewline
%
\isamarkupcmt{Using the above definitions, I can define the murderer's maxim as, ``When at your door, I will knowingly
utter the falsehood that I am not a murderer in order to find my intended victim." Now I will repeat
the same process for your maxim.%
}\isanewline
\isanewline
\isacommand{consts}\isamarkupfalse%
\ victim{\isacharunderscore}not{\isacharunderscore}home{\isacharcolon}{\isacharcolon}t\isanewline
%
\isamarkupcmt{This statement is the lie that you tell the murderer: that their intended victim is not at home.%
}\isanewline
\isacommand{abbreviation}\isamarkupfalse%
\ murderer{\isacharunderscore}at{\isacharunderscore}door{\isacharcolon}{\isacharcolon}t\ \isakeyword{where}\ \isanewline
{\isachardoublequoteopen}murderer{\isacharunderscore}at{\isacharunderscore}door\ {\isasymequiv}\ W\ murderers{\isacharunderscore}maxim\ murderer{\isachardoublequoteclose}\isanewline
%
\isamarkupcmt{These are the circumstances that you are in: the murderer has willed their maxim and thus lied to you.%
}\isanewline
\isacommand{consts}\isamarkupfalse%
\ protect{\isacharunderscore}victim{\isacharcolon}{\isacharcolon}t\isanewline
%
\isamarkupcmt{Your goal is to protect the murderer's intended victim.%
}\isanewline
\isacommand{abbreviation}\isamarkupfalse%
\ my{\isacharunderscore}maxim{\isacharcolon}{\isacharcolon}maxim\ \isakeyword{where}\ \isanewline
{\isachardoublequoteopen}my{\isacharunderscore}maxim\ {\isasymequiv}\ {\isacharparenleft}murderer{\isacharunderscore}at{\isacharunderscore}door{\isacharcomma}\ \ {\isasymlambda}s{\isachardot}\ knowingly{\isacharunderscore}utter{\isacharunderscore}falsehood\ s\ victim{\isacharunderscore}not{\isacharunderscore}home{\isacharcomma}\ protect{\isacharunderscore}victim{\isacharparenright}{\isachardoublequoteclose}\isanewline
%
\isamarkupcmt{Using these definitions, I construct your maxim, which is ``When a murderer is at my door, I will 
knowingly utter the falsehood that their intended victim is not at home in order to protect the victim.%
}%
\begin{isamarkuptext}%
Now that I have defined the maxims at hand, I can begin reasoning about them. First, I will show 
that, using the same convention of lying common sense fact as above, the murderer's maxim is prohibited.
Effectively, this tests that the assumption is indeed strong enough to prohibit lying.%
\end{isamarkuptext}\isamarkuptrue%
\isacommand{lemma}\isamarkupfalse%
\ murderers{\isacharunderscore}maxim{\isacharunderscore}prohibited{\isacharcolon}\isanewline
\ \ \isakeyword{assumes}\ {\isachardoublequoteopen}{\isasymforall}w{\isachardot}\ well{\isacharunderscore}formed\ murderers{\isacharunderscore}maxim\ murderer\ w{\isachardoublequoteclose}\isanewline
%
\isamarkupcmt{Initial set-up, the murderer's maxim is well-formed.%
}\isanewline
\ \ \isakeyword{assumes}\ {\isachardoublequoteopen}{\isasymTurnstile}\ {\isacharparenleft}find{\isacharunderscore}victim\ \isactrlbold {\isasymrightarrow}\ {\isacharparenleft}believe\ me\ not{\isacharunderscore}a{\isacharunderscore}murderer{\isacharparenright}{\isacharparenright}{\isachardoublequoteclose}\isanewline
%
\isamarkupcmt{Assumption that, in order for the murderer to find their victim, you must not believe that they are a murderer.
This is an example of the kind of situation specific common sense reasoning necessary to use my system.
Again, this is an uncontroversial statement.%
}\isanewline
\ \ \isakeyword{assumes}\ {\isachardoublequoteopen}{\isasymforall}t\ w{\isachardot}\ {\isacharparenleft}{\isacharparenleft}{\isasymforall}p{\isachardot}\ utter{\isacharunderscore}falsehood\ p\ t\ w{\isacharparenright}\ {\isasymlongrightarrow}\ {\isacharparenleft}{\isasymforall}p{\isachardot}\ \isactrlbold {\isasymnot}\ {\isacharparenleft}believe\ p\ t{\isacharparenright}\ w{\isacharparenright}{\isacharparenright}{\isachardoublequoteclose}\isanewline
%
\isamarkupcmt{The convention of lying common sense assumption from above.%
}\isanewline
\ \ \isakeyword{assumes}\ {\isachardoublequoteopen}{\isasymforall}w{\isachardot}\ when{\isacharunderscore}at{\isacharunderscore}my{\isacharunderscore}door\ w{\isachardoublequoteclose}\isanewline
%
\isamarkupcmt{Restrict our focus to worlds in which the circumstance of the murderer being at my door holds, as 
these are the morally interesting worlds for this example. An irrelevant technical detail.%
}\isanewline
\ \ \isakeyword{shows}\ {\isachardoublequoteopen}{\isasymTurnstile}\ {\isacharparenleft}prohibited\ murderers{\isacharunderscore}maxim\ murderer{\isacharparenright}{\isachardoublequoteclose}\ \ \ \isanewline
%
\isadelimproof
%
\endisadelimproof
%
\isatagproof
\isacommand{proof}\isamarkupfalse%
\ {\isacharminus}\ \isanewline
%
\isamarkupcmt{Again, this proof is too heavy for Isabelle to finish on its own, so I needed to specify some
intermediate steps. The same intermediate steps as above sufficed, effectively providing a pattern for 
the proof. Isabelle does allow defining `proof methods,' so a more robust version of my system could define
this proof pattern and apply it in cases involving lies.%
}\isanewline
\ \ \isacommand{have}\isamarkupfalse%
\ {\isachardoublequoteopen}{\isacharparenleft}{\isasymforall}p\ w{\isachardot}\ {\isacharparenleft}W\ murderers{\isacharunderscore}maxim\ p{\isacharparenright}\ w{\isacharparenright}\ {\isasymlongrightarrow}\ {\isacharparenleft}{\isasymTurnstile}\ {\isacharparenleft}when{\isacharunderscore}at{\isacharunderscore}my{\isacharunderscore}door\ \isactrlbold {\isasymrightarrow}\ {\isacharparenleft}\isactrlbold {\isasymnot}\ find{\isacharunderscore}victim{\isacharparenright}{\isacharparenright}{\isacharparenright}{\isachardoublequoteclose}\isanewline
\ \ \ \ \isacommand{using}\isamarkupfalse%
\ assms{\isacharparenleft}{\isadigit{2}}{\isacharparenright}\ assms{\isacharparenleft}{\isadigit{4}}{\isacharparenright}\ \isacommand{by}\isamarkupfalse%
\ auto\isanewline
\ \ \ \isacommand{have}\isamarkupfalse%
\ {\isachardoublequoteopen}not{\isacharunderscore}universalizable\ murderers{\isacharunderscore}maxim\ murderer{\isachardoublequoteclose}\isanewline
\ \ \ \ \ \isacommand{using}\isamarkupfalse%
\ assms{\isacharparenleft}{\isadigit{2}}{\isacharparenright}\ assms{\isacharparenleft}{\isadigit{4}}{\isacharparenright}\ \isacommand{by}\isamarkupfalse%
\ auto\isanewline
\ \ \ \isacommand{thus}\isamarkupfalse%
\ {\isacharquery}thesis\isanewline
\ \ \ \ \ \isacommand{using}\isamarkupfalse%
\ FUL\ assms{\isacharparenleft}{\isadigit{1}}{\isacharparenright}\ \isacommand{by}\isamarkupfalse%
\ blast\isanewline
\ \isacommand{qed}\isamarkupfalse%
%
\endisatagproof
{\isafoldproof}%
%
\isadelimproof
%
\endisadelimproof
%
\begin{isamarkuptext}%
I will now formalize Korsgaard's argument for the permissibility of lying to a liar. She modifies
the convention of lying assumption above when she argues that, if the murderer believes that you don't 
believe they're a murderer, they will think that you won't lie to them. Precisely, she claims that, 
``it is because the murderer supposes you do not know what circumstances you are in - that is, that 
you do not know you are addressing a murderer - and so does not conclude from the fact that people 
in those circumstances always lie that you will lie" \cite[6]{KorsgaardRTL}. The concept of belief 
defined above will become much more crucial to formalize this version of the convention of lying 
assumption. I formalize this argument below.%
\end{isamarkuptext}\isamarkuptrue%
\isacommand{lemma}\isamarkupfalse%
\ lying{\isacharunderscore}to{\isacharunderscore}liar{\isacharunderscore}permissible{\isacharcolon}\isanewline
\ \ \isakeyword{assumes}\ {\isachardoublequoteopen}{\isasymTurnstile}\ {\isacharparenleft}well{\isacharunderscore}formed\ murderers{\isacharunderscore}maxim\ murderer{\isacharparenright}{\isachardoublequoteclose}\isanewline
\ \ \isakeyword{assumes}\ {\isachardoublequoteopen}{\isasymTurnstile}\ {\isacharparenleft}well{\isacharunderscore}formed\ my{\isacharunderscore}maxim\ me{\isacharparenright}{\isachardoublequoteclose}\isanewline
%
\isamarkupcmt{Assume that we're working with well-formed maxims.%
}\isanewline
\ \ \isakeyword{assumes}\ {\isachardoublequoteopen}{\isasymTurnstile}\ {\isacharparenleft}protect{\isacharunderscore}victim\ \isactrlbold {\isasymrightarrow}\ {\isacharparenleft}murderer\ believes\ victim{\isacharunderscore}not{\isacharunderscore}home{\isacharparenright}{\isacharparenright}{\isachardoublequoteclose}\isanewline
%
\isamarkupcmt{In order for you to successfully protect the victim, the murderer must believe that the victim is not
home. This is a noncontroversial assumption about the specific act at hand.%
}\isanewline
\ \ \isakeyword{assumes}\ {\isachardoublequoteopen}{\isasymforall}sentence{\isacharcolon}{\isacharcolon}t{\isachardot}\ {\isasymforall}p{\isadigit{1}}{\isacharcolon}{\isacharcolon}s{\isachardot}\ {\isasymforall}p{\isadigit{2}}{\isacharcolon}{\isacharcolon}s{\isachardot}\ {\isasymforall}w{\isacharcolon}{\isacharcolon}i{\isachardot}\ {\isacharparenleft}{\isacharparenleft}p{\isadigit{1}}\ believes\ {\isacharparenleft}utter{\isacharunderscore}falsehood\ p{\isadigit{2}}\ sentence{\isacharparenright}{\isacharparenright}\ w{\isacharparenright}\ {\isasymlongrightarrow}\ {\isacharparenleft}{\isasymnot}\ {\isacharparenleft}p{\isadigit{1}}\ believes\ sentence{\isacharparenright}\ w{\isacharparenright}{\isachardoublequoteclose}\isanewline
%
\isamarkupcmt{This is one of two assumptions that encode Korsgaard's core argument. If person1 believes that person2 
utters a sentence as a falsehood, then person1 won't believe that sentence. This is a modification of the 
convention of lying assumption from above, and I will refer to it as the ``convention of belief" assumption.
Again, like the convention of lying assumption, this assumption is uncontroversial.%
}\isanewline
\ \ \isakeyword{assumes}\ {\isachardoublequoteopen}{\isasymforall}c\ a\ g\ w{\isachardot}\ {\isacharparenleft}universalized\ {\isacharparenleft}c{\isacharcomma}\ a{\isacharcomma}\ g{\isacharparenright}\ w{\isacharparenright}\ {\isasymlongrightarrow}\ {\isacharparenleft}{\isacharparenleft}person{\isadigit{1}}\ believes\ {\isacharparenleft}person{\isadigit{2}}\ believes\ c{\isacharparenright}{\isacharparenright}\ \isactrlbold {\isasymrightarrow}\ {\isacharparenleft}person{\isadigit{1}}\ believes\ {\isacharparenleft}a\ person{\isadigit{2}}{\isacharparenright}{\isacharparenright}{\isacharparenright}\ w{\isachardoublequoteclose}\isanewline
%
\isamarkupcmt{This is the second key common sense assumption. If the maxim (c, a, g) is universalized, then if person1 believes 
person2 believes c, then person1 believes person2 performs the act. In other words, person1 will believe that 
person2 wills the maxim. I will refer to this as the ``convention of willing" assumption. This follows
directly from Korsgaard's conception of universalizability: when a maxim is universalized, everyone 
wills it. Korsgaard assumes that if everyone wills a maxim, they also believe that everyone else wills it.%
}\isanewline
\ \ \isakeyword{assumes}\ {\isachardoublequoteopen}{\isasymforall}w{\isachardot}\ murderer{\isacharunderscore}at{\isacharunderscore}door\ w{\isachardoublequoteclose}\isanewline
%
\isamarkupcmt{Restrict our focus to worlds in which the circumstance of the murderer being at my door holds, as 
these are the morally interesting worlds for this example. An irrelevant technical detail.%
}\isanewline
\ \ \isakeyword{shows}\ {\isachardoublequoteopen}{\isasymTurnstile}\ {\isacharparenleft}permissible\ my{\isacharunderscore}maxim\ me{\isacharparenright}{\isachardoublequoteclose}\isanewline
%
\isadelimproof
\ \ %
\endisadelimproof
%
\isatagproof
\isacommand{using}\isamarkupfalse%
\ assms{\isacharparenleft}{\isadigit{1}}{\isacharparenright}\ assms{\isacharparenleft}{\isadigit{6}}{\isacharparenright}\ \isacommand{by}\isamarkupfalse%
\ auto\isanewline
%
\isamarkupcmt{Notice the use of the first and sixth assumption in this automatically generated proof. Essentially, 
the common sense assumptions given are not strong enough to generate a prohibition against lying to a liar,
and are thus unused in this proof.%
}%
\endisatagproof
{\isafoldproof}%
%
\isadelimproof
%
\endisadelimproof
%
\begin{isamarkuptext}%
The above lemma shows that, with a more nuanced set of common sense facts, my system can show that 
lying to a liar is permissible. Moreover, I know that this set of assumptions is correct because it 
can also show that the murderer's maxim is prohibited. I show this in the lemma below.%
\end{isamarkuptext}\isamarkuptrue%
\isacommand{lemma}\isamarkupfalse%
\ murderers{\isacharunderscore}maxim{\isacharunderscore}prohibited{\isadigit{2}}{\isacharcolon}\isanewline
\ \ \isakeyword{assumes}\ {\isachardoublequoteopen}{\isasymforall}w{\isachardot}\ well{\isacharunderscore}formed\ murderers{\isacharunderscore}maxim\ murderer\ w{\isachardoublequoteclose}\isanewline
%
\isamarkupcmt{The murderer's maxim is a well-formed maxim composed of some circumstances, act, and goal.%
}\isanewline
\ \ \isakeyword{assumes}\ {\isachardoublequoteopen}{\isasymTurnstile}\ {\isacharparenleft}find{\isacharunderscore}victim\ \isactrlbold {\isasymrightarrow}\ {\isacharparenleft}believe\ me\ not{\isacharunderscore}a{\isacharunderscore}murderer{\isacharparenright}{\isacharparenright}{\isachardoublequoteclose}\isanewline
%
\isamarkupcmt{Assumption that, in order for the murderer to find their victim, you must not believe that they are a murderer.%
}\isanewline
\ \ \isakeyword{assumes}\ {\isachardoublequoteopen}{\isasymforall}sentence{\isacharcolon}{\isacharcolon}t{\isachardot}\ {\isasymforall}p{\isadigit{1}}{\isacharcolon}{\isacharcolon}s{\isachardot}\ {\isasymforall}p{\isadigit{2}}{\isacharcolon}{\isacharcolon}s{\isachardot}\ {\isasymforall}w{\isacharcolon}{\isacharcolon}i{\isachardot}\ {\isacharparenleft}{\isacharparenleft}p{\isadigit{1}}\ believes\ {\isacharparenleft}utter{\isacharunderscore}falsehood\ p{\isadigit{2}}\ sentence{\isacharparenright}{\isacharparenright}\ w{\isacharparenright}\ {\isasymlongrightarrow}\ {\isacharparenleft}{\isasymnot}\ {\isacharparenleft}p{\isadigit{1}}\ believes\ sentence{\isacharparenright}\ w{\isacharparenright}{\isachardoublequoteclose}\isanewline
%
\isamarkupcmt{The convention of belief assumption from above.%
}\isanewline
\ \ \isakeyword{assumes}\ {\isachardoublequoteopen}{\isasymforall}c\ a\ g\ w{\isachardot}\ {\isacharparenleft}universalized\ {\isacharparenleft}c{\isacharcomma}\ a{\isacharcomma}\ g{\isacharparenright}\ w{\isacharparenright}\ {\isasymlongrightarrow}\ {\isacharparenleft}{\isacharparenleft}person{\isadigit{1}}\ believes\ {\isacharparenleft}person{\isadigit{2}}\ believes\ c{\isacharparenright}{\isacharparenright}\ \isactrlbold {\isasymrightarrow}\ {\isacharparenleft}person{\isadigit{1}}\ believes\ {\isacharparenleft}a\ person{\isadigit{2}}{\isacharparenright}{\isacharparenright}{\isacharparenright}\ w{\isachardoublequoteclose}\isanewline
%
\isamarkupcmt{The convention of willing assumption from above.%
}\isanewline
\ \ \isakeyword{assumes}\ {\isachardoublequoteopen}{\isasymforall}w{\isachardot}\ when{\isacharunderscore}at{\isacharunderscore}my{\isacharunderscore}door\ w{\isachardoublequoteclose}\isanewline
%
\isamarkupcmt{Restrict our focus to worlds in which the circumstance of the murderer being at my door holds, as 
these are the morally interesting worlds for this example. Basically an irrelevant technical detail.%
}\isanewline
\ \ \isakeyword{shows}\ {\isachardoublequoteopen}{\isasymTurnstile}\ {\isacharparenleft}prohibited\ murderers{\isacharunderscore}maxim\ murderer{\isacharparenright}{\isachardoublequoteclose}\isanewline
%
\isadelimproof
%
\endisadelimproof
%
\isatagproof
\isacommand{proof}\isamarkupfalse%
\ {\isacharminus}\ \isanewline
\ \ \isacommand{have}\isamarkupfalse%
\ {\isachardoublequoteopen}{\isacharparenleft}{\isasymforall}p\ w{\isachardot}\ {\isacharparenleft}W\ murderers{\isacharunderscore}maxim\ p{\isacharparenright}\ w{\isacharparenright}\ {\isasymlongrightarrow}\ {\isacharparenleft}{\isasymTurnstile}\ {\isacharparenleft}when{\isacharunderscore}at{\isacharunderscore}my{\isacharunderscore}door\ \isactrlbold {\isasymrightarrow}\ {\isacharparenleft}\isactrlbold {\isasymnot}\ find{\isacharunderscore}victim{\isacharparenright}{\isacharparenright}{\isacharparenright}{\isachardoublequoteclose}\isanewline
\ \ \ \ \isacommand{using}\isamarkupfalse%
\ assms{\isacharparenleft}{\isadigit{2}}{\isacharparenright}\ \isacommand{by}\isamarkupfalse%
\ auto\isanewline
\ \ \isacommand{have}\isamarkupfalse%
\ {\isachardoublequoteopen}not{\isacharunderscore}universalizable\ murderers{\isacharunderscore}maxim\ murderer{\isachardoublequoteclose}\isanewline
\ \ \ \ \isacommand{using}\isamarkupfalse%
\ assms{\isacharparenleft}{\isadigit{2}}{\isacharparenright}\ assms{\isacharparenleft}{\isadigit{5}}{\isacharparenright}\ case{\isacharunderscore}prod{\isacharunderscore}beta\ fst{\isacharunderscore}conv\ internal{\isacharunderscore}case{\isacharunderscore}prod{\isacharunderscore}def\ old{\isachardot}prod{\isachardot}case\ old{\isachardot}prod{\isachardot}exhaust\ snd{\isacharunderscore}conv\ \isacommand{by}\isamarkupfalse%
\ auto\isanewline
\ \ \isacommand{thus}\isamarkupfalse%
\ {\isacharquery}thesis\isanewline
\ \ \ \ \isacommand{using}\isamarkupfalse%
\ FUL\ assms{\isacharparenleft}{\isadigit{1}}{\isacharparenright}\ \isacommand{by}\isamarkupfalse%
\ blast\ \isanewline
\isacommand{qed}\isamarkupfalse%
%
\endisatagproof
{\isafoldproof}%
%
\isadelimproof
%
\endisadelimproof
%
\begin{isamarkuptext}%
This concludes my examination of the maxim of lying to a liar. I was able to show that, by slightly
modifying the common sense facts used, my system can show that lying to a liar is permissible, but lying 
in order to find a victim is not. The assumptions used in this example are a little more robust, but still
ultimately uncontroversial. Indeed, they are direct consequences of Korsgaard's definition of willing 
and of ordinary definitions of lying. These thin assumptions were sufficient to generate moral conclusions
that Kantian scholars debate robustly. Indeed, armed with this common sense, my system generated 
a conclusion that many critics of Kant failed to see. 

While this example demonstrates that power of my system (when equipped with some common sense), it 
also shows how vital the role of the common sense reasoning is. Slight, intuitive changes in the common
sense facts can achieve totally different conclusions about lying. This represents an obstacle to 
fully automated ethical reasoning, because such an agent would need a trusted database of common 
sense facts, which is still an unsolved problem. My work is one step towards such an agent, but the 
importance of common sense means that much progress must be made in order to completely automate ethics.
In the next section, I examine the question of whether or not this work is even a good idea, and what role
automated ethics should actually play in our ordinary moral lives.%
\end{isamarkuptext}\isamarkuptrue%
%
\isadelimdocument
%
\endisadelimdocument
%
\isatagdocument
%
\isamarkupsubsection{Philosophical Analysis: Is Automated Ethics a Good Idea?%
}
\isamarkuptrue%
%
\endisatagdocument
{\isafolddocument}%
%
\isadelimdocument
%
\endisadelimdocument
%
\isadelimtheory
%
\endisadelimtheory
%
\isatagtheory
%
\endisatagtheory
{\isafoldtheory}%
%
\isadelimtheory
%
\endisadelimtheory
%
\end{isabellebody}%
\endinput
%:%file=~/Desktop/cs91r/paper/application.thy%:%
%:%24=6%:%
%:%36=8%:%
%:%37=9%:%
%:%38=10%:%
%:%39=11%:%
%:%40=12%:%
%:%41=13%:%
%:%42=14%:%
%:%43=15%:%
%:%44=16%:%
%:%45=17%:%
%:%46=18%:%
%:%47=19%:%
%:%48=20%:%
%:%49=21%:%
%:%50=22%:%
%:%51=23%:%
%:%52=24%:%
%:%53=25%:%
%:%54=26%:%
%:%55=27%:%
%:%56=28%:%
%:%57=29%:%
%:%58=30%:%
%:%59=31%:%
%:%60=32%:%
%:%61=33%:%
%:%62=34%:%
%:%63=35%:%
%:%64=36%:%
%:%65=37%:%
%:%66=38%:%
%:%67=39%:%
%:%68=40%:%
%:%69=41%:%
%:%70=42%:%
%:%71=43%:%
%:%72=44%:%
%:%73=45%:%
%:%74=46%:%
%:%75=47%:%
%:%76=48%:%
%:%77=49%:%
%:%78=50%:%
%:%79=51%:%
%:%80=52%:%
%:%81=53%:%
%:%82=54%:%
%:%83=55%:%
%:%84=56%:%
%:%85=57%:%
%:%86=58%:%
%:%87=59%:%
%:%88=60%:%
%:%89=61%:%
%:%90=62%:%
%:%91=63%:%
%:%92=64%:%
%:%93=65%:%
%:%94=66%:%
%:%95=67%:%
%:%96=68%:%
%:%97=69%:%
%:%98=70%:%
%:%99=71%:%
%:%100=72%:%
%:%101=73%:%
%:%102=74%:%
%:%103=75%:%
%:%104=76%:%
%:%105=77%:%
%:%106=78%:%
%:%115=80%:%
%:%127=82%:%
%:%128=83%:%
%:%129=84%:%
%:%130=85%:%
%:%131=86%:%
%:%132=87%:%
%:%133=88%:%
%:%134=89%:%
%:%135=90%:%
%:%136=91%:%
%:%137=92%:%
%:%138=93%:%
%:%139=94%:%
%:%140=95%:%
%:%141=96%:%
%:%142=97%:%
%:%144=99%:%
%:%145=99%:%
%:%147=100%:%
%:%148=101%:%
%:%149=102%:%
%:%150=103%:%
%:%151=104%:%
%:%152=105%:%
%:%153=106%:%
%:%154=106%:%
%:%155=107%:%
%:%156=108%:%
%:%157=108%:%
%:%159=109%:%
%:%160=109%:%
%:%161=110%:%
%:%162=110%:%
%:%163=111%:%
%:%165=112%:%
%:%166=112%:%
%:%167=113%:%
%:%168=113%:%
%:%169=114%:%
%:%171=115%:%
%:%172=116%:%
%:%173=117%:%
%:%174=117%:%
%:%175=122%:%
%:%176=123%:%
%:%177=123%:%
%:%178=124%:%
%:%180=125%:%
%:%181=126%:%
%:%182=127%:%
%:%185=129%:%
%:%186=130%:%
%:%187=131%:%
%:%188=132%:%
%:%189=133%:%
%:%190=134%:%
%:%191=135%:%
%:%195=137%:%
%:%196=138%:%
%:%198=140%:%
%:%199=140%:%
%:%201=141%:%
%:%202=141%:%
%:%203=142%:%
%:%204=142%:%
%:%206=143%:%
%:%207=144%:%
%:%208=144%:%
%:%209=145%:%
%:%210=145%:%
%:%212=146%:%
%:%215=148%:%
%:%216=149%:%
%:%217=150%:%
%:%218=151%:%
%:%220=153%:%
%:%221=153%:%
%:%222=154%:%
%:%223=155%:%
%:%225=156%:%
%:%226=156%:%
%:%227=157%:%
%:%229=158%:%
%:%230=159%:%
%:%231=159%:%
%:%232=160%:%
%:%234=161%:%
%:%235=162%:%
%:%236=163%:%
%:%237=164%:%
%:%238=165%:%
%:%239=166%:%
%:%240=166%:%
%:%241=167%:%
%:%243=168%:%
%:%244=169%:%
%:%245=169%:%
%:%246=170%:%
%:%253=171%:%
%:%254=171%:%
%:%255=172%:%
%:%256=172%:%
%:%257=173%:%
%:%258=173%:%
%:%260=174%:%
%:%261=175%:%
%:%262=176%:%
%:%263=177%:%
%:%264=177%:%
%:%265=178%:%
%:%266=178%:%
%:%267=179%:%
%:%268=179%:%
%:%269=180%:%
%:%270=180%:%
%:%271=181%:%
%:%272=181%:%
%:%273=181%:%
%:%274=182%:%
%:%284=184%:%
%:%285=185%:%
%:%286=186%:%
%:%287=187%:%
%:%288=188%:%
%:%290=190%:%
%:%291=190%:%
%:%292=191%:%
%:%294=193%:%
%:%295=194%:%
%:%296=195%:%
%:%297=196%:%
%:%298=197%:%
%:%299=198%:%
%:%300=199%:%
%:%301=200%:%
%:%302=201%:%
%:%303=202%:%
%:%305=204%:%
%:%306=204%:%
%:%307=205%:%
%:%308=206%:%
%:%310=207%:%
%:%311=207%:%
%:%312=208%:%
%:%314=209%:%
%:%315=210%:%
%:%316=210%:%
%:%317=211%:%
%:%319=212%:%
%:%320=212%:%
%:%321=213%:%
%:%323=214%:%
%:%324=215%:%
%:%325=215%:%
%:%326=216%:%
%:%329=217%:%
%:%333=217%:%
%:%334=217%:%
%:%343=219%:%
%:%344=220%:%
%:%345=221%:%
%:%346=222%:%
%:%347=223%:%
%:%348=224%:%
%:%349=225%:%
%:%350=226%:%
%:%351=227%:%
%:%352=228%:%
%:%353=229%:%
%:%362=233%:%
%:%374=235%:%
%:%375=236%:%
%:%376=237%:%
%:%377=238%:%
%:%378=239%:%
%:%379=240%:%
%:%380=241%:%
%:%381=242%:%
%:%382=243%:%
%:%383=244%:%
%:%384=245%:%
%:%385=246%:%
%:%386=247%:%
%:%387=248%:%
%:%388=249%:%
%:%389=250%:%
%:%391=252%:%
%:%392=252%:%
%:%394=253%:%
%:%395=253%:%
%:%396=254%:%
%:%397=254%:%
%:%399=255%:%
%:%400=256%:%
%:%401=257%:%
%:%402=257%:%
%:%403=258%:%
%:%404=258%:%
%:%406=259%:%
%:%407=259%:%
%:%408=260%:%
%:%409=260%:%
%:%411=261%:%
%:%412=261%:%
%:%413=262%:%
%:%414=262%:%
%:%415=263%:%
%:%417=264%:%
%:%418=265%:%
%:%419=266%:%
%:%420=266%:%
%:%421=267%:%
%:%422=268%:%
%:%423=268%:%
%:%425=269%:%
%:%426=269%:%
%:%427=270%:%
%:%428=270%:%
%:%429=271%:%
%:%431=272%:%
%:%432=272%:%
%:%433=273%:%
%:%434=273%:%
%:%436=274%:%
%:%437=274%:%
%:%438=275%:%
%:%439=275%:%
%:%440=276%:%
%:%442=277%:%
%:%443=278%:%
%:%446=280%:%
%:%447=281%:%
%:%448=282%:%
%:%450=284%:%
%:%451=284%:%
%:%452=285%:%
%:%454=286%:%
%:%455=286%:%
%:%456=287%:%
%:%458=288%:%
%:%459=289%:%
%:%460=290%:%
%:%461=290%:%
%:%462=291%:%
%:%464=292%:%
%:%465=292%:%
%:%466=293%:%
%:%468=294%:%
%:%469=295%:%
%:%470=295%:%
%:%471=296%:%
%:%478=297%:%
%:%479=297%:%
%:%481=298%:%
%:%482=299%:%
%:%483=300%:%
%:%484=301%:%
%:%485=301%:%
%:%486=302%:%
%:%487=302%:%
%:%488=303%:%
%:%489=303%:%
%:%490=303%:%
%:%491=304%:%
%:%492=304%:%
%:%493=305%:%
%:%494=305%:%
%:%495=305%:%
%:%496=306%:%
%:%497=306%:%
%:%498=307%:%
%:%499=307%:%
%:%500=307%:%
%:%501=308%:%
%:%511=310%:%
%:%512=311%:%
%:%513=312%:%
%:%514=313%:%
%:%515=314%:%
%:%516=315%:%
%:%517=316%:%
%:%518=317%:%
%:%520=319%:%
%:%521=319%:%
%:%522=320%:%
%:%523=321%:%
%:%525=322%:%
%:%526=322%:%
%:%527=323%:%
%:%529=324%:%
%:%530=325%:%
%:%531=325%:%
%:%532=326%:%
%:%534=327%:%
%:%535=328%:%
%:%536=329%:%
%:%537=330%:%
%:%538=330%:%
%:%539=331%:%
%:%541=332%:%
%:%542=333%:%
%:%543=334%:%
%:%544=335%:%
%:%545=336%:%
%:%546=336%:%
%:%547=337%:%
%:%549=338%:%
%:%550=339%:%
%:%551=339%:%
%:%552=340%:%
%:%555=341%:%
%:%559=341%:%
%:%560=341%:%
%:%561=341%:%
%:%563=342%:%
%:%564=343%:%
%:%565=344%:%
%:%575=346%:%
%:%576=347%:%
%:%577=348%:%
%:%579=350%:%
%:%580=350%:%
%:%581=351%:%
%:%583=352%:%
%:%584=352%:%
%:%585=353%:%
%:%587=354%:%
%:%588=354%:%
%:%589=355%:%
%:%591=356%:%
%:%592=356%:%
%:%593=357%:%
%:%595=358%:%
%:%596=358%:%
%:%597=359%:%
%:%599=360%:%
%:%600=361%:%
%:%601=361%:%
%:%602=362%:%
%:%609=363%:%
%:%610=363%:%
%:%611=364%:%
%:%612=364%:%
%:%613=365%:%
%:%614=365%:%
%:%615=365%:%
%:%616=366%:%
%:%617=366%:%
%:%618=367%:%
%:%619=367%:%
%:%620=367%:%
%:%621=368%:%
%:%622=368%:%
%:%623=369%:%
%:%624=369%:%
%:%625=369%:%
%:%626=370%:%
%:%636=372%:%
%:%637=373%:%
%:%638=374%:%
%:%639=375%:%
%:%640=376%:%
%:%641=377%:%
%:%642=378%:%
%:%643=379%:%
%:%644=380%:%
%:%645=381%:%
%:%646=382%:%
%:%647=383%:%
%:%648=384%:%
%:%649=385%:%
%:%650=386%:%
%:%651=387%:%
%:%660=389%:%