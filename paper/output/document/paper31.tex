%
\begin{isabellebody}%
\setisabellecontext{paper{\isadigit{3}}{\isadigit{1}}}%
%
\isadelimtheory
%
\endisadelimtheory
%
\isatagtheory
%
\endisatagtheory
{\isafoldtheory}%
%
\isadelimtheory
%
\endisadelimtheory
%
\isadelimdocument
%
\endisadelimdocument
%
\isatagdocument
%
\isamarkupsection{Prior Formalizations of The Categorical Imperative%
}
\isamarkuptrue%
%
\endisatagdocument
{\isafolddocument}%
%
\isadelimdocument
%
\endisadelimdocument
%
\begin{isamarkuptext}%
In this section, I present two formalizations of the categorical imperative and a testing framework
to evaluate them. In Section \ref{sec:naive}, I will consider an intuitive but naive formalization of 
the formula of universal law. This formalization is equivalent to a theorem in my base logic (DDL), so 
thus does not actually increase the power of my base logic. In effect, this formalization serves as a control
group that I use to present the testing architecture used to evaluate following formalizations.
In Section \ref{sec:kroy}, I will explore Moshe Kroy's partial formalization of 
the categorical imperative.%
\end{isamarkuptext}\isamarkuptrue%
%
\isadelimdocument
%
\endisadelimdocument
%
\isatagdocument
%
\isamarkupsubsection{Naive Formalization of the Formula of Universal Law \label{sec:naive}%
}
\isamarkuptrue%
%
\endisatagdocument
{\isafolddocument}%
%
\isadelimdocument
%
\endisadelimdocument
%
\begin{isamarkuptext}%
This section presents a simple and intuitive formalization of the Formula of Universal Law (FUL). 
This naive formalization will hold in the base logic itself, so this formalization does not actually
improve upon a plain deontic logic at all. This section serves two purposes. First, the naive formalization
is a toy example that demonstrates the implementation and testing process that will be used for the more 
complex formalizations presented later in Chapters 2 and 3. Second, this formalization is effectively
a control group used to determine which properties of obligaition hold in the base logic. Future formalizations
will improve on the base logic by passing more tests, or equivalently, proving more properties of 
obligation than the base logic can.


The FUL roughly states that, if a maxim cannot be willed in a world where it is universalized, it is prohibited. One
reading of this rule is that a maxim is only permissible if it is necessarily permissible. To formalize
a reading of the FUL like this naive one, I will first represent the reading as a sentence in my logic
and then add this sentence as an axiom in my logic.%
\end{isamarkuptext}\isamarkuptrue%
%
\isadelimdocument
%
\endisadelimdocument
%
\isatagdocument
%
\isamarkupsubsubsection{Formalization \label{sec:naive_form}%
}
\isamarkuptrue%
%
\endisatagdocument
{\isafolddocument}%
%
\isadelimdocument
%
\endisadelimdocument
%
\begin{isamarkuptext}%
Many of the formalizations of the categorical imperative that I present in this thesis require
some logical background. This naive formalization requires that I define the notion of permissibility,
where an action is permissible if and only if it is not prohibited.%
\end{isamarkuptext}\isamarkuptrue%
\isacommand{abbreviation}\isamarkupfalse%
\ ddlpermissable{\isacharcolon}{\isacharcolon}{\isachardoublequoteopen}t{\isasymRightarrow}t{\isachardoublequoteclose}\ {\isacharparenleft}{\isachardoublequoteopen}P{\isacharunderscore}{\isachardoublequoteclose}{\isacharparenright}\isanewline
\ \ \isakeyword{where}\ {\isachardoublequoteopen}{\isacharparenleft}P\ A{\isacharparenright}\ {\isasymequiv}\ {\isacharparenleft}\isactrlbold {\isasymnot}{\isacharparenleft}O\ {\isacharbraceleft}\isactrlbold {\isasymnot}A{\isacharbraceright}{\isacharparenright}{\isacharparenright}{\isachardoublequoteclose}\isanewline
%
\isamarkupcmt{An act $A$ is permissible if its negation is not obligated. For example, buying a red folder is 
permissible because I am not required to refrain from buying a red folder.%
}%
\begin{isamarkuptext}%
This naive formalization requires no additional logical machinery, but more complex
formalizations may require additional logical concepts. 

Let's now consider a naive reading of the Formula of Universal Law (FUL): ``act only in accordance 
with that maxim through which you can at the same time will that it become a universal law" \citep{groundwork}.
An immediate translation to DDL is that if $A$ is not necessary permissible then it is prohibited. In other
words, if we cannot universalize $P A$ (where universalizing is represented by the modal necessity 
operator), then $A$ is prohibited. This sentence is formalized in the abbreviation below:%
\end{isamarkuptext}\isamarkuptrue%
\isacommand{abbreviation}\isamarkupfalse%
\ FUL{\isacharunderscore}naive\ \isakeyword{where}\ {\isachardoublequoteopen}FUL{\isacharunderscore}naive\ {\isasymequiv}\ {\isasymlambda}A{\isachardot}\ {\isacharparenleft}{\isacharparenleft}\isactrlbold {\isasymnot}{\isacharparenleft}{\isasymbox}\ {\isacharparenleft}P\ A{\isacharparenright}{\isacharparenright}{\isacharparenright}\ \isactrlbold {\isasymrightarrow}\ {\isacharparenleft}O\ {\isacharbraceleft}{\isacharparenleft}\isactrlbold {\isasymnot}A{\isacharparenright}{\isacharbraceright}{\isacharparenright}{\isacharparenright}{\isachardoublequoteclose}\isanewline
%
\isamarkupcmt{For a given maxim `A', the FUL states that if A is not necessarily permissible, it is prohibited.%
}%
\begin{isamarkuptext}%
This naive formalization holds as a theorem of DDL. I show this using Isabelle below:%
\end{isamarkuptext}\isamarkuptrue%
\isacommand{lemma}\isamarkupfalse%
\ {\isachardoublequoteopen}{\isasymforall}A{\isachardot}\ {\isasymTurnstile}\ {\isacharparenleft}FUL{\isacharunderscore}naive\ A{\isacharparenright}{\isachardoublequoteclose}\isanewline
%
\isadelimproof
\ \ %
\endisadelimproof
%
\isatagproof
\isacommand{by}\isamarkupfalse%
\ simp\isanewline
%
\isamarkupcmt{In this short and simple proof, the statement ``by simp" demonstrates that the proof is completed
using the ``simp" tool, which is Isabelle's term rewriting engine. In this case, the result follows from
the definitions of the modal operators in DDL, so term rewriting suffices to complete the proof.%
}%
\endisatagproof
{\isafoldproof}%
%
\isadelimproof
%
\endisadelimproof
%
\begin{isamarkuptext}%
The general process of an implementation of a formalization of the FUL will be to represent the 
formalization as a sentence in my logic, as above, and then to add the formalization as an axiom to 
the logic. Kant's ethical theory is rule based, so it involves applying the categorical
imperative to solve ethical dilemmas. In logic, this is equivalent to adopting the categorical imperative as 
an axiom and then reasoning in the newly formed logic to come to ethical conclusions. Adding the categorical
imperative as an axiom makes it impossible to violate it and thus represents the categorical imperative
as the supreme, unviolable law of morality. 

Note that under this approach, reasoning about violations of obligation is difficult. Any violation of the 
categorical imperative immediately results in a contradiction. Developing a Kantian account of contrary-
to-duty obligations is a much larger philosophical project that is still open \cite{KorsgaardRTL}. This paper will focus 
on the classical Kantian notion of an ideal moral world, and thus does not reason about violations 
of the moral law \cite{idealtheory}.

Because my naive formalization holds in the base logic, adding it as an axiom does not make the logic
any more powerful. No new theorems can be derived using the naive formalization that could not already
be derived in the base logic. Thus, this section serves as a``control group." Tests performed in this
section establish which properties of obligation don't hold in the base logic. The fact that these 
tests will pass for the later, more sophisticated formalizations will serve as evidence for the superiority
of these formalizations over the base logic.%
\end{isamarkuptext}\isamarkuptrue%
\isacommand{axiomatization}\isamarkupfalse%
\ \isakeyword{where}\isanewline
FUL{\isacharunderscore}{\isadigit{1}}{\isacharcolon}\ {\isachardoublequoteopen}{\isasymTurnstile}\ {\isacharparenleft}{\isacharparenleft}\isactrlbold {\isasymnot}{\isacharparenleft}{\isasymbox}\ {\isacharparenleft}P\ A{\isacharparenright}{\isacharparenright}{\isacharparenright}\ \isactrlbold {\isasymrightarrow}\ {\isacharparenleft}O\ {\isacharbraceleft}{\isacharparenleft}\isactrlbold {\isasymnot}A{\isacharparenright}{\isacharbraceright}{\isacharparenright}{\isacharparenright}{\isachardoublequoteclose}%
\begin{isamarkuptext}%
Once I add a formalization of the FUL as an axiom to my system, I will test the formalization.
Each test will take the form of a lemma which I expect to either hold or be disproven by the categorical
imperative. For example, one test might be the lemma ``murder is wrong." I will evaluate formalizations
based on their ability to prove expected properties of the categorical imperative, as determined by 
philosophical literature. These tests fall into two categories: metaethical tests, which focus on 
abstract properties of the ethical system, and application tests, which simulate the kind of practical reasoning 
that an agent would actually perform by specifying a simple model. 

One way to understand computational ethics is as translational work that seeks
to translate an ethical theory presented by a philosopher to something that a computer can reason about.
My testing architecture evaluates how faithful a particular formalization is to the ethical theory that 
it translate. This testing approach is not specific to my ethical theory and could be used to evaluate
other formalizations of other theories as well.%
\end{isamarkuptext}\isamarkuptrue%
%
\isadelimdocument
%
\endisadelimdocument
%
\isatagdocument
%
\isamarkupsubsubsection{Metaethical Tests \label{sec:meta_tests_naive}%
}
\isamarkuptrue%
%
\endisatagdocument
{\isafolddocument}%
%
\isadelimdocument
%
\endisadelimdocument
%
\begin{isamarkuptext}%
First, I present metaethical tests for the naive formalization (or equivalently the base logic). 
These tests evaluate abstract properies of the system, independent of a particular agent, situation, or 
act. For example, one metaethical test may be to determine if the system is capable of generating models 
in which actions are obligated. If the system can never obligate anything, this indicates that it is 
not a good ethical system.%
\end{isamarkuptext}\isamarkuptrue%
%
\begin{isamarkuptext}%
$\textbf{Preliminary Tests}$%
\end{isamarkuptext}\isamarkuptrue%
%
\begin{isamarkuptext}%
The immediate test for any formalization is consistency, or the property of being free of contradictions. 
An inconsistent formalization is immediately useless, because all sentences are true in an inconsistent
logic. Nitpick, Isabelle's model checker, offers a handy way of checking consistency. Specifically, 
if Nitpick can find a model that satisfies all the axioms of the logic, then the logic is consistent.%
\end{isamarkuptext}\isamarkuptrue%
\isacommand{lemma}\isamarkupfalse%
\ True\ \isacommand{nitpick}\isamarkupfalse%
\ {\isacharbrackleft}satisfy{\isacharcomma}user{\isacharunderscore}axioms{\isacharcomma}format{\isacharequal}{\isadigit{2}}{\isacharbrackright}%
\isadelimproof
\ %
\endisadelimproof
%
\isatagproof
\isacommand{oops}\isamarkupfalse%
\isanewline
%
\isamarkupcmt{\color{blue} Nitpick found a model for card i = 1:

  Empty assignment \color{black}%
}\isanewline
%
\isamarkupcmt{Nitpick tells us that the FUL is consistent\footnote{``oops" at the end of a lemma indicates that the 
proof is left unfinished. It does not indicate that an error occurred. In this case, we aren't interested
in proving True (the proof is trivial and automatic), hence the oops.}%
}\isanewline
\isanewline
%
\endisatagproof
{\isafoldproof}%
%
\isadelimproof
%
\endisadelimproof
%
\isadelimproof
%
\endisadelimproof
%
\isatagproof
%
\endisatagproof
{\isafoldproof}%
%
\isadelimproof
%
\endisadelimproof
%
\isadelimproof
%
\endisadelimproof
%
\isatagproof
%
\endisatagproof
{\isafoldproof}%
%
\isadelimproof
%
\endisadelimproof
%
\isadelimproof
%
\endisadelimproof
%
\isatagproof
%
\endisatagproof
{\isafoldproof}%
%
\isadelimproof
%
\endisadelimproof
%
\isadelimproof
%
\endisadelimproof
%
\isatagproof
%
\endisatagproof
{\isafoldproof}%
%
\isadelimproof
%
\endisadelimproof
%
\begin{isamarkuptext}%
An initial property that we might be interested in is the possibility of permissibility, or 
whether or not the system can generate models in which certain acts are permissible. In modern ethics, 
permissibility is a well-accepted phenomenon.
An ethical theory that doesn't allow for permissibility would require that every action is either obligatory or 
prohibited. If that is the case, many counterintuitive theorems follow, including that all 
permissible actions are obligatory.\footnote{Proof is in the appendix.} Therefore, I will include 
the possibility of permissibility as one test for my formalizations.%
\end{isamarkuptext}\isamarkuptrue%
\ \ \isacommand{lemma}\isamarkupfalse%
\ permissible{\isacharcolon}\isanewline
\ \ \ \ \isakeyword{shows}\ {\isachardoublequoteopen}{\isasymexists}A{\isachardot}\ {\isacharparenleft}{\isacharparenleft}\isactrlbold {\isasymnot}\ {\isacharparenleft}O\ {\isacharbraceleft}A{\isacharbraceright}{\isacharparenright}{\isacharparenright}\ \isactrlbold {\isasymand}\ {\isacharparenleft}\isactrlbold {\isasymnot}\ {\isacharparenleft}O\ {\isacharbraceleft}\isactrlbold {\isasymnot}\ A{\isacharbraceright}{\isacharparenright}{\isacharparenright}{\isacharparenright}\ w{\isachardoublequoteclose}\isanewline
\ \ \ \ \isacommand{nitpick}\isamarkupfalse%
\ {\isacharbrackleft}user{\isacharunderscore}axioms{\isacharcomma}\ falsify{\isacharequal}false{\isacharbrackright}%
\isadelimproof
\ %
\endisadelimproof
%
\isatagproof
\isacommand{oops}\isamarkupfalse%
\isanewline
%
\isamarkupcmt{\color{blue}Nitpick found a model for card i = 1 and card s = 1:

  Skolem constant:
    A =($\lambda x. \_$)($i_1$ := False) \color{black}%
}\isanewline
%
\isamarkupcmt{I want to show that there exists a model where there is some formula A that is permissible, or, 
in English, that permissibility is possible. Nitpick finds a model where the above formula holds, 
so permissibility is indeed possible.%
}%
\endisatagproof
{\isafoldproof}%
%
\isadelimproof
%
\endisadelimproof
%
\begin{isamarkuptext}%
Another similar property is that for any arbitrary action A, there is a model that makes 
it permissible. This property is actually not desirable, because if A is "murder" then the CI should require that 
it be prohibited in every world. Therefore, in order for this test to pass, Nitpick should \emph{not}
be able to find a satisfying model for this formula.%
\end{isamarkuptext}\isamarkuptrue%
\isacommand{lemma}\isamarkupfalse%
\ fixed{\isacharunderscore}formula{\isacharunderscore}is{\isacharunderscore}permissible{\isacharcolon}\isanewline
\ \ \isakeyword{fixes}\ A\isanewline
\ \ \isakeyword{shows}\ {\isachardoublequoteopen}{\isacharparenleft}{\isacharparenleft}\isactrlbold {\isasymnot}\ {\isacharparenleft}O\ {\isacharbraceleft}A{\isacharbraceright}{\isacharparenright}{\isacharparenright}\ \isactrlbold {\isasymand}\ {\isacharparenleft}\isactrlbold {\isasymnot}\ {\isacharparenleft}O\ {\isacharbraceleft}\isactrlbold {\isasymnot}\ A{\isacharbraceright}{\isacharparenright}{\isacharparenright}{\isacharparenright}\ w{\isachardoublequoteclose}\isanewline
\ \ \isacommand{nitpick}\isamarkupfalse%
\ {\isacharbrackleft}user{\isacharunderscore}axioms{\isacharcomma}\ falsify{\isacharequal}false{\isacharbrackright}%
\isadelimproof
\ %
\endisadelimproof
%
\isatagproof
\isacommand{oops}\isamarkupfalse%
\isanewline
%
\isamarkupcmt{\color{blue}Nitpick found a model for card i = 1:

  Free variable:
    A = ($\lambda x. \_$)($i_1$ := False)\color{black}%
}\isanewline
%
\isamarkupcmt{Because Nitpick finds a satisfying model for this formula, this test fails for the naive interpretation.%
}%
\endisatagproof
{\isafoldproof}%
%
\isadelimproof
%
\endisadelimproof
%
\begin{isamarkuptext}%
Another initial property is that arbitary actions should not be obligated. No sensible ethical
theory would require that any arbitrary action A is obligated, because A may be something obviously wrong,
like murder. In order for this test to pass, Nitpick must disprove the formula below by finding a counterexample.%
\end{isamarkuptext}\isamarkuptrue%
\isacommand{lemma}\isamarkupfalse%
\ arbitrary{\isacharunderscore}obligations{\isacharcolon}\isanewline
\ \ \isakeyword{fixes}\ A{\isacharcolon}{\isacharcolon}{\isachardoublequoteopen}t{\isachardoublequoteclose}\isanewline
\ \ \isakeyword{shows}\ {\isachardoublequoteopen}O\ {\isacharbraceleft}A{\isacharbraceright}\ w{\isachardoublequoteclose}\isanewline
\ \ \isacommand{nitpick}\isamarkupfalse%
\ {\isacharbrackleft}user{\isacharunderscore}axioms{\isacharequal}true{\isacharbrackright}%
\isadelimproof
\ %
\endisadelimproof
%
\isatagproof
\isacommand{oops}\isamarkupfalse%
\isanewline
%
\isamarkupcmt{\color{blue} Nitpick found a counterexample for card i = 1:

  Free variable:
    A = ($\lambda x. \_$)($i_1$ := False) \color{black}%
}\isanewline
%
\isamarkupcmt{Nitpick finds a counterexample disproving the statement that any arbitrary action is obligatory, so
this test passes.%
}%
\endisatagproof
{\isafoldproof}%
%
\isadelimproof
%
\endisadelimproof
%
\begin{isamarkuptext}%
$\textbf{Conflicting Obligations}$%
\end{isamarkuptext}\isamarkuptrue%
%
\begin{isamarkuptext}%
The next set of tests will focus on conflicting obligations. There is some debate about Kant's
personal stance on conflicting obligations, but neo-Kantian agree that the FUL itself cannot obligate
conflicting actions. For more complete discussion of conflicting obligations in Kantian literature, 
see \ref{sec:priorgoals}. I will first test whether or not, for some arbitrary action, Nitpick can find
a model in which that action is both obligated and prohibited.%
\end{isamarkuptext}\isamarkuptrue%
\isacommand{lemma}\isamarkupfalse%
\ conflicting{\isacharunderscore}obligations{\isacharcolon}\isanewline
\ \ \isakeyword{fixes}\ A\isanewline
\ \ \isakeyword{shows}\ {\isachardoublequoteopen}{\isacharparenleft}O\ {\isacharbraceleft}A{\isacharbraceright}\ \isactrlbold {\isasymand}\ O\ {\isacharbraceleft}\isactrlbold {\isasymnot}\ A{\isacharbraceright}{\isacharparenright}\ w{\isachardoublequoteclose}\isanewline
\ \ \isacommand{nitpick}\isamarkupfalse%
\ {\isacharbrackleft}user{\isacharunderscore}axioms{\isacharcomma}\ falsify{\isacharequal}false{\isacharbrackright}%
\isadelimproof
\ %
\endisadelimproof
%
\isatagproof
\isacommand{oops}\isamarkupfalse%
\isanewline
%
\isamarkupcmt{\color{blue} Nitpick found a model for card i = 2:

  Free variable:
    A = ($\lambda x. \_$)($i_1$ := False, $i_2$ := True) \color{black}%
}\isanewline
%
\isamarkupcmt{Nitpick found a model with conflicting obligations, so this tests fails.%
}\isanewline
%
\endisatagproof
{\isafoldproof}%
%
\isadelimproof
%
\endisadelimproof
%
\isadelimproof
%
\endisadelimproof
%
\isatagproof
%
\endisatagproof
{\isafoldproof}%
%
\isadelimproof
%
\endisadelimproof
%
\begin{isamarkuptext}%
The above is a rather weak notion of contradictory obligations. Korsgaard additionally argues that Kantian 
ethics also has the stronger property that if two maxims imply a contradiction, they must not be willed \cite{KorsgaardFUL}.
I test this property below. Because this property is stronger than the previous test, and the previous 
test failed, this test will also fail.%
\end{isamarkuptext}\isamarkuptrue%
\isacommand{lemma}\isamarkupfalse%
\ implied{\isacharunderscore}contradiction{\isacharcolon}\isanewline
\ \ \isakeyword{fixes}\ A{\isacharcolon}{\isacharcolon}{\isachardoublequoteopen}t{\isachardoublequoteclose}\isanewline
\ \ \isakeyword{fixes}\ B{\isacharcolon}{\isacharcolon}{\isachardoublequoteopen}t{\isachardoublequoteclose}\ \isanewline
\ \ \isakeyword{assumes}\ {\isachardoublequoteopen}{\isasymTurnstile}{\isacharparenleft}\isactrlbold {\isasymnot}\ {\isacharparenleft}A\ \isactrlbold {\isasymand}\ B{\isacharparenright}{\isacharparenright}{\isachardoublequoteclose}\isanewline
\ \ \isakeyword{shows}\ {\isachardoublequoteopen}{\isasymTurnstile}{\isacharparenleft}\isactrlbold {\isasymnot}\ {\isacharparenleft}O\ {\isacharbraceleft}A{\isacharbraceright}\ \isactrlbold {\isasymand}\ O\ {\isacharbraceleft}B{\isacharbraceright}{\isacharparenright}{\isacharparenright}{\isachardoublequoteclose}\isanewline
\ \ \isacommand{nitpick}\isamarkupfalse%
\ {\isacharbrackleft}user{\isacharunderscore}axioms{\isacharbrackright}\isanewline
%
\isadelimproof
%
\endisadelimproof
%
\isatagproof
\isacommand{proof}\isamarkupfalse%
\ {\isacharminus}\ \isanewline
\ \ \isacommand{have}\isamarkupfalse%
\ {\isachardoublequoteopen}{\isasymTurnstile}{\isacharparenleft}\isactrlbold {\isasymnot}{\isacharparenleft}{\isasymdiamond}{\isacharparenleft}A\ \isactrlbold {\isasymand}\ B{\isacharparenright}{\isacharparenright}{\isacharparenright}{\isachardoublequoteclose}\isanewline
\ \ \ \ \isacommand{by}\isamarkupfalse%
\ {\isacharparenleft}simp\ add{\isacharcolon}\ assms{\isacharparenright}\isanewline
\ \ \isacommand{then}\isamarkupfalse%
\ \isacommand{have}\isamarkupfalse%
\ {\isachardoublequoteopen}{\isasymTurnstile}{\isacharparenleft}\isactrlbold {\isasymnot}\ {\isacharparenleft}O\ {\isacharbraceleft}A\ \isactrlbold {\isasymand}\ B{\isacharbraceright}{\isacharparenright}{\isacharparenright}{\isachardoublequoteclose}\ \isacommand{by}\isamarkupfalse%
\ {\isacharparenleft}smt\ O{\isacharunderscore}diamond{\isacharparenright}\isanewline
%
\isamarkupcmt{Notice that this is $\textbf{almost}$ the property we are interested in. In fact, if $O \{ A \wedge B \}$
is equivalent to $O\{A\} \wedge O\{B\}$, then the proof is complete.%
}\isanewline
\ \ \isacommand{thus}\isamarkupfalse%
\ {\isacharquery}thesis\ \isacommand{oops}\isamarkupfalse%
\isanewline
%
\isamarkupcmt{\color{blue} Nitpick found a counterexample for card i = 2:

  Free variables:
    A = ($\lambda x. \_$)($i_1$ := True, $i_2$ := False)
    B = ($\lambda x. \_$)($i_1$ := False, $i_2$ := True) \color{black}%
}\isanewline
%
\isamarkupcmt{Sadly the property I'm actually interested in doesn't follow.%
}%
\endisatagproof
{\isafoldproof}%
%
\isadelimproof
%
\endisadelimproof
%
\begin{isamarkuptext}%
The above proof yields an interesting observation.  $O \{ A \wedge B \} $ is not equivalent to 
$O\{A\} \wedge O\{B\}$. The rough English translation of  $O \{ A \wedge B \} $ is ``you are obligated to 
do both A and B". The rough English translation of $O\{A\} \wedge O\{B\}$ is ``you are obligated to do A 
and you are obligated to do B." We think those English sentences mean the same thing, so they should mean 
the same thing in our logic as well. This ``distributive" property of obligation is another test.%
\end{isamarkuptext}\isamarkuptrue%
\isacommand{lemma}\isamarkupfalse%
\ distribute{\isacharunderscore}obligations{\isacharcolon}\isanewline
\ \ \isakeyword{assumes}\ \ {\isachardoublequoteopen}{\isasymTurnstile}\ {\isacharparenleft}O\ {\isacharbraceleft}A{\isacharbraceright}\ \isactrlbold {\isasymand}\ O\ {\isacharbraceleft}B{\isacharbraceright}{\isacharparenright}{\isachardoublequoteclose}\isanewline
\ \ \isakeyword{shows}\ {\isachardoublequoteopen}{\isasymTurnstile}\ O\ {\isacharbraceleft}A\ \isactrlbold {\isasymand}\ B{\isacharbraceright}{\isachardoublequoteclose}\isanewline
\ \ \isacommand{nitpick}\isamarkupfalse%
\ {\isacharbrackleft}user{\isacharunderscore}axioms{\isacharbrackright}%
\isadelimproof
\ %
\endisadelimproof
%
\isatagproof
\isacommand{oops}\isamarkupfalse%
\isanewline
%
\isamarkupcmt{\color{blue} Nitpick found a counterexample for card i = 2:

  Free variables:
    A = ($\lambda x. \_$)($i_1$ := True, $i_2$ := False)
    B = ($\lambda x. \_$)($i_1$ := False, $i_2$ := True)\color{black}
Once again, this tests fails in the control group.%
}%
\endisatagproof
{\isafoldproof}%
%
\isadelimproof
%
\endisadelimproof
%
\begin{isamarkuptext}%
$\textbf{Miscellaneous Properties}$%
\end{isamarkuptext}\isamarkuptrue%
%
\begin{isamarkuptext}%
The last set of metaethical tests involve miscellaneous properties of the categorical 
imperative. First, I show that the naive formalization is equivalent to the below theorem, which clearly
fails to track intuition about ethics.%
\end{isamarkuptext}\isamarkuptrue%
\isacommand{lemma}\isamarkupfalse%
\ FUL{\isacharunderscore}alternate{\isacharcolon}\isanewline
\ \ \isakeyword{shows}\ {\isachardoublequoteopen}{\isasymTurnstile}\ {\isacharparenleft}{\isacharparenleft}{\isasymdiamond}\ {\isacharparenleft}O\ {\isacharbraceleft}\isactrlbold {\isasymnot}\ A{\isacharbraceright}{\isacharparenright}{\isacharparenright}\ \isactrlbold {\isasymrightarrow}\ {\isacharparenleft}O\ {\isacharbraceleft}\isactrlbold {\isasymnot}\ A{\isacharbraceright}{\isacharparenright}{\isacharparenright}{\isachardoublequoteclose}\isanewline
%
\isadelimproof
\ \ %
\endisadelimproof
%
\isatagproof
\isacommand{by}\isamarkupfalse%
\ simp\isanewline
%
\isamarkupcmt{This means that if something is possibly prohibited, it is in fact prohibited.%
}\isanewline
%
\isamarkupcmt{This is a direct consequence\footnote{For a manual proof, see the Appendix.} of the naive formalization, but it's not clear to me that this is 
actually how we think about ethics. For example, imagine an alternate universe where smiling at 
someone is considered an incredibly rude and disrespectful gesture. In this universe, I am probably 
prohibited from smiling at people, but this doesn't mean that in this current universe, smiling is 
morally wrong.%
}%
\endisatagproof
{\isafoldproof}%
%
\isadelimproof
%
\endisadelimproof
%
\begin{isamarkuptext}%
The ``ought implies can" principle is attributed to Kant\footnote{The exact philosophical credence of this view is disputed, but the rough idea holds nonetheless. See \cite{kohl} for more.}
 and is rather intuitive: you can't be obligated to do the impossible. Deontic 
logics evolved specifically from this principle, so this should hold in the base logic \cite{cresswell}.%
\end{isamarkuptext}\isamarkuptrue%
\isacommand{lemma}\isamarkupfalse%
\ ought{\isacharunderscore}implies{\isacharunderscore}can{\isacharcolon}\isanewline
\ \ \isakeyword{shows}\ {\isachardoublequoteopen}{\isasymforall}A{\isachardot}\ {\isasymTurnstile}\ {\isacharparenleft}O\ {\isacharbraceleft}A{\isacharbraceright}\ \isactrlbold {\isasymrightarrow}\ {\isacharparenleft}{\isasymdiamond}A{\isacharparenright}{\isacharparenright}{\isachardoublequoteclose}\isanewline
%
\isadelimproof
\ \ %
\endisadelimproof
%
\isatagproof
\isacommand{using}\isamarkupfalse%
\ O{\isacharunderscore}diamond\ \isacommand{by}\isamarkupfalse%
\ blast\isanewline
%
\endisatagproof
{\isafoldproof}%
%
\isadelimproof
%
\endisadelimproof
%
\isadelimproof
%
\endisadelimproof
%
\isatagproof
%
\endisatagproof
{\isafoldproof}%
%
\isadelimproof
%
\endisadelimproof
%
\isadelimproof
%
\endisadelimproof
%
\isatagproof
%
\endisatagproof
{\isafoldproof}%
%
\isadelimproof
%
\endisadelimproof
%
\isadelimdocument
%
\endisadelimdocument
%
\isatagdocument
%
\isamarkupsubsubsection{Application Tests \label{sec:app_tests_naive}%
}
\isamarkuptrue%
%
\endisatagdocument
{\isafolddocument}%
%
\isadelimdocument
%
\endisadelimdocument
%
\begin{isamarkuptext}%
STARTHERE One category of tests involves specified models to encode certain facts 
into the system and then ask questions about obligations. Without specifying the model, we are limited 
to showing high-level metaethical facts. Let's start with analyzing an obvious example - that murder is 
wrong.%
\end{isamarkuptext}\isamarkuptrue%
\isacommand{consts}\isamarkupfalse%
\ M{\isacharcolon}{\isacharcolon}{\isachardoublequoteopen}t{\isachardoublequoteclose}\isanewline
\isacommand{abbreviation}\isamarkupfalse%
\ murder{\isacharunderscore}wrong{\isacharcolon}{\isacharcolon}{\isachardoublequoteopen}bool{\isachardoublequoteclose}\ \isakeyword{where}\ {\isachardoublequoteopen}murder{\isacharunderscore}wrong\ {\isasymequiv}\ {\isasymTurnstile}{\isacharparenleft}O\ {\isacharbraceleft}\isactrlbold {\isasymnot}\ M{\isacharbraceright}{\isacharparenright}{\isachardoublequoteclose}\isanewline
%
\isadelimproof
%
\endisadelimproof
%
\isatagproof
%
\endisatagproof
{\isafoldproof}%
%
\isadelimproof
\isanewline
%
\endisadelimproof
\isacommand{abbreviation}\isamarkupfalse%
\ possibly{\isacharunderscore}murder{\isacharunderscore}wrong{\isacharcolon}{\isacharcolon}{\isachardoublequoteopen}bool{\isachardoublequoteclose}\ \isakeyword{where}\ {\isachardoublequoteopen}possibly{\isacharunderscore}murder{\isacharunderscore}wrong\ {\isasymequiv}\ {\isacharparenleft}{\isasymdiamond}\ {\isacharparenleft}O\ {\isacharbraceleft}\isactrlbold {\isasymnot}\ M{\isacharbraceright}{\isacharparenright}{\isacharparenright}\ cw{\isachardoublequoteclose}\isanewline
%
\isamarkupcmt{These are very simple properties. \isa{poss{\isacharunderscore}murder{\isacharunderscore}wrong} is an abbreviation for the axiom
that there is some world where murder might be prohibited. Even this is quite a strong assumption - 
ideally we'd want to give the system nonmoral facts about murder (like a definition) and then make 
moral claims.%
}\isanewline
\isanewline
\isacommand{lemma}\isamarkupfalse%
\ wrong{\isacharunderscore}if{\isacharunderscore}possibly{\isacharunderscore}wrong{\isacharcolon}\isanewline
\ \ \isakeyword{shows}\ {\isachardoublequoteopen}possibly{\isacharunderscore}murder{\isacharunderscore}wrong\ {\isasymlongrightarrow}\ murder{\isacharunderscore}wrong{\isachardoublequoteclose}\isanewline
%
\isadelimproof
\ \ %
\endisadelimproof
%
\isatagproof
\isacommand{by}\isamarkupfalse%
\ simp\isanewline
%
\isamarkupcmt{This lemma gets to the ``heart" of this naive interpretation. We really want to say that if something
 isn't necessarily obligated, it's not obligated anywhere.%
}%
\endisatagproof
{\isafoldproof}%
%
\isadelimproof
%
\endisadelimproof
%
\begin{isamarkuptext}%
The above example does exactly what we expect it to: we show that if something is wrong somewhere 
it's wrong everywhere. That being said, it seems like quite a weak claim. We assumed a very strong, moral 
fact about murder (that it is wrong somewhere), so it's not surprise that we were able to reach our desired conclusion.%
\end{isamarkuptext}\isamarkuptrue%
%
\begin{isamarkuptext}%
Let's try a weaker assumption: Not everyone can lie.%
\end{isamarkuptext}\isamarkuptrue%
\isacommand{typedecl}\isamarkupfalse%
\ person\isanewline
\isacommand{consts}\isamarkupfalse%
\ lie{\isacharcolon}{\isacharcolon}{\isachardoublequoteopen}person{\isasymRightarrow}t{\isachardoublequoteclose}\isanewline
\isacommand{consts}\isamarkupfalse%
\ me{\isacharcolon}{\isacharcolon}{\isachardoublequoteopen}person{\isachardoublequoteclose}\isanewline
%
\isamarkupcmt{Notice that this machinery is quite empty. We don't give axioms about what a person can or can't do.%
}\isanewline
\isanewline
\isacommand{abbreviation}\isamarkupfalse%
\ lying{\isacharunderscore}not{\isacharunderscore}universal{\isacharcolon}{\isacharcolon}{\isachardoublequoteopen}bool{\isachardoublequoteclose}\ \isakeyword{where}\ {\isachardoublequoteopen}lying{\isacharunderscore}not{\isacharunderscore}universal\ {\isasymequiv}\ {\isasymforall}w{\isachardot}\ {\isasymnot}\ {\isacharparenleft}{\isacharparenleft}{\isasymforall}x{\isachardot}\ lie{\isacharparenleft}x{\isacharparenright}\ w{\isacharparenright}\ {\isasymand}\ {\isacharparenleft}lie{\isacharparenleft}me{\isacharparenright}\ w{\isacharparenright}{\isacharparenright}{\isachardoublequoteclose}%
\begin{isamarkuptext}%
This is a rough translation of failure of the universalizability test: we will test the maxim universally,
as represented by the universal quantifier in the first conjunct, and simultaneously \cite{simul}, as represented by 
the second conjunct. The FUL tells us that if this sentence is true, then lying should be prohibited. 
Let's test it.%
\end{isamarkuptext}\isamarkuptrue%
\isacommand{lemma}\isamarkupfalse%
\ breaking{\isacharunderscore}promises{\isacharcolon}\isanewline
\ \ \isakeyword{assumes}\ lying{\isacharunderscore}not{\isacharunderscore}universal\isanewline
\ \ \isakeyword{shows}\ {\isachardoublequoteopen}{\isacharparenleft}O\ {\isacharbraceleft}\isactrlbold {\isasymnot}\ {\isacharparenleft}lie{\isacharparenleft}me{\isacharparenright}{\isacharparenright}{\isacharbraceright}{\isacharparenright}\ cw{\isachardoublequoteclose}\isanewline
\ \ \isacommand{nitpick}\isamarkupfalse%
\ {\isacharbrackleft}user{\isacharunderscore}axioms{\isacharbrackright}\isanewline
%
\isadelimproof
\ \ %
\endisadelimproof
%
\isatagproof
\isacommand{oops}\isamarkupfalse%
\isanewline
%
\isamarkupcmt{\color{blue} Nitpick found a counterexample for card person = 2 and card i = 2:

  Free variable:
    lie = ($\lambda x. \_$)($p_1$ := ($\lambda x. \_$)($i_1$ := True, $i_2$ := False), $p_2$ := ($\lambda x. \_$)($i_1$ := False, $i_2$ := False)) \color{black}%
}\isanewline
%
\isamarkupcmt{Quick note on how to read Nitpick results. Nitpick will either say that it found a ``model" or a ``counterexample" in 
the first line. It will then provide a model by specifying model components. For readability, all except for the 
free variables are hidden. This model has cardinality 2 for the person and world (i) types. The term \texttt{lie} is 
defined for people $p_1$ and $p_2$. $p_1$ lies at world $i_1$ and does not lie at world $i_2$. $p_2$ does
the opposite.%
}\isanewline
%
\isamarkupcmt{These details will be elided for most Nitpick examples, but this provides guidance on how to interpret
the output.%
}%
\endisatagproof
{\isafoldproof}%
%
\isadelimproof
%
\endisadelimproof
%
\begin{isamarkuptext}%
This formula isn't valid. While the FUL should tell us that lying is prohibited, the fact that it 
doesn't demonstrates the weakness of this naive formulation of the categorical imperative. Kant's version of
the FUL universalizes across people, as we did in the definition of \isa{lying{\isacharunderscore}not{\isacharunderscore}universal\ {\isasymequiv}\ {\isasymforall}w{\isachardot}\ {\isasymnot}\ {\isacharparenleft}{\isacharparenleft}{\isasymforall}x{\isachardot}\ lie\ x\ w{\isacharparenright}\ {\isasymand}\ lie\ me\ w{\isacharparenright}}. The 
naive formalization universalizes across worlds using the $\Box$ operator, so it makes sense that it can't
handle this example appropriately.

The above implies that the FUL should prescribe consistent obligations across people. If our formalization
doesn't, clearly something has gone wrong somewhere. Let's test that!%
\end{isamarkuptext}\isamarkuptrue%
\isacommand{lemma}\isamarkupfalse%
\ universalizability{\isacharcolon}\isanewline
\ \ \isakeyword{assumes}\ {\isachardoublequoteopen}{\isasymTurnstile}\ O\ {\isacharbraceleft}{\isacharparenleft}lie{\isacharparenleft}me{\isacharparenright}{\isacharparenright}{\isacharbraceright}{\isachardoublequoteclose}\isanewline
\ \ \isakeyword{shows}\ {\isachardoublequoteopen}{\isasymforall}x{\isachardot}\ {\isasymTurnstile}\ {\isacharparenleft}O\ {\isacharbraceleft}{\isacharparenleft}lie{\isacharparenleft}x{\isacharparenright}{\isacharparenright}{\isacharbraceright}{\isacharparenright}{\isachardoublequoteclose}\isanewline
\ \ \isacommand{nitpick}\isamarkupfalse%
\ {\isacharbrackleft}user{\isacharunderscore}axioms{\isacharbrackright}%
\isadelimproof
\ %
\endisadelimproof
%
\isatagproof
\isacommand{oops}\isamarkupfalse%
\isanewline
%
\isamarkupcmt{\color{blue} Nitpick found a counterexample for card person = 2 and card i = 2:

  Free variable:
    lie = ($\lambda x. \_$)($p_1$ := ($\lambda x. \_$)($i_1$ := False, $i_2$ := True), $p_2$ := ($\lambda x. \_$)($i_1$ := False, $i_2$ := False))
  Skolem constant:
    x = $p_2$ \color{black}%
}%
\endisatagproof
{\isafoldproof}%
%
\isadelimproof
%
\endisadelimproof
%
\begin{isamarkuptext}%
This lemma demonstrates the problem with the naive interpretation. The FUL universalizes across people
but the naive formalization universalizes across worlds. Because this interpretation is so naive, it is 
limited in its power. However, this serves as an example of the kind of reasoning that 
Isabelle empowers us to do. Even this simple argument has philosophical consequences. It tells us that
reading the FUL as a claim about consistency across possible worlds, instead of consistency across 
agents, leads to counterintuitive conclusions.%
\end{isamarkuptext}\isamarkuptrue%
%
\isadelimtheory
%
\endisadelimtheory
%
\isatagtheory
\isacommand{end}\isamarkupfalse%
%
\endisatagtheory
{\isafoldtheory}%
%
\isadelimtheory
%
\endisadelimtheory
\isanewline
%
\end{isabellebody}%
\endinput
%:%file=~/Desktop/cs91r/paper/paper31.thy%:%
%:%24=7%:%
%:%36=9%:%
%:%37=10%:%
%:%38=11%:%
%:%39=12%:%
%:%40=13%:%
%:%41=14%:%
%:%42=15%:%
%:%51=17%:%
%:%63=19%:%
%:%64=20%:%
%:%65=21%:%
%:%66=22%:%
%:%67=23%:%
%:%68=24%:%
%:%69=25%:%
%:%70=26%:%
%:%71=27%:%
%:%72=28%:%
%:%73=29%:%
%:%74=30%:%
%:%75=31%:%
%:%76=32%:%
%:%85=34%:%
%:%97=36%:%
%:%98=37%:%
%:%99=38%:%
%:%101=40%:%
%:%102=40%:%
%:%103=41%:%
%:%105=42%:%
%:%106=43%:%
%:%109=45%:%
%:%110=46%:%
%:%111=47%:%
%:%112=48%:%
%:%113=49%:%
%:%114=50%:%
%:%115=51%:%
%:%116=52%:%
%:%118=54%:%
%:%119=54%:%
%:%121=55%:%
%:%124=57%:%
%:%126=59%:%
%:%127=59%:%
%:%130=60%:%
%:%134=60%:%
%:%135=60%:%
%:%137=61%:%
%:%138=62%:%
%:%139=63%:%
%:%149=65%:%
%:%150=66%:%
%:%151=67%:%
%:%152=68%:%
%:%153=69%:%
%:%154=70%:%
%:%155=71%:%
%:%156=72%:%
%:%157=73%:%
%:%158=74%:%
%:%159=75%:%
%:%160=76%:%
%:%161=77%:%
%:%162=78%:%
%:%163=79%:%
%:%164=80%:%
%:%165=81%:%
%:%166=82%:%
%:%167=83%:%
%:%168=84%:%
%:%170=86%:%
%:%171=86%:%
%:%172=87%:%
%:%174=89%:%
%:%175=90%:%
%:%176=91%:%
%:%177=92%:%
%:%178=93%:%
%:%179=94%:%
%:%180=95%:%
%:%181=96%:%
%:%182=97%:%
%:%183=98%:%
%:%184=99%:%
%:%185=100%:%
%:%186=101%:%
%:%195=103%:%
%:%207=105%:%
%:%208=106%:%
%:%209=107%:%
%:%210=108%:%
%:%211=109%:%
%:%215=111%:%
%:%219=113%:%
%:%220=114%:%
%:%221=115%:%
%:%222=116%:%
%:%224=119%:%
%:%225=119%:%
%:%226=119%:%
%:%228=119%:%
%:%232=119%:%
%:%233=119%:%
%:%235=120%:%
%:%236=121%:%
%:%237=122%:%
%:%238=122%:%
%:%240=123%:%
%:%241=124%:%
%:%242=125%:%
%:%243=125%:%
%:%244=126%:%
%:%306=173%:%
%:%307=174%:%
%:%308=175%:%
%:%309=176%:%
%:%310=177%:%
%:%311=178%:%
%:%312=179%:%
%:%314=181%:%
%:%315=181%:%
%:%316=182%:%
%:%317=183%:%
%:%318=183%:%
%:%320=183%:%
%:%324=183%:%
%:%325=183%:%
%:%327=184%:%
%:%328=185%:%
%:%329=186%:%
%:%330=187%:%
%:%331=187%:%
%:%333=188%:%
%:%334=189%:%
%:%335=190%:%
%:%345=192%:%
%:%346=193%:%
%:%347=194%:%
%:%348=195%:%
%:%350=197%:%
%:%351=197%:%
%:%352=198%:%
%:%353=199%:%
%:%354=200%:%
%:%355=200%:%
%:%357=200%:%
%:%361=200%:%
%:%362=200%:%
%:%364=201%:%
%:%365=202%:%
%:%366=203%:%
%:%367=204%:%
%:%368=204%:%
%:%370=205%:%
%:%380=207%:%
%:%381=208%:%
%:%382=209%:%
%:%384=211%:%
%:%385=211%:%
%:%386=212%:%
%:%387=213%:%
%:%388=214%:%
%:%389=214%:%
%:%391=214%:%
%:%395=214%:%
%:%396=214%:%
%:%398=215%:%
%:%399=216%:%
%:%400=217%:%
%:%401=218%:%
%:%402=218%:%
%:%404=219%:%
%:%405=220%:%
%:%415=222%:%
%:%419=224%:%
%:%420=225%:%
%:%421=226%:%
%:%422=227%:%
%:%423=228%:%
%:%425=230%:%
%:%426=230%:%
%:%427=231%:%
%:%428=232%:%
%:%429=233%:%
%:%430=233%:%
%:%432=233%:%
%:%436=233%:%
%:%437=233%:%
%:%439=234%:%
%:%440=235%:%
%:%441=236%:%
%:%442=237%:%
%:%443=237%:%
%:%445=238%:%
%:%446=238%:%
%:%469=253%:%
%:%470=254%:%
%:%471=255%:%
%:%472=256%:%
%:%474=258%:%
%:%475=258%:%
%:%476=259%:%
%:%477=260%:%
%:%478=261%:%
%:%479=262%:%
%:%480=263%:%
%:%481=263%:%
%:%488=264%:%
%:%489=264%:%
%:%490=265%:%
%:%491=265%:%
%:%492=266%:%
%:%493=266%:%
%:%494=267%:%
%:%495=267%:%
%:%496=267%:%
%:%497=267%:%
%:%499=268%:%
%:%500=269%:%
%:%501=269%:%
%:%502=270%:%
%:%503=270%:%
%:%504=270%:%
%:%506=271%:%
%:%507=272%:%
%:%508=273%:%
%:%509=274%:%
%:%510=275%:%
%:%511=275%:%
%:%513=276%:%
%:%523=278%:%
%:%524=279%:%
%:%525=280%:%
%:%526=281%:%
%:%527=282%:%
%:%529=284%:%
%:%530=284%:%
%:%531=285%:%
%:%532=286%:%
%:%533=287%:%
%:%534=287%:%
%:%536=287%:%
%:%540=287%:%
%:%541=287%:%
%:%543=288%:%
%:%544=289%:%
%:%545=290%:%
%:%546=291%:%
%:%547=292%:%
%:%548=293%:%
%:%558=295%:%
%:%562=297%:%
%:%563=298%:%
%:%564=299%:%
%:%566=301%:%
%:%567=301%:%
%:%568=302%:%
%:%571=303%:%
%:%575=303%:%
%:%576=303%:%
%:%578=304%:%
%:%579=304%:%
%:%581=305%:%
%:%582=306%:%
%:%583=307%:%
%:%584=308%:%
%:%585=309%:%
%:%595=311%:%
%:%596=312%:%
%:%597=313%:%
%:%599=315%:%
%:%600=315%:%
%:%601=316%:%
%:%604=317%:%
%:%608=317%:%
%:%609=317%:%
%:%610=317%:%
%:%651=337%:%
%:%663=339%:%
%:%664=340%:%
%:%665=341%:%
%:%666=342%:%
%:%668=344%:%
%:%669=344%:%
%:%670=345%:%
%:%671=345%:%
%:%683=353%:%
%:%686=354%:%
%:%687=354%:%
%:%689=355%:%
%:%690=356%:%
%:%691=357%:%
%:%692=358%:%
%:%693=358%:%
%:%694=359%:%
%:%695=360%:%
%:%696=360%:%
%:%697=361%:%
%:%700=362%:%
%:%704=362%:%
%:%705=362%:%
%:%707=363%:%
%:%708=364%:%
%:%718=366%:%
%:%719=367%:%
%:%720=368%:%
%:%724=370%:%
%:%726=372%:%
%:%727=372%:%
%:%728=373%:%
%:%729=373%:%
%:%730=374%:%
%:%731=374%:%
%:%733=375%:%
%:%734=375%:%
%:%735=376%:%
%:%736=377%:%
%:%737=377%:%
%:%739=379%:%
%:%740=380%:%
%:%741=381%:%
%:%742=382%:%
%:%744=384%:%
%:%745=384%:%
%:%746=385%:%
%:%747=386%:%
%:%748=387%:%
%:%749=387%:%
%:%752=388%:%
%:%756=388%:%
%:%757=388%:%
%:%759=389%:%
%:%760=390%:%
%:%761=391%:%
%:%762=392%:%
%:%763=392%:%
%:%765=393%:%
%:%766=394%:%
%:%767=395%:%
%:%768=396%:%
%:%769=397%:%
%:%770=397%:%
%:%772=398%:%
%:%773=399%:%
%:%783=401%:%
%:%784=402%:%
%:%785=403%:%
%:%786=404%:%
%:%787=405%:%
%:%788=406%:%
%:%789=407%:%
%:%790=408%:%
%:%792=410%:%
%:%793=410%:%
%:%794=411%:%
%:%795=412%:%
%:%796=413%:%
%:%797=413%:%
%:%799=413%:%
%:%803=413%:%
%:%804=413%:%
%:%806=414%:%
%:%807=415%:%
%:%808=416%:%
%:%809=417%:%
%:%810=418%:%
%:%811=419%:%
%:%821=421%:%
%:%822=422%:%
%:%823=423%:%
%:%824=424%:%
%:%825=425%:%
%:%826=426%:%
%:%834=430%:%
%:%842=430%:%