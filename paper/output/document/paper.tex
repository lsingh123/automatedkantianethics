%
\begin{isabellebody}%
\setisabellecontext{paper}%
%
\isadelimtheory
%
\endisadelimtheory
%
\isatagtheory
%
\endisatagtheory
{\isafoldtheory}%
%
\isadelimtheory
%
\endisadelimtheory
%
\isadelimdocument
%
\endisadelimdocument
%
\isatagdocument
%
\isamarkupsection{Introduction%
}
\isamarkuptrue%
%
\endisatagdocument
{\isafolddocument}%
%
\isadelimdocument
%
\endisadelimdocument
%
\begin{isamarkuptext}%
As artifical reasoners become increasingly powerful, computers become capable of
performing complex ethical reasoning. Moreover, as regulators and academics call for ``ethical AI,"
the field of machine ethics becomes increasingly popular. Much prior work on machine ethics implements
relatively simple philosophical theories, partially as an artifact of this work emerging out of
Computer Science or Mathematics departments \cite{mesurvey}. Such work rarely capitalizes on the 
centuries of philosophical debate discussing the ethical theories in question. In this thesis, I attempt
to automate Kantian ethics, a sophisticated ethical theory, while staying faithful to philosophical
literature on the subject.

Automating a sophisticated ethical theory is interesting for two reasons. First, the proliferation of 
artifically autonomous agents is creating and will continue to create a demand for 
automated ethics. These agents must be able to reason about complex ethical theories 
that withstand philosophical scrutiny. Second, just as automated mathematical reasoning is a tool
for mathematicians, automated ethical reasoning is a tool that philosophers can use 
when reasoning about ethics. I argue that computational ethics can serve as another tool in a 
philosopher's toolbox, like a thought experiment or counterexample.

Modelling ethics without sacrificing the intracacies of an ethical theory is a 
challenging computational and philosophical problem. Simple and intuitive computational approaches, 
such as encoding ethics as a constraint satisfaction problem, fail to capture
the complexity of most philosophically plausible systems. Constraint satisfaction systems often default to some version of utiliarianism, 
the principle of doing the most good for the most people. Alternatively, they model basic moral 
principles such as ``do not kill," without modelling the theory that these principles originated from.
Modelling a more complex ethical theory will not only enable smarter philosophical machines, it will
also empower philosophers to study more complex ethical issues with the computer's help. The entire
field of philosophy is devoted to developing and testing robust ethical theories. Plausible machine
ethics must draw on plausible moral philosophy. Despite the importance of formalizing complex moral theories,
it is not immediately clear how to formalize complex theories like virtue ethics.

Kantian ethics, often described as ``formal" due to its rigid, rule-based structure, has been often 
floated as an attractive theory to automate \cite{powers, BL, lin}. This project's objective is to 
automate Kantian ethics. Roughly, my approach is to represent Kantian
ethics as an axioms in Carmo and Jones' Dyadic Deontic Logic (DDL), a modal logic designed to reason
about concepts like obligation \cite{CJDDL}. Such a representation of Kantian ethics in a logic
is called a ``formalization." I then embed this DDL formalization in higher-order logic 
(HOL) and implement it in the Isabelle/HOL theorem prover  \cite{isabelle}. I can then use Isabelle 
to automatically prove or disprove theorems (such as, ``murder is wrong") in my custom formalization, generating results
derived from the categorical imperative. Essentially, the computer is performing ethical reasoning using
the framework of Kantian ethics.

First, I recreate prior work implementing Dyadic Deontic Logic in Isabelle/HOL. Next, I present two 
prior attempts to formalize Kantian ethics in modal logic. The first is a naive
interpretation of Kant's categorical imperative that collapses to the base logic itself. The second is 
Moshe Kroy's partial formalization of the categorical imperative. As part of this presentation, I also
create a testing framework that can be used to evaluate these formalizations. In effect, my tests formalize
expected properties of the categorical imperative (such as the fact that it prohibits murder) and test 
whether or not these properties hold in a given formalization. My testing framework offers a method for
evaluating different formalizations against moral intuitions and philosophical literature. Second, based on
lessons learned from these prior formalization attempts, I contribute my own custom formalization of
the categorical imperative. My testing framework demonstrates how my custom formalization improves upon
the prior work. Third, I apply my system to an ethical dilemma to demonstrate its potential use. I also
consider the philosophical implications and value of automated ethics, studying questions about the 
possibility and responsibility of automated ethics.

I contribute implementations of two different interpretations of the categorical imperative, 
examples of how each implementation can be used to model and solve ethical scenarios, and tests that
examine ethical and logical properties of the system, including logical consistency, consistency
of obligation, and possibility of permissibility. I also contribute a logical formalization of the categorical
imperative that improves on previous work, an implementation of this formalization, and evidence of 
its improvement. Lastly, I demonstrate how such a system could be applied. The implementations themselves are usable models 
of ethical principles and the tests represent the kind of philosophical work that formalized ethics 
can contribute.

The rest of this project is structured as follows. In Chapter 1, I introduce the logical, computational, and 
philosophical frameworks underlying my project. In Chapter 2, I implement and test prior formalizations of the categorical
imperative. In Chapter 3, I contribute my own custom formalization and test it. Finally, in Chapter 4, 
I apply my custom formalization to an example ethical dilemma.%
\end{isamarkuptext}\isamarkuptrue%
%
\begin{isamarkuptext}%
In the rest of this chapter, I present the background necessary to undersand my work. The goal 
of this project is to automate sophisticated ethical reasoning. This requires three
components. First, the choice of an ethical theory that is both intuitively attractive and 
lends itself to formalization. Second, the choice of formal logic to model the theory in. Third, the 
choice of automation engine to implement the formal model in. Section \ref{sec:kant} introduces Kantian
ethics, Section \ref{sec:DDL} explains Carmo and Jones's Dyadic Deontic Logic \cite{CJDDL} as a 
base logic, and Section \ref{sec:isabelle} presents the Isabelle/HOL implementation of the logic.%
\end{isamarkuptext}\isamarkuptrue%
%
\isadelimdocument
%
\endisadelimdocument
%
\isatagdocument
%
\isamarkupsubsection{Kantian Ethics \label{sec:kant}%
}
\isamarkuptrue%
%
\endisatagdocument
{\isafolddocument}%
%
\isadelimdocument
%
\endisadelimdocument
%
\begin{isamarkuptext}%
Kantian ethics is an attractive choice of ethical theory to automate. Kant's writings inspired much of Western analytic philosophy. 
In addition to the rich neo-Kantian program, almost all major philosophical traditions after Kant have
engaged with his work. Much of Western libertarian political thought is inspired by Kant's deontology,
and his ethics have bled into household ethical thought. Deontology is seen as one of the three major 
schools of Western analytic ethics.

Understanding the ethic's potential for formalization requires understanding Kant's system. Kant argues 
that if morality exists, it must take the form of an inviolable rule, which he calls the ``categorical
imperative." He presents three formulations of the categorical imperative, as well as a robust defense
of them in his seminal work on ethics \cite{groundwork}. He argues that all three formulations are 
equivalent. 

The first formulation of the categorical imperative is the ``Formula of Universal Law." 

$\textbf{Definition}$ \emph{(Formula of Universal Law)}

\emph{I ought never to act except in such a way that I could also will that my maxim should become a universal law} \cite{groundwork}

A ``maxim" is a moral rule such as, ``I can murder someone to take their job". ``Willing" a maxim is 
equivalent to acting on that rule. The FUL creates a thought experiment called the universalizability 
test: to decide if a maxim is permissible, imagine what would happen if everyone willed that maxim. 
If your imagined world yields a contradiction, the maxim is prohibited\footnote{There is another case here.
What happens if the imagined world does not yield a logical contradiction, but is still undesirable? Kant 
distinguishes between these cases as contradictions in conception and contradictions in will. Both yield moral
prohibitions, but contradictions in conception generate stronger contradictions and therefore stronger obligations. 
This paper will focus entirely on contradictions in conception, for this is the only kind of contradiction 
for which a strict logical interpretation makes sense. For a complete treatment of the difference, see \cite{groundwork, KorsgaardFUL}}. Intuitively, the FUL asks 
the question, "What would happen if everyone did that?" \cite{KorsgaardFUL}. 

As an example, let's apply the universalizability test to the maxim of murdering others to take their job.
Universalized, anyone who wants a job can murder the person currently holding that job. It is contradictory
to simultaneously will that I acquire a job (through murder) and also that someone can take that job from me 
whenever they want (also by murder). The maxim thwarts its own end, and is thus internally contradictory.
 Therefore, this maxim is prohibited.

The universalizability test makes the ``formal" nature of Kant's ethics immediately clear. Formal 
logic has the tools to universalize a maxim (apply a universal quantifier) and to test for 
contradictions (test for inconsistency). 

Kant also presents two additional formulations of the categorical imperative. 

$\textbf{Definition}$ \emph{(Formula of Humanity)}

\emph{The Formula of Humanity (FUH)
is to act in such a way ``that you use humanity, whether in your own person or in the person
of any other, always at the same time as an end, never merely as a means."}\cite{groundwork}

\medskip 

$\textbf{Definition}$ \emph{(Kingdom of Ends)}

\emph{The third formulation of the categorical imperative states that ``we should so act that we may 
think of ourselves as legislating universal laws through our maxims."}\cite{KorsgaardFUL}

The last two formulations are not as obviously formal as the FUL, but they can still be
modelled in logic. Because Kantian ethics presents a series of rules, a logical system can encode 
the theory by modelling each rule as an axiom.

The above outline is a brief and incomplete picture of a rich philosophical tradition. Kantian scholars
debate the meaning of each formulation of the categorical imperative and develop views far more 
nuanced than those above. For the purposes of this paper, I will adopt Kant's three original 
formulations presented above. Note additionally that Kantian ethics is widely disputed. I do not present 
a defense of Kant's ethic in this paper. My approach to formalizing ethics can be applied to other 
theories as well.  

This paper aims to formalize Kant's ethic as faithfully as possible. This is an important choice. 
While it is tempting to modify or simplify an ethical theory in seemingly insignificant way, these 
choices often have ramifications. The entire field of neo-Kantian thought 
has been exploring versions of Kant's ethical theory for centuries. I will not attempt to present a 
radically new conception of Kant's ethic, but will instead draw on philosophical expertise regarding
the content and justification of the categorical imperative.%
\end{isamarkuptext}\isamarkuptrue%
%
\isadelimdocument
%
\endisadelimdocument
%
\isatagdocument
%
\isamarkupsubsection{Dyadic Deontic Logic \label{sec:DDL}%
}
\isamarkuptrue%
%
\isamarkupsubsubsection{Deontic Logic%
}
\isamarkuptrue%
%
\endisatagdocument
{\isafolddocument}%
%
\isadelimdocument
%
\endisadelimdocument
%
\begin{isamarkuptext}%
Traditional modal logics include the necessitation operator, denoted as $\Box$. In simple modal logic
using the Kripke semantics s \cite{cresswell}, $\Box p$ is true at a world $w$ if $p$ is true at all of $w$'s neighbors. 
These logics also usually contain the $\diamond$ operator, representing possibility, where
 $\diamond p \iff \sim \Box \sim p$. Additionally, modal logics include operators of propositional 
logic like $\sim, \wedge, \vee, \rightarrow$.

A deontic logic is a special kind of modal logic designed to reason about obligation. Standard deontic
logic \cite{cresswell, sep-logic-deontic} replaces $\Box$ with the obligation operator
$O$, and $\diamond$ with the permissibility operator $P$. Using the Kripke semantics for $O$, $O p$ 
is true at $w$ if $p$ is true at all  ideal deontic alternatives to $w$. The $O$ operator in SDL
takes a single argument (the formula that is obligatory), and is thus called a monadic deontic operator.

 While SDL is appreciable for its simplicity, it suffers a variety of well-documented paradoxes, 
including contrary-to-duty paradoxes \footnote{The paradigm case of a contrary-to-duty paradox is the 
Chisholm paradox. Consider the following statements: \begin{enumerate}
\item It ought to be that Tom helps his neighbors
\item It ought to be that if Tom helps his neighbors, he tells them he is coming
\item If Tom does not help his neighbors, he ought not tell them that he is coming
\item Tom does not help his neighbors
\end{enumerate} 
These premises contradict themselves, because items (2)-(4) imply that Tom ought not help his neighbors. The 
contradiction results because the logic cannot handle violations of duty mixed with
conditionals. \cite{chisholm, ctd}
}. 
In situations where duty is violated, the logic breaks down 
and produces paradoxical results. Thus, I use an improved deontic logic instead of SDL for this work.%
\end{isamarkuptext}\isamarkuptrue%
%
\isadelimdocument
%
\endisadelimdocument
%
\isatagdocument
%
\isamarkupsubsubsection{Dyadic Deontic Logic%
}
\isamarkuptrue%
%
\endisatagdocument
{\isafolddocument}%
%
\isadelimdocument
%
\endisadelimdocument
%
\begin{isamarkuptext}%
I use as my base logic Carmo and Jones's dyadic deontic logic, or DDL, which improves on SDL \cite{CJDDL}. 
It introduces a dyadic obligation operator $O\{A \vert B\}$ 
to represent the sentence ``A is obligated in the context B". This gracefully handles contrary-to-duty
conditionals. The obligation operator uses a neighborhood semantics \cite{neighborhood1, neighborhood2}, instead of the Kripke semantics. 
Carmo and Jones define a function $ob$ that maps from worlds to sets of sets of worlds. Intuitively, 
each world is mapped to the set of propositions obligated at that world, where a proposition $p$ is defined as 
the worlds at which the $p$ is true.

DDL also includes other modal operators. In addition to $\Box$ and $\diamond$, DDL also has a notion
of actual obligation and possible obligation, represented by operators $O_a$ and $O_p$ respectively. 
These notions are accompanied by the corresponding modal operators $\Box_a, \diamond_a, \Box_p, \diamond_p$. 
These operators use a Kripke semantics, with the functions $av$ and $pv$ mapping a world $w$ to the set 
of corresponding actual or possible versions of $w$. 

For more of fine-grained properties of DDL see \cite{CJDDL} or this project's source code. DDL is a heavy logic and contains modal operators 
that aren't necessary for my analysis. While this expressivity is powerful, it may also cause performance
impacts. DDL has a large set of axioms involving quantification over complex higher-order
logical expressions. Proofs involving these axioms will be computationally expensive.  Benzmueller 
and Parent warned me that this may become a problem if Isabelle's automated proof tools begin to time out.%
\end{isamarkuptext}\isamarkuptrue%
%
\isadelimtheory
%
\endisadelimtheory
%
\isatagtheory
%
\endisatagtheory
{\isafoldtheory}%
%
\isadelimtheory
%
\endisadelimtheory
%
\end{isabellebody}%
\endinput
%:%file=~/Desktop/cs91r/paper/paper.thy%:%
%:%24=6%:%
%:%36=8%:%
%:%37=9%:%
%:%38=10%:%
%:%39=11%:%
%:%40=12%:%
%:%41=13%:%
%:%42=14%:%
%:%43=15%:%
%:%44=16%:%
%:%45=17%:%
%:%46=18%:%
%:%47=19%:%
%:%48=20%:%
%:%49=21%:%
%:%50=22%:%
%:%51=23%:%
%:%52=24%:%
%:%53=25%:%
%:%54=26%:%
%:%55=27%:%
%:%56=28%:%
%:%57=29%:%
%:%58=30%:%
%:%59=31%:%
%:%60=32%:%
%:%61=33%:%
%:%62=34%:%
%:%63=35%:%
%:%64=36%:%
%:%65=37%:%
%:%66=38%:%
%:%67=39%:%
%:%68=40%:%
%:%69=41%:%
%:%70=42%:%
%:%71=43%:%
%:%72=44%:%
%:%73=45%:%
%:%74=46%:%
%:%75=47%:%
%:%76=48%:%
%:%77=49%:%
%:%78=50%:%
%:%79=51%:%
%:%80=52%:%
%:%81=53%:%
%:%82=54%:%
%:%83=55%:%
%:%84=56%:%
%:%85=57%:%
%:%86=58%:%
%:%87=59%:%
%:%88=60%:%
%:%89=61%:%
%:%90=62%:%
%:%91=63%:%
%:%92=64%:%
%:%93=65%:%
%:%94=66%:%
%:%95=67%:%
%:%96=68%:%
%:%97=69%:%
%:%98=70%:%
%:%99=71%:%
%:%100=72%:%
%:%101=73%:%
%:%102=74%:%
%:%106=77%:%
%:%107=78%:%
%:%108=79%:%
%:%109=80%:%
%:%110=81%:%
%:%111=82%:%
%:%112=83%:%
%:%121=85%:%
%:%133=87%:%
%:%134=88%:%
%:%135=89%:%
%:%136=90%:%
%:%137=91%:%
%:%138=92%:%
%:%139=93%:%
%:%140=94%:%
%:%141=95%:%
%:%142=96%:%
%:%143=97%:%
%:%144=98%:%
%:%145=99%:%
%:%146=100%:%
%:%147=101%:%
%:%148=102%:%
%:%149=103%:%
%:%150=104%:%
%:%151=105%:%
%:%152=106%:%
%:%153=107%:%
%:%154=108%:%
%:%155=109%:%
%:%156=110%:%
%:%157=111%:%
%:%158=112%:%
%:%159=113%:%
%:%160=114%:%
%:%161=115%:%
%:%162=116%:%
%:%163=117%:%
%:%164=118%:%
%:%165=119%:%
%:%166=120%:%
%:%167=121%:%
%:%168=122%:%
%:%169=123%:%
%:%170=124%:%
%:%171=125%:%
%:%172=126%:%
%:%173=127%:%
%:%174=128%:%
%:%175=129%:%
%:%176=130%:%
%:%177=131%:%
%:%178=132%:%
%:%179=133%:%
%:%180=134%:%
%:%181=135%:%
%:%182=136%:%
%:%183=137%:%
%:%184=138%:%
%:%185=139%:%
%:%186=140%:%
%:%187=141%:%
%:%188=142%:%
%:%189=143%:%
%:%190=144%:%
%:%191=145%:%
%:%192=146%:%
%:%193=147%:%
%:%194=148%:%
%:%195=149%:%
%:%196=150%:%
%:%197=151%:%
%:%198=152%:%
%:%199=153%:%
%:%200=154%:%
%:%201=155%:%
%:%202=156%:%
%:%203=157%:%
%:%212=161%:%
%:%216=163%:%
%:%228=165%:%
%:%229=166%:%
%:%230=167%:%
%:%231=168%:%
%:%232=169%:%
%:%233=170%:%
%:%234=171%:%
%:%235=172%:%
%:%236=173%:%
%:%237=174%:%
%:%238=175%:%
%:%239=176%:%
%:%240=177%:%
%:%241=178%:%
%:%242=179%:%
%:%243=180%:%
%:%244=181%:%
%:%245=182%:%
%:%246=183%:%
%:%247=184%:%
%:%248=185%:%
%:%249=186%:%
%:%250=187%:%
%:%251=188%:%
%:%252=189%:%
%:%253=190%:%
%:%262=192%:%
%:%274=194%:%
%:%275=195%:%
%:%276=196%:%
%:%277=197%:%
%:%278=198%:%
%:%279=199%:%
%:%280=200%:%
%:%281=201%:%
%:%282=202%:%
%:%283=203%:%
%:%284=204%:%
%:%285=205%:%
%:%286=206%:%
%:%287=207%:%
%:%288=208%:%
%:%289=209%:%
%:%290=210%:%
%:%291=211%:%
%:%292=212%:%