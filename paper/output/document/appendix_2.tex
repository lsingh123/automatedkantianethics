%
\begin{isabellebody}%
\setisabellecontext{appendix{\isacharunderscore}{\isadigit{2}}}%
%
\isadelimtheory
%
\endisadelimtheory
%
\isatagtheory
%
\endisatagtheory
{\isafoldtheory}%
%
\isadelimtheory
%
\endisadelimtheory
%
\isadelimdocument
%
\endisadelimdocument
%
\isatagdocument
%
\isamarkupsection{Additional Tests \label{weirdtests}%
}
\isamarkuptrue%
%
\endisatagdocument
{\isafolddocument}%
%
\isadelimdocument
%
\endisadelimdocument
%
\begin{isamarkuptext}%
In this section, I show that my system can correctly show prohibitions against actions that 
are impossible to universalize, against conventional acts, and against natural acts. In the process
of running these tests, I discover and resolve an ambiguity in Korsgaard's canonical example of a 
prohibited maxim. I show that her maxim actually has two readings, one reading under which it is a 
natural act, and another under which it is a conventional act. My formalization can correctly
handle both readings. The recognition of this ambiguity is another example of the power of 
computational ethics, and demonstrates that the process of making a philosophical argument precise 
enough to represent to a machine can generate philosophical insights.

In this section, I show that the maxim, ``When strapped for cash, falsely promise to pay your friend back
to get some easy money," is prohibited. Korsgaard uses this example when arguing for the practical
contradiction interpretation of the FUL \citep{KorsgaardFUL}. She argues that this maxim describes
a conventional act, or an act that is possible due to some pre-existing social system, and is thus
within reach for the logical contradiction interpretation. Natural acts, on the other hand, are acts
that are possible simply due to the laws of nature, such as murder, and the logical contradiction
interpretation cannot correctly handle such acts. 

I argue that, in addition to Korsgaard's reading of this maxim as a conventional act, there is also
another reading of the maxim as a natural act. 
Under Korsgaard's reading, the act of falsely promising is read as
entering a pre-existing, implicit, social system of promising with no intention of upholding your 
promise. Under the second reading, the act of falsely promising is equivalent to uttering the worlds 
``I promise X'' without intending to do X. There is a difference
between promising as an act with meaning in a larger social structure and merely uttering the words
``I promise,'' so these two readings are distinct.

Under Korsgaard's reading, the maxim fails because falsely promising is no longer possible in a world where 
everyone everyone does so, or because the action of falsely promising is literally impossible to
universalize. Recall that this is how the logical contradiction interpretation prohibits this maxim—falsely 
promising is no longer possible when universalized because the institution of promising breaks down. 
First, I formalize this argument and show that my system can show the wrongness of the false
promising maxim under Korsgaard's reading. This also shows that my system can show the wrongness of a
maxim that is impossible to universalize. 

To formalize this argument, I first define the relevant maxim.%
\end{isamarkuptext}\isamarkuptrue%
\isacommand{consts}\isamarkupfalse%
\ when{\isacharunderscore}strapped{\isacharunderscore}for{\isacharunderscore}cash{\isacharcolon}{\isacharcolon}t\isanewline
%
\isamarkupcmt{This constant represents the circumstances ``when strapped for cash.''%
}\isanewline
\isacommand{consts}\isamarkupfalse%
\ falsely{\isacharunderscore}promise{\isacharcolon}{\isacharcolon}os\isanewline
%
\isamarkupcmt{This constant represents the act ``make a false promise to pay a loan back.''%
}\isanewline
\isacommand{consts}\isamarkupfalse%
\ to{\isacharunderscore}get{\isacharunderscore}easy{\isacharunderscore}cash{\isacharcolon}{\isacharcolon}t\isanewline
%
\isamarkupcmt{This constant represents the goal ``to get some money.''%
}\isanewline
\isacommand{abbreviation}\isamarkupfalse%
\ false{\isacharunderscore}promising{\isacharcolon}{\isacharcolon}maxim\ \isakeyword{where}\ \isanewline
{\isachardoublequoteopen}false{\isacharunderscore}promising\ {\isasymequiv}\ {\isacharparenleft}when{\isacharunderscore}strapped{\isacharunderscore}for{\isacharunderscore}cash{\isacharcomma}\ falsely{\isacharunderscore}promise{\isacharcomma}\ to{\isacharunderscore}get{\isacharunderscore}easy{\isacharunderscore}cash{\isacharparenright}{\isachardoublequoteclose}\isanewline
%
\isamarkupcmt{Armed with the circumstances, act, and goal above, I can define the example false promising maxim as a tuple.%
}%
\begin{isamarkuptext}%
The logical objects above are empty or thin, in the sense that I haven't specified any of their 
relevant properties, such as a robust definition of promising or any system of currency. I define 
only the properties absolutely necessary for my argument as assumptions and show that, if the maxim 
above satisfies the assumed properties, it is prohibited. Specifically, I am interested in Korsgaard's
reading of this maxim, under which promising is a social convention that breaks down when abused. Instead
of formally defining a conventional act, which requires wading into complex debates about trust and 
social contracts, I merely focus on the fact that, unders this reading, not everyone can falsely 
promise universally. Whatever kind of social convention promising is, my argument merely relies on
the impossibility of breaking it.%
\end{isamarkuptext}\isamarkuptrue%
\isacommand{abbreviation}\isamarkupfalse%
\ everyone{\isacharunderscore}can{\isacharprime}t{\isacharunderscore}lie\ \isakeyword{where}\ \isanewline
{\isachardoublequoteopen}everyone{\isacharunderscore}can{\isacharprime}t{\isacharunderscore}lie\ {\isasymequiv}\ {\isasymforall}w{\isachardot}\ {\isasymnot}\ {\isacharparenleft}{\isasymforall}s{\isachardot}\ falsely{\isacharunderscore}promise{\isacharparenleft}s{\isacharparenright}\ w{\isacharparenright}\ {\isachardoublequoteclose}\isanewline
%
\isamarkupcmt{The above formula reads, ``At all worlds, it is not the case that everyone falsely promises.''%
}%
\begin{isamarkuptext}%
With this abbreviation, I show that if not everyone can falsely promise simultaneously, then
the constructed maxim about falsely promising is prohibited.%
\end{isamarkuptext}\isamarkuptrue%
\isacommand{lemma}\isamarkupfalse%
\ falsely{\isacharunderscore}promising{\isacharunderscore}korsgaard{\isacharunderscore}interpretation{\isacharcolon}\isanewline
\ \ \isakeyword{assumes}\ {\isachardoublequoteopen}{\isasymforall}w{\isachardot}\ when{\isacharunderscore}strapped{\isacharunderscore}for{\isacharunderscore}cash\ w{\isachardoublequoteclose}\isanewline
%
\isamarkupcmt{Restrict our focus to worlds in which the circumstance of being strapped for cash holds. 
A technical detail.%
}\isanewline
\ \ \isakeyword{assumes}\ {\isachardoublequoteopen}{\isasymforall}s{\isachardot}\ {\isasymTurnstile}\ {\isacharparenleft}well{\isacharunderscore}formed\ false{\isacharunderscore}promising\ s{\isacharparenright}{\isachardoublequoteclose}\isanewline
%
\isamarkupcmt{Initial set-up: the falsely promising maxim is well-formed.%
}\isanewline
\ \ \isakeyword{assumes}\ everyone{\isacharunderscore}can{\isacharprime}t{\isacharunderscore}lie\isanewline
%
\isamarkupcmt{The assumption that this is Korsgaard's reading of the maxim, in which everyone cannot falsely promise
simultaneously.%
}\isanewline
\ \ \isakeyword{shows}\ {\isachardoublequoteopen}{\isasymforall}s{\isachardot}\ {\isasymTurnstile}\ {\isacharparenleft}prohibited\ false{\isacharunderscore}promising\ s{\isacharparenright}{\isachardoublequoteclose}\isanewline
%
\isadelimproof
%
\endisadelimproof
%
\isatagproof
\isacommand{proof}\isamarkupfalse%
{\isacharminus}\isanewline
\ \ \isacommand{have}\isamarkupfalse%
\ {\isachardoublequoteopen}{\isasymforall}s{\isachardot}\ not{\isacharunderscore}universalizable\ false{\isacharunderscore}promising\ s{\isachardoublequoteclose}\isanewline
\ \ \ \ \isacommand{by}\isamarkupfalse%
\ {\isacharparenleft}simp\ add{\isacharcolon}\ assms{\isacharparenleft}{\isadigit{3}}{\isacharparenright}\ assms{\isacharparenleft}{\isadigit{1}}{\isacharparenright}{\isacharparenright}\isanewline
%
\isamarkupcmt{As in the proofs in Chapter \ref{applications}, once I split this proof into this intermediate lemma,
Isabelle can automatically complete the proof.%
}\isanewline
\ \ \isacommand{thus}\isamarkupfalse%
\ {\isacharquery}thesis\isanewline
\ \ \ \ \isacommand{using}\isamarkupfalse%
\ FUL\ assms{\isacharparenleft}{\isadigit{2}}{\isacharparenright}\ \isacommand{by}\isamarkupfalse%
\ blast\ \isanewline
\isacommand{qed}\isamarkupfalse%
%
\endisatagproof
{\isafoldproof}%
%
\isadelimproof
%
\endisadelimproof
%
\begin{isamarkuptext}%
The above lemma shows that, under Korsgaard's reading of promising as a conventional act, 
my system can show that falsely promising is prohibited. This means that my system passes both
the conventional act test and the test that requires showing the wrongness of actions that are 
impossible to universalize. Next, I show that my system can show a prohibition against this maxim
even under the second reading, which understands it as a natural act.

Under the second reading of this maxim, the act ``falsely promising'' refers to uttering the
sentence ``I promise to do X'' with no intention of actually doing X. This is a natural act because the act of uttering a sentence does not rely 
on any conventions, merely the laws of nature governing how your mouth and vocal cords behave.\footnote{
Linguistic relativists may take issue with this claim and may argue that if the English language had 
never developed, then making this utterance would be impossible. Even if this is true, the laws of 
nature itself would not prohibit making the sounds corresponding to the English pronounciation of 
this phrase, so the act would still not be impossible in the way that a conventional act can be.} 

The logical
contradiction interpretation cannot prohibit this version of the maxim because making an utterance
is always logically possible, even if everyone else makes the same utterance. However,
under this reading, the practical contradiction interpretation prohibits this maxim because, in a world 
where false promising is universalized, no one believes promises anymore, so the utterance is no longer 
an effective way to get money. Because my system implements the stronger practical contradiction 
interpretation of the FUL, it can show the wrongness of this maxim even under this reading. First, 
I formalize this reading of the maxim.%
\end{isamarkuptext}\isamarkuptrue%
\isacommand{consts}\isamarkupfalse%
\ believed{\isacharcolon}{\isacharcolon}os\ \isanewline
\isacommand{abbreviation}\isamarkupfalse%
\ false{\isacharunderscore}promising{\isacharunderscore}not{\isacharunderscore}believed\ \isakeyword{where}\ \isanewline
{\isachardoublequoteopen}false{\isacharunderscore}promising{\isacharunderscore}not{\isacharunderscore}believed\ {\isasymequiv}\ {\isasymforall}w\ s{\isachardot}\ {\isacharparenleft}falsely{\isacharunderscore}promise{\isacharparenleft}s{\isacharparenright}\ w\ {\isasymlongrightarrow}\ {\isasymnot}\ believed{\isacharparenleft}s{\isacharparenright}\ w{\isacharparenright}{\isachardoublequoteclose}\isanewline
%
\isamarkupcmt{This abbreviation formalizes the idea that if everyone falsely promises, then no one is believed
when promising.%
}\isanewline
\isanewline
\isacommand{abbreviation}\isamarkupfalse%
\ need{\isacharunderscore}to{\isacharunderscore}be{\isacharunderscore}believed\ \isakeyword{where}\ \isanewline
{\isachardoublequoteopen}need{\isacharunderscore}to{\isacharunderscore}be{\isacharunderscore}believed\ {\isasymequiv}\ {\isasymforall}w\ s{\isachardot}\ {\isacharparenleft}{\isasymnot}\ believed{\isacharparenleft}s{\isacharparenright}\ w\ {\isasymlongrightarrow}\ \isactrlbold {\isasymnot}{\isacharparenleft}{\isacharparenleft}falsely{\isacharunderscore}promise\ s{\isacharparenright}\ \isactrlbold {\isasymrightarrow}\ to{\isacharunderscore}get{\isacharunderscore}easy{\isacharunderscore}cash{\isacharparenright}w{\isacharparenright}{\isachardoublequoteclose}\isanewline
%
\isamarkupcmt{This abbreviation formalizes the idea that if a promise is not believed, then it is not an effective
way of getting easy cash.%
}%
\begin{isamarkuptext}%
Once again, I avoid giving robust definitions of hotly debates concepts like belief. Instead, 
I represent the bare minimum logical background: false promises won't be believed when universalized,
and promises must be believed to be effective.%
\end{isamarkuptext}\isamarkuptrue%
\isacommand{lemma}\isamarkupfalse%
\ falsely{\isacharunderscore}promising{\isacharunderscore}bad{\isacharunderscore}natural{\isacharunderscore}act{\isacharcolon}\isanewline
\ \ \isakeyword{assumes}\ {\isachardoublequoteopen}{\isasymforall}w{\isachardot}\ when{\isacharunderscore}strapped{\isacharunderscore}for{\isacharunderscore}cash\ w{\isachardoublequoteclose}\isanewline
%
\isamarkupcmt{Restrict our focus to worlds in which the circumstance of being strapped for cash holds. 
A technical detail.%
}\isanewline
\ \ \isakeyword{assumes}\ {\isachardoublequoteopen}{\isasymforall}s{\isachardot}\ {\isasymTurnstile}\ {\isacharparenleft}well{\isacharunderscore}formed\ false{\isacharunderscore}promising\ s{\isacharparenright}{\isachardoublequoteclose}\isanewline
%
\isamarkupcmt{Initial set-up: the falsely promising maxim is well-formed.%
}\isanewline
\ \ \isakeyword{assumes}\ false{\isacharunderscore}promising{\isacharunderscore}not{\isacharunderscore}believed\isanewline
\ \ \isakeyword{assumes}\ need{\isacharunderscore}to{\isacharunderscore}be{\isacharunderscore}believed\isanewline
%
\isamarkupcmt{The two assumptions above.%
}\isanewline
\ \ \isakeyword{shows}\ {\isachardoublequoteopen}{\isasymforall}s{\isachardot}\ {\isasymTurnstile}\ {\isacharparenleft}prohibited\ false{\isacharunderscore}promising\ s{\isacharparenright}{\isachardoublequoteclose}\isanewline
%
\isadelimproof
%
\endisadelimproof
%
\isatagproof
\isacommand{proof}\isamarkupfalse%
{\isacharminus}\isanewline
\ \ \isacommand{have}\isamarkupfalse%
\ {\isachardoublequoteopen}{\isasymforall}s{\isachardot}\ not{\isacharunderscore}universalizable\ false{\isacharunderscore}promising\ s{\isachardoublequoteclose}\isanewline
\ \ \ \ \isacommand{using}\isamarkupfalse%
\ assms{\isacharparenleft}{\isadigit{1}}{\isacharparenright}\ assms{\isacharparenleft}{\isadigit{2}}{\isacharparenright}\ assms{\isacharparenleft}{\isadigit{3}}{\isacharparenright}\ \isacommand{by}\isamarkupfalse%
\ auto\isanewline
\ \ \isacommand{thus}\isamarkupfalse%
\ {\isacharquery}thesis\isanewline
\ \ \ \ \isacommand{using}\isamarkupfalse%
\ FUL\ assms{\isacharparenleft}{\isadigit{2}}{\isacharparenright}\ \isacommand{by}\isamarkupfalse%
\ blast\isanewline
\isacommand{qed}\isamarkupfalse%
\isanewline
%
\isamarkupcmt{With some help, Isabelle is able to show that the maxim is prohibited under this reading as well.%
}%
\endisatagproof
{\isafoldproof}%
%
\isadelimproof
%
\endisadelimproof
%
\begin{isamarkuptext}%
These proofs demonstrate that my formalization is able to correctly prohibit this maxim, whether 
it is understood as a conventional act or a natural act. 
Korsgaard argues that the practical contradiction interpretation outperforms other interpretations
of the FUL because it can show the wrongness of both conventional and natural acts. Therefore, the 
fact that my interpretation can correctly show the wrongness of both conventional and natural acts
is evidence for its correctness as a formalization of the practical contradiction interpretation.%
\end{isamarkuptext}\isamarkuptrue%
%
\isadelimproof
%
\endisadelimproof
%
\isatagproof
%
\endisatagproof
{\isafoldproof}%
%
\isadelimproof
%
\endisadelimproof
%
\isadelimdocument
%
\endisadelimdocument
%
\isatagdocument
%
\endisatagdocument
{\isafolddocument}%
%
\isadelimdocument
%
\endisadelimdocument
%
\isadelimtheory
%
\endisadelimtheory
%
\isatagtheory
%
\endisatagtheory
{\isafoldtheory}%
%
\isadelimtheory
%
\endisadelimtheory
%
\end{isabellebody}%
\endinput
%:%file=~/Desktop/cs91r/paper/appendix_2.thy%:%
%:%24=5%:%
%:%36=7%:%
%:%37=8%:%
%:%38=9%:%
%:%39=10%:%
%:%40=11%:%
%:%41=12%:%
%:%42=13%:%
%:%43=14%:%
%:%44=15%:%
%:%45=16%:%
%:%46=17%:%
%:%47=18%:%
%:%48=19%:%
%:%49=20%:%
%:%50=21%:%
%:%51=22%:%
%:%52=23%:%
%:%53=24%:%
%:%54=25%:%
%:%55=26%:%
%:%56=27%:%
%:%57=28%:%
%:%58=29%:%
%:%59=30%:%
%:%60=31%:%
%:%61=32%:%
%:%62=33%:%
%:%63=34%:%
%:%64=35%:%
%:%65=36%:%
%:%66=37%:%
%:%67=38%:%
%:%68=39%:%
%:%69=40%:%
%:%70=41%:%
%:%72=43%:%
%:%73=43%:%
%:%75=44%:%
%:%76=44%:%
%:%77=45%:%
%:%78=45%:%
%:%80=46%:%
%:%81=46%:%
%:%82=47%:%
%:%83=47%:%
%:%85=48%:%
%:%86=48%:%
%:%87=49%:%
%:%88=49%:%
%:%89=50%:%
%:%91=51%:%
%:%94=53%:%
%:%95=54%:%
%:%96=55%:%
%:%97=56%:%
%:%98=57%:%
%:%99=58%:%
%:%100=59%:%
%:%101=60%:%
%:%102=61%:%
%:%104=63%:%
%:%105=63%:%
%:%106=64%:%
%:%108=65%:%
%:%111=67%:%
%:%112=68%:%
%:%114=70%:%
%:%115=70%:%
%:%116=71%:%
%:%118=72%:%
%:%119=73%:%
%:%120=73%:%
%:%121=74%:%
%:%123=75%:%
%:%124=75%:%
%:%125=76%:%
%:%127=77%:%
%:%128=78%:%
%:%129=78%:%
%:%130=79%:%
%:%137=80%:%
%:%138=80%:%
%:%139=81%:%
%:%140=81%:%
%:%141=82%:%
%:%142=82%:%
%:%144=83%:%
%:%145=84%:%
%:%146=84%:%
%:%147=85%:%
%:%148=85%:%
%:%149=86%:%
%:%150=86%:%
%:%151=86%:%
%:%152=87%:%
%:%162=89%:%
%:%163=90%:%
%:%164=91%:%
%:%165=92%:%
%:%166=93%:%
%:%167=94%:%
%:%168=95%:%
%:%169=96%:%
%:%170=97%:%
%:%171=98%:%
%:%172=99%:%
%:%173=100%:%
%:%174=101%:%
%:%175=102%:%
%:%176=103%:%
%:%177=104%:%
%:%178=105%:%
%:%179=106%:%
%:%180=107%:%
%:%181=108%:%
%:%182=109%:%
%:%183=110%:%
%:%185=112%:%
%:%186=112%:%
%:%187=113%:%
%:%188=113%:%
%:%189=114%:%
%:%191=115%:%
%:%192=116%:%
%:%193=116%:%
%:%194=117%:%
%:%195=118%:%
%:%196=118%:%
%:%197=119%:%
%:%199=120%:%
%:%200=121%:%
%:%203=123%:%
%:%204=124%:%
%:%205=125%:%
%:%207=127%:%
%:%208=127%:%
%:%209=128%:%
%:%211=129%:%
%:%212=130%:%
%:%213=130%:%
%:%214=131%:%
%:%216=132%:%
%:%217=132%:%
%:%218=133%:%
%:%219=134%:%
%:%221=135%:%
%:%222=135%:%
%:%223=136%:%
%:%230=137%:%
%:%231=137%:%
%:%232=138%:%
%:%233=138%:%
%:%234=139%:%
%:%235=139%:%
%:%236=139%:%
%:%237=140%:%
%:%238=140%:%
%:%239=141%:%
%:%240=141%:%
%:%241=141%:%
%:%242=142%:%
%:%243=142%:%
%:%245=143%:%
%:%255=145%:%
%:%256=146%:%
%:%257=147%:%
%:%258=148%:%
%:%259=149%:%
%:%260=150%:%