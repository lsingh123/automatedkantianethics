%
\begin{isabellebody}%
\setisabellecontext{paper{\isadigit{2}}{\isadigit{2}}{\isadigit{4}}}%
%
\isadelimtheory
%
\endisadelimtheory
%
\isatagtheory
%
\endisatagtheory
{\isafoldtheory}%
%
\isadelimtheory
%
\endisadelimtheory
%
\isadelimdocument
%
\endisadelimdocument
%
\isatagdocument
%
\endisatagdocument
{\isafolddocument}%
%
\isadelimdocument
%
\endisadelimdocument
%
\begin{isamarkuptext}%
\textbf{Modus Ponens}%
\end{isamarkuptext}\isamarkuptrue%
\isacommand{lemma}\isamarkupfalse%
\ modus{\isacharunderscore}ponens{\isacharcolon}\ \isakeyword{assumes}\ {\isachardoublequoteopen}{\isasymTurnstile}\ A{\isachardoublequoteclose}\ \isakeyword{assumes}\ {\isachardoublequoteopen}{\isasymTurnstile}\ {\isacharparenleft}A\ \isactrlbold {\isasymrightarrow}\ B{\isacharparenright}{\isachardoublequoteclose}\isanewline
\ \ \isakeyword{shows}\ {\isachardoublequoteopen}{\isasymTurnstile}B{\isachardoublequoteclose}\isanewline
%
\isadelimproof
\ \ %
\endisadelimproof
%
\isatagproof
\isacommand{using}\isamarkupfalse%
\ assms{\isacharparenleft}{\isadigit{1}}{\isacharparenright}\ assms{\isacharparenleft}{\isadigit{2}}{\isacharparenright}\ \isacommand{by}\isamarkupfalse%
\ blast\isanewline
%
\isamarkupcmt{Because I have not defined a ``derivable" operator, inference rules are written using assumptions.%
}\isanewline
%
\isamarkupcmt{The rule {\color{blue} blast} is a classical reasoning method that comes with Isabelle out of the box. \cite{isabelle}%
}\isanewline
%
\isamarkupcmt{This is an example of a metalogical proof in this system using the validity operator.%
}%
\endisatagproof
{\isafoldproof}%
%
\isadelimproof
%
\endisadelimproof
%
\isadelimproof
%
\endisadelimproof
%
\isatagproof
%
\endisatagproof
{\isafoldproof}%
%
\isadelimproof
%
\endisadelimproof
%
\isadelimproof
%
\endisadelimproof
%
\isatagproof
%
\endisatagproof
{\isafoldproof}%
%
\isadelimproof
%
\endisadelimproof
%
\isadelimproof
%
\endisadelimproof
%
\isatagproof
%
\endisatagproof
{\isafoldproof}%
%
\isadelimproof
%
\endisadelimproof
%
\isadelimdocument
%
\endisadelimdocument
%
\isatagdocument
%
\endisatagdocument
{\isafolddocument}%
%
\isadelimdocument
%
\endisadelimdocument
%
\isadelimproof
%
\endisadelimproof
%
\isatagproof
%
\endisatagproof
{\isafoldproof}%
%
\isadelimproof
%
\endisadelimproof
%
\isadelimproof
%
\endisadelimproof
%
\isatagproof
%
\endisatagproof
{\isafoldproof}%
%
\isadelimproof
%
\endisadelimproof
%
\isadelimdocument
%
\endisadelimdocument
%
\isatagdocument
%
\endisatagdocument
{\isafolddocument}%
%
\isadelimdocument
%
\endisadelimdocument
%
\begin{isamarkuptext}%
Another relevant operator for our purposes is $\Box$, the modal necessity operator. In 
this system, $\Box$ behaves as an S5 \cite{cresswell} modal necessity operator.%
\end{isamarkuptext}\isamarkuptrue%
\isacommand{lemma}\isamarkupfalse%
\ K{\isacharcolon}\isanewline
\ \ \isakeyword{shows}\ {\isachardoublequoteopen}{\isasymTurnstile}\ {\isacharparenleft}{\isacharparenleft}{\isasymbox}{\isacharparenleft}A\ \isactrlbold {\isasymrightarrow}\ B{\isacharparenright}{\isacharparenright}\ \isactrlbold {\isasymrightarrow}\ {\isacharparenleft}{\isacharparenleft}{\isasymbox}A{\isacharparenright}\ \isactrlbold {\isasymrightarrow}\ {\isacharparenleft}{\isasymbox}B{\isacharparenright}{\isacharparenright}{\isacharparenright}{\isachardoublequoteclose}%
\isadelimproof
\ %
\endisadelimproof
%
\isatagproof
\isacommand{by}\isamarkupfalse%
\ blast%
\endisatagproof
{\isafoldproof}%
%
\isadelimproof
%
\endisadelimproof
\isanewline
\isanewline
\isacommand{lemma}\isamarkupfalse%
\ T{\isacharcolon}\isanewline
\ \ \isakeyword{shows}\ {\isachardoublequoteopen}{\isasymTurnstile}\ {\isacharparenleft}{\isacharparenleft}{\isasymbox}A{\isacharparenright}\ \isactrlbold {\isasymrightarrow}A{\isacharparenright}{\isachardoublequoteclose}%
\isadelimproof
\ %
\endisadelimproof
%
\isatagproof
\isacommand{by}\isamarkupfalse%
\ blast%
\endisatagproof
{\isafoldproof}%
%
\isadelimproof
%
\endisadelimproof
\isanewline
\isanewline
\isacommand{lemma}\isamarkupfalse%
\ {\isadigit{5}}{\isacharcolon}\isanewline
\ \ \isakeyword{shows}\ {\isachardoublequoteopen}{\isasymTurnstile}\ {\isacharparenleft}{\isacharparenleft}{\isasymdiamond}A{\isacharparenright}\ \isactrlbold {\isasymrightarrow}\ {\isacharparenleft}{\isasymbox}{\isacharparenleft}{\isasymdiamond}A{\isacharparenright}{\isacharparenright}{\isacharparenright}{\isachardoublequoteclose}%
\isadelimproof
\ %
\endisadelimproof
%
\isatagproof
\isacommand{by}\isamarkupfalse%
\ blast\isanewline
%
\endisatagproof
{\isafoldproof}%
%
\isadelimproof
%
\endisadelimproof
%
\isadelimdocument
%
\endisadelimdocument
%
\isatagdocument
%
\endisatagdocument
{\isafolddocument}%
%
\isadelimdocument
%
\endisadelimdocument
%
\begin{isamarkuptext}%
As mentioned earlier, the obligation operator is most interesting for my purposes. Here are some 
of its properties.%
\end{isamarkuptext}\isamarkuptrue%
\isacommand{lemma}\isamarkupfalse%
\ O{\isacharunderscore}diamond{\isacharcolon}\isanewline
\ \ \isakeyword{shows}\ {\isachardoublequoteopen}{\isasymTurnstile}\ {\isacharparenleft}O{\isacharbraceleft}A{\isacharbar}B{\isacharbraceright}\ \isactrlbold {\isasymrightarrow}\ {\isacharparenleft}{\isasymdiamond}{\isacharparenleft}B\ \isactrlbold {\isasymand}\ A{\isacharparenright}{\isacharparenright}{\isacharparenright}{\isachardoublequoteclose}\isanewline
%
\isadelimproof
\ \ %
\endisadelimproof
%
\isatagproof
\isacommand{using}\isamarkupfalse%
\ ax{\isacharunderscore}{\isadigit{5}}b\ ax{\isacharunderscore}{\isadigit{5}}a\isanewline
\ \ \isacommand{by}\isamarkupfalse%
\ metis\isanewline
%
\isamarkupcmt{A is only obligatory in a context if it can possibly be true in that context. This is meant to 
prevent impossible obligations.%
}\isanewline
%
\endisatagproof
{\isafoldproof}%
%
\isadelimproof
%
\endisadelimproof
%
\isadelimproof
%
\endisadelimproof
%
\isatagproof
%
\endisatagproof
{\isafoldproof}%
%
\isadelimproof
%
\endisadelimproof
%
\isadelimproof
%
\endisadelimproof
%
\isatagproof
%
\endisatagproof
{\isafoldproof}%
%
\isadelimproof
%
\endisadelimproof
%
\isadelimproof
%
\endisadelimproof
%
\isatagproof
%
\endisatagproof
{\isafoldproof}%
%
\isadelimproof
%
\endisadelimproof
%
\isadelimproof
%
\endisadelimproof
%
\isatagproof
%
\endisatagproof
{\isafoldproof}%
%
\isadelimproof
\isanewline
%
\endisadelimproof
\isacommand{lemma}\isamarkupfalse%
\ O{\isacharunderscore}nec{\isacharcolon}\isanewline
\ \ \isakeyword{shows}\ {\isachardoublequoteopen}{\isasymTurnstile}{\isacharparenleft}O{\isacharbraceleft}B{\isacharbar}A{\isacharbraceright}\ \isactrlbold {\isasymrightarrow}\ {\isacharparenleft}{\isasymbox}O{\isacharbraceleft}B{\isacharbar}A{\isacharbraceright}{\isacharparenright}{\isacharparenright}{\isachardoublequoteclose}\isanewline
%
\isadelimproof
\ \ %
\endisadelimproof
%
\isatagproof
\isacommand{by}\isamarkupfalse%
\ simp\isanewline
%
\isamarkupcmt{Obligations are necessarily obligated. This axiom is faithful to Kant's interpretation of ethics 
and is evidence of DDL's power in representing Kant's theory. Kant claimed that the categorical
imperative was not contingent on any facts about the world, but instead a property of the concept of 
morality itself \cite{groundwork}. Under this view, obligation should not be world-specific.%
}\isanewline
%
\endisatagproof
{\isafoldproof}%
%
\isadelimproof
%
\endisadelimproof
%
\isadelimproof
%
\endisadelimproof
%
\isatagproof
%
\endisatagproof
{\isafoldproof}%
%
\isadelimproof
%
\endisadelimproof
%
\begin{isamarkuptext}%
Below is an example of a more involved proof in Isabelle. This proof was almost completely automatically
generated. The property itself here is not very interesting for my purposes because I will rarely mix the dyadic
and monadic obligation operators.%
\end{isamarkuptext}\isamarkuptrue%
\isacommand{lemma}\isamarkupfalse%
\ O{\isacharunderscore}to{\isacharunderscore}O{\isacharcolon}\isanewline
\ \ \isakeyword{shows}\ {\isachardoublequoteopen}{\isasymTurnstile}{\isacharparenleft}O{\isacharbraceleft}B{\isacharbar}A{\isacharbraceright}\isactrlbold {\isasymrightarrow}O{\isacharbraceleft}{\isacharparenleft}A\isactrlbold {\isasymrightarrow}B{\isacharparenright}{\isacharbar}\isactrlbold {\isasymtop}{\isacharbraceright}{\isacharparenright}{\isachardoublequoteclose}\isanewline
%
\isadelimproof
%
\endisadelimproof
%
\isatagproof
\isacommand{proof}\isamarkupfalse%
{\isacharminus}\isanewline
\ \ \isacommand{have}\isamarkupfalse%
\ {\isachardoublequoteopen}{\isasymforall}X\ Y\ Z{\isachardot}\ {\isacharparenleft}ob\ X\ Y\ {\isasymand}\ {\isacharparenleft}{\isasymforall}w{\isachardot}\ X\ w\ {\isasymlongrightarrow}\ Z\ w{\isacharparenright}{\isacharparenright}\ {\isasymlongrightarrow}\ ob\ Z\ {\isacharparenleft}{\isasymlambda}w{\isachardot}{\isacharparenleft}Z\ w\ {\isasymand}\ {\isasymnot}X\ w{\isacharparenright}\ {\isasymor}\ Y\ w{\isacharparenright}{\isachardoublequoteclose}\isanewline
%
\isamarkupcmt{I had to manually specify this subgoal, but once I did Isabelle was able to prove it automatically.%
}\isanewline
\ \ \ \ \isacommand{by}\isamarkupfalse%
\ {\isacharparenleft}smt\ ax{\isacharunderscore}{\isadigit{5}}d\ ax{\isacharunderscore}{\isadigit{5}}b\ ax{\isacharunderscore}{\isadigit{5}}b{\isacharprime}{\isacharprime}{\isacharparenright}\isanewline
%
\isamarkupcmt{Isabelle's proof-finding tool, Sledgehammer  \cite{sledgehammer}, comes with out-of-the-box support for smt solving \cite{smt}.%
}\isanewline
\ \ \isacommand{thus}\isamarkupfalse%
\ {\isacharquery}thesis\isanewline
\ \ \isacommand{proof}\isamarkupfalse%
\ {\isacharminus}\isanewline
\ \ \ \ \isacommand{have}\isamarkupfalse%
\ f{\isadigit{1}}{\isacharcolon}\ {\isachardoublequoteopen}{\isasymforall}p\ pa\ pb{\isachardot}\ {\isacharparenleft}{\isacharparenleft}{\isasymnot}\ {\isacharparenleft}ob\ p\ pa{\isacharparenright}{\isacharparenright}\ {\isasymor}\ {\isacharparenleft}{\isasymexists}i{\isachardot}\ {\isacharparenleft}p\isactrlbold {\isasymand}{\isacharparenleft}\isactrlbold {\isasymnot}\ pb{\isacharparenright}{\isacharparenright}\ i{\isacharparenright}{\isacharparenright}\ {\isasymor}\ {\isacharparenleft}ob\ pb\ {\isacharparenleft}{\isacharparenleft}pb\isactrlbold {\isasymand}{\isacharparenleft}\isactrlbold {\isasymnot}\ p{\isacharparenright}{\isacharparenright}\isactrlbold {\isasymor}\ \ pa{\isacharparenright}{\isacharparenright}{\isachardoublequoteclose}\isanewline
\ \ \ \ \ \ \isacommand{using}\isamarkupfalse%
\ {\isacartoucheopen}{\isasymforall}X\ Y\ Z{\isachardot}\ ob\ X\ Y\ {\isasymand}\ {\isacharparenleft}{\isasymTurnstile}{\isacharparenleft}X\isactrlbold {\isasymrightarrow}Z{\isacharparenright}{\isacharparenright}\ {\isasymlongrightarrow}\ ob\ Z\ {\isacharparenleft}\ {\isacharparenleft}Z\isactrlbold {\isasymand}{\isacharparenleft}\isactrlbold {\isasymnot}\ X{\isacharparenright}{\isacharparenright}\isactrlbold {\isasymor}\ Y{\isacharparenright}{\isacartoucheclose}\ \isacommand{by}\isamarkupfalse%
\ force\isanewline
\ \ \ \ \isacommand{obtain}\isamarkupfalse%
\ ii\ {\isacharcolon}{\isacharcolon}\ {\isachardoublequoteopen}{\isacharparenleft}i\ {\isasymRightarrow}\ bool{\isacharparenright}\ {\isasymRightarrow}\ {\isacharparenleft}i\ {\isasymRightarrow}\ bool{\isacharparenright}\ {\isasymRightarrow}\ i{\isachardoublequoteclose}\ \isakeyword{where}\isanewline
\ \ \ \ \ \ {\isachardoublequoteopen}{\isasymforall}x{\isadigit{0}}\ x{\isadigit{2}}{\isachardot}\ {\isacharparenleft}{\isasymexists}v{\isadigit{3}}{\isachardot}\ {\isacharparenleft}x{\isadigit{2}}\isactrlbold {\isasymand}{\isacharparenleft}\isactrlbold {\isasymnot}\ x{\isadigit{0}}{\isacharparenright}{\isacharparenright}\ v{\isadigit{3}}{\isacharparenright}\ {\isacharequal}\ {\isacharparenleft}x{\isadigit{2}}\isactrlbold {\isasymand}{\isacharparenleft}\isactrlbold {\isasymnot}\ x{\isadigit{0}}{\isacharparenright}{\isacharparenright}\ {\isacharparenleft}ii\ x{\isadigit{0}}\ x{\isadigit{2}}{\isacharparenright}{\isachardoublequoteclose}\isanewline
\ \ \ \ \ \ \isacommand{by}\isamarkupfalse%
\ moura\isanewline
\ \ \ \ \isacommand{then}\isamarkupfalse%
\ \isacommand{have}\isamarkupfalse%
\ {\isachardoublequoteopen}{\isasymforall}p\ pa\ pb{\isachardot}\ {\isacharparenleft}{\isacharparenleft}{\isasymnot}\ ob\ p\ pa{\isacharparenright}\ {\isasymor}\ {\isacharparenleft}p\isactrlbold {\isasymand}{\isacharparenleft}\isactrlbold {\isasymnot}\ pb{\isacharparenright}{\isacharparenright}\ {\isacharparenleft}ii\ pb\ p{\isacharparenright}{\isacharparenright}\ {\isasymor}\ ob\ pb\ {\isacharparenleft}\ {\isacharparenleft}pb\isactrlbold {\isasymand}{\isacharparenleft}\isactrlbold {\isasymnot}\ p{\isacharparenright}{\isacharparenright}\isactrlbold {\isasymor}\ pa{\isacharparenright}{\isachardoublequoteclose}\isanewline
\ \ \ \ \ \ \isacommand{using}\isamarkupfalse%
\ f{\isadigit{1}}\ \isacommand{by}\isamarkupfalse%
\ presburger\isanewline
\ \ \ \ \isacommand{then}\isamarkupfalse%
\ \isacommand{show}\isamarkupfalse%
\ {\isacharquery}thesis\isanewline
\ \ \ \ \ \ \isacommand{by}\isamarkupfalse%
\ fastforce\isanewline
\ \ \isacommand{qed}\isamarkupfalse%
\isanewline
%
\isamarkupcmt{This entire Isar style proof was automatically generated using Sledgehammer.%
}\isanewline
\isacommand{qed}\isamarkupfalse%
%
\endisatagproof
{\isafoldproof}%
%
\isadelimproof
%
\endisadelimproof
%
\begin{isamarkuptext}%
The implementation of DDL showcases some of the useful features of Isabelle. Abbreviations allow
us to embed the syntax of DDL into HOL without defining an entire abstract sytax tree. Automated 
support for proof-finding using Sledgehammer makes proving lemmas trivial, and proving more complex theorems
far easier. Nitpick's model finding ability is useful to check for consistency and create countermodels.%
\end{isamarkuptext}\isamarkuptrue%
%
\isadelimdocument
%
\endisadelimdocument
%
\isatagdocument
%
\endisatagdocument
{\isafolddocument}%
%
\isadelimdocument
%
\endisadelimdocument
%
\isadelimproof
%
\endisadelimproof
%
\isatagproof
%
\endisatagproof
{\isafoldproof}%
%
\isadelimproof
%
\endisadelimproof
%
\isadelimproof
%
\endisadelimproof
%
\isatagproof
%
\endisatagproof
{\isafoldproof}%
%
\isadelimproof
%
\endisadelimproof
%
\isadelimdocument
%
\endisadelimdocument
%
\isatagdocument
%
\endisatagdocument
{\isafolddocument}%
%
\isadelimdocument
%
\endisadelimdocument
%
\isadelimproof
%
\endisadelimproof
%
\isatagproof
%
\endisatagproof
{\isafoldproof}%
%
\isadelimproof
%
\endisadelimproof
%
\isadelimproof
%
\endisadelimproof
%
\isatagproof
%
\endisatagproof
{\isafoldproof}%
%
\isadelimproof
%
\endisadelimproof
%
\isadelimdocument
%
\endisadelimdocument
%
\isatagdocument
%
\endisatagdocument
{\isafolddocument}%
%
\isadelimdocument
%
\endisadelimdocument
%
\isadelimproof
%
\endisadelimproof
%
\isatagproof
%
\endisatagproof
{\isafoldproof}%
%
\isadelimproof
%
\endisadelimproof
%
\isadelimproof
%
\endisadelimproof
%
\isatagproof
%
\endisatagproof
{\isafoldproof}%
%
\isadelimproof
%
\endisadelimproof
%
\isadelimproof
%
\endisadelimproof
%
\isatagproof
%
\endisatagproof
{\isafoldproof}%
%
\isadelimproof
%
\endisadelimproof
%
\isadelimproof
%
\endisadelimproof
%
\isatagproof
%
\endisatagproof
{\isafoldproof}%
%
\isadelimproof
%
\endisadelimproof
%
\isadelimproof
%
\endisadelimproof
%
\isatagproof
%
\endisatagproof
{\isafoldproof}%
%
\isadelimproof
%
\endisadelimproof
%
\isadelimproof
%
\endisadelimproof
%
\isatagproof
%
\endisatagproof
{\isafoldproof}%
%
\isadelimproof
%
\endisadelimproof
%
\isadelimproof
%
\endisadelimproof
%
\isatagproof
%
\endisatagproof
{\isafoldproof}%
%
\isadelimproof
%
\endisadelimproof
%
\isadelimproof
%
\endisadelimproof
%
\isatagproof
%
\endisatagproof
{\isafoldproof}%
%
\isadelimproof
%
\endisadelimproof
%
\isadelimtheory
%
\endisadelimtheory
%
\isatagtheory
%
\endisatagtheory
{\isafoldtheory}%
%
\isadelimtheory
%
\endisadelimtheory
%
\end{isabellebody}%
\endinput
%:%file=~/Desktop/cs91r/paper/paper224.thy%:%
%:%32=13%:%
%:%34=15%:%
%:%35=15%:%
%:%36=16%:%
%:%39=17%:%
%:%43=17%:%
%:%44=17%:%
%:%45=17%:%
%:%47=18%:%
%:%48=18%:%
%:%50=19%:%
%:%51=19%:%
%:%53=20%:%
%:%154=54%:%
%:%155=55%:%
%:%157=56%:%
%:%158=56%:%
%:%159=57%:%
%:%161=57%:%
%:%165=57%:%
%:%166=57%:%
%:%173=57%:%
%:%174=58%:%
%:%175=59%:%
%:%176=59%:%
%:%177=60%:%
%:%179=60%:%
%:%183=60%:%
%:%184=60%:%
%:%191=60%:%
%:%192=61%:%
%:%193=62%:%
%:%194=62%:%
%:%195=63%:%
%:%197=63%:%
%:%201=63%:%
%:%202=63%:%
%:%225=69%:%
%:%226=70%:%
%:%228=72%:%
%:%229=72%:%
%:%230=73%:%
%:%233=74%:%
%:%237=74%:%
%:%238=74%:%
%:%239=75%:%
%:%240=75%:%
%:%242=76%:%
%:%243=77%:%
%:%244=77%:%
%:%302=101%:%
%:%305=102%:%
%:%306=102%:%
%:%307=103%:%
%:%310=104%:%
%:%314=104%:%
%:%315=104%:%
%:%317=105%:%
%:%318=106%:%
%:%319=107%:%
%:%320=108%:%
%:%321=108%:%
%:%344=116%:%
%:%345=117%:%
%:%346=118%:%
%:%348=120%:%
%:%349=120%:%
%:%350=121%:%
%:%357=122%:%
%:%358=122%:%
%:%359=123%:%
%:%360=123%:%
%:%362=124%:%
%:%363=124%:%
%:%364=125%:%
%:%365=125%:%
%:%367=126%:%
%:%368=126%:%
%:%369=127%:%
%:%370=127%:%
%:%371=128%:%
%:%372=128%:%
%:%373=129%:%
%:%374=129%:%
%:%375=130%:%
%:%376=130%:%
%:%377=130%:%
%:%378=131%:%
%:%379=131%:%
%:%380=132%:%
%:%381=133%:%
%:%382=133%:%
%:%383=134%:%
%:%384=134%:%
%:%385=134%:%
%:%386=135%:%
%:%387=135%:%
%:%388=135%:%
%:%389=136%:%
%:%390=136%:%
%:%391=136%:%
%:%392=137%:%
%:%393=137%:%
%:%394=138%:%
%:%395=138%:%
%:%397=139%:%
%:%398=139%:%
%:%399=140%:%
%:%409=142%:%
%:%410=143%:%
%:%411=144%:%
%:%412=145%:%