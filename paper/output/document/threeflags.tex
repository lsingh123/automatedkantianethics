%
\begin{isabellebody}%
\setisabellecontext{threeflags}%
%
\isadelimtheory
%
\endisadelimtheory
%
\isatagtheory
%
\endisatagtheory
{\isafoldtheory}%
%
\isadelimtheory
%
\endisadelimtheory
%
\isadelimdocument
%
\endisadelimdocument
%
\isatagdocument
%
\isamarkupsection{Choice to Formalize the FUL%
}
\isamarkuptrue%
%
\endisatagdocument
{\isafolddocument}%
%
\isadelimdocument
%
\endisadelimdocument
%
\begin{isamarkuptext}%
In \emph{Groundwork of the Metaphysics of Morals}, Kant presents three formulations, or versions, 
of what he calls the ``supreme law of morality." I will focus on the first of these three formulations, 
and below I explain the formulations and defend my choice.

Kant argues that if  morality 
exists, it must take the form of a categorical imperative or a law that holds unconditionally. Categorical
imperatives are contrasted with hypothetical imperatives, which take the form of conditionals as in, 
``If I want to get good grades, I must study hard." Hypothetical imperatives only have force so long
as the antecedent holds, but the categorical imperative is unconditionally binding $\cite[28]{groundwork}$. In the first 
half of \emph{Groundwork}, Kant examines what the categorical imperative, if such a thing exists and has force,
must be. He concludes that there are three ``formulations" of the categorical imperative, or three ways 
of articulating the supreme law of morality. 

The first formulation of the categorical imperative is the
formula of universal law (FUL), which reads, ``act only according to that maxim through which you can 
at the same time will that it become a universal law." $\cite[34]{groundwork}$ This formulation
generates the universalizability test, which ``tests" the moral value of a maxim by 
imagining a world in which it becomes a universal law and attempting to will the maxim in that world. The 
second formulation of the categorical imperative is the formula of humanity (FUH): ``So act that you use humanity, 
in your own person, as well as in the person of any other, always at the same time as an end, never merely 
as a means." $\cite[41]{groundwork}$. This formulation is often understood as requiring us to 
acknowledge and respect the dignity of every other person. The third formulation of the categorical 
imperative is the formula of autonomy (FOA), which Korsgaard summarizes in her introduction to the Groundwork 
as, ``we should so act that we may think of ourselves as legislating universal laws through our 
maxims." $\cite[28]{korsgaardintro}$ While closely related to the FUL, the FOA presents morality as the activity of 
perfectly rational agents in an ideal ``kingdom of ends," guided by what Kant calls the ``laws of freedom."

I choose to focus on formalizations of Kant's first formulation of the categorical imperative,
the formula of universal law (FUL), because it is the most formal and thus the easiest to formalize and implement. 
Onora O'Neill explains that the formalism of the FUL allows 
for greater precision in philosophical arguments analyzing its implications and power $\cite[33]{actingonprinciple}$. This precision 
is particularly useful in a computational context because any formalism necessarily makes its content 
precise. The FUL's existing precision reduces ambiguity, allowing me to remain faithful to Kant's writing and 
philosophical interpretations of it. Precision reduces the need to make choices to resolve debates 
and ambiguities. Some of these choices may be well-studied and grounded in literature, 
but some may be unique to formalizing the FUL and thus understudied. Minimizing these choices minimizes 
arbitrariness in my formalization and puts it on solid philosophical footing. Given that this thesis is a proof-of-concept, 
the formalism of the FUL is attractive because it reduces both the computational and philosophical complexity of my work. 

While some criticize the FUL for its formalism and percieved ``sterility" $\cite[33]{actingonprinciple}$, 
Kantian constructivists embrace it $\cite[173]{ebelsduggan}$. My project is not committed to Kantian constructivism. 
I believe that computational ethics is likely a valuable tool for any ethicist, and I make the case 
for Kantian ethics specifically. Nonetheless, Kantian constructivists may find the focus on 
the FUL particularly appealing. 

Though Kantians study all formulations of the categorical imperative, Kant argues in Groundwork 
that the three formulations of the categorical imperative are equivalent \cite{groundwork}. While this 
argument is disputed \cite{sepkant}, for those who believe it, the
stakes for my choice of the FUL are greatly reduced. If all formulations are equivalent, then a formalization of the FUL
lends the exact same power as a formalization of the second or third formulation of the categorical 
imperative. In fact, future work could formalize the other formulas and try to prove that they 
are identical. Kant believes that his argument for the equality of the formulas is analytical, and
if he is correct, it should be possible to recreate the argument in logic.%
\end{isamarkuptext}\isamarkuptrue%
%
\isadelimdocument
%
\endisadelimdocument
%
\isatagdocument
%
\isamarkupsection{Definition of a Maxim%
}
\isamarkuptrue%
%
\endisatagdocument
{\isafolddocument}%
%
\isadelimdocument
%
\endisadelimdocument
%
\begin{isamarkuptext}%
The central unit of evaluation for the universalizability test is a ``maxim," which Kant defines 
in a footnote in \emph{Groundworkd} as ``the subjective principle of willing," or the principle that 
the agent acts on $\cite[16]{groundwork}$. Modern Kantians differ in their interpretations of this definition. The naive view 
is that a maxim is an act, but Korsgaard adopts the more sophisticated view that a maxim is composed
of an act and the agent's purpose for acting \cite{actingforareason}. She also compares a maxim 
to Aristotle's logos, which includes these components and information about the circumstances and methods 
of the act. O'Neill concludes that Kant's examples imply that a maxim must also include circumstances \cite{actingonprinciple}, and 
Kitcher \cite{whatisamaxim} uses textual evidence from the Groundwork to argue for the inclusion of a maxim's purpose 
or motivation. In order to formalize the notion of a maxim, I must adopt a specific definition and 
defend my choice.

I define a maxim as a circumstance, act, goal tuple (C, A, G), read 
as ``In circumstances C, act A for goal G." Isabelle's strict typing rules mean that the choice of the 
type of each member of this tuple is significant. A circumstance is represented as a set of worlds 
$t$ where that circumstance holds. A goal is also a term because it can be true or false at a world if it 
is realized or not. An act is an open sentence because an act itself is not the kind of thing that can 
be true or false (as in, an act is not truth-apt), but the combination of a subject performing an act 
can be true or false at a world depending on whether or not the act is indeed performed by that subject. 
For example, ``running" is not truth-apt, but ``Sara runs" is truth-apt.

My definition of a maxim is inspired by O'Neill's work on maxims. I will defend my representation
below and consider an additional component that Kitcher argues for.%
\end{isamarkuptext}\isamarkuptrue%
%
\isadelimdocument
%
\endisadelimdocument
%
\isatagdocument
%
\isamarkupsubsection{O'Niell's Original Schematic and The Role of Practical Judgement%
}
\isamarkuptrue%
%
\endisatagdocument
{\isafolddocument}%
%
\isadelimdocument
%
\endisadelimdocument
%
\begin{isamarkuptext}%
O'Neill $\cite[37]{actingonprinciple}$ presents what Kitcher \cite{whatisamaxim}  calls the widely accepted 
view that a maxim is a circumstance, act, goal tuple. A maxim 
is an action-guiding rule and thus naturally includes an act and the circumstances under which 
it should be performed, which are often referred to as ``morally relevant circumstances." 

She also includes a purpose, end, or goal in the maxim because Kant includes this in many of his 
example maxims and because Kant argues that human activity, because it is guided by a rational will, 
is inherently purposive $\cite[4:428]{groundwork}$. A rational will does not act randomly (else it would not be rational), 
but instead in the pursuit of ends which it deems valuable. This inclusion is also essential for the version of the universalizability test 
that I will implement, explained in Section ??.

O'Neill's inclusion of circumstances is potentially controversial because it leaves open the question of what qualifies as a 
relevant circumstance for a particular maxim. This is gives rise to ``the tailoring objection" $\cite[217]{whatisamaxim} \footnote{Kitcher
cites \cite{kantsethicalthought}  as offering an example of a false positive due to this objection.}$, 
under which maxims are arbitrarily specified to pass the FUL. For example, the maxim ``When my name is Lavanya Singh,
I will lie to get some easy money," is universalizable, but is clearly a false positive. One solution to 
this problem is to argue that the circumstance ``When my name is Lavanya Singh" is not morally relevant 
to the act and goal. This solution requires some discussion of what qualifies as a relevant circumstance.

O'Neill seems to acknowledge the difficulty of determining relevant circumstances when she concedes that a maxim cannot include all 
of the infinitely many circumstances in which the agent may perform the action$\cite[4:428]{actingonprinciple}$. She argues that this is 
an artifact of the fact that maxims are rules of practical reason, the kind of reason that helps us decide what to do 
and how to do it \cite{bok}. Like any practical rule, 
maxims require the exercise of practical judgement to determine in which circumstances they should be applied. 
This judgement, applied in both choosing when to exercise the maxim and in the formulation of the maxim 
itself, is what determines the ``morally relevant circumstances."

The upshot for computational ethics is that the computer cannot perform all ethical activity alone. 
Human judgement and the exercise of practical reason are essential to both formulate maxims and 
determine when the actual conditions of life coincide with the circumstances in which the maxim is relevant. 
Choosing when to exercise a maxim is less relevant to my project because analyzing a formal representation of the FUL requires 
making the circumstances in a given scenario precise, but will be important for applications of 
computational ethics to guiding AI agents. The difficulty in formulating a maxim, on the other hand, demonstrates 
the important fact that ethics, as presented here, is not a solely computational activity. A
human being must create a representation for the dilemma they wish to test, effectively translating 
a complex, real situation into a flat logical structure. This parallels the challenge that programmers 
face when translating the complexity of reality to a programming langauge or computational representation. Not only will some of the situation's complexity
inevitably be lost, the outcome of the universalizability test will depend on how the human formulates the maxim
and whether or not this formulation does indeed include morally relevant circumstances. If the human puts 
garbage into the test, the test will return garbage out.

While this may appear to be a weakness of my system, I believe that it actually
allows my system to retain some of the human complexity that many philosophers agree cannot be automated away.\footnote{Powers presents 
the determination of morally relevant circumstances as an obstacle to the automation of Kantian ethics \cite{powers}.}
Ethics is a fundamentally human activity. Kant argues that the categorical imperative is a statement 
about the properties of rational wills. In fact, Korsgaard argues that morality derives its authority over us, 
or normativity, only because is it a property of a rational will, and we, as human beings, are rational wills.
If ethics is meant to guide human behavior, the role of the computer becomes clear as not a replacement for our will,
but instead as a tool to help guide our wills and reason more efficiently 
and more effectively. Just as calculators don't render mathematicians obsolete, computational ethics
does not render human judgement or philosophy obsolete. Chapter 4 Section ?? will be devoted to a more complete discussion 
of this issue.%
\end{isamarkuptext}\isamarkuptrue%
%
\isadelimdocument
%
\endisadelimdocument
%
\isatagdocument
%
\isamarkupsubsection{Exclusion of Motive%
}
\isamarkuptrue%
%
\endisatagdocument
{\isafolddocument}%
%
\isadelimdocument
%
\endisadelimdocument
%
\begin{isamarkuptext}%
Kitcher begins with O'Niell's circumstance, act, goal view and expands it to include the motive 
behind performing the maxim \cite{whatisamaxim}. This additional component is read 
as ``In circumstance C, I will do A in order to G because of M," where M may be ``duty" or ``self-love."
Kitcher argues that the inclusion of motive is necessary for the fullest, most general form of a maxim
in order to capture Kant's idea that an action derives its moral worth from being done for the sake of duty itself.
Under this view, the FUL would obligate maxims of the form 
``In circumstance C, I will do A in order to G because I can will that I and everyone else simultaneously
will do A in order to G in circumstance C." In other words, if Kant is correct in arguing that moral 
actions must be done from the motive of duty, the affirmative result of the FUL becomes 
the motive for a moral action.

While Kitcher's conception of a maxim captures Kant's idea of acting for duty's own sake, I will not implement it 
because it is not necessary for putting maxims through the FUL. Indeed, Kitcher acknowledges that 
O'Neill's formulation suffices for the universalizability test, but is not the general notion of a maxim.
In order to pass the maxim through the FUL, it suffices to know the circumstance, act, and goal. The FUL
derives the motive that Kitcher bundles into the maxim, so automating the FUL does not require 
including a motive. The ``input" to the FUL is the circumstance, act, goal tuple. My project takes 
this input and returns the motivation that the dutiful, moral agent would adopt. Additionally, doing
justice to the rich notion of motive requires modelling the operation of practical reason itself, 
which is outside the scope of this project. My work focuses on the universalizability test, but future work that 
models the process of practical reason may use my implementation of the FUL as a ``library." Combined 
with a logic of practical reason, an implementation of the FUL can move from evaluating a maxim to 
evaluating an agent's behavior, since that's when ``acting from duty" starts to matter.%
\end{isamarkuptext}\isamarkuptrue%
%
\isadelimdocument
%
\endisadelimdocument
%
\isatagdocument
%
\isamarkupsection{Practical Contradiction Interpretation%
}
\isamarkuptrue%
%
\endisatagdocument
{\isafolddocument}%
%
\isadelimdocument
%
\endisadelimdocument
%
\begin{isamarkuptext}%
Kantians debate the correct interpretation of the formula of universal law because Kant appears to 
interpret the universalizability test in different ways. My project uses Korsgaard's practical contradiction 
interpretation, broadly accepted as correct within the philosophical community $\cite[177]{ebelsduggan}$.
Below, I briefly reconstruct Korsgaard's argument for the practical contradiction interpretation. While 
she believes that the text partially supports this interpretation, her argument is philosophical and 
derives its strength from the plausibility of the practical contradiction interpretation.

Recall that the formula of universal law is “act only in accordance with that maxim through which you can at the 
same time will that it become a universal law” $\cite[4:421]{groundwork}$. To determine if a maxim can be willed as a 
universal law, one must use the “universalizability test,” which requires imagining a world in which 
everyone for all of time has willed the maxim. If willing the maxim in such a world generates a contradiction, 
then the action is prohibited. There are three interpretations of what sort of contradiction is necessary: 
(1) the teleological view, prohibiting actions that conflict with some assumed teleological end when 
universalized, (2) the logical contradiction view, prohibiting maxims that are logically impossible 
when universalized, and (3) the practical contradiction view, prohibiting maxims that are self-defeating 
when universalized.

Under the logical contradiction interpretation, falsely promising to repay a loan to get some quick cash
fails the universalizability test because, in such a world, the practice of promising would die out so 
making a false promise would be impossible. Korsgaard appeals to Dietrichson \cite{dietrichson} to construct the example of 
a mother killing her children that tend to cry more than average so that she can get some 
sleep at night. Universalizing this maxim does not generate a logical contradiction, but it is clearly 
morally wrong. The problem here is that killing is a natural action, which Korsgaard distinguishes from 
a practice, like promising. Natural actions will never be logically impossible, so the logical contradiction 
view fails to prohibit them.

Under the teleological contradiction interpretation, a maxim is prohibited if it undercuts some natural 
or assigned purpose for some practice, act, or object. For example, the purpose of promising is to 
create a system of mutual trust and false promising undercuts this purpose and is thus prohibited. The problem 
with this view is that it assumes that the agent is committed, either because of their own goals or 
because of some property of a rational will, to some teleological system. Acton formulates Hegel's argument that \cite{acton},
an agent doesn't have to be committed to promising as a system of mutual trust. Korsgaard concludes that 
assigning teleological purposes to actions is difficult because ``such purposes may have
nothing to do with what the agent wants or ought rationally to want, or even with what
any human being wants." If the agent is not committed to the purpose, then will not see a contradiction 
in willing an act that violates this purpose.

This difficulty with the teleological contradiction interpretation drives Korsgaard to look for purposes
that an agent must necessarily be committed to, and she concludes that this must be the purpose of the 
maxim itself. By willing a maxim, an agent commits themselves to the goal of the maxim, and thus cannot 
rationally will a system in which this goal is undercut. This system satisfactorily handles natural actions
like those of the sleep-deprived mother: in willing the end of sleeping through the night, she is 
implicitly willing that she be alive in order to secure and enjoy her sleep. If any mother is allowed to kill
any loud child, then she cannot be secure in the possession of her life, because her own mother may have 
grown frustrated with her crying. Her willing this maxim thwarts the end that she sought to secure. 

The practical contradiction interpretation not only addresses the problems with the first two 
interpretations, it also offers a much more satisfying explanation of why certain maxims are immoral. 
The problem is not the existence of a contradiction itself, but instead the fact that these maxims 
involve parasitic behavior on social conditions that the agent seeks to benefit from. The false promiser 
simultaneously wants to abuse the system of promising and benefit from it, and is thus making an exception 
of themselves. It is this kind of free-riding that the universalizability test seeks to draw out. The test
raises the same kinds of objections that the question ``What if everyone did that?" seeks to draw out.%
\end{isamarkuptext}\isamarkuptrue%
%
\isadelimtheory
%
\endisadelimtheory
%
\isatagtheory
%
\endisatagtheory
{\isafoldtheory}%
%
\isadelimtheory
%
\endisadelimtheory
%
\end{isabellebody}%
\endinput
%:%file=~/Desktop/cs91r/paper/threeflags.thy%:%
%:%24=9%:%
%:%36=12%:%
%:%37=13%:%
%:%38=14%:%
%:%39=15%:%
%:%40=16%:%
%:%41=17%:%
%:%42=18%:%
%:%43=19%:%
%:%44=20%:%
%:%45=21%:%
%:%46=22%:%
%:%47=23%:%
%:%48=24%:%
%:%49=25%:%
%:%50=26%:%
%:%51=27%:%
%:%52=28%:%
%:%53=29%:%
%:%54=30%:%
%:%55=31%:%
%:%56=32%:%
%:%57=33%:%
%:%58=34%:%
%:%59=35%:%
%:%60=36%:%
%:%61=37%:%
%:%62=38%:%
%:%63=39%:%
%:%64=40%:%
%:%65=41%:%
%:%66=42%:%
%:%67=43%:%
%:%68=44%:%
%:%69=45%:%
%:%70=46%:%
%:%71=47%:%
%:%72=48%:%
%:%73=49%:%
%:%74=50%:%
%:%75=51%:%
%:%76=52%:%
%:%77=53%:%
%:%78=54%:%
%:%79=55%:%
%:%80=56%:%
%:%81=57%:%
%:%82=58%:%
%:%83=59%:%
%:%84=60%:%
%:%85=61%:%
%:%86=62%:%
%:%87=63%:%
%:%88=64%:%
%:%97=66%:%
%:%109=68%:%
%:%110=69%:%
%:%111=70%:%
%:%112=71%:%
%:%113=72%:%
%:%114=73%:%
%:%115=74%:%
%:%116=75%:%
%:%117=76%:%
%:%118=77%:%
%:%119=78%:%
%:%120=79%:%
%:%121=80%:%
%:%122=81%:%
%:%123=82%:%
%:%124=83%:%
%:%125=84%:%
%:%126=85%:%
%:%127=86%:%
%:%128=87%:%
%:%129=88%:%
%:%130=89%:%
%:%139=91%:%
%:%151=93%:%
%:%152=94%:%
%:%153=95%:%
%:%154=96%:%
%:%155=97%:%
%:%156=98%:%
%:%157=99%:%
%:%158=100%:%
%:%159=101%:%
%:%160=102%:%
%:%161=103%:%
%:%162=104%:%
%:%163=105%:%
%:%164=106%:%
%:%165=107%:%
%:%166=108%:%
%:%167=109%:%
%:%168=110%:%
%:%169=111%:%
%:%170=112%:%
%:%171=113%:%
%:%172=114%:%
%:%173=115%:%
%:%174=116%:%
%:%175=117%:%
%:%176=118%:%
%:%177=119%:%
%:%178=120%:%
%:%179=121%:%
%:%180=122%:%
%:%181=123%:%
%:%182=124%:%
%:%183=125%:%
%:%184=126%:%
%:%185=127%:%
%:%186=128%:%
%:%187=129%:%
%:%188=130%:%
%:%189=131%:%
%:%190=132%:%
%:%191=133%:%
%:%192=134%:%
%:%193=135%:%
%:%194=136%:%
%:%195=137%:%
%:%196=138%:%
%:%197=139%:%
%:%198=140%:%
%:%199=141%:%
%:%200=142%:%
%:%201=143%:%
%:%202=144%:%
%:%211=146%:%
%:%223=148%:%
%:%224=149%:%
%:%225=150%:%
%:%226=151%:%
%:%227=152%:%
%:%228=153%:%
%:%229=154%:%
%:%230=155%:%
%:%231=156%:%
%:%232=157%:%
%:%233=158%:%
%:%234=159%:%
%:%235=160%:%
%:%236=161%:%
%:%237=162%:%
%:%238=163%:%
%:%239=164%:%
%:%240=165%:%
%:%241=166%:%
%:%242=167%:%
%:%243=168%:%
%:%244=169%:%
%:%245=170%:%
%:%254=172%:%
%:%266=174%:%
%:%267=175%:%
%:%268=176%:%
%:%269=177%:%
%:%270=178%:%
%:%271=179%:%
%:%272=180%:%
%:%273=181%:%
%:%274=182%:%
%:%275=183%:%
%:%276=184%:%
%:%277=185%:%
%:%278=186%:%
%:%279=187%:%
%:%280=188%:%
%:%281=189%:%
%:%282=190%:%
%:%283=191%:%
%:%284=192%:%
%:%285=193%:%
%:%286=194%:%
%:%287=195%:%
%:%288=196%:%
%:%289=197%:%
%:%290=198%:%
%:%291=199%:%
%:%292=200%:%
%:%293=201%:%
%:%294=202%:%
%:%295=203%:%
%:%296=204%:%
%:%297=205%:%
%:%298=206%:%
%:%299=207%:%
%:%300=208%:%
%:%301=209%:%
%:%302=210%:%
%:%303=211%:%
%:%304=212%:%
%:%305=213%:%
%:%306=214%:%
%:%307=215%:%
%:%308=216%:%
%:%309=217%:%
%:%310=218%:%
%:%311=219%:%
%:%312=220%:%
%:%313=221%:%
%:%314=222%:%
%:%315=223%:%
%:%316=224%:%
%:%317=225%:%
%:%318=226%:%