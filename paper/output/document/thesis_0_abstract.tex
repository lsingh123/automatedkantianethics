%
\begin{isabellebody}%
\setisabellecontext{thesis{\isacharunderscore}{\isadigit{0}}{\isacharunderscore}abstract}%
%
\isadelimtheory
%
\endisadelimtheory
%
\isatagtheory
%
\endisatagtheory
{\isafoldtheory}%
%
\isadelimtheory
%
\endisadelimtheory
%
\begin{isamarkuptext}%
AI agents are beginning to make decisions without human supervision in increasingly consequential 
contexts like healthcare, policing, and driving. These decisions are inevitably ethically tinged, 
but most AI agents navigating the world today are not explicitly guided by ethics.
Warnings from regulators, philosophers, and computer scientists about the 
dangers of unethical artificial intelligence, from science-fiction killer robots to criminal
sentencing algorithms prejudiced against people of color, have spurred interest in automated ethics, or the development 
of machines that can perform ethical reasoning. Much prior work in automated ethics approaches the 
problem from a computational perspective and rarely engages with philosophical literature on ethics, despite
its relevance to the development of AI agents that can
responsibly navigate the world. If automated ethics draws on sophisticated philosophical literature, the ethical reasoning
underlying such decisions will be more nuanced, precise, consistent, and trustworthy. However, faithfully translating complex ethical theories
from natural language to the rigid syntax of a computer program poses technical and philosophical 
challenges. 

In this thesis, I present an implementation of automated Kantian
ethics that is faithful to the Kantian philosophical tradition. Of the three major ethical
traditions, Kant's categorical imperative is the most natural to formalize because it is an inviolable 
formal rule that requires less situational awareness than other ethical theories. I formalize Kant's categorical imperative 
in Carmo and Jones's Dyadic Deontic Logic, implement this formalization 
in the Isabelle/HOL theorem prover, and develop a testing framework to evaluate how well 
my implementation coheres with expected properties of Kantian ethics, as established in the literature. 
I also use my system to reason about two ethical dilemmas used to criticize Kantian ethics: the difference
between lying and joking and the example of a murderer knocking on your door asking about the location of their
intended victim. Finally, I examine the philosophical implications of this system, exploring its limitations 
and its potential to help both AI agents and human beings better reason about ethics. This work serves 
as an early proof-of-concept for philosophically mature AI agents and is one step towards the development 
of responsible, trustworthy artificial intelligence.%
\end{isamarkuptext}\isamarkuptrue%
%
\isadelimtheory
%
\endisadelimtheory
%
\isatagtheory
%
\endisatagtheory
{\isafoldtheory}%
%
\isadelimtheory
%
\endisadelimtheory
%
\end{isabellebody}%
\endinput
%:%file=~/Desktop/cs91r/paper/thesis_0_abstract.thy%:%
%:%19=6%:%
%:%20=7%:%
%:%21=8%:%
%:%22=9%:%
%:%23=10%:%
%:%24=11%:%
%:%25=12%:%
%:%26=13%:%
%:%27=14%:%
%:%28=15%:%
%:%29=16%:%
%:%30=17%:%
%:%31=18%:%
%:%32=19%:%
%:%33=20%:%
%:%34=21%:%
%:%35=22%:%
%:%36=23%:%
%:%37=24%:%
%:%38=25%:%
%:%39=26%:%
%:%40=27%:%
%:%41=28%:%
%:%42=29%:%
%:%43=30%:%
%:%44=31%:%
%:%45=32%:%