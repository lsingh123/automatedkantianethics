%
\begin{isabellebody}%
\setisabellecontext{thesis{\isacharunderscore}{\isadigit{1}}{\isacharunderscore}intro}%
%
\isadelimtheory
%
\endisadelimtheory
%
\isatagtheory
%
\endisatagtheory
{\isafoldtheory}%
%
\isadelimtheory
%
\endisadelimtheory
%
\isadelimdocument
%
\endisadelimdocument
%
\isatagdocument
%
\isamarkupsection{Introduction%
}
\isamarkuptrue%
%
\endisatagdocument
{\isafolddocument}%
%
\isadelimdocument
%
\endisadelimdocument
%
\begin{isamarkuptext}%
The development of autonomous artificial agents has spurred interest in computers that can perform 
ethical reasoning, known as automated moral agents. AI agents are making decisions in increasingly 
consequential contexts, such as medical diagnoses, driving, and criminal sentencing, and therefore 
must perform ethical reasoning in order to navigate moral dilemmas responsibly. For example, self-driving
cars may face less extreme versions of the following moral dilemma: an autonomous vehicle approaching 
an intersection fails to notice pedestrians in the crosswalk until it is too late to brake. The car 
can either continue on its course, running over and killing three pedestrians, or it can swerve to 
hit the car in the next lane, killing the single passenger inside it. While this example is (hopefully) 
not typical of the operation of a self-driving car, every decision that such an AI agent makes, from 
avoiding congested freeways to carpooling, is morally tinged. Because ethics is the study of navigating 
the world responsibly, AI agents routinely make moral decisions without explicitly performing ethical
reasoning. Moreover, artifical agents are making many such moral decisions without
human supervision, such as hiring algorithms that filter job applicants' resumes, a decision that can
impact people's livlihoods and often involves implicit prejudices. Machine ethics, also called 
automated ethics, is the study of how to develop machines that can perform robust, sophisticated ethical reasoning. 

Machine ethicists recognize the need for automated ethics and have made both theoretical 
(\citep{moralmachineonline} \citep{davenport} \citep{moralmachine} \citep{gabriel}) and practical progress 
(\citep{logicprogramming} \citep{biology} \citep{delphi} \citep{winfield}) towards automating ethics. 
However, prior work in machine ethics using popular ethical theories like deontology (\citep{deon2} \citep{deon1}), 
consequentialism (\citep{util1} \citep{util2} \citep{cloos}), and virtue ethics (\citep{berberich}) rarely 
engages with philosophical literature and thus misses philosophers' insights. The above example of 
the malfunctioning self-driving car is an instance of Phillipa Foot's trolley problem \citep{foot}, 
in which a bystander watching a runaway trolley can pull a lever to kill one instead of three. 
Decades of philosophical debate have developed ethical theories that can offer nuanced and 
consistent answers to the trolley problem. The trolley problem demonstrates that the moral dilemmas 
that AI faces are not entirely new, so solutions to these problems should take advantage of philosophical 
progress. The more faithful that automated ethics is to philosophical literature, the more reliable and 
nuanced it will be.

A lack of engagement with prior philosophical literature also makes automated moral agents less 
explainable, or interpretable by human observers, as seen in the example of Delphi, which uses deep 
learning to make moral judgements based on a training dataset of ethical decisions made by humans \citep{delphi}. 
Early versions of Delphi often gave unexpected results, such as declaring that the user should commit 
genocide if it makes everyone happy \citep{verge}. Moreover, because no explicit ethical theory underpins 
Delphi's judgements, human beings cannot analytically determine why Delphi thinks genocide is obligatory. 
Machine learning approaches like Delphi often cannot explain their decisions to a human being and, 
in the extreme case, are black box algorithms. This reduces human trust in a machine's controversial 
ethical judgements. If a machine prescribes killing one person to save three without justifying this 
decision, acting on this judgement becomes difficult. The high stakes of automated ethics require 
explainability to build trust and catch mistakes. 

While automated ethics should draw on philosophical literature, in practice, automating an ethical 
theory is a technical and philosophical challenge. Intuitive computational approaches explored 
previously, such as representing ethics as a constraint satisfaction problem \citep{csp} or reinforcement 
learning algorithm \citep{util1}, fail to capture philosophically plausible ethical theories. For 
example, encoding ethics as a Markov Decision Process assumes that ethical reward can be aggregated 
according to some discounted sum, but many philosophers reject this notion of aggregation \citep{consequentialismsep}. 
Ethical theories are almost always described in natural language, so automated ethics must first 
make ethics precise enough to represent to a computer. Even once ethics is translated from natural 
language to program syntax, the factual background given to the machine, such as the description of 
an ethical dilemma, is equally as important in determining the machine's decisions. Another complication
is that philosophers do not agree on a single ``correct" ethical theory. Even philosophers who
agree that a specific ethical theory, like Kantian ethics, is true, debate the theory's 
details.\footnote{For examples of these debates in the case of Kantian ethics, see Section Joking
and Section Murderer.} Even once reasoning within a 
particular ethical theory is automated, those who disagree with that theory will disagree with the 
system's judgements.

This thesis presents a proof-of-concept implementation of philosophically faithful automated ethics
according to Kantian ethics. I formalize Kant's categorical imperative, or moral rule, as an axiom 
in Carmo and Jones' Dyadic Deontic Logic (DDL), a modal logic designed to reason about 
obligation \citep{CJDDL}. I implement my formalization in Isabelle/HOL, an interactive theorem prover 
that can automatically verify and generate proofs in user-defined logics \citep{isabelle}. Finally, 
I use Isabelle to automatically prove theorems (such as, ``murder is wrong'') in my new logic, 
generating results derived from the categorical imperative. Because my system automates reasoning in 
a logic that represents Kantian ethics, it automates Kantian ethical reasoning. Once equipped with 
minimal factual background, it can classify actions as prohibited, permissible or obligatory. I 
make the following contributions:%
\end{isamarkuptext}\isamarkuptrue%
%
\begin{isamarkuptext}%
\begin{enumerate}
\item In Section ??, I make a philosophical argument for why Kantian ethics is the most natural of the three major
ethical traditions (deontology, virtue ethics, utilitarianism) to formalize.

\item In Section ??, I present a formalization of the practical contradiction interpretation of the 
Formula of Universal Law in Dyadic Deontic Logic. I implement this formalization in the Isabelle/HOL
theorem prover. My implementation includes axioms and definitions such that my system, when given an appropriately
represented input, can prove that the input is permissible, obligatory, or prohibited. It can also return
a list of facts used in the proof and, in some cases, an Isar-style human readable proof. 

\item In Sections ?? and ?, I demonstrate my system's power and flexibility by 
using it to produce nuanced answers to two well-known Kantian ethical dilemmas. I show that, because 
my system draws on definitions of Kantian ethics presented in philosophical literature, it is able 
to perform sophisticated moral reasoning. 

\item In Section ??, I present a testing framework that can generally evaluate how faithful an implementation 
of automated Kantian ethics is. My framework includes meta-ethical tests and application tests inspired by philosophical
literature. My testing framework shows that my formalization substantially improves on prior work and can 
be generalized to any implementation of automated Kantian ethics.

\item In Section ??, I present new ethical insights discovered using my system and argue that
computational methods like the one presented in this paper can help philosophers address ethical problems.
Not only can my system help machines reason about ethics, it can also help philosophers make philosophical
progress.
\end{enumerate}%
\end{isamarkuptext}\isamarkuptrue%
%
\begin{isamarkuptext}%
I present a faithful implementation of automated Kantian ethics, a testing framework to evaluate how 
well my implementation coheres with expected properties of Kantian ethics (derived from philosophical 
literature), and examples in which my system performs sophisticated moral reasoning. My system consists 
of a logic in which I formalize an ethical theory and an implementation of this formalization in an 
interactive theorem prover. 

I choose to formalize Kant's moral rule in Carmo and Jones' Dyadic Deontic Logic (DDL) \citep{CJDDL}. Deontic 
logic is a modal logic that can express obligation, or morally binding requirements. Traditional modal 
logics include the necessitation operator, denoted as $\Box$. In modal logic using the Kripke semantics, 
$\Box p$ is true at world $w$ if $p$ is true at all worlds that neighbor $w$ \citep{cresswell}. Modal 
logics  also contain the possibility operator $\diamond$, where $\diamond \, p \iff \neg (\Box (\neg p))$ 
and operators of propositional logic like $\neg, \wedge, \vee, \rightarrow$. I use DDL, in which
the dyadic obligation operator $O\{A \vert B\}$ to represent the sentence ``A is obligated in the context B.'' 
The introduction of context allows DDL to reason about violations of duty. DDL is both deontic and modal, 
so sentences like $O\{A \vert B\}$ are terms that can be true or false at a world. For example, the 
sentence $O \{ \text{steal} \vert \text{when rich}\}$ is true at a world if stealing when rich is 
obligated at that particular world. 

I automate Kantian ethics because it is the most natural to formalize, as I argue in Section WhyKant. 
Kant presents three versions of a single moral rule, known as the categorical imperative, from which 
all moral judgements can be derived. I implement a version of this rule called the Formula of Universal 
Law (FUL), which states that people should only act on those principles that can be acted on by all 
people without contradiction. For example, in a world where everyone falsely promises to repay a loan, 
lenders will no longer believe these promises and will stop offering loans. Therefore, not everyone 
can simultaneously falsely promise to repay a loan, so the FUL thus prohibits this act.

Prior work by Benzmüller, Farjami, and Parent \citep{logikey, BFP} implements DDL in Isabelle/HOL and 
I add the Formula of Universal Law as an axiom on top of their library. The resulting Isabelle theory 
can automatically or semi-automatically generate proofs in a new logic that has the categorical 
imperative as an axiom. Because proofs in this logic are derived from the categorical imperative, 
they judge actions as obligated, prohibited, or permissible. Moreover, because interactive 
theorem provers are designed to be interpretable, my system is explainable. Isabelle can list 
the axioms and facts it used to generate an ethical judgement, and, in some cases, construct 
human-readable proofs. In Sections Joking and Murderer, I use my system to arrive at 
sophisticated solutions to two ethical dilemmas often used in critiques of Kantian ethics. Because 
my system is faithful to philosophical literature, it is able to provide nuanced answers to these paradoxes. 

In addition to presenting the above logic and implementation, I also contribute a testing framework 
that evaluates how well my formalization coheres with philosophical literature. I formalize expected 
properties of Kantian ethics as sentences in my logic, such as the property that obligations cannot 
contradict each other. I represent each of these properties as a sentence in my logic that my system 
should be able to prove or refute. I run the tests by using Isabelle to automatically find proofs or 
countermodels for the test statements. For example, my implementation passes the contradictory 
obligations test because it is able to prove the sentence $\neg (O\{A|B\} \wedge O\{\neg A | B\})$. 
I find that my system outperforms raw DDL and Moshe Kroy's prior attempt at formalizing Kantian ethics 
in deontic logic \citep{kroy}.

As it stands, my implementation can evaluate the moral status of sentences represented in my logic. 
Given an appropriate input, my project returns a value indicating if the action is obligatory 
(its negation violates the FUL), permissible (consistent with the FUL), or prohibited (violates the FUL) 
by proving or refuting a theorem in my logic. 

A machine that can evaluate the moral status of a maxim can not only help machines better reason about ethics, 
but it can also help philosophers 
better study philosophy. I argue for ``computational ethics," or the use of computational tools to 
make philosophical progress. To this end, I present a philosophical insight about which kinds of
maxims are appropriate for ethical consideration that I discovered using my system. 
The process of building and interacting with a computer that can reason about ethics helped me, a 
human philosopher, arrive at a philosophical conclusion that has implications for practical
reason and philosophy of doubt. Thus, my system can be used in two distinct ways. First, to help
automated agents navigate the world, which I will refer to as automated ethics or machine ethics interchangeably. Second, 
to help human philosophers reason about philosophy, which I call computational ethics.%
\end{isamarkuptext}\isamarkuptrue%
%
\isadelimtheory
%
\endisadelimtheory
%
\isatagtheory
%
\endisatagtheory
{\isafoldtheory}%
%
\isadelimtheory
%
\endisadelimtheory
%
\end{isabellebody}%
\endinput
%:%file=~/Desktop/cs91r/paper/thesis_1_intro.thy%:%
%:%24=6%:%
%:%36=9%:%
%:%37=10%:%
%:%38=11%:%
%:%39=12%:%
%:%40=13%:%
%:%41=14%:%
%:%42=15%:%
%:%43=16%:%
%:%44=17%:%
%:%45=18%:%
%:%46=19%:%
%:%47=20%:%
%:%48=21%:%
%:%49=22%:%
%:%50=23%:%
%:%51=24%:%
%:%52=25%:%
%:%53=26%:%
%:%54=27%:%
%:%55=28%:%
%:%56=29%:%
%:%57=30%:%
%:%58=31%:%
%:%59=32%:%
%:%60=33%:%
%:%61=34%:%
%:%62=35%:%
%:%63=36%:%
%:%64=37%:%
%:%65=38%:%
%:%66=39%:%
%:%67=40%:%
%:%68=41%:%
%:%69=42%:%
%:%70=43%:%
%:%71=44%:%
%:%72=45%:%
%:%73=46%:%
%:%74=47%:%
%:%75=48%:%
%:%76=49%:%
%:%77=50%:%
%:%78=51%:%
%:%79=52%:%
%:%80=53%:%
%:%81=54%:%
%:%82=55%:%
%:%83=56%:%
%:%84=57%:%
%:%85=58%:%
%:%86=59%:%
%:%87=60%:%
%:%88=61%:%
%:%89=62%:%
%:%90=63%:%
%:%91=64%:%
%:%92=65%:%
%:%93=66%:%
%:%94=67%:%
%:%95=68%:%
%:%96=69%:%
%:%97=70%:%
%:%98=71%:%
%:%99=72%:%
%:%100=73%:%
%:%101=74%:%
%:%102=75%:%
%:%103=76%:%
%:%104=77%:%
%:%108=80%:%
%:%109=81%:%
%:%110=82%:%
%:%111=83%:%
%:%112=84%:%
%:%113=85%:%
%:%114=86%:%
%:%115=87%:%
%:%116=88%:%
%:%117=89%:%
%:%118=90%:%
%:%119=91%:%
%:%120=92%:%
%:%121=93%:%
%:%122=94%:%
%:%123=95%:%
%:%124=96%:%
%:%125=97%:%
%:%126=98%:%
%:%127=99%:%
%:%128=100%:%
%:%129=101%:%
%:%130=102%:%
%:%131=103%:%
%:%132=104%:%
%:%136=106%:%
%:%137=107%:%
%:%138=108%:%
%:%139=109%:%
%:%140=110%:%
%:%141=111%:%
%:%142=112%:%
%:%143=113%:%
%:%144=114%:%
%:%145=115%:%
%:%146=116%:%
%:%147=117%:%
%:%148=118%:%
%:%149=119%:%
%:%150=120%:%
%:%151=121%:%
%:%152=122%:%
%:%153=123%:%
%:%154=124%:%
%:%155=125%:%
%:%156=126%:%
%:%157=127%:%
%:%158=128%:%
%:%159=129%:%
%:%160=130%:%
%:%161=131%:%
%:%162=132%:%
%:%163=133%:%
%:%164=134%:%
%:%165=135%:%
%:%166=136%:%
%:%167=137%:%
%:%168=138%:%
%:%169=139%:%
%:%170=140%:%
%:%171=141%:%
%:%172=142%:%
%:%173=143%:%
%:%174=144%:%
%:%175=145%:%
%:%176=146%:%
%:%177=147%:%
%:%178=148%:%
%:%179=149%:%
%:%180=150%:%
%:%181=151%:%
%:%182=152%:%
%:%183=153%:%
%:%184=154%:%
%:%185=155%:%
%:%186=156%:%
%:%187=157%:%
%:%188=158%:%
%:%189=159%:%
%:%190=160%:%
%:%191=161%:%
%:%192=162%:%
%:%193=163%:%
%:%194=164%:%
%:%195=165%:%
%:%196=166%:%
%:%197=167%:%