%
\begin{isabellebody}%
\setisabellecontext{paper}%
%
\isadelimtheory
%
\endisadelimtheory
%
\isatagtheory
\ \ \isakeyword{imports}\ Main\isanewline
%
\endisatagtheory
{\isafoldtheory}%
%
\isadelimtheory
%
\endisadelimtheory
%
\isadelimdocument
%
\endisadelimdocument
%
\isatagdocument
%
\isamarkupsection{Introduction%
}
\isamarkuptrue%
%
\endisatagdocument
{\isafolddocument}%
%
\isadelimdocument
%
\endisadelimdocument
%
\begin{isamarkuptext}%
As artifical reasoners become increasingly powerful, computers become increasingly able to 
perform complex ethical reasoning. The field of machine ethics \cite{mesurvey} is attractive for two reasons.
First, the proliferation of artifically autonomous agents is creating and will continue to create a demand for 
ethical autonomous agents. These agents must be able to model and reason about complex ethical theories 
that withstand philosophical scrutiny. Second, just as automated mathematical reasoning gives mathematicians
new powers and proof-finding ability, automated ethical reasoning is a tool that philosophers can use 
when reasoning about ethics. Many contradictions or paradoxes with an ethical system may not be 
immediately obvious to the human eye, but can be easily tested using an automated theorem prover.

Modelling ethics without sacrificing the intracacies and complexities of an ethical theory is a 
challenging computational and philosophical problem. Simple and intuitive computational approaches, 
such as encoding ethical rules as constraints in a constraint satisfaction problem, fail to capture
the complexity of most philosophically plausible systems. On the other hand, it is not immediately clear
how to formalize many complex moral theories, like virtue ethics.

Computational ethics also requires a sophisticated 
ethical theory to model. Constraint satisfaction systems often default to some version of utiliarianism, 
the principle of doing the most good for the most people. Alternatively, they model basic moral 
principles such as ``do not kill," without modelling the theory that these principles originated from.
Modelling a more complex ethical theory will not only enable smarter philosophical machines, it will
also empower philosophers to study more complex ethical issues with the computer's help. The entire
field of philosophy is devoted to developing and testing robust ethical theories. Plausible machine
ethics must draw on plausible moral philosophy.

The ideal candidate ethical theory will be both philosophically interesting and easy to 
formalize. Kantian ethics, often described as ``formal," has been often floated as such a candidate \cite{powers} \cite{BL}. 
The categorical imperative, Kant's universal law of morality, is a moral rule that can be used to 
guide action. 

This project's objective is to automate reasoning about Kantian ethics. I present two different
formalizations of Kant's categorical imperative implemented and tested in the Isabelle/HOL \cite{isabelle} theorem prover.
Each formalization is modelled as an extension of Carmo and Jones' Dyadic Deontic Logic (DDL). The corresponding 
DDL formalization is then embedded in higher-order logic (HOL) and implemented in Isabelle. Section 
3.1 implements and tests the naive formalization, a control group that is equivalent to DDL 
itself. Section 3.2 examines a more sophisticated formalization inspired by Moshe Kroy's partial 
formalization of thecategorical imperative. 

I contribute implementations of two different interpretations of the categorical imperative, 
examples of how each implementation can be used to model and solve ethical scenarios, and tests that
examine ethical and logical properties of the system, including logical consistency, consistency
of obligation, and possibility of permissibility. The implementations themselves are usable models 
of ethical principles and the tests represent the kind of philosophical work that formalized ethics 
can contribute.%
\end{isamarkuptext}\isamarkuptrue%
%
\isadelimdocument
%
\endisadelimdocument
%
\isatagdocument
%
\isamarkupsection{System Overview \label{sec:overview}%
}
\isamarkuptrue%
%
\endisatagdocument
{\isafolddocument}%
%
\isadelimdocument
%
\endisadelimdocument
%
\begin{isamarkuptext}%
The goal of this project is to automate sophisticated ethical reasoning. This requires three
components. First, the choice of an ethical theory that is both intuitively attractive and 
lends itself to formalization. Second, the choice of formal logic to model the theory in. Third, the 
choice of automation engine to implement the formal model in. Section 2.1 introduces Kantian
ethics, Section 2.2 explains Carmo and Jones's Dyadic Deontic Logic \cite{CJDDL} as a 
base logic, and Section 2.3 presents the Isabelle/HOL implementation of the logic.%
\end{isamarkuptext}\isamarkuptrue%
%
\isadelimdocument
%
\endisadelimdocument
%
\isatagdocument
%
\isamarkupsubsection{Kantian Ethics \label{sec:kant}%
}
\isamarkuptrue%
%
\endisatagdocument
{\isafolddocument}%
%
\isadelimdocument
%
\endisadelimdocument
%
\begin{isamarkuptext}%
Kantian ethics is an attractive choice of ethical theory to automate for two reasons. The 
first is that Kant's writings have been a major inspiration for much of Western analytic philosophy. 
In addition to the rich neo-Kantian program, almost all major philosophical traditions after Kant have
engaged with his work. Much of Western libertarian political thought is inspired by Kant's deontology,
and his ethics have bled into household ethical thought. Deontology is seen as one of the three major 
schools of Western normative philosophy.

Understanding the ethic's potential for formalization requires understanding Kant's system. Kant argues 
that if morality exists, it must take the form of an inviolable rule, which he calls the ``categorical
imperative." He presents three formulations of the categorical imperative, as well as a robust defense
of them in his seminal work on ethics \cite{groundwork}. He argues that all three formulations are 
equivalent. 

The first formulation of the categorical imperative is the ``Formula of Universal Law." 

$\textbf{Definition}$ \emph{(Formula of Universal Law)}

\emph{I ought never to act except in such a way that I could also will that my maxim should become a universal law} \cite{groundwork}

A ``maxim" is roughly a moral rule like ``I can murder someone to take their job". ``Willing" a maxim is 
equivalent to acting on that rule. The FUL creates a thought experiment called the universalizability 
test: to decide if a maxim is permissible, imagine what would happen if everyone willed that maxim. 
If your imagined world yields a contradiction, the maxim is prohibited. Intuitively, the FUL asks 
the question, "What would happen if everyone did that?" \cite{KorsgaardFUL}.

The universalizability test makes the ``formal" nature of Kant's ethics immediately clear. Formal 
logic has the tools to universalize a maxim (apply a universal quantifier) and to test for 
contradictions (test for inconsistency). 

Kant also presents two additional formulations of the categorical imperative. 

$\textbf{Definition}$ \emph{(Formula of Humanity)}

\emph{The Formula of Humanity (FUH)
is to act in such a way "that you use humanity, whether in your own person or in the person
of any other, always at the same time as an end, never merely as a means."}\cite{groundwork}

\medskip 

$\textbf{Definition}$ \emph{(Kingdom of Ends)}

\emph{The third formulation of the categorical imperative states that "we should so act that we may 
think of ourselves as legislating universal laws through our maxims."}\cite{KorsgaardFUL}

The last two formulations are not as obviously formal as the FUL, but they can still be
modelled in logic. Because Kantian ethics presents a series of rules, a logical system can encode 
the theory by modelling each rule as an axiom.

The above outline is a brief and incomplete picture of a rich philosophical tradition. Kantian scholars
debate the meaning of each formulation of the categorical imperative and develop views far more 
nuanced than those above. For the purposes of this paper, I will adopt Kant's three original 
formulations presented above. Note additionally that Kantian ethics is widely disputed. I do not present 
a defense of Kant's ethic in this paper. My approach to formalizing ethics can be applied to other 
theories as well.  

This paper aims to formalize Kant's ethic as faithfully as possible. This is an important choice. 
While it is tempting to modify or simplify an ethical theory in seemingly insignificant way, these 
choices often have ramifications. The entire field of neo-Kantian thought 
has been exploring versions of Kant's ethical theory for centuries. I will not attempt to present a 
radically new conception of Kant's ethic, but will instead draw on philosophical expertise regarding
the content and justification of the categorical imperative.%
\end{isamarkuptext}\isamarkuptrue%
%
\isadelimdocument
%
\endisadelimdocument
%
\isatagdocument
%
\isamarkupsubsection{Dyadic Deontic Logic%
}
\isamarkuptrue%
%
\isamarkupsubsubsection{Deontic Logic%
}
\isamarkuptrue%
%
\endisatagdocument
{\isafolddocument}%
%
\isadelimdocument
%
\endisadelimdocument
%
\begin{isamarkuptext}%
Traditional modal logics include $\Box$ operators to represent necessity. In simple modal logics like S5 \cite{cresswell} 
using the Kripke semantics, $\Box p$ is true at a world $w$ if $p$ is true at all of $w$'s neighbors. 
These logics also usually contain the $\diamond$ operator, representing possibility, where
 $\diamond p \iff \sim \Box \sim p$. Additionally, modal logics support operators of propositional 
logic like $\sim, \wedge, \vee, \rightarrow$.

A deontic logic is a special kind of modal logic designed to reason about obligation. Standard deontic
logic \cite{cresswell} \cite{sep-logic-deontic} essentially replaces $\Box$ with the obligation operator
$O$, and $\diamond$ with the permissibility operator $P$. Using the Kripke semantics for $O$, $O p$ 
is true at $w$ if $p$ is true at all  ideal deontic alternatives to $w$. The $O$ operator in SDL
takes a single argument (the formula that is obligatory), and is thus called a monadic deontic operator.

 While SDL is appreciable for its simplicity, it suffers a variety of well-documented paradoxes, 
including contrary-to-duty paradoxes \cite{ctd}. In situations where duty is violated, the logic breaks down 
and produces paradoxical results. Thus, I use an improved deontic logic instead of SDL for this work.%
\end{isamarkuptext}\isamarkuptrue%
%
\isadelimdocument
%
\endisadelimdocument
%
\isatagdocument
%
\isamarkupsubsubsection{Dyadic Deontic Logic%
}
\isamarkuptrue%
%
\endisatagdocument
{\isafolddocument}%
%
\isadelimdocument
%
\endisadelimdocument
%
\begin{isamarkuptext}%
I use as my base logic Carmo and Jones's dyadic deontic logic, or DDL, which improves on SDL \cite{CJDDL}. 
It introduces a dyadic obligation operator $O\{A \vert B\}$ 
to represent the sentence ``A is obligated in the context B". This gracefully handles contrary-to-duty
conditionals. The obligation operator uses a neighborhood semantics \cite{neighborhood1}\cite{neighborhood2}, instead of the Kripke semantics. 
Carmo and Jones define a function $ob$ that maps from worlds to sets of sets of worlds. Intuitively, 
each world is mapped to the set of propositions obligated at that world, where a proposition $p$ is defined as 
the worlds at which the $p$ is true.

DDL also includes many other modal operators. In addition to $\Box$ and $\diamond$, DDL also has a notion
of actual obligation and possible obligation, represented by operators $O_a$ and $O_p$ respectively. 
These notions are accompanied by the corresponding modal operators $\Box_a, \diamond_a, \Box_p, \diamond_p$. 
These operators use a Kripke semantics, with the functions $av$ and $pv$ mapping a world $w$ to the set 
of corresponding actual or possible versions of $w$. 

For more of fine-grained properties of DDL see \cite{CJDDL} or this project's source code. DDL is quite a heavy logic and contains many modal operators 
that aren't necessary for my analysis. While this expressivity is powerful, it may also cause performance
impacts. In particular, DDL has a large set of axioms involving quantification over complex higher-order
logical expressions. Proofs involving these axioms will be computationally expensive.  Benzmueller 
and Parent warned me that this may become a problem if Isabelle's automated proof tools begin to time out.%
\end{isamarkuptext}\isamarkuptrue%
%
\isadelimdocument
%
\endisadelimdocument
%
\isatagdocument
%
\endisatagdocument
{\isafolddocument}%
%
\isadelimdocument
%
\endisadelimdocument
%
\isadelimtheory
%
\endisadelimtheory
%
\isatagtheory
%
\endisatagtheory
{\isafoldtheory}%
%
\isadelimtheory
%
\endisadelimtheory
%
\end{isabellebody}%
\endinput
%:%file=~/Desktop/cs91r/paper/paper.thy%:%
%:%10=2%:%
%:%25=6%:%
%:%37=8%:%
%:%38=9%:%
%:%39=10%:%
%:%40=11%:%
%:%41=12%:%
%:%42=13%:%
%:%43=14%:%
%:%44=15%:%
%:%45=16%:%
%:%46=17%:%
%:%47=18%:%
%:%48=19%:%
%:%49=20%:%
%:%50=21%:%
%:%51=22%:%
%:%52=23%:%
%:%53=24%:%
%:%54=25%:%
%:%55=26%:%
%:%56=27%:%
%:%57=28%:%
%:%58=29%:%
%:%59=30%:%
%:%60=31%:%
%:%61=32%:%
%:%62=33%:%
%:%63=34%:%
%:%64=35%:%
%:%65=36%:%
%:%66=37%:%
%:%67=38%:%
%:%68=39%:%
%:%69=40%:%
%:%70=41%:%
%:%71=42%:%
%:%72=43%:%
%:%73=44%:%
%:%74=45%:%
%:%75=46%:%
%:%76=47%:%
%:%77=48%:%
%:%78=49%:%
%:%79=50%:%
%:%88=52%:%
%:%100=54%:%
%:%101=55%:%
%:%102=56%:%
%:%103=57%:%
%:%104=58%:%
%:%105=59%:%
%:%114=61%:%
%:%126=63%:%
%:%127=64%:%
%:%128=65%:%
%:%129=66%:%
%:%130=67%:%
%:%131=68%:%
%:%132=69%:%
%:%133=70%:%
%:%134=71%:%
%:%135=72%:%
%:%136=73%:%
%:%137=74%:%
%:%138=75%:%
%:%139=76%:%
%:%140=77%:%
%:%141=78%:%
%:%142=79%:%
%:%143=80%:%
%:%144=81%:%
%:%145=82%:%
%:%146=83%:%
%:%147=84%:%
%:%148=85%:%
%:%149=86%:%
%:%150=87%:%
%:%151=88%:%
%:%152=89%:%
%:%153=90%:%
%:%154=91%:%
%:%155=92%:%
%:%156=93%:%
%:%157=94%:%
%:%158=95%:%
%:%159=96%:%
%:%160=97%:%
%:%161=98%:%
%:%162=99%:%
%:%163=100%:%
%:%164=101%:%
%:%165=102%:%
%:%166=103%:%
%:%167=104%:%
%:%168=105%:%
%:%169=106%:%
%:%170=107%:%
%:%171=108%:%
%:%172=109%:%
%:%173=110%:%
%:%174=111%:%
%:%175=112%:%
%:%176=113%:%
%:%177=114%:%
%:%178=115%:%
%:%179=116%:%
%:%180=117%:%
%:%181=118%:%
%:%182=119%:%
%:%183=120%:%
%:%184=121%:%
%:%185=122%:%
%:%186=123%:%
%:%195=127%:%
%:%199=129%:%
%:%211=131%:%
%:%212=132%:%
%:%213=133%:%
%:%214=134%:%
%:%215=135%:%
%:%216=136%:%
%:%217=137%:%
%:%218=138%:%
%:%219=139%:%
%:%220=140%:%
%:%221=141%:%
%:%222=142%:%
%:%223=143%:%
%:%224=144%:%
%:%225=145%:%
%:%234=147%:%
%:%246=149%:%
%:%247=150%:%
%:%248=151%:%
%:%249=152%:%
%:%250=153%:%
%:%251=154%:%
%:%252=155%:%
%:%253=156%:%
%:%254=157%:%
%:%255=158%:%
%:%256=159%:%
%:%257=160%:%
%:%258=161%:%
%:%259=162%:%
%:%260=163%:%
%:%261=164%:%
%:%262=165%:%
%:%263=166%:%
%:%264=167%:%